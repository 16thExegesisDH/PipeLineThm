% !TeX TS-program = lualatex
\documentclass{article}
\usepackage[T1]{fontenc}
\usepackage{microtype}% Pour l'ajustement de la mise en page
\usepackage[pdfusetitle,hidelinks]{hyperref}
\usepackage[english]{french} % Pour les règles typographiques du français
\usepackage{polyglossia}
\setotherlanguage{greek}
\usepackage[series={},nocritical,noend,noeledsec,nofamiliar,noledgroup]{reledmac}
\usepackage{reledpar} % Package pour l'édition

\usepackage{fontspec} % 
\setmainfont{Liberation Serif}

\usepackage{sectsty}
\usepackage{xcolor}

% Redefine \section font
\sectionfont{\normalfont\scshape\color{gray}}


\begin{document}

\date{}
        \title{Commentarii In Epistolas D. Pavli Ad Timotheum : [Aretius, Benedictu], [1580]}
\maketitle

\begin{pages} 
\beginnumbering
        
\section*{COMMENTARII }
\marginpar{[ p.8 ]}\pstart Sexto, quomodo cum seruis agendum, contra schismaticos, cum diuitibus, capite sexto, etc.  \pend\pstart Tertia pars Epilogi est capite sexto, versu vicesimo, repetit autem saepe iam iteratam exhortationem vt seductores diligenter caueat, et sanam doctrinam constanter tueatur.  \pendCAPVT I. \pstart C OMPLECTITVR salutationem, et primum locum institutionis de sana doctrina constanter tuenda contra corruptelas Legis, itemque de vero vsu Legis.  \pend
\textit{Salutatio. }\pstart Titulus honesta et apposita ad rem praesentem constat salutatione: quia enim instituere voluit quomodo vtiliter huic Ecclesiae praeesse debeat, operaepretium fuit initio amica salutatione, qua grata ipsi Timotheo exprimeret, caueretque contraria. Et ipsa natura in omni sermone primum locum tribuit salutationibus, post  \pend
\textbf{}
\section*{IN I. EPIST. AD TIMOTH. }
\marginpar{[ p.9 ]}\pstart quas demum solent inferre animi conceptionem.  \pend\pstart Tribus partibus constat haec salutatio, vt etiam in superioribus epistolis. Primum de se loquitur. Deinde, Timotheum ornat debi ta sua laude. Tertio, comprecatur eidem Domini dona, sine quibus, munus iniunctum feliciter non poterat exequi. Primum ad personam scribentis, alterum ad personam cui seribitur, tertium ad comprecationem pertinet.  \pend
\textit{Vers.1. Paulus Apostolus Iesu Christi secundum mandatum Dei saluatoris nostri, et Domini Iesu Christi spei nostrae. }\pstart De se primum loquitur pro more epistolari, secundum quem nomen scribentis primo exprimendum erat, idque adeo constanter seruatum est, vt infimi quoque maximis Regibus et Imperatoribus scribentes, sua nommna Regum titulis praeponerent: donec praepostera illa adulatio regnare in Ecclesia Christi coepit, et hypocrisis sese extulit in lo cum Dei, hinc quaesita est gloria et splendoris dignitas in nominum praerogatiua, de qua alibi:nunc ad rem. Apostolus se commendat titulis muneris sui diuinitus accepti, non quod de vocatione sua dubitaret Timotheus, sed vt rebus maior accederet dignitas. Deinde quod vbique in omnibus suis epistolis hoc  \pend
\section*{COMMENTARII }
\marginpar{[ p.10 ]}\pstart seruet Apostolus: non fuit igitur ratio cur idem hic non praestaret. Itaque in rebus grauibus, loco et tempore pium est titulis sui mu neris sese commendare et insinuare aliis.  \pend\pstart Primum suum nomen exprimit: de nomi ne Pauli diximus in Prolegomenis in Epistolam ad Rom. Certum autem est Romanum esse nomen: quando autem hoc assumpserit, qua de causa, vti, certo sciri non potest: pobabilia tamen sunt quae de his tum afferebamus. omittenda igitur harum rerum grauior disputatio, quia ad salutem nihil prodest, et curiosi est hominis anxie de his inquirere.  \pend\pstart Apostolus sonat legatum, qui superioris alicuius domini profert, non sua mandata:estque in nouo Testamento personacum summo imperio S. Sancti ad omnia quae pertinent ad functionem Ecclesiasticam ornata: adeo vt Prophetarum etiam munere fungerentur, et quidquid Euangelistae, Episcopi, presbyteri, aliique gradus praestarent, ipsi ordinarent, vbique docebant, nouas Ecclesias instituebant his praeficiebant ministros, ordinabant interdum seniores, praescribebant formulam doctrinae et morum, et Sacramentorum. Itaque nullo certo loco manebant, sed vbique respiciebant Ecclesias, nouasque ordinabant. Pastores vero in certis locis  \pend
\section*{IN I. EPIST. AD TIMOTH. }
\marginpar{[ p.11 ]}\pstart manebant, quos vocat Episcopos Apostolus, et postera aetas. Euangelistae iidem videntur cum doctoribus qui Euangelium circunferebant: sed nos eos vocamus hodie proprie Euangelistas, qui historiam Christi viuentis in his terris conscripserunt: et doctores prius scholarum est vocabulum: minister hodie solum Ecclesiae relictum est, de quibus alibi dictum est.  \pend
\textit{Secundum ordinationem. }\pstart Vocationis suae ad Apostolatum profert autorem Deum Patrem et Filium eius Iesum Christum. Deus Pater vt fons est omnis boni, ex quo manauit aeternum Dei consilium de saluandis electis. Ideo Pater dicitur saluator, quia per Christum filium suum redemit mundum : filius autem dicitur spes, quia hunc vt mediatorem nobis propositumspe amplectimur. Est igitur spes, indubitata expectatio vitae aeternae, ex Dei misericordia, quae in nos per filium mediatorem tansfunditur.  \pend\pstart Ἐπιταγὴ, autem dicitur mandatum, instructio, significans Apostoli doctrinam diuinitus esse traditam, non natam apud homines: ideo non potest non bona esse doctrina quam hic proferet. Superioris asfert mandata, hoc est, Dei mittentis et ad hanc doctrinam ligat se, et Timotheum:  \pend
\section*{COMMENTARII }
\marginpar{[ p.12 ]}\pstart ideo successores Apostolorum etiam hanc solam doctrinam proferre debent, et qui ab hac Ἐπιταγῇ  recedunt, non Apostolorum, sed apostatarum successores habendi sunt.  \pend
\textit{2 Timotheo germano filio in fide. }\pstart Alter locus est salutationis, qui laudes Timothei continet, quem ornat titulis conuenientibus, hoc est, virtutem illius celebrat. Est enim laus sermo magnitudinem virtutis exprimens, Ἔπαινος ἐστὶ λόγος ἐμφανίζων μέγεθος τῆς ἀρετῆς: et virtutem declaramus vel proferendo efficientia virtutem, vel eiusdem effecta. Virtutem efficiunt parentes, pia educatio: quanquam non semper pui parentes pios alant liberos. Effecta autem virtutis sunt facta illustria. Ideo Timotheum hic laudans filium suum facit, significams pia institutione ad hanc felicitatem peruenisse, nec infeliciter impendisse hos labores, sed germanum factum, hoc est, in fide et religione Christiana ingenuum et sincerum.  \pend\pstart Nomen Timothei, significat eum qui Deum honorat et colit. Indubie a piis parentibus, auia Loide, et matre Eunica, ei impositum fuit: non enim solenne erat adhuc, vt conuersi ad Christianismum alia assumerent nomina. Nomen pium religioni conforme est, muneri respondens, quia diuino consecratus est ministerio: quae res a Deo  \pend
\section*{IN I. EPIST. AD TIMOTH. }
\marginpar{[ p.13 ]}\pstart gubernatur, vt piorum parentum vota in impositione nominum dirigat ad felicem euentum.  \pend\pstart Germanum autem vocat, hoc est, ἄδολον, minime fucatum, simpliciter pium et vere Christianum, qui nihil aliud spectet quam cursum Christi. Magna est laus per se in integritate.  \pend\pstart Filius dicitur in religione, item affectu: quia enim opera Pauli fuit recte institutus in causa fidei, praestitit ei id quod pit parentes suis liberis imprimis solent praestare, hoc est, piam educationem et institutionem. Ideo pater illi fuit, quo sensu praeceptores discipulorum parentes dici possunt: deinde quia non minori affectu illum prosequutus est, et contra ipse filioli animo coluit Apostolum. Philipp.2. velut cum patre filius mecum seruiuit in Euangelio. Inde ibidem dicitur ἰσόψυχος, hoc est, pari in me affectu, quo ego in illum praeditus sum.  \pend
\textit{ln fide. }\pstart Particula distinguit inter filios: alii enim sunt natur ales, sic Eunicae proprie fuit filius, et cuiusdam viri Graeci qui ordinarius fuit Eunicae maritus. alii institutione patres sunt, animo et studio iuuandi:sic praeceptores, tutores, omnesque benefactores parentum sunt loco. Sic Paulus in fidei causa pater eius fuit.  \pend
\section*{COMMENTARII }
\marginpar{[ p.14 ]}\pstart Ac fides accipiatur pro tota religionis causa q.d.in religione Christiana, in Christianismo filius meus est.  \pend\pstart His laudibus illum ornat vt beneficiorum Dei eum admoneat: nihil enim hic commemorat de quo per se Timotheus gloriari possit. Deinde admonitus de his donis, eo magis inflammabitur, et excitabitur studio ea retinendi, et erat amoris testificatio quoque non contemnenda: solent enim amici in amicis libenter agnoscere virtutes, easque magi facere.  \pend
\textit{Gratia, misericordia et pax.. }\pstart Tertius locus salutationis, continens comprecationem. Porro tria comprecatur propter morem hactenus vsurpatum in superioribus epistolis, forte amoris vehementia erga Timotheum excitatus: sunt autem omnia tria conseruantia dona Dei in Timotheo, adeoque talia sine quibus munus in Ecclesia Ephesina non possit vtiliter expedire.  \pend\pstart Xάρις, gratia, est vltronea Dei beneuolentia in Christo amplectentis nos nihil tale meritos. Ratio est, quia gratis facere dicuntur, quando rogati, non pro re aliqua, aut vt sibi quicquam qui faciunt subueniant, sed ei solum cui faciunt: ideoque vltronea est beneuolentia, alioquin gratia non esset gratia. Ἔλεος , misericordia est. Apud Hebr. Chefed et  \pend
\section*{IN I. EPIST. AD TIMOTH. }
\marginpar{[ p.15 ]}\pstart Rechem, commiseratio, dolor de alieno maloquod malum ad nos quoque pertingere posse videatur. Sic quidem Philosophi definiunt, sed non satis commode ad hunc locum. Malum enim humani generis nihil ad Deum, hoc est, non ad eum pertingere poterat vt eum laederet: sed tota vis corrumpentis mali effundebatur in hominis solius perniciem: Interim Deus commiserationis sensu tactus est. Itaque sic definiamus, Misericordia est dolor de alieno malo, quod indignum videtur Deo, vt totum hominemcorrumpat: ideo malo remedium protulit.  \pend\pstart Pax est omnimoda felicitas, qua animus redditur pacatus: vult enim natura sibi bene esse vt in actionibus sit εὐπραξία, in consiliis ὀρθότης, in vitae necessariis εὐπορία, quae omnia conspirent in εὐδαιμονίαν quandam et αὐτάρκειαν, huius sensus reddit animum pacatum. Ideo in solo Deo reperitur illa pax, ob id prudenter haec omnia ad Deum Patrem et Iesum Christum mediatorem refert. Pater reconciliatur nobis in filio, ideo filius est pacificator. Hinc pii in mediis periculis et aerumnis sese ad Deum hilares referunt, quia pax illa inuiolata manet, et ex hac promanans omnimoda felicitas.  \pend\pstart Hanc securitatem animi et summam felicitatem  \pend
\section*{COMMENTARII }
\marginpar{[ p.16 ]}\pstart ignorat mundus, ridetque pios vt stupidos, ac stulte transfert nomen felicitatis ac pacis ad externa quaedam subsidia opum, gloriae, formae, roboris, sanitatis et similium, in quibus nihil nec spei, nec constantiae reperiri potest in veris animi terroribus et aerumnis.  \pend
\textit{3. Quemadmodum te rogaui. }\pstart Secunda pars epistolae, quae complectitur instructionem secundum quam Timotheus omnia in Ecclesia Ephesina debet quam vigilantissime administrare.  \pend\pstart Ac primus eius locus est, de sana doctrina retinenda ac propaganda: cuius illa hic sunt membra. Initio quomodo haec defendenda sit, et quae sint praecipuae eius labes. Deinde, agit de vero vsu Legis. Tertio, de sua vocatione et doctrinae veritate. Quarto, hinc erumpit in δοξολογίαν Dei. Quinto, redit ad Timotheum, quem nouis argumentis excitat ad vigilantiam in officio adhibendam. Haec quinque membra huius loci sunt. Nos de primo nunc agemus.  \pend\pstart Propositio est, Timotheo et omni Ecclesiarum ministro quam studiosissime curandum est, vt sanam doctrinam contra quasuis corruptelas quam constantissime tueantur. Et quia Apostolus hoc mandaturus est, prius idoneam personam ei Ecclesiae praefecit. Ideo discimus omnibus Ecclesiis (vt sint  \pend
\section*{IN I. EPIST. AD TIMOTH. }
\marginpar{[ p.17 ]}\pstart tutiores contra corruptelas) necessarium esse vt habeant manentes, et in eis haerentes miuistros idoneos. Ratio est, quia hoc Apostolus in Ephesina Ecclesia fecit, qui reliquit Timotheum manentem illic aliquandiu ministrum. Sic in Creta reliquit Titum. Tit.1. vers.5. et Actor.14. vers.21.testatur in omnibus Ecclesiis Paulum cum Barnaba constituisse pastores et presbyteros.  \pend\pstart Propositionis ratio prima hinc petatur a fineconsilii Apostolici. Hoc fine relictus est Timotheus Ephesi, ergo in hoc diligenter debetincumbere. Sic valet ad generalem propositionem. Hoc fine constituuntur ministri in Ecclesiis vt sanam doctrinam tueantur contra corruptelas: ergo in hoc debent vigilanter incumbere. Quod hoc fine sit istic relictus hinc probatur: quia proficiscens eum rogauit Vt maneret Ephesi, Act.2o. de hac profectio ne videtur loqui. Alii ad illam quae est 16. vel etiam ad hanc quae est cap.19. Act. referunt. Lector diiudicet ad quam possit aptissime accommodari. Porro quod rogauit, modestiae Apostolicae argumentum est:rogat disci pulum cui mandare poterat, sed voluit prom ptum et spontaneum illis praeficere ministrum, ideo vsus est potius suasoriis argumemtis, quam Apostolica autoritate praecipiendo. Altera ratio est a genere corruptae do\pend
\section*{COMMENTARII }
\marginpar{[ p.18 ]}\pstart ctrinae, seu forma ac materia, quae est έτεροδι- δασκαλεῖν : genus enim videtur significare omnium corruptelarum, quibus omnibus hoc proprium est quod dicitur ἕτερον, id est, alienum et peregrinum, cui opponitur γνήσιον, genuinum, quod consonum est sacrae Scripturae et conforme menti S.Sancti. Proprium sacrae doctrinae est comprehendi sacris literis: proprium falsae doctrinae, est ab eo quod comprehensum est sacris literis alienum esse. Et illud ἕτερον vel ad formam, vel ad materiam referri potest. Ad materiam si referas, tum sensus erit, ne alia doctrina, quam sacris literis expressa proferatur, hoc est, in his quaestionib subsistamus: quid.n.ad nos quando conditi sint Angeli, quis sit illorum sermo, quae inter illos politia et ordinum gradus? etc. Si ad formam referas, vetabit Apostolus hic, ne locos Scripturae noua interpretatione corrumpamus. Itaque et materia et formula explicandi ex Apostolica doctrina est depromenda: siquid aliunde afferatur siue in modo, siue in materia, id totum ἕτερον elt, et ob ld adusterinum ac pernicioium. Deinde expendatur contra quos valeat  \pend\pstart ista admonitio personam indefinito pronomine notat, τισὶ quibusdam, significans non solum animi sui modestiam in taxandis reprobis, sed spem de illorum emendatione su\pend
\section*{IN I. EPIST. AD TIMOTH. }
\marginpar{[ p.19 ]}\pstart peresse: nam mox infra desperatos ipsis nominibus traducit, Alexandrum et Hymenaeum indubie quia hi iam plane reprobi et inememdabiles facti erant. isti autem quos indefinite hic taxat, nondum tales erant redditi. Praeterea anceps est quoque hoc modo, quia tam ad auditores quam doctores referri potest. Et verbum ἐτεροδιδασκαλέῖν non abhorret ab vtraque corruptela, et pronomem τισὶ quoque. Itaque vtrinq.erat periculum huic Ecclesiae: certin erant falsi doctores Euamgelii corruptores per obseruationem Legis, et inter auditores qui faciles in hac re praeberent corruptoribus aures. Vtrique malo vna est et eadem posita medicina, vigilantia scilicet Timothei.  \pend
\textit{4 Nec attedere fabulis. }\pstart Secundi argumenti est declaratio, nam illud ἕτερον deducit in suas vicinas species, quas in illa Ecclesia metuebat. Μυθικὸν est fictum et fabulosum, cui opponitur το ἀληθὲς verum ac firmum, quod est perpetuum. His fabulis attendebent primum auditores. qui eis oblectabantur: deinde doctores qui do ctrinam ad fabulas accommodabant. De his fabulis Iudaicis vide adhuc Hebraeorum commenta in Legem, et Prophetas, quid de Adamo, de serpente, de Eua, de Angelis fabulentur notum est. Sic de Messia,  \pend
\section*{COMMENTARII }
\marginpar{[ p.20 ]}\pstart de Elia, de Leuiathan, et boue conuiuii magni in aduentu Messiae, de ouo, de Sole, Luna, pudendas babent fabulas.  \pend\pstart Alterum est genealogiae praeposterum stu dium: dicuntur autem libri seu catalogi generis certi per medios auos et attauos vsque ad primam stirpis originem, tales libri et catalogi erant necessarii ante Christi aduentum, vt constaret de certa Messiae linea. Ex his enim contexi debuit Christi Genealogia, et ex his scire poterat Paulus, quod esset de Tribu Beniamin. Post vero Christum natum abdendi erant, quia tota illa politia delenda fuit, et in Christo neque Iudaeus, neque Graecus quicquam valet, sed noua in Christo creatura: ideo diuina dispensatione hic damnantur, et olim Romae ab Imperatoribus exusti sunt. Vide locum de Genealogiis in Problematibus nostris.  \pend\pstart Tertia ratio est, a qualitate falsae doctrinae, quia ei adhaeret ἀπέραντον, infinitum: sunt autem ἀπέραντα, ἀκαταληπτὰ, hoc est, comprehendi non possunt nec vsu, nec intellectus certitudine. Omne autem tale est inutile et nocens, finita enim est et comprehensa salutis ratio: ergo omne infinitum est fugiendum.  \pend\pstart Quarto, quaestiones exhibent magis quam aedificationem Dei in fide. Argumentum fir\pend
\section*{IN I. EPIST. AD TIMOTH. }
\marginpar{[ p.21 ]}\pstart mum est ab vsu. Finis totius doctrinae est certa vtilitas, hoc est, aedificatio: sed hanc non praestat aduersariorum doctrina, est igitur stu diose cauenda. ζητήσεις et οἰκοδομίαν opponit vt contraria: ideo ζίτησις hic est inutilis quaestio, quales sunt nugae Iudaeorum, et poetarum, et bona pars narrationum in legenda patrum, ac vitis. Porro aedificatio consueta est Apostolo metaphora, qua significat profectum in melius. Quo admouemur studium et conatum proficiendi in hac vita debere esse perpetuum: qui enim aedificat, non habet absolutam et perfectam domum. Ideo vt simus aliquando vera et viua Dei templa semper elaborandum est.  \pend\pstart Est autem ille profectus Deo referendus acceptus, ideo hic Dei dicitur aedificatio, cum mali semper in peius proficiant. Et est in fide, quia fides religionis nostrae fundamentum est, quod Christo nititur lapide angulari. Ideo aedificium nostrum in fide crescit, ad quam tota doctrina est accommodanda.  \pend
\textit{5 Finis autem mandati est charitas ex puro corde et conscientia bona etc. }\pstart Quinta ratio a fine, ad quem tota doctrina est accommodanda: loquitur autem hic de fine intermedio, cum dicit finem mandati esse dilectionem. Est autem finis perfectio  \pend
\section*{COMMENTARII }
\marginpar{[ p.22 ]}\pstart rei, cuius gratia aliquid fit. Si finem igitur perfectionem accipias, tum finis Legis est Christus, vt loquitur ad Rom.10.versu 4. Sed est finis etiam intermedius, sic finis Legis est dilectio proximi. Est autem dilectio beneuolentia qua proximum amplectimur studio Dei, vel nostri quaerentes. Ad hanc beneuolentiam refertur tota lex, vt est Rom. 13.versu 9.10. et 1. Corinth.13. versu 4.5. Galat.5. versu 14. Porro alius potest hic quoque esse sensus: nam παραγγελία videtur sumi pro mandato Apostolico, hoc est, interpretandi formula, recte interpretandi Legem, quam potestatem dedit Timotheo, vt mandaret quibusdam ne aliam doctrinam docerent. Itaque libenter hic non pro Lege, quod tamen omnes faciunt, sed pro illo mandato Apostoli acciperem, vt hic esset sensus, Finis autem mandati veri de recta Legis interpretatione, ac sinceritate doctrinae est dilectio: huc enim omnis doctrina referenda est, vt prosit proximo. Ideo dilectionis gradus seu causas primarias explicat, quae sunt cordis puritas. Dicitur cor cogitationum et sensuum fons, is igitur fons purus esse debet-vt puram hinc aquam proximus haurire possit : purum dicitur cor minime fucatum, dolo carens, candide et ingenue agens.  \pend
\section*{IN I. EPIST. AD TIMOTH. }
\marginpar{[ p.23 ]}\pstart Deinde proximus est gradus, conscientia bona, hoc est, integra. Est conscientia pars memoriae, qua de actib. nostris bene aut male conscii sumus: haec quia mille testium est, ideo nos vel absoluit, vel reos agit coram Deo. Hac damnabantur Iudaei, Rom.2.V.1.qui ipsi faciebant, quae in aliis damnabant. Eadem Apololum Paulum absoluit, 1 Cor.4. versu 4. Nullius mihi conscius sum, sed in hoc non iustificatus sum. 2. Corinthiorum 1. versu 12. testimonium bonae conscientiae se habere gloriatur, hoc est, quae ipsum absoluat coram Deo.  \pend\pstart Tertio loco additur, FIDES NON FICTA, hoc est, quae viuida sit, significat fidei vocabulum late patere. Est enim quaedam vera, quae Iacobo dicitur viua : alia perosa, ornata bonis fructibus bonorum operum, et per hanc imprimis declaratur llectio. Alia est hypocritica, quae Iacobo mortua dicitur, atque illa vel impudens est, carens omni studio bonorum operum, qualis est in prophanis Christianis, qui de fide, hoc est, professione gloriantur: interim autem petulanter viuunt, vt qui hodie de fidei iustitia gloriantur dissolute viuentes. Vel est fuco tecta, qualis est in Pharisaica vita, hanc remouet hic. His tribus gradibus vera dilectio est  \pend
\section*{COMMENTARII }
\marginpar{[ p.24 ]}\pstart instructa. Argumentum est contra seductores qui in docenda lege hos fines non obseruant, ideo aberrant a dilectione, et hic toto caelo.  \pend
\textit{6 A quibus cam quidam. etc. }\pstart Sexta ratio ab exemplo et experientia. Vim habet argumentum a consequentibus et effectis. Aberrant a dilectionis proprietatibus, hinc inciderunt in vaniloquentiam: ergo illorum doctrina fugienda est, sed geminum est argumentum in verbis ἀστοχεῖν, significat scopo ordinario aberrare, cui opponitur εὐτο- χεῖν, bene collimare, scopum probe attingere. Scopus doctrinae est aedificatio per dilectionem, quo illi aberrant: igitur studiose cauendi sunt. Alterum ab effectis, MATAEOLOGIFACTI SVNT, μάταιον vitium est in omni re culpabile, vanitas, hinc ματαιολογία, sermo nis inanitas, cui nihil veri nec solidi subest. Hanc vanitatem parit is error doctorum legis: igitur cauenda illorum doctrina. Est etiam vis in verbo ἐκτρέπειν, quod sonat euertere vt igi tur a scopo legitimo aberrant, sic euertuntur a veritate, et excidunt a via regia, incidunt contra in vanitatem illam culpabilem.  \pend
\textit{7 Volentes esse legis doctores, etc. }\pstart Septimum argumeutum, quo vanitati addit ignorantiam crassam et satis impuden tem. Doctores Legis esse volunt, quo significat eos titulum hunc venari passim apud  \pend
\section*{IN I. EPIST. AD TIMOTH. }
\marginpar{[ p.25 ]}\pstart fideles, vt ipsi magni doctores habeantur, Rabbi dicantur, et magistri, quos confutat D. Christus Matth.23. Deinde cum hac ambitione coniuncta est insignis ignorantia, quae vbique quidem turpis est, turpissima autem in his qui religionis volunt esse doctores. Sic ad Gal.6. ver.13. Volunt vos circuncidi vt de carne vestra glorientur. Et paulo ante, Volunt spaciosi videri: et hoc loco ambitionem in eis notat. Ignorantia duobus modis notatur, μὴτι ἂ λέγουσι, μὴτε περι τίνων διαβεβαιοῦνται. Prius refero ad illorum explicationes de vsu Legis, ceremoniarum, circuncisionis, lotionum, ieiunorum, et similium, in quorum laudes et commendationes prolixe solebant declamare. Posterius vero, περὶ τίνων, ad locos controuersos cum Apostolo, vt de satisfactione Christi, de merito mortis eius, de abrogatione ceremoniarum, de circuncisione abolita, de vocatione Gentium ad gratiam. Vtrunque ignorant, hoc est, causas rerum tantarum nesciunt, vt cur data sit Lex, ad quid profuerit citcuncisio, quare Christus aduenerit: in summa causas veras ignorant.  \pend\pstart Inde fit vt tanquam caeci contendant de rerum vmbris, corpus ipsum interim amittunt. Ac est illa ignorantia distinguenda: in aliis enim fuit affectata et crassa, vt adhuc est in multis, qui agnitae veritati contumaciter  \pend
\section*{COMMENTARII }
\marginpar{[ p.26 ]}\pstart resistunt, quales videntur fuisse isti Nomodidascali, et bona pars Pharisaeorum apud Iu daeos. In aliis vero erat ex zelo Legis et gloriae Dei, qui in his putabant violari gloriam et voluntatem Domini, de quorum numero etiam fuit Paulus, vt mox sequetur.  \pend
\textit{8 Scimus autem Legem esse bonam, etc. }\pstart Alterum membrum de vero Legis vsu subiicitur superioribus per praeoccupationem. Nam Legis doctores statim excipiebant superiora, Paulum damnare Legem, adeoque totam abolere. Respondet igitur huic calumniae hoc loco, se id non facere, sed verum vsum, et veram interpretationem requirere: munit autem occupationem concessione, quod scilicet Lex sit bona. Hoc dixit se scire, q.d.in hac re se non opus habere doctore, scire id poterat ex institutione, qua ad pedes Gamalielis erat in Pharisaismo probe institutus. Deinde bonam sciebat, quia a Deo autore latam sciebat: non potest autem a Deo aliquid mali hominibus dari. Praeterea vsum verumet finem Legis nouit, quem aduersarii ignorabant, ideo totum hoc quond Lex bona sit, mul to melius nouit ipsis aduersariis. Deinde concessionem munit correctione, non enim absolute concedit bonam, sed cum conditione, S1 QVIS ILLA LEGITIME VTATVR, h.e. vsu bona est, sic abusu mala redditur: ideo ad vsum referem\pend
\section*{IN I. EPIST. AD TIMOTH. }
\marginpar{[ p.27 ]}\pstart da sunt omnia. legitime vtitur Lege qui mentem legislatoris in illa respicit, et ad hanc mentem omnia refert. Mens autem Dei erat, vt Lex illa paedagogus esset ad Christum, quo veniente cessaret tota illa paedagogia, Gal.3.ideo conuin cuntur abusus omnes illi doctores Legis, qui Christo iam exhibito obtrudunt circuncisionem alasq. ceremonias tamquam ad salutem necessarias. Deinde alter vsus Legis, est monstrare pecca tum, de quo imprimis etiam constare debet. Verum hunc vsum ita exercet, vt ad Christum liberatorem a peccato impellat. Hinc Legi tribuitur, iram operari, ostendere peccatum, paedagogum esse ad Christum.  \pend
\textit{9 Sciens hoc, quod iusto lex non sit posita, etc. }\pstart Explicat propositum suum de vero vsu Legis, patefaciendo ad quos non ptineat. Non pertinet ad iustum, inquit, IVSTO LEX NON EST POSI TA, h.e. quatenus iustus est nemo Lege vrgemdus est, sed quatenus peccator. Iam renati in Christo iusti sunt, imputatione beneficii Chri sti igitur ceremoniis non sunt onerandi, quod moralia attinet, nouos quoque motus in illis efficit S. Sanctus, vt sponte illa praestent quan tum homini est possibile. Itaque illis Lex non est posita. Iustus igitur hic accipitur pro illo qui Christi merito, et imputatione iam a damnatione et iure peccati absolutus est, et liber pronunciatus. Sunt enim iustorum tres ordines. Alii iusti sunt natura, talis Adam erat ante  \pend
\section*{COMMENTARII }
\marginpar{[ p.28 ]}\pstart lapsum: sed absolutissimum in humana natura habemus exemplum, filium Dei hominem Iesum Christum. Secundi ordinis sunt hypocritae, opinione iusti, de quibus dixit Christus Marc.2. Non veni vocare iustos, scilicet sua opinione, cum tales minime sint natura. Tertii ordinis sunt imputatione iusti, de quibus hic loquitur, his Lex non est posita, cum in Christo finem Legis consecuti sint.  \pend\pstart Deinde sequitur in hoc versu θέσις, quibus scilicet Lex posita sit, hoc est, in quos et contra quos eius vsus imprimis valeat. Constat locus enumeratione. Ἄνομοι sunt exleges, qui aliis praescribunt quod ipsi minime praestant: vt in imperiis tyranni solent, et in Ecclesia Christi Pontifex exlex esse vult manifesto argumento tyrannidis Ecclesiasticae.  \pend\pstart Ἀνυποτακτοὶ, qui iugum ferre nesciunt, nullius imperio assueti, contumaces ad omnem disciplinam.  \pend\pstart Ἀσεβεῖς, impii, irreligiosi. Religio studium et solicita cura est sese approbandi Deo. Ab hac cura illi alieni sunt.  \pend\pstart Ἁμαρτωλοὶ, notorii sunt peccatores, quorum species quasdam mox referet, quales erant Publicani, et alii Christi conuictores, quos Pharisaei ei loco probri obiiciunt. Ἀνόσιοι, impii, qui studium honestae vitae negligunt quo Deo inseruiant. Βέβηλοι, prophani, impuri:  \pend
\section*{IN I. EPIST. AD TIMOTH. }
\marginpar{[ p.29 ]}\pstart Βήλος dicitur purus, βέβηλος, impurus: vt erant apud Graecos quaedam sacra βέβηλα nominata, ad quae impurissimis quibusque licebat accedere. Πατραλώαι, parricidae: μη- τραλώαι, matricidae dicuntur. Quorum genus est ἀνδροφονία.  \pend
\textit{10 Scortatoribus, etc. }\pstart Persequitur catalogum impium. Scortatoribus, inquit, posita est, 2. ἀρσενοκοίταις: de his 1.Cor. 6.ver.9. 3. ἀνδραποδισταῖς, plaglariis, ἀνδραποδίζειν significat virum pedicis irretire. Significans est scelus quo liberi homines circunueniuntur, et per vim in seruitutem rapiuntur. 4. ψεύσταις, mendacibus. 5. ἐπίορκος, periuris, de quibus in locis plenius. Hactenus enumeratio est absoluta: sequitur illius καθολικὴ σύλλειψις, vniuersalis clausula, vt siquid aliud sanae doctrinae oppositum est. Opponi sacrae doctrinae est ἑτερο- διθασκαλεῖν, quod prius νομοδιδασκάλοις tribuit, cuius notae sunt τὸ μυθικόν, τὸ ἀπέραντον, τὸ  ζητητικόν, τὸ ἀνοικοδόμητόν. His coniungit iam quidquid in moribus est ἄνομον, qualia sunt enumerata iam capita.  \pend
\textit{II Secundum Euangelium, etc. }\pstart Ostendit vnde sana doctrina sit petenda, scilicet ex Euangelio. Itaque et sana Legis interpretatio hinc petenda est. Dicitur  \pend
\section*{COMMENTARII }
\marginpar{[ p.30 ]}\pstart autem Euangelium gloriae, vel quia est gloriosa doctrina, hoc est, praestantissima, nec vn quam sati laudata.  \pend\pstart Et sane praestantissima est, quia integrales duas partes sanae doctrinae complectitur, poenitentiam scilicet et remissionem: patefecit enim peccatum et ob id Legem docet, dein de monstrat peccatorum remedium et remissionem in Christo ostendit. Deinde haec capita clare et pure docet, sine omnibus typis ac ceremoniis, secus faciebat vetus Testamentum: sed potest hic etiam esse sensus, Euangelium gloriae dici a subiecto, hoc est, quod Dei gloriam imprimis illustrat, atq. huc facit cohaerens comma, BEATI DEI, q. d. explicat Dei beati veram gloriam. Sed et hinc conuincitur, quod haec doctrina sit praestantissima: ratio est, quia Dei gloriam illustrat. Sita autem est gloria Dei in claritate, amplitudine etefficacia, quibus modis luculenter sese patefecit in Euangelio. Beatus dicitur Deus, quia nos beatos ef ficit per praedicationem Euangelii, et haec beamtudo reuciata noois elt ln Euangeno.  \pend\pstart Ad hanc doctrinam igitur, vt sanam alligat Timotheum et omnes pios, cuius ministerium ipsi concreditum est.  \pend
\textit{12. Et gratiam habeo, etc. }\pstart Tertium membrum de sua vocatione disserit. Cohaeret autem pulchre superioribus,  \pend
\section*{IN I. EPIST. AD TIMOTH. }
\marginpar{[ p.31 ]}\pstart in quibus doctrinae sanitatem suo Euangelio alligauit. Ideo iam de vocatione nonnihil disserit: praedicat autem hic insignem Dei misericordiam, ideo a gratiarum actione orditur. Obseruandus est hic vocationis Apostolicae author, Christus Iesus: ideo non potest non augustum et magnificum esse munus. Grauae autem non meriti esse indicat phrasis, πιςτὸν ἡγήσατο, fidelem reputauit, scil. quum natura talis nonessem: ideo supra dixit ἑπιςτεόθην passiua forma Alterum Christi beneficium, quod cum officio nouo, nouas addit vires, quo pertinet illud τοῖς ἀδυναμώσαντι: hoc non praestare possunt homines, qui officium quidem conmittunt, fed ingenium et vires non possunt addere. Hunc ordinem Apostolus mutauit, et praedicat primum posterius, h.e. vires in docendo, qua mirabilis erat successus coniunctus doctrinae Paulinae. Hinc demũ ascendit ad eius veram causam.  \pend
\textit{13 Qui prius erant, etc. }\pstart Vt illustrissimum reddat Dei in se beneficium, se prius ad ima deprimit. Habes hic illustre exemplum confessionis peccatorum. Tria autem enormia fatetur peccata: primo, quod fuerit blasphemus: deinde, quod persecutor, ettertio calumniator. His tantis scele ribus nihil oppo nit aliud quam nudam Dei misericordiam. Ἠλεή- θην, inquit, misericordiam sum assecutus, non repudiando oblatam misericordiam. Addit tamen aliquam rationem, QVIA IGNORANS FECI,  \pend
\section*{COMMENTARII }
\marginpar{[ p.32 ]}\pstart ignorantia non crassa et affectata, et zelo Legis patriae excitata: leuius tamen peccat qui per ignorantiam labitur: attamem et is fatetur peccatum suum, ideo ne his quidem aliud subuenire potest, quam nuda Dei miseratio. In Iudaeis cum ignorantia coniunctum fuit odium bonae causae, ideo nec oblatam gratiam admiserunt. Obseruetur hic discrimen peccantium:alia nuda ignorantia, alii etiam odio peccant.  \pend
\textit{14 Superabundauit autem gratia, etc. }\pstart Iterum peccatis suis opponit, non aliquod meritum, sed nudam Dei misericordiam, vt doceat quae esset vera causa remissionis peccatorum. Obserua hic antithesin peccatorum et gratiae: quanquam illa grauissima sint, tamen gratia longe maior est, quod eleganter exprimit verbum ὐπερεπλεόνασε, quasi dicat, infinitis modis maior est gratia, potentior, amplior, validior: adeo vt maximo peccato longe maior fit. Fit autem in regeneratione, quando corda sanctorum illustrantur S.Sancto vrgente ipsos fide et dilectione.  \pend
\textit{15 Fidus est sermo, etc. }\pstart Regulam generalem tradit de remissione peccatorum, ac proponit in ea veram causam efficientem, scilicet Christum. Ac membra illius duo sunt: remedium et eius  \pend
\section*{IN I. EPIST. AD TIMOTH. }
\marginpar{[ p.33 ]}\pstart laus, a laude incipit. FIDVS SERMO, πιστὸς λό- γος , hoc est, doctrina fide digna, vel ἀξιόπιστος, digna cui credatur: ideo πιστὸν interpretatur per posterius conma, omni receptione dignus, hoc est, praestantissima haec est doctrina, et aurea sententia, digna quae scribatur aureis literis, ac publice decantetur: nulla enim linqua, nullus sermo, nulla eloquentia pro merius hanc efferre potest. Summa, omni humana commendatione superior est.  \pend\pstart Sequitur laudem ipsa sententia. IESVS CHRISTVS VENIT IN HVNC MVNDVM PECCATORES SALVOS FACERE. Haec vera est doctrina de remissione peccatorum coram Deo, ac alterum caput sanae doctrinae, scilicet de remedio peccati. Legis doctores solum peccatum monstrabant, nihil de Christi merito, in quo verum latet remedium: ideo huc trahit etiam Legis verum vsum, veramque eius interpretationem.  \pend\pstart Loquitur de Christi aduentu in carnem, quem commendat a fructu et vsu, seu fine, qui est saluos facere peccatores. Ἁμαρτωλοὶ dicuntur notorii peccatores, quales Publicani erant apud Iudaeos, fornicaria illa mulier, et similes qui peccatum suum negare non poterant : tales etiam soli apud Christum remedium receperunt, hoc est, qui suam iniustitiam fatentur. Reliqui quanuis et ipsi peccatores  \pend
\section*{COMMENTARII }
\marginpar{[ p.34 ]}\pstart sint, tamen non prodest illis Christus, quandoquidem peccatum non agnoscunt, imo sibi iusti videntur.  \pend\pstart Sonat Christus vnctum, et recte, habet enim vnctionem multiplicem. Prima humanitatis est in humilitate, secundum quam homo verus natus est. Altera est ministerii et officii propria, quae declarata est in Baptismo: tum enim nobis vt doctor commendatus est a Deo Patre. Tertia, est dignitatis, secundum quam vt rex vnctus est. Est enim non solum Pontifex noster, sed etiam rex. Hanc declarauit in die Palmarum, quando se Iudaeis tanquam rex obtulit, talis olim redibit iudex.  \pend\pstart De prima et secunda vnctione hic loquitur, VENIT SALVARE PECCATORES. Sic Ioan.3.ver.14. Vt Moses exaltauit serpentem, sic oportet exaltari filium hominis, vt omnis qui credit in illum non pereat, sed habeat vitam aeternam. Hinc dicitur saluator mundi, hostia pro peccatis nostris, redemptor, etc.  \pend
\textit{Quorum ego sum primus. }\pstart Communi regulae sese inuoluit: atque hic verus scripturae vsus est, vt nobis applicemus salutaria, quo spem habeamus de salute nostra. Obserua autem in confessione peccatorum promptitudinem: dicit, EGO SVM PRIMVS PECCATORVM, non vti\pend
\section*{IN I. EPIST. AD TIMOTH. }
\marginpar{[ p.35 ]}\pstart que primus erat, ante illum infiniti fuerant, et fuerat primus Adam ante quem erat peccatrix Eua. Quos omnes praeterit ac se primum facit non ordine, sed magnitudine: ac si dicat, maximus sum, μέγισος. Iudicio certe suo, cui mirum in modum displicebat prior illa conuersatio fuerat blasphemus, persecutor ac calumniator : maximus igitur erat in suis oculis, adeo vt reliquorum delicta parua viderentur : ratio, quia illa Paulum non premebant, cuique enim grauissimum est onus quod suis impositum est humeris. Deinde primus dicitur, quia non dubitat primus esse in ordine confitentium: vt si censura sit subeunda ad tribunal Christi, paratus est primo loco accedere, et fateri sua scelera, simulque ambire Domini gratiam. Discamus igitur exemplo Apostoli, libenter fateri Domino peccata nostra, et ex animo, sic facilior erit venia. Contra in donis celebrandis libenter nos humiliemus: sic Apostolus vt hic se maximum facit peccatorem, ita alibi minimum Apostolorum, et qui non sit dignus qui vocetur Apostolus. Interim de effectu gloria tur in Domino, quod plus aliis laborarit. In peccato igitur maximus, in ministerio minimus, vbiq. tamem inter homines magnus erat.  \pend
\section*{COMMENTARII }
\marginpar{[ p.36 ]}\pstart Repetit suum exemplum, vt aliis prosit. Ostendit autem fines Dei quos respexerit in Pauli vocatione. Duo sunt cohaerentes: prior est, vt extaret exemplum illustre diuinae longanimitatis, quae vbique homines ad poenitentiam inuitat. Rom.2.ver.4. hanc quasi totam in Pauli vocatione effudit: amplificat enim peccatum suum, vt illustris sit Dei miseratio. Et obserua Christo tribui illam summam misericordiae potestatem, ex quo Deus probetur: solius enim Dei est summo peccatori, qualis fuerat Paulus summam communicare gratiam. Alter finis est, vt pii haberent formulam iustificationis per fidem. Prior finis ad omnes pertinet etiam incredulos: nam longanimitas omnibus offertur, sic Pharaoni, Iudae, Sauli, similibusque oblata est. Sed fide illam amplecti non omnium est, ideo hic finis ad electos tantum spectat. Ac suum exemplum vocat ὐποτόπωσιν, hoc est, illustrem et viuam rei demonstrationem, quales sunt picturae Mathematicorum, quibus demonstrant suas propositiones: vocant hoc πρὸ ὀμμάτων ποιέῖν, rem ad viuum oculis subiicere. Tale exemplum est etiam Matthaei, Zacchaei, latronis in cruce, et similium: sic de Abrahae exemplo loquitur Rom.4.ver.24. Colligantur hic causae Iustificationis ac Re\pend\pstart missionis peccatorum: vera causa efficiens  \pend
\section*{IN I. EPIST. AD TIMOTH. }
\marginpar{[ p.37 ]}\pstart est Dei misericordia: organica est Christi satisfactio: formalis a parte peccatoris peccasse per ignorantiam in ignorantia, vers.13. finalis autem, extare exempla apud posteritatem verae doctrinae in lac quaestione, etc.  \pend
\textit{17 Regi autem seculorum, etc. }\pstart Quartum membrum, quo loco superiorem concludit, de gratuita remissione peccatorum per Christi meritum. Habet autemlocus hic illustrem δοξολόγιαν, in quam crumpit ἐκφωνήσει: nam animus consideratione tanti beneficii superatur, vt non habeat quodaddat. Cum igitur verbis destituitur, rerum autem pondere premitur exclamat, Obseruanda hic sunt Dei epitheta, seu potius Domini Christi, ad quem malo haec referre. Nam de filio dixit hactenus, cui tribuit facultatem remittendi peccata, summam misericordiam, longanimitatem, et sanctorum in psum spem. Ideo de filio etiam hanc exclamationem accipio. Rex dicitur quia sui populi fundamentum est et vere basis, Βάσις τοῦ λαοῦ ἀιτοῦ. Seculorum autem rex est, hoc est, omnium aetatum: et αἰώνων ante conditum mundum, quasi dicat, aeterno. Ἀόρατος dicitur, non solum respectu diuinae naturae, qua Deum nemo vidit vnquam, sed etiam nunc respectu humanae, quia ex humanis subduxit se, residens ad dextram Deiin caelestibus, vnde aliquan\pend
\section*{COMMENTARII }
\marginpar{[ p.36 ]}\pstart do visibilem expectamus. Ἄφθαpτος autem vtriusque naturae respectu quatenus erat Deus semper fuit et est ἄρθαρτος: sed quatenus homo natus fuit et deputatus ad sacrificium redemptionis nostrae, mortalis fuit homo, vereque mortuus, quo respectu corruptioni potest dici obnoxius: corruptione pro morte accepta, vsu vero et duratione satisfactionis vere est ἄφθαρτος. Nunc quoque in gloria, Christus homo vere est, eritque semper ἄφθαρτος, quales nos post resurrectionem corporum quoque erimus. SOLI SAPIEN TI DEO. Deus vere est et dicitur Christus: solus autem sapiens, quia Deus, cuius respectu omnium  hominum sapientia nihil est, mera stultitia est. Eandem phrasim vide ad Rom.16.ver.27. Basilius epistola 141. interpretatur μόνον per ἕνα, ac μόνος et εἷς  de natura diuina dici affirmat, vt euertat Arrianorum interpretatio nem.  \pend
\textit{18 Hoc mandatum, etc. }\pstart Quintum membrum, quo tandem re dit ad principalem scopum orationis, qui est, Timotheo commendare curam sanae doctrinae. Itaque repetatur hic propositio, de qua sup.ver.3. Timotheus quam vigilantissimus debet esse custos sanae doctrinae. Mandatum accipiatur de Apostoli hoc praecepto, mandantis ei conseruationem sanae doctrinae  \pend
\section*{IN I. EPIST. AD TIMOTH. }
\marginpar{[ p.39 ]}\pstart contra omnes corruptelas. Hoc illi tradit quasi depositum: in eo igitur debet esse vigilans et fidelis. Rationes igitur sunt Propositionis: Quia sana doctrina depositum est Apostolicum, hoc est, diuinitus traditum et mandatum, vt purum ad posteritatem transmittatur. Deinde specialis est a qualitate vocauonis, quia non temere, sed graui iudicio magnorum virorum ad hoc vocatus est: debetigitur operam dare, vt spei satisfaciat. Hocsensu accipio hic, Prophetias in ipsum Timotheum deducentes. Prophetia iudicium est piorum virorum praedicentium eum fore custodem fidelem sanae doctrinae.  \pend\pstart Siquaeras vbi fint talia iudicia, profero tibi ex Act.16. Vbi fratres Iconii et Lystris ei bonum de derunt testimonium: inter hoc intelligo fuisse aliquos Prophetico spiritu praeditos, qualia sunt testimonia Act.12.et 13. Et est Prophetia non solum de futuris, sed de praesentibus quoque recte statuere, et in electione ministrorum imprimis valet illa facultas recte coniiciendi, qui vtiles sint futuri Ecclesiae, qui minus: tali iudicio fuit electus Timotheus. Tertio, quia functio ista est militia quaedam: igitur vigilantem esse oportet. Militiae titulus de periculis nos admonet: ministri milites sunt antesignani, et duces auditorum : in qua re maius etiam est  \pend
\section*{COMMENTARII }
\marginpar{[ p.40 ]}
\textit{19 Habens fidem et bonam etc. }\pstart Vt pulchra possit esse Timothei militia circunscribit illius quasi limites, q.d. duo imprimis spectabis in officio tuo, fidem scilicet et bonam conscientiam, quibus seruatis gloriosa erit tua militia, vt eisdem amissis foe da et turpis sequetur fuga. Et faciunt haec eadem ad argumenta: sunt enim ista praesidia maxima quibus nititur in exequendo suo officio.  \pend\pstart Quarta igitur ratio est. Quia habes fidem, hoc est, rationem sanae doctrinae, quam hic omnino per sidem accipere conuenit, quasi dicat, nota est tibi formula sanae doctrinae, ergo illam vigilanter retine in tua Ecclesia, facilius autem conseruatur si nota sit sana doctrina: nota autem ei erat ex Apostoli institutione: notam habes, ergo facile defendes si fueris vigilans.  \pend\pstart Quinto, ET BONAM CONSCIENTIAM. Praesidium alterum est ad victoriam, et meta recte agendi in hac militia. Bona  \pend
\section*{IN I. EPIST. AD TIMOTH. }
\marginpar{[ p.41 ]}\pstart conscientia est in ministro qua nihil agit petulanter, nihil ambitiose, nihil temere: hanc siconseruat, ingens praesidium ad res vtiliter administrandi in Ecclesia. Hanc commendat a periculoso. QVAM NONNVLLI REPVDIANTES, CIRCA FIDEM NAVFRACIVM FECERVNT: ergo studiose conseruanda est consciem dae integritas, quia hac amissa, simul perit fides hoc est, formula sanae doctrinae. Id confiderent diligenter ministri.  \pend
\textit{20 Ex quibus est Hymenaeus, etc. }\pstart Rem illustrat exemplis: tristissimi sunt illi casus in ministerio, praesertim in grauibus viris quales isti fuerunt. De Hymenaeo vide infra 2. Timoth. cap. secundo, negauit resurrectionem carnis, quod illi irr epsit ex amissa bona conscientia. Sequitur hinc poena corruptae doctrinae, et sauciatae conscientiae. TRADIDIT EOS SATANAE, quae erat potestas Apostolis tradita, puniendi palam contumaces et vice politici magistratus. Postea aucta Ecclesia, et reddito magistratu politico desiit vt alia multa. Postea ad Excommunicationem etiam hoc applicarunt, ae distinxerunt in minorem et maiorem: minoris hos gradus constituerunt, priuatam admonitionem, admonitionem factam per mnistros Ecclesiae, demum submouere a  \pend
\section*{COMMENTARII }
\marginpar{[ p.42 ]}\pstart Sacramentis donec resipiscat.  \pend\pstart Maiorem vocarunt seclusionem ab omni licito actu: non multum differt ab anathemate, quod pronunciari debebat solum in eos, qui in S.Sanctum finaliter peccassent, damnationisque suae aeternae non dubia dedissent argumenta. Id poterant Apostoli, postera Ecclesia non ita facile in homines priuatos pronuntiauit, quia semper spes Ecclesiae est de emendatione, etc. Vide locum de Excommunicatione.  \pend
\textit{CAPVT II. }
\textbf{D }\pstart Vo sunt fidelium officia in publicis coetibus: audire sanam doctrinam ex verbo Dei, et cum toto coetu preces ad Dominum fundere. De priore iam actum est capite primo. Nunc ad alterum descendit, admonens Timotheum vt suis author sit in Ephesina Ecclesia, vt in precibus seruent εὐταξίαν pietati congruentem, praesertim quum Ecclesiae adhuc omnes sub infidelibus agerent magistratibus.  \pend\pstart Et quia in coetibus publicis quoque orant  \pend
\section*{IN I. EPIST. AD TIMOTH. }
\marginpar{[ p.33 ]}\pstart mulieres, adhuc tamenforte Ethnico more, parum decore sese gererent, his praescribit suum quoddam decorum, quod eis omnino seruandum est, vt in vestitu omnis desit luxus, et superbiae suspicio, in religione dominandi ambitio. Atque haec duo membra isto capite explicantur.  \pend\pstart Propositio illa sit: Timotheo, et ob id omni Ecclesiarum ministro curandum est vt preces in coetibus publicis decenter et pie fiant. Fient autem decenter, 1.si certam habeant formulam. 2. si fiant pro his, pro quibus orandum est. 3.si certo et ordinario fine. 4. si honestis de causis impellantur homines ad orandum. 5.si ab his fiant, quorum interest imprimis.6.si denique suo loco et modo.  \pend\pstart Quum igitur Apostolus de oratione admoneret, has circunstantias omnes, breuiter quidem, sed diligenter explicuit, quas nos etiam ordine notabimus.  \pend
\textit{Vers.1. Adhortor igitur ante omnia vt fiant deprecationes, etc. }\pstart Exordium secundi loci ducit ab ipsa statim propositione: curandum ei, vt preces fiant. Miscet propositioni suam authoritatem, vt hortetur ad rem tam piam: necessariam autem indicat dum hortatur vt fiant πρῶτον πάντων.i.ante omnia, non quidem sine  \pend
\section*{COMMENTARII }
\marginpar{[ p.44 ]}\pstart cura sanae doctrinae, quae vtique prima esse debet, sed cum illa semper orationis habendam esse rationem: quod ita constanter semper seruauit et adhuc seruet Ecclesia, vt nunquam ad audiendum verbum conueniat quin preces simul tum ante sanae doctrinae explicationem, tum etiam post illam, fundantur ad Dominum. Particulam igitur πρῶτον πάν- των, simpliciter ad rei necessitatem refero et coniungo explicationi sanae doctrinae: aliis etiam minutioribus conuentum occupationibus praefero ista duo. Hactenus propositio quam explicatius priùs posuimus.  \pend\pstart Deinde sequitur circunstantiarum consideratio, vt enim decenter et pie fiant, scire oportet primo quid in oratione nobis sit concipiendum. Id ostendit Apostolus quadam specierum enumeratione. Ordiamur rem a toto genere.  \pend\pstart Oratio est actio piae mentis confugientis ad Dominum pro remissione peccatorum et auxilio pie et feliciter quoque viuendi in hac terra. Species eius hic exprimuntur quatuor: δέησις orationis species est, qua animus petit a malis vrgentibus per Dominum liberari, nec refert vtrum malum hoc ad ipsum, vel ad alium, adeoque totam Ecclesiam, aut Remp. concernat. Προσευχὴ autem est qua bona largiri nobis  \pend
\section*{IN I. EPIST. AD TIMOTH. }
\marginpar{[ p.45 ]}\pstart petimus, imprimis corpori necessaria quae in oratione solet animus magis necessariis subiicere quasi secundaria in votis.  \pend\pstart Ἐντεύξις, qua de iniuria illata conquerimur, proque aduersariis simul intercedimus. Εὐχαριστία pro acceptis beneficiis Deo agit gratias, vt simul se Deo approbet grati animi fignificatione. Haec diuisio ex Apostolo hic colligitur, ac presenti instituto facile accommodari potest: nam ἐντεύξις proprie pro Ecclesiis fiebat, quae in persecutione erant aut tandem veniebant: δέησις pro imperio, pro magistratu singularum Ecclesiarum: προσευχὴ ingenere pro beneficiis accipiendis, quemiadmodum εὐχαρισςτία pro acceptis gratias agit. Aliam diuisionem vide in Locis nostris.  \pend
\textit{Pro omnibus hominibus. }\pstart Alterum est scitu necessarium, pro quibus orandum sit. Fideles primi ex Iudaeis vix admittebant in communionem salutis Ethnicos, vana adhuc laborantes persuasione de Iudaeorum praerogatiua in Messiae beneficiis. Ideo facile etiam pro aliis infidelibus non orabant: hunc igitur scrupulum hic iubet deponere, cum docet pietatis esse pro omnibus hominibus orare. Atque hac re Christiani superant alias omnes sectas. Iudaei ad huc pro Israële fundunt preces, reliquos  \pend
\section*{COMMENTARII }
\marginpar{[ p.46 ]}\pstart omnes execrantur in suis precibus : idem faciunt Turcae, soli Christiani pro infidelibus orant. Itaque pietatis nostrae, adeoque verae proprium est pro omnibus hominibus orare, quod accipio non solum de specie in genere, sed de personis, hoc est, nullum esse debere mihi tam abominabilem hostem pro cuius salute non debeam Deum rogare.  \pend
\textit{2 Pro regibus et omnibus in eminentia constitutis: vt placidam ac quietam vitam degamus cum omni pietate, etc. }\pstart Totum posuit, id iam deducit ad eam speciem quae poterat esse controuersa. Orare pro domesticis fidei non iubet, quia sine monitore hoc faciendum esse nouerant: verum pro impio Caesare, pro infidelibus magistratibus orandum esse non satis illis notum fuisse apparet, de hoc igitur admonet. Reges accipio hic imprimis imperii Romani administratorem Caesarem, deinde ei subiectos monarchas, et si qui erant extra imperii fines noti et ignoti monarchae. In eminentia vere constituti sunt Proconsules prouinciarum, Praetores, singularum etiam vrbium proprii magistratus et officiarii: qui omnes precibus sunt Domino commendandi. Primum hic astruitur magistratus, vt cuius non possit non diuina esse ordinatio, cum pro his  \pend
\section*{IN I. EPIST. AD TIMOTH. }
\marginpar{[ p.47 ]}\pstart Deus sit rogandus. Deinde impios non indignos esse nostra intercessione, item de nullis reprobis facile desperandum cum indiscriminatim pro his omnibus orandum sit. Complemus hoc modo Christi mandatum, vt hosti benefaciamus, praecipientis.  \pend
\textit{Vt tranquillam et quietam, etc. }\pstart Tertio loco indicat quo fine sit orandum. Primo vt tramquillam vitam degamus, hoc est, pro pace publica Reip. nam sub illius alis tuti possunt esse fideles. Ideo rogandus est Deus, vt semper aliquod hospitium piis excitet, vbi tuto illi seruire possint. H'ρεμία, tramquillitas est, qualis in mari caelo sereno apparet, vbi mare placidum est, et sibi relictum ab externis causis motuum liberum : ἠσυχία autem tramquillitas animi, vbi intra se etiam nihil habet quod perturbationi dare possit occasionem: pax igitur tutissima est, vbi ab externis et domesticis seditionibus tuti sumus. Εὐσέβεια, certa est colendi Deum ratio Σεμινότης autem honestae vitae ac morum. Itaque ad hos duos fines acconmodari debent preces piorum ad conseruationem verae pietatis, et ad honestatem vitae: et quia ad hoc etiam pro magistratibus orandum est, patet magistratum vtriusque custodem esse, verae scilicet doctrinae, et honestatis vitae, in vtraque conseruanda est subditis pax. Similia habes  \pend
\section*{COMMENTARII }
\marginpar{[ p.48 ]}\pstart Ierem.2.9. vbi iubet pro statu et pace Babylonis orare Iudaeos, vt et ipsi pacem habeant.  \pend
\textit{3 Hoc enim bonumn est,etc. }\pstart Quarto loco admonet de causis impellentibus ad orandum. Colligendae sunt hic rationes propter quas ad preces compellimur: nos ex Apostolo eas indicabimus. Primum igitur rei honestas nos ad id debet mo uere. Καλὸν dicitur natura honestum, pulchrum et vtile: orare pro omnibus natura est honestum, igitur studiose est faciendum. Deinde, quia est Deo gratum et acceptum: debent autem pii tanquam filii Dei imprimis respicere ad Dei voluntatem, vt omnia sua ad hanc accommodent. Tertio, quia Deus omnes homines vult saluos fieri: ergo nos pro omnium salute et conuersione orare debemus, cur enim non orabis pro eo quem Deus velit saluari? Et omnes homines non solum de tota specie accipio, hoc est, vt ex ommb. nationib. aliquos velit saluari, sed de indiuiduis omnib. vult. n.omnes saluos fieri reuera, sed conditione hac vt poenitentiam agant, et se conuertant ad Christum filium Dei: sic voluit saluari etiam Cainum, Pharaonem, Iudam, verum quia in desperatione manserunt, contempta omni pia admonitione, merito damnantur. Sic interpretatur hanc clausulam D. Ambrosius, vt dicat sub\pend
\section*{IN I. EPIST. AD TIMOTH. }
\marginpar{[ p.49 ]}\pstart esse conditionem, secundum quam prorsus nemini salutem negat Deus, ea omnino necessaria est subintelligenda. Quarto, non solum omnes vult saluari simpliciter, sed in hac vita quoque omnes vult peruenire ad co gnitionem veritatis. Veritas Christus est, ipso dicente, Ego sum via et veritas. Itaque ad Christi cognitionem omnes vult perue nre:in hac latet etiam verus Dei cultus, vera adorat io, sui cognitio, peccati, beneficiorum Christi, et ratio vera salutis. Haec tanta bona Deus omnibus offert sine inuidia aut personarum respectu: quod vero non omnes ea bona amplectuntur, non Dei offerentis, sed hominum est repudiantium.  \pend\pstart Quinto, VNVS TANTVM EST DEVS, scilicet omnium nationum et hominum: ergo pro omnibus ad vnum Deum est orandum. Videtur respicere Apostolus ad Iudaeorum falsam persuasionem cupientium Gentes excludere a Dei gratia: sic Rom.3.inquit, Num Iudaeorum saltem Deus est, an non et Gentium? vtique et Gentium, igitur salus Gentium Deo quoque curae est, et ideo pro his infidelibus adhuc serio est orandum. Sexto, VNVS ETIAM SOLVM EST MEDIATOR : ergo ab hoc non sunt ar cendae Gentes. Vt enim vnus tantum est Deus omnium hominum: ex quo efficitur  \pend
\section*{COMMENTARII }
\marginpar{[ p.50 ]}\pstart curam omnium ad hunc vnum pertinere, sic quia vnus tantum est mediator: consequens est, omnium peccatorum causam ad hunc vnum esse deferendam. Deus ille vnus omnes homines damnandi habet ius, quia in illum omnes peccarunt, nunc si multi essent mediatores, alii aliis modis possent reconciliari, nec esset necessum Christianos pro omnibus orare: at solum vnus est mediator, ille Christus Dei filius a quo pii dicuntur Christiani, ergo vt omnes ad hunc vnum mediatorem conuertantur pro illis orandum est. Ac mediatoris officium in Christo accipio, de tota persona Christi, hoc est, vt sit mediator respectu vtriusque naturae, diuinae et humanae. Sic interpretatur locum hunc Chrysostomus, et alii, etc. Vide Locos.  \pend
\textit{6 Qui dedit semetipsum, etc. }\pstart Explicat plenius hanc causam formalem salutis nostrae: λύτρον dicitur redemptionis pretium: ἀντίλυτρον vero cum alius soluit quod reus non poterat, vt in bello cum caput capite, vita vnius alterius morte redimitur, etc.  \pend\pstart Ac latet hic nouum argumentum, Cur pro omnibus sit orandum. Ratio est, quia D. Christus pro omnibus passus est. Cur igitur dubitares pro omnibus orare , quum Christus non dubitarit pro omnibus |mori?  \pend
\section*{IN I. EPIST. AD TIMOTH. }
\marginpar{[ p.5 ]}\pstart Mortuus est autem pro omnibus, vt hic loquitur Apostolus, nec impedit hoc quod non omnes mortis beneficium amplectuntur. Quia suo malo et vitio naturae suae oblatam gratiam aspernantur, non aliqua Dei culpa. It aque nec nos pigeat pro omnibus orare etiam pro perditissimis nebulonibus, et tyrannis, quanuis nihilo reddantur meliores 1e ue nostris.  \pend
\textit{Temporibus propriis. }\pstart Accipio defignata tempora, quando scili cet Christus nasci, pati pro nobis, et praedicari in toto orbe debuit: sic loquitur ad Tit. Leverte\pend
\textit{7 In quod ego constitutus sum etc. }\pstart Digressiunculam continuat etiam sui muneris annexa mentione. Sumatur hinc octaua ratio, Cur pro impiis Ethnicis sit orandum. Ratio est, quia Deus illi singularem Apostolum constituit, per quem conuerterentur:ergo digni sunt pro quibus oretur. Item dignos iudicat Apostolatu peculiari: et Apostolus Paulus dignos iudicat pro quibus tot obeat labores:ergo nos dignos iudicemus no stris precibus.  \pend
\textit{8 Volo igitur viro orare, etc. }\pstart Redit ad institutum, ac infert πρoς τὴν θέσιν rationes suas. Est quinta orationis circunstantia, a quibus orandum sit. A viris  \pend
\section*{COMMENTARII }
\marginpar{[ p.52 ]}\pstart simul ac mulieribus, hoc est, a toto coetu consentienter orandum est pro omnibus hominibus. Ac hac diuisione sexus includit omnes fideles omnium etiam aetatum, vt sit sensus, vt pro omnibus orandum sit, ab omnibus faciendum est : neminem excludit Deus in orando. Omnium pater est et Deus: omnibus igitur liberum concessit aditum.  \pend\pstart Sexta orationis circunstantia est de loco idoneo adorandum. Ex disciplina Mosis ac Prophetarum vnus certus erat cultui diuino deputatus locus, quem postea confirmauit, templi Hierosolymis constructi maiestas, adeo vt solum illic Deum adorare liceret, quo qui peruenire non possent, saltem versus Hierosolymam conuersi orarent, quod adhuc Iudaei hodie seruant. Hinc Ioannis capite 4. Quaerit mulier illa de loco orationis, Iudaei dicunt oportere Hierosolymis adorare, patres autem in monte Samariae adorasse: id discrimen valuit vsque ad natum Christum.2. Nodum hunc soluit ipse Messias, Ioannis quarto versu vicesimo primo: deinceps sublatum est, et hic etiam testatur Apostolus in omni loco pium est Deum inuocare: locus enim nec addit, nec adimit pietati.  \pend\pstart Septima, Quomodo orandum sit, qui modus proprie ad pietatem pertinet. Huc re\pend
\section*{IN I. EPIST. AD TIMOTH. }
\marginpar{[ p.53 ]}\pstart fero quod hic scribitur, TOLLENTES PIAS MANVS ABS QVE IRAET DISCEPTATIONE. Tollere manus in oratione gestus est animi ad Deum respicientis, qui sic est comparatus, vt in omni statu externis gestibus prodat affectum suum, laetitiam risu et vultu hilari prodit: tristitiam contra moerore, submissione manuum complicatione, et similibus gestibus. Ex quo discimus, a pietate non esse alienum in oratione ardenti gestus externos addere conformes affectibus animi. Hinc alii prosternunt se humi, et Iudaei cinere et sacco se deformant in argumentum doloris, vestes lacerant. Nos genua flectimus, caput nudamus, complicamus manus, contundimus pectora. Patribus nostris per Heluetiam moris fuit expansis manibus orare, quod indubie a maioribus acceperunt: et in summis periculis belli adhuc ante conflictum sic orant.  \pend\pstart Sic aliae aliis nationes vtuntur gestibus, quos dictat et exprimit animi affectus, nec facile leges admittit perturbatus in orando animus.  \pend\pstart Hoc saltem memoratu dignum est, qualesquales sint affectus, vt manus sint purae, vel, vt inquit, sanctae, hoc est, lotae in innocentia, opponitur enim sanguinolentis, auaris, vsurariis. Deinde manus sint etiam alienae ab ira:  \pend
\section*{COMMENTARII }
\marginpar{[ p.5 ]}\pstart quanquam ad totum hominem referatur forte melius, q.d. abesse debet studium vindictae, ne contra aduersarium tuum ores vt Deus illum male perdat, sed sanam ei det mentem: aut absit ira quae indignatur Deo per impatientiam: nam solent homines fracti mora Deo irasci quod non liberet quando ipsis videtur. Vtrunque vitium abesse debet a pia oratione. Tertio, absit etiam disceptatio, hoc est, dubitatio, quae promissionibus Dei detrahit, et incipit disceptare in vtramque partem, vt ostendit Apostolus Rom.2. ver.15.  \pend
\textit{9 Similiter et mulieres, etc. }\pstart Alter locus huius capitis de decoro a mulieribus seruando in publicis coetibus. Accommodat autem quaestionem ad duo. Primum, quales debeant esse in precibus. Deinde, in reliqua conversatione.  \pend\pstart De precibus ait, Mulieres similiter debere orare: sic enim accipio illa, VOLO EAS SIMILITER ORARE: aequantur viris studio et ardore orandi, vt conditionibus illis omnibus viris pares esse debeant. Singularia autem sunt, quae afferre debent de suo decoro, modestia in vestitu, et modestia in discendo: duo enim hic taxat vitia in muliebri sexu. Superbiam in habitu, natura enim mulier est animal φιλόκοσμον. Deinde studium do\pend
\section*{IN I. EPIST. AD TIMOTH. }
\marginpar{[ p.55 ]}\pstart minandi, vult enim videri tanquam et ipsa sapiat.  \pend\pstart De priori inquit, IN AMICTV HONESTO, scilicet orare illas volo: καταστολὴ dicitur stola qua totum corpus tegitur vndique, ne quid indecorum compareat. Interim eis concedit mundiciem, non enim vult sordidas, sed in mundicie pietatis ratio habenda est.  \pend\pstart Deinde, ADSIT EIS αἶδος, PVDOR, oratio elegantissima est, quae pudicam monstrat mulierem, ne sit effrons, ne deponatur rubor, sed potius declaretur  \pend\pstart Tertio, adsit sobrietas, quae custos est pudoris, Graecis dicta etiam ὅτι σώζει τὴν φρώνησιν- σιν, sapient iam seruat sobrietas in omni quidem sexu, imprimis tamen in muliebri.  \pend\pstart Sequuntur deinceps notae luxus, vt est πλeγμα, hoc est, crinium ornatus tortus, quum instar torui incuruatur pilus, sub quo omnis crinium fucus facile potest intelligi; sed potest πλέγμα quoque accipi de ornatu vestium, vnde nos dicimus plegii: luxuriat enim muliebris vestitus in primis hoc modo, vt fimbriis varia plicatura addatur: plegii vndenn Zũ vndenn, in mittenn vff den roskem obenn sicleit mitt siluneem: also vff den ermtenn, golleamm: sint πλέγματα, schnur spi zlii, et similes nugae, tamen adauctum est, hoc loco nobis dici,mu blegii, vomi belegenn non a Graeco;  \pend
\section*{COMMENTARII }
\marginpar{[ p.56 ]}\pstart cui tamen affine est πλέγμα, et Graecum πλέ- γμα de reticulo crinium et fuco crispandi pilum proprie dici.  \pend\pstart Auro ornari mulierem piam non decet, in eo enim luxus est superfluus: damnatur hic luxus aurum intexendi vestibus, annulorum, torquium, fibularum, amuletorum, quo genere nos insigniter peccamus: non solum matro nae, quae catenis, annulis, reticulis aurum produnt, sed viri etiam. Alii auribus infigunt, alii potant, edunt, nihil non in perniciem et luxum auri comminiscuntur.  \pend\pstart Vnionibus eadem imminet turpitudo ab hominibus. Cleopatra insignem luxum vorandis vnionibus, et Antonius prodiderunt: nostrae matronae crines et reticula eis ornant. VESTITV PRECIOSO, quae per partes et species damnauit, nunc in genere perstringit, q.d.in omni vestitu qualisqualis sit cauendum esse luxum: sic rusticus in suo lacero sagulo decorum seruet, ciuis in toga vrbana, Senator, Consul, Nobilis, in summa singuli decorum seruent. Eadem ratio est in vestitu muliebri, de quo proprie hic loquitur Apostolus.  \pend
\textit{10 Sed quod decet, etc. }\pstart Iterum affirmatiue loquitur: summa, τό πρέπον seruandum est in omni habitu. Decorum autem hoc regatur a pietatis regula, non iudicio humano. PER OPERA BONA, addit  \pend
\section*{IN I. EPIST. AD TIMOTH. }
\marginpar{[ p.57 ]}\pstart Apostolus, quia operis rationem habet, quocunque tandem aedificatur proximus.  \pend
\textit{11 Mulier in silentio discat, etc. }\pstart Alterum membrum huius institutionis de muliebri decoro in conuersatione Ecclesiastica. Praeceptum repetatur ex 1.Cor.14. quo mulieri docere non permittitur in Ecdesia. Hic ostendit praecipuas virtutes domesticas in causa religionis, quales sunt ἡσυχία, modestia in discendo sine scrupulositate: ὑπο- ταγὴ studium parendi legitimo imperio.  \pend
\textit{12 Mulieri autem docere, etc. }\pstart Contra studium dominandi agit: quod declarant mulieres vel praesumptione, quando doctrices esse volunt: vel priuatim plus sapere viris suis sibi videntur, vtrunque prohibet. DOCERE NON CONCEDO, inquit, NEC αὐθεντεῖν, hoc est, imperiose agere in virum. Hinc authentica dicuntur quae habent authoritatem, vt libri authentici. Nec frangit regulam hanc, quod aliquando mulieres prae fuerint rebus, vt Debora, Anna, et similes, quia extraordinaria exempla non framgunt legem communem.  \pend
\textit{13 Adam enim primus, etc. }\pstart Confirmat cur mulieri non sit permittendum imperium : ratio est, ab ordine creationis. Primo, Adam conditus est, ergo fa\pend
\section*{COMMENTARII }
\marginpar{[ p.58 ]}\pstart pientior et prudentior Eua: valet ratio in eadem specie, alioquin infirma, cum pecudes et plantae ante Adamum conditae sint: sed in eadem specie videtur Deus praestantiora primo loco condidisse, vt inter homines Adamum: Eua hoc gradu inferior est.  \pend
\textit{14 Et Adam non est deceptus, etc. }\pstart Altera ratio ab euentu, nam primi hominis imperium mulieri concessum infelix fuit: igitur deinceps mulier ab imperio sibi caueat. ADAM NON DECEPTVS EST, scil. primo loco, alioquin vtiq. deceptus est, sed secundario loco, cuius culpam fert Eua.  \pend
\textit{15 Seruabitur autem etc. }\pstart Respondet obiectioni tacitae: nam posterius argumentum sumptum erat a lapsu, et significationem habebat poenae: nam ob illam infelicitatem consilii Euae, poenae loco mancipata est in seruitutem quodammodo viri. Nunc cogitabit mulier, ergone misera eris semper et damnata ob infelicitatem illam? Respondet Apostolus, casus quidem miserãdus est qui dominandi studium merito possit dissuadere: interim tamen paratum est huic quoq. remedium, commune scil. in Christi merito. Deinde poenae loco impositum, quas poenas si patienter ferat, saluti suae consulit: illud posterius est τεκνογονία, sic infra 5.  \pend
\section*{IN I. EPIST. AD TIMOTH. }
\marginpar{[ p.59 ]}\pstart cap. de viduis iunioribus loquitur, quas vult nubere, liberos procreare, et domum custodire. Addit hinc, SI MANSERINT IN FIDE: mulieres scilicet: nam collectiue mulieris voce totum sexum intelligit. Heterosis numeri effecit vt hoc quidam ad filias referrent. Sed de mulieribus loquitur, SI MANSERINT IPSAE IN FIDE, hoc est, verae doctrinae professione, et dilectione proximi, benefaciendi studio, amore Dei, et sanctificationehoc est, vitae innocentia qua ornatur professio: modestia vero est σεμνότης illa ciuilis, quaein habitu et excessu elucet.  \pend
\textit{CAPVT III. }
\textbf{G }\pstart VBERNATIO Ecclesiastica docet quodnam sit omnium ordinum officium in coetu fidelium, quae vt recte administretur respiciendum est ad doctrinam et disciplinam: in doctrina sanitas requiritur, sana autem est quae habet congruam loquendi interpretationem conuenientem in Euangelio: disciplina autem mores doctrinae sanae conformes requirit, vtrunque procurare debent praesides Ecclesiarum, ac de sana  \pend
\section*{COMMENTARII }
\marginpar{[ p.60 ]}\pstart doctrina admonuit primo capite:mox ad disciplinam de εὐταξία tramsiuit capite 2. nunc vero tertio loco admonet de ipsis personis per quas vtrunque in Ecclesia confirmari et vtiliter administrari debet: monet igitur de his qui eligendi sunt, vt discant iuniores quasnam virtutes ad ministerium sacrum afferre debeant. Deinde vt Timotheus et omnes quibus Ecclesiarum cura est demandata, sciant quales admittere oporteat, aut quinam tanquam indigni sint arcendi ab hoc munere.  \pend\pstart Propositio sit, Timotheo, et omnibus, qui cura Ecclesiarum incumbunt, curandum est Vt semper adsint idonei homines per quos sana doctrina et tota pietatis cura administretur: ad quam functionem non nisi aptos, et canoni huic respondentes maiori ex parte admittere debent.  \pend\pstart Partes sunt: initium et finis συσατικὰ sunt. Commendat enim initio munus hoc a sua dignitate, quo aditum facit ad praeceptionem: sic finis quoque commendat curam hanc Timotheo, vt iam grauissimam et dignam in quam diligenter et accurate incumbat. Media sunt Canonica, hoc est, praescribunt formam, quae in electionibus seruanda est. Ac primum de ipsis ministris. Deinde de ministrorum diaconis. Tertio de ipsorum familia et vxoribus. Quarto de altero diaconorum ge\pend
\section*{IN I. EPIST. AD TIMOTH. }
\marginpar{[ p.61 ]}\pstart nere, quod eleemosynis praeest. Sunt igitur sex huius capitis partes. Principium συστατι- κὸν: Deinde, idea boni ministri: Tertio, de virtutibus diaconi: Quarto, de virtutibus vxorum, ministrorum, et diaconorum: Quinto, de diaconis eleemosynariis: Sexto, epilogus adhortatorius ad ipsum Timotheum.  \pend
\textit{Vers.1. Fidelis sermo, etc. }\pstart Aditus ad nouum locum de personis ad defensionem sanae doctrinae et disciplinam morum ordinandis: loquitur autem de praestantia ministerii. Propositio est, ad ministerii functionem ideo homini operam suam loco et tempore offerre licet: quod liceat probamus, quia appetere concedit et desiderare: non autem concedit ambitionem, igitur loco et tempore solum concedit, et solum idoneo, quia non idoneum enciuuunt iequentesnote.  \pend\pstart Vocat λόγον πιστὸν, hoc est, ἀξιόπιστον, dignum cui fidem habeamus, vel αὐτόπιστον, cui per se assentiamur, sine operosa demonstratione: et re vera in Ecclesia Christi sententia haec est extra controuersiam idoneo homini operam suam deferre licere ecclesiae, saluo illius iudicio.  \pend\pstart Επισκοπὴ, munus docendi dicitur, hinc Episcopi, qui hodie ministri et simpliciter concionatores dicuntur: nomen habent ab inspi\pend
\section*{COMMENTARII }
\marginpar{[ p.62 ]}\pstart ciendo, quia attendunt quid singuli doceant, et agant, et viuant in Ecclesia singuli ordines. Recte igitur dicitur nomen laboris ac diligemtiae. Sumptum est vocabulum a Graeco loquendi vsu, nam Attici vocabant Episcopos, magistratum, reficiendis publicis aedificiis praefectum, legatos quoque in prouincias missos, quales Lacones vocabant ἁρμοςτὰς, et in certaminibus publicis iudices quidam dicebantur Episcopi. In Ecclesia Christi ministri sic appellati sunt a vigilantia et cura. Postea vox coacta est ad meram dignitatem et pompam neglecta cura docendi. Videant igitur ne sint κατασκοποὶ potius et ὑποσκοποὶ, quam Episcopi.  \pend\pstart Hoc munus appetere licet: ὀρέγεσθαι dicitur, quemadmodum cibum et potum affectamus natura et necessitate compulsi: sic deferre operam suam licet, afferenti mediocris industriae conscientiam cum videt Ecclesiam opera boni pastoris indigere sic licet operam suam offerre ad alios honestos labores : sic interpretor verbum etiam omiθυμεῖν.  \pend\pstart Dicitur autem καλὸν ἔργον, BONVM OPVS, hoc est, ex ordine τῶν καλῶν, quia a Deo est institutum. Deinde καλὸν, quia Ecclesiae summopere est vtile et necessarium: καλὸν igitur est etiam, quia χρήσιμον, et deniq. quia arduũ  \pend
\section*{IN I. EPIST. AD TIMOTH. }
\marginpar{[ p.63 ]}\pstart ac difficile, quemadmodum dicere solemus, Difficilia quae pulchra:nam Episcopi primi rapiebantur ad supplicia in persecutionibus: sic in pace illis incumbebat cura sanae doctrinae, disciplina morum, et alia infinita.  \pend
\textit{2 Oportet igitur episcopum irreprehensibilem esse, vnius vxoris maritum, etc. }\pstart Duas boni et idonei ministri virtutes recenset. Primum καταρατικῶς, deinde etiam ἀποφατικῶς. Ac primum ait oportere illum esse ἀνεπίληπτον, infra ad Tit.I. vocat ἀνέγκλητος, quod significat hominem qui criminis non possit conuinci, vt sit apud Ecclesiam extra crimen: non enim vult sine peccato esse, tum enim nullus idoneus esse posset minister, sed atroci crimine vacare. Deinde apud homines, nam ad Deum si rem conferamus, vtique omnes sumus ἐπίληπτοι. Deinde VNIVS VXORIS VIRVM, hoc est, non polygamus esse debet, vt tum omnes Orientales, imprimis Iudaei plures alebant vxores: hoc vitium in ministro non esse ferendumostendit. Itaque damnatur primum hic polygamia: deinde concubinatus, vbi nulla legitima habetur, vt multae aut variae concubinae. Praeterea si vna legitima, aliae accedunt concubinae. Praeterea damnatur etiam coelibatus, vt fine illo non possit esse idoneus  \pend
\section*{COMMENTARII }
\marginpar{[ p.64 ]}\pstart minister non enim requirit ἄγαμον, sed contentum sua legitima vxore. .  \pend\pstart Constat igitur vxoratum posse ministrum esse bonum, sed quaerunt de vna vxore: alii vna defuncta non concedunt secundam, sed quia mors tollit vinculum matrimonii, desinit enim per mortem esse vxor : manifestum est vnam dici vxorem, quae vno tempore vna semper est, ne plures simul et eodem tempore legitimae aut illegitimae habeantur. Qui de vna Ecclesia vnius episcopi interpretantur, nihil dicunt ad praesentem locum idoneum, eoque nomine magis ineptiunt, quod vnus episcopus plura semper inuadit beneficia, vt ne sic praecepto Apostoli satisfaciant.  \pend\pstart Tertio, νηφάλαιον OPORTET ESSE, hoc est, sobrium, quae est non tam virtus quam virtutis proprietas: reprimitur hac proprietate nimius potus.  \pend\pstart Quarto, σώφρονα, temperantem, moderatio est cibi et potus, adeoque omnium cupiditatum: temperans facilius acquiescit pietati. Hinc σωφροσύνην dicitur εὐσεβείας γείτων, conseruatrix quoque rationis est, vt ministro imprimis sit necessaria.  \pend\pstart Quinto, κόσμιος dicitur condecens habitus, qui professionem ornat, ne militaris sit vestitus, vel ad iustitiae studium referatur vt respondeat ἐγκρατῇ, ad Tit.1. sed malo de ornatu  \pend
\section*{IN I. EPIST. AD TIMOTH. }
\marginpar{[ p.65 ]}\pstart sui decori accipere, non sordidum vult, nec indecenter vestiri, sed vt ex habitu cognosci possit.  \pend\pstart Sexto, φιλόξενος, hospitalis: cum enim Ecclesiae bona ad ministros deferrentur, vt prius ad Apostolorum pedes, merito hospitales esse. debent : peregrini enim hinc et pauperes alendi erant. Nunc multo magis tales esse debent constitutis stipendiis ordinariis, ex quibus et ipsi et familia viuere possunt, pauperum cura publica in pium magistratum reiecta. Septimo, aptus ad docendum, διδακτικὸς, instructus ratione docendi facili et perspicua, qualis est methodica, vt ostensa rei natura ac Vtilitate ad illam possit hortari, vt contra proposita scelerum nota ab his potenter possit dehortari: sic in defendenda sana doctrina didacticus est si pure et perspicue veritatem proponat prolatis veris causis, errores perspicue detegat et confutet, aduersarios conuincat, haec δείνοσις donum S.Sancti est.  \pend
\textit{3. Non vinosum, etc. }\pstart Hactenus affirmatiue dixit qualis debeat esse minister, nunc ἀποφαρικῶς eandem rem persequitur. Primum igitur non debet esse πάροινος, vinosus: species est vel proprietas sobrietatis, de qua iam diximus. Significat autem πάροινος, vino quodammodo affixum, assidentem ei semper: et παροινεῖν est ex vino  \pend
\section*{COMMENTARII }
\marginpar{[ p.66 ]}\pstart lasciuire, peccare, quemadmodum Nerotemulentus saltat inter scenicos, libidinatur, corrumpit puellas et matronas, canit, etc. Sic Alexander inter pocula occidit Clitum et alia nefanda committit: vetat igitur hic vini amorem, et lasciuiam inde promanantem.  \pend
\textit{2. Πλήκτης dicitur pugnax, paratior manu quam ratione, Ein schlegler, boccher. }\pstart 3. Αἰσχροκερδὴς, turpis lucri cupidus, vel turpis quaestus studiosus: quod fieri potest, vel si ex rebus per se turpibus colligatur, vt venalem habere linguam, vocationes muneribus vendere, doctrinam ad opes comparandas accommodare: vel si per accidens turpis fiat quaestus, qui extra ministerium poterat esse honestus, vt si minister idem sit mercator, institor, agricola, negotiator, et similes tra ctet artes, quae idiotae forte concedi poterant, ministro autem minime, auari hominis argumenta sunt.  \pend\pstart 4. Επιεικὴς, moderationis dicitur studiosus, non auidus poenarum, sed qui commode fratrum delicta interpretetur in iure administrando et exequendo, ne sit nimium rigidus: summũ ius enim summa solet esse imuria..  \pend\pstart 5. Ἄμαχος, prius dixerat μὴ πλήκτὴν, non pugnacem, nunc alienum a pugnis: sed latius etiam patet ἄμαχον esse, hoc est, non debet etiam verbis esse pugnax, contentiosus dispu\pend
\section*{IN I. EPIST. AD TIMOTH. }
\marginpar{[ p.67 ]}\pstart tator, sed placidus doctor potius.  \pend\pstart 6 Ἀφιλάργυρον, alienum ab auaritia, prius dixerat hospitalem, nunc radicem notat ipsam auaritiam.  \pend
\textit{4 Qui suae domui, etc. }\pstart Hactenus fere recemsuit insitas virtutes, nunc externas profert: ac primum, VT DOMVI SVAE BENE PRAESIT, domus gubernatio argumentum est gubernationis Ecclesiae, quae et ipsa domus Dei dicitur.  \pend\pstart 2. LIBEROS HABEAT IN SVBIECTIONE, ET HONESTATE: in familia episcopi duo requirit, subiectionem et honestatem. Ὑποταγὴ est obedientia ordinem suum seruans, quo dicto sunt audientes inferiores. Σεμνότης autem familiae decorum et grauitatem indicat in vestitu, incessu, sermone, factis, totamq. vita:illa honestas mirifice ornat ministrũ Ecclesiae apud auditores.  \pend
\textit{5 Si autem quis suae domui, etc. }\pstart Argumentum a minori addit, quo confirmat prius membrum de facultate oeconomica. Domus hic opponitur Ecclesiae, minus maiori: si id quod minus est non potest, ergo maius administrare non poterit, vt si, nauiculam in flumine modico non potest administrare:ergo non nauem in Oceano regere poterit: vera est ratio, non tamem conuertitur, vt is ꝗ domui pont conmode praeesse possit etia Ecclesiam regere: non enim statim maiora potest, ꝗ minora nouit admini  \pend
\section*{COMMENTARII }
\marginpar{[ p.68 ]}\pstart strare: negatiue igitur recte colligit Apostolus, ac defectus scientiae includit etiam defectum facultatis, quasi dicat, non potest plane vtilis esse talis Eeclesiae.  \pend
\textit{6 Non nouitium, etc. }\pstart Νεέφυτος dicitur recens plantatus : veο- κατηχητης, nouiter admissus ad religionis profesionem, quales dicebantur Catechumeni: igitur nouitius dicitur, qui in his quae ad ministerium necessaria sunt, rudis adhuc est Huic imminent duo pericula, prius a seipso posterius ab auditorib. Ex seipso, ne fiat infla tus, hinc arrogans et aliorum contemptor, completur in eo quod dici solet, magistratus virum arguit. Ex auditoribus autem periculum est, inuidia, nam ωθόνος πρὸς τόν ἔχοντα έρπος, tamen Apostolus illa conuertit quasi posterius ex priore fluat, et reuera inflati ruunt in calumniatoris iudicium: potest igitur et calumniator esse Satan, qui nouitios praecipitat er vno in aliud peccatum, et commode dicitur calumniator, quia Deum apud hominem traducit, vt testatur historia paradisi, et hominem apud Deum, vt est videre in Iobo, et deniq. homines apud homines: in hoc tam varium et graue iudicium Satanae cadunt isti.  \pend
\textit{7 Oportet autem ipsum etiam, etc. }\pstart Bono testimonio ministrum munit contra  \pend
\section*{IN I. EPIST. AD TIMOTH. }
\marginpar{[ p.69 ]}\pstart calumnias. Geminum autem potest esse vel a domesticis fidei, id non requirit hic, quanquam prorsus sit necessarium: consistit autem id in superiorib. virtutib. de quib. testabumtur domestici fidei. Aliud testimonium petitur ab externis, h.e. ab incredulis adhuc cum quib. prius vixit. Hoc necessarium est ob sequentes causas. Primum NE INCIDAT IN EXPROBATIONEM DIABOLI, hoc est, ne prius vitae genus illi in probrum cedat: exprobrabitur illi si mala vita prior fuit, vt si fuerit scor tator, fur, vsurarius, aut tale quid, id totum cadet ei in ὀνειδισμόν, in exprobrationem in noua religione. Deinde, NE CADAT IN LAQVEVM DIABOLI, hoc est, vt vitae genus prius ei sit impedimento in officio suo, detrahetur aliquid de autoritate sua, quae vt vitentur petendum est testimonium ab externis. Hactenus de virtutibus episcopi, hoc est, boni et idonei ministri, quas hoc ordine digeremus in bono et idoneo ministro, aliae virtutes sunt insitae, aliae externae. Insitas voco quae in ipsius ingenio quaeruntur et moribus. Externas vero, quae in aliis cum sint illi cedunt ornamento. Insitae vel sunt facultatis vel assuefactionis: facultas aut est doctrinae, aut gubernationis doctrinae facultas est, vt sit διδακτικὸς gubernationis, et suae domui bene praesit, h.e.bonus sit oeonomus.  \pend
\section*{COMMENTARII }
\marginpar{[ p.70 ]}\pstart Assuefactionis moralis reliquae sunt, quae vel facit ad temperantiam, vt repri mere ardorem venereum, SIT VNIVS VXORIS MARITVS. 2. moderationem potus, sit sobrius, non vino deditus. 3. moderationem cibi, vt iudicii sit σώφρον. 4. moucrauonem cultus, sit κόσμιoς.  \pend\pstart II.Ad liberalitatem, sit φιλόξενος. 2. sit ἀφιλάργυρος. 3. sit αἰσχροκερδής.  \pend\pstart III. Ad reprimendam temeritatem, μἡ πληκτής. 2. ἐπιεικής 3. ἄμαχος.  \pend\pstart IIII. Contra ambitionem, μὴ νεόφυτος, μὴ τυφωτείς.  \pend\pstart Externae virtutes sunt, liberos habere bene institutos, quo pertinet subiectio illorum, et honestas. Item bonum habere testimoniumab externis, etc.  \pend
\textit{8 Diaconos itidem compositos, non bilingues, non multo vino deditos, etc. }\pstart mosynam in pauperes, etc. In his σεμνότης reTertio, loco praecipit de diaconis in quibus easdem virtutes requirit, quas tribuit episcopo: deinde quasdam specialiter. Porro diaconos, intelligo post episcopum, hoc est, primarium ministrum Ecclesiae inferiorem, qui episcopo docendo, administrando Sacramenta, scribendo etiam, aliisque curandis rebus adhibetur, vt visitare aegros, curare elec\pend
\section*{IN I. EPIST. AD TIMOTH. }
\marginpar{[ p.75 ]}\pstart quiritur, hoc est, grauitas pietati congruens: est enim in quolibet officio sua quaedam grauitas qua ornatur muneris professio. Hinc σέμενια dicuntur collegia pietati deputata.  \pend\pstart 2. non bilingues. δίλογοι dicuntur qui aliud stantes, aliud sedentes loquuntur, sibi pugnantia afferunt, incerta, et ob id saepe falsa. Quia igitur diaconus episcopi est quasi oculus et auris, merito vult non bilingues:nam deferebant populi peccata et errores ad episcopum, hinc episcopus multa carpebat in populo saepe ex diaconi ore, in qua re prudentem oportet esse. Admonemur illo in deferendo cautos esse oportere et non nisi vera deferre ne odio et inuidia fratrum errores nimium augeamus. Hinc delatoris nomen omnibus nationibus odiosum et inuisum redditum est.  \pend\pstart Similiter ministri discant ne facile huiusmodi delatoribus credant, nec statim traducant nisi re prius probe comperta. 3. NON DEDITOS VINO, idem in episcopo damnauit, quem voluit esse πάροινον, quasi vino assidentem semper: sic diaconus non debet προσέχειν, hoc est, non de bet attendere, et inhiare vino, ne vino corrumpatur, etiam in deferendo aliorum peccata, alioquin vinum per se non damnat, sed πόλλον, hoc est, copiam, et προσέχειν, studium.  \pend
\section*{COMMENTARII }
\marginpar{[ p.72 ]}\pstart 4. NON LVCRI CVPIDOIS, quae ratio valuit supra, hic etiam repetatur, lucrum turpe est in diacono, si ad munera respiciat et compotatoribus pareat, aliis iniquior: turpe est vel ex re per se turpi, vt Vespasianus ex lotio colligebat vectigal, dicens, Bonus odor lucri ex re qualibet. Sic hodie Ro mae ingens lucrum colligitur ex publicis prostibulis, lucrum turpe est. Deinde per accidens fit turpe, vt si minister mercaturam exerceat  \pend
\textit{9 Tenentes mysterium, etc. }\pstart Vitiis opponit virtutes in diacono necessarias. Prima est, MYSTERIVM FIDEI, hoc est, sanae doctrinae formulam et regulam nouisse, illud est ἔχεη μυςτήριον, tenere, nouisse possidentis est. Doctrina sana recte mysterium dicitur, quia res est mundo incognita et diuinitus patefacta. Deinde quia ab impiis non agnoscitur, nec etiam sine poenitentia admittuntur. Praeterea propter rationem Sacramentorum: hinc in communione dicebatur, catechumeni exeunto. Debet igitur diaconus fundamenta doctrinae et formam eius nouisse. Deinde cognitioni addit puram conscientiam, quasi dicat, non solum peritus in religione debet esse, sed castis moribus, ne polluat conscientiam suam: conscientia se habet vt nuda tabella, bona autem vel mala fit studio ac vitae qualitate. Virtutum  \pend
\section*{IN I. EPIST. AD TIMOTH. }
\marginpar{[ p.73 ]}\pstart studio et bonorum operum exercitio bona nascitur: mala autem vitae petulantia ac im probitate: recte igitur fidei coniungit con scientiae puritatem.  \pend
\textit{10 Et hi probentur, etc. }\pstart Virtutibus addit probationem, nam vir tutes cognosci debent publice. Probari autem accipio, pro examinari, rogantur de volutate et facultate velintne et possint Ecclesiae vtiliter seruire. Facultas cognoscetur ex articulis fidei, et dexteritate concionandi, au diatur ergo quomodo verbum Dei interpre tetur, qualis sit illius vox, qui gestus, habitus, methodus in tractandis locis, etc.  \pend\pstart Post examen officii est inauguratio, ministrorum Ecclesiae, scil. docendo, consolando, visitando aegrotos: hinc suo episcopo, et aliis fratribus. Tertio, repetit virtutes in genere omnes, SI SIT IRREPREHENSIBILIS: conditio est necessaria, non enimeruditio, et eloquentia hîc rei satisfacit.  \pend
\textit{11 Vxores similiter, etc }\pstart Quarto, loquitur de vxoribus ministronrum et diaconorum. Nam quod alii de diaconissis hoc accipiunt, non satis firmum est, minus etiam placet eorum sententia, qui in genere de sexu muliebri haec interpretantur. Prius enim cap.2.versu 9. harum decorum descripsit, vt nonsit opus hic repetere. Itaque  \pend
\section*{COMMENTARII }
\marginpar{[ p.74 ]}\pstart vxores intelligo episcoporum et diaconorum: debent istae esse primum σεμναὶ, vt ornent maritorum suorum conditionem. Discrimen igitur debet esse inter ministri vxorem, et alterius de plebe, modestia, habitu, moribus, totoq. vitae genere, reliquis exemplo esse debet. Deinde, NON SINT CALVMMEATRICES, διαβολοὶ, nam solent istae etiam populi -errata ad maritos deferre vt corrigantur in concionibus, ne igitur innocentes deferant, vt prius de diaconis dixit, ne sint διλoγοί. Deinde in genere ne sint garrulae, nugiuendae.  \pend\pstart Tertio, SOBRIAE, sic maritis coniungit sobrietate, et requiritur quidem in omnibus piis ista virtus vt in genere virtus in omnibus requirenda est, in ministro autem et sua fami lia postulatur excellentia quaedam et exemplum.  \pend\pstart Quarto, FIDELES IN OMNIBVS, hoc est, in tota causa religionis, deinde deferendis aliis, idem in fide matrimoniali, in administra tione rei domesticae, in regenda familia, in summa, omnia dirigat ad pietatis auctionem.  \pend
\textit{12 Diaconi sint, etc. }\pstart Quinto loco loquitur de altero diaconorum genere, quod scil. administrabat opes Ecclesiasticas, erant hi instituti primitus ab Apostolis, Act.6. Hinc videtur in aliis Ec\pend
\section*{IN I. EPIST. AD TIMOTH. }
\marginpar{[ p.75 ]}\pstart clesiis passim obseruatum, vt ministris adessent diaconi quidam de senioribus selecti, quibus committebatur aerarium Ecclesiae: hi pauperum, peregrinorum, viduarum, pupillorum, aegrotorum habebant curam, distribuebantque illis pro ratione facultatum et consilio seniorum. Hinc Ambrosius censet in qualibet ciuitate debere esse episcopum vnum, duos diaconos primi generis, qui simul do ceant interdum et septem secundi ordinis, qui Ecclesiae bona procurent. Nos ministros habemus, et primi generis diaconos quoque duos, secundi autem generis nullos, quia illa pars ciuilis facta est, et a politico administratur magistratu.  \pend
\textit{13 Qui enim bene ministrant, etc }\pstart Inuitat spe praemiorum ad diligentiam. Ac refero in genere ad totum ministerium: debet enim haec ratio valere apud episcopum, diaconum, eleemosynarium, ac si qui alii sunt gradus in Ecclesiastica administratione. GRA DVM SIBI AC QVIRVNT BONVM, non accipio de diacono, quasi spe episcopatus debeat diligens esse: sed gradum bonum apud Deum, qui solet assiduis et vigilanter officio suo intentis augere sua dona, secundum illud, Serue bone fuisti in modico fidelis, constituam te super maiora.  \pend\pstart Item habenti dabitur, et ab co qui non  \pend
\section*{COMMENTARII }
\marginpar{[ p.76 ]}\pstart habet, auferetur etiam id quod habet. Diligentes igitur habent quo sperent donorum auctionem. Deinde bonus gradus est apud Ecclesiam cui seruit, illa enim commendabit virtutem hominis, laudabit modestiam, gloria igitur et publica authoritas quasi prae mium virtutis bonus erit gradus. Tertio, apud semetipsum, nam conscientia libera erit, et intrepide versabitur in ministerio suo cum nullius sceleris sibi sit conscius. Quarto, gradus hic firmissimus est ad vitam aeternam, diligenter vocationi suae adesse, etc. Addit alterum argumentum, et multam libertatem, hoc est, si fuerit assiduus in officio, valebit etiam libertate docen di, quasi dicat, si suis fuerit ornatus virtutibus, contraria vitia intrepide carpet, vt ebrietatem, scortationem, otium, et similes pestes: fieri enim non potest vt commode ebrios corrrigat minister qui et ipse vinosus est: sic scortator non potest bona conscientia contra scortatores agere, idem fit in aliis, vt igitur ministro sua maneat salua παῤῥησία dabit operam vt in officio quam diligentissimus sit.  \pend
\textit{14 Haec tibi scribo, etc. }\pstart Sexto loco sequitur epilogus: docet autem cur scripserit de his rebus breuius: deinde cur etiam simpliciter scribat. nam videtur du\pend
\section*{IN I. EPIST. AD TIMOTH. }
\marginpar{[ p.77 ]}\pstart plex occupatio: primum enim de breuitate inre tam difficili non immerito quaeri poterat: deinde quia respondet se scribere breuius quia mox ipse sit vent urus: iterum quaeritur, cur ergo scribis cum sis ipse mox venturus? et huic respondet, hinc petendam esse guberna tionis Ecclesiasticae formulam. Itaque hic priori quaestioni respondet de breuitate scripti, SPERO ME BREVI VENTVRVM, ideo breuior sum in scribendo..  \pend
\textit{15 Sin vero tardauero, etc. }\pstart Alteri occupationi occurrit, cur ergo scribat: respondet, si non venero, hinc tu habebis formulam gubernationis Ecclesiasticae.  \pend\pstart Disce Apostolum non sui fuisse iuris, sed saepe contra animi consilium alio ductum Dei consilio, vt hic promittit aduentum suum, de quo tamem dubius est. Proponit igitur etiam sanctus homo, sed Deus disponit vt ipsi videbitur. Deinde non absolute de futuris hominum consiliis statuendum est: sic Apostolus hic conditionem prae seruat, ideo concedit fieri posse vt diuinitus impediatur, quod suo consilio nihil derogat. Vtvt cadat, interim epistola haec suum habet vsum, scilicet, vt hine modum gubernationis Ecclesiasticae depromat.  \pend\pstart Propositio igitur sit, Timotheus, et ob id  \pend
\section*{COMMENTARII }
\marginpar{[ p.78 ]}\pstart omnis Ecclesiae Christi minister gubernationis formulam non nisi hinc debet petere. Petantur hinc argumenta suasoria ad diligentiam: primum, quia versantur in domo Dei, hoc est, sunt oeconomi in Ecclesia Dei. Quanto qui praeest domui praestantiori, tanto vigilantior debet esse in officio: sed hi sunt in praestantissima, vtpote Dei domo oeconomi, Ergo, etc. Domus Dei dicitur, quia continet domesticos Dei, hoc est, electos et pios, qui profitentur veram doctrinam. Hos omnes complectitur vniuersalis Ecclesia, cuius particu lae sunt priuatae Ecclesiae, vt singulares coetus. Et qua ratione Ecclesia dicitur domus Dei, eadem ministri dicuntur oeconomi Dei. Itaque vt oeconomus in familia praecipuos fert labores, primus surgit, vltimus confert se cubitum: primus est in laboribus, vltimus inde recedit: sic ministri in Ecclesia praecipuum ferunt olus,quare instructione imprimis opus habent.  \pend\pstart Alterum argumentum est, quia domus haec est Dei viuentis, quasi dicat, non stupidi aut stulti domini familia est, sed Dei viuentis, hoc est, omnia scientis, omnia coram inspicientis, quem nihil latet vspiam, vigilandum igitur. Nam oeconomi saepe ludunt impune, et securiter stertunt heris improuidis, aut non satis callidis et attentis ad rem, secus hic ratio se habet. Deus viuens huic familiae  \pend
\section*{IN I. EPIST. AD TIMOTH. }
\marginpar{[ p.79 ]}\pstart praeest, falli igitur non potest. Ac videtur Deum opponere hominibus, viuum vero idolis: plus enim Deus quam homo metui solet: sed verus Deus etiam plus quam idola metuendus est, quia idola stupida sunt, vita carent: Deus autem vita est.  \pend\pstart Tertio, domus illa columna est, et stabilimentum veritatis, hoc est, fundata in veritatem quae est Christus, hinc columna veritatis quasi ex veritate excitata, alioquin Christus fundamentum est, vt loquitur 2. Ephes.ver.20. huic fundamento superstructa Ecclesia dicitur columna, quasi dicat, firmissima et inexpugnabilis. Deinde dicitur veritatis columna et firmamentum, quia Deus per Ecclesiam propagat veritatem, estque testis eius, et huic innititur quasi marpesia cautes. Iam valet argumentum etiam hinc a conditione seu proprietate Ecclesiae, quae est semper indiuulse adhaerere veritati, ergovigilandum est ministro, ne suo vitio ab illa exorbitent auditores.  \pend
\textit{16 Et sine controuersia magnum est pietatis mysterium, etc. }\pstart Digreditur longius in commendationem doctrinae Ecclesiasticae ac enumerat veritatis illius tam firmae praecipuos locos. eosque imprimis, qui latent rationem: piis autem sunt ὁμολογούμενοι. Et petatur hinc  \pend
\section*{COMMENTARII }
\marginpar{[ p.80 ]}\pstart quartum argumentum ad diligentiam, qui a ministri tractant mysteria solis piis ὁμολογούμενα: ergo vigilandum illis ne tanta mysteria suo vitio polluant. Discamus ex hoc argumento quae sit vera Ecclesia Christi in his terris, nimirum quae inter alia etiam quaedam habet confessa et concessa dogmata, αὐτοπιςτὰ et ὁμο- λογούμενα, vt hic loquitur Apostolus, hoc est, talia de quibus piis non licet dubitare. Nam dubitando produnt se non esse pios. Interim de his dubitat et disputat totus mundus. Sic omnes facultates suas habent hypotheses, quas non licet in dubium vocare. Hypotheses igitur Christianae pietatis hoc loco sunt.  \pend
\textit{1. Deus manifestatus est in carne. }\pstart Hoc est, Deus carnem assumpsit. Ioan.1. Deus caro factus est. In Christo, igitur duae vnitae sunt naturae diuina et humana in vna persona:atque hanc hypothesin qui negant, principia pietatis conuellunt: ratio est, quia Apostolo est ὁμολογούμενον : videant ergo quid agant Arriani, et similes qui hoc totum in dubium vocant.  \pend
\textit{3 Iustificatus in spiritu. }\pstart Hoc est, Christus Dei filius homo factus iustitiae laudem obtinuit, quod sit pura et idonea hostia pro hominum peccatis delendis, iustificatus dicitur absolutus, et liber pro\pend
\section*{IN I. EPIST. AD TIMOTH. }
\marginpar{[ p.81 ]}\pstart nuntiatus, vere iustus, vide Matt.11.Luc.7. ver.35. Psal.51.  \pend
\textit{3 Visus Angelis. }\pstart Id quidem verum est secundum historiam, nam in natiuitate, ieiunio, passione, resurrectione, ascensione semper adsunt Angeli spectatores: sed mihi videtur de maiore quadam visione loqui, qua scilicet contra An gelorum opinionem Christus mirabili ratione duas naturas vniuit in vna persona: quod nec homines, nec Angeli satis explicare possunt, et ob id summo stupore viderunt: et viderunt id in natiuitate, totoque Christi historia, plenius tamen in ascensione, stupendum mysterium Christum hominem ad dextram Dei Patris ascendere. Hinc Theodoretus dialogo 2.refert ex Hyppolito veteri et Ecclesiastico scriptore, Angelos in Ascensione Christi hoc pronuntiasse quod scribitur Psal.24. Aperite principes portas vestras vt ingrediatur rex ille magnus, etc. quem locum sic etiam interpretatur Iustinus Martyr con tra Tryphonem. Haec mysteria sunt quae rident Prophani et impii homines, ludunt de Christo tanquam homine nudo, et precario Deo.  \pend\pstart Quarto, PRAEDICATVS EST GENTIBVS, scil. incredibili successu contra spem rationis: homines piscatores dementant totum mundum.  \pend
\section*{COMMENTARII }
\marginpar{[ p.82 ]}\pstart Vocatio igitur Gentium hypothesis est. Quinto, CREDITVM EST ILLI IN MVNDo. Locum habuit Euangelica doctrina in mundo contra Philosophorum decreta, contra Iudaeorum furores, contra magistratuum omnium edicta: habebit igitur hospitium aliquod semper in his terris vsque ad extremum iudicium.  \pend\pstart Sexto, RECEPTVS EST IN GL. scil.idem ille Deus incarnatus: sedet ad dextram Dei Patris, intercedit pro Ecclesia sua, illam tuetur potentia sua, gubernat Spiritu suo, flectitur illi omne genu in terra, caelo, et sub terra. Haec maiestas Christi est, quidquid dicant con temptores eius.  \pend
\textit{CAPVT IIII. }
\textbf{L }\pstart OCVS grauissimus hic explicatur de erroribus circa fidem, qui postremis temporibus simplicitatem fidei corrupturi sunt. Locum proponit more Prophetico: est enim oraculum Christi Ecclesiae relictum ad vtilitatem. Simile habemus in 2. ad Timotheum capite tertio versu quinto, et in 2.Petri capite 2.  \pend
\section*{IN I. EPIST. AD TIMOTH. }
\marginpar{[ p.83 ]}\pstart et in epistola Iudae versu decimo octauo. Hic explicat oraculum per circunstantias, tandem accommodat ad corruptae doctrinae capita praecipua, qualia sunt prohibere matrimonia, et ciborum discrimina inducere contra Dei institutionem.  \pend\pstart Haec proponit Timotheo non solum cauenda, sed vt alios de his quoque praemoneat. Hinc transit ad alia praecepta, vt de anilibus fabulis cauendis, de exercitio verae pietatis, de officio ministri, quo debet in ommbus virtutibus exemplar esse subditorunf: nem de lectione, de non negligendis donis Dei, et similibus, quae omnia ad vigilantiam pios ministros excitant.  \pend\pstart Propositio est, Timotheo et omni pio ministro quam vigilantissime curandum est ne qua labes contra sanam doctrinam inuehatur in Ecclesiam Christi, quales tamen extabunt vltimis temporibus.  \pend\pstart Partes capitis tres sunt. Prima, de statu Ecclesiae post ipsius Apostoli exitum. Altera, de cauendis fabulis, et verae pietatis vero exercitio. Tertia, episcopum per omnia exemplar debere esse omnium virtutum suis auditoribus. Cui adduntur quae dam etiam specialia ad Timotheum sic pertinentia, vt in illius persona omnium ministrorum idea formetur.  \pend
\section*{COMMENTARII }
\marginpar{[ p.84 ]}
\textit{Vers.1. Spiritus autem manifeste dicit, quod in posterioribus temporibus desciscent quidam a fide. }\pstart Tramsitio ad quartum locum huius epistolae, quem locum simplici narratione proponit, quemadmodum de hoc admonitus fuerat diuinitus. Ideo diximus oraculum esse de futuro statu Ecclesiae post Apostoli mortem. Ideo primum hic consideretur quis illius oraculi sit author, Spiritus scil. Sanctus. dicit enim, Spiritus expresse dicit, hoc est, Spiritus Sanctus autor est admonitionis illius, ad Ecclesiae vtilitatem, vt pii sciant quae doctrina cauenda, aut sequenda sit.  \pend\pstart Ac alludit ad Propheticum morem, quibus hoc erat frequens, Dominus loquitur, Deus dicit, Spiritus Domini loquutus est, et similia, quo significabant se non authores, sed organa tantum esse:sic Apostolus authotitatem praemonitioni conciliat, dum non se, sed Dei spiritum authorem inducit. Sic2. Pet.I. dicitur non voluntate hominum, sed Dei Spiritu impulsos, locutos fuisse sanctos Dei homines.  \pend\pstart Deinde addit, quod hoc loquatur ῥητῶς, hoc est, expresse, quasi dicat, sine ambagibus, sed nuda et perspicua oratione: perspicuitas igitur est testimonium veritatis,  \pend
\section*{IN I. EPIST. AD TIMOTH. }
\marginpar{[ p.85 ]}\pstart cum mendacium inuolucris vtatur, et variis coloribus sese pingat. Sic oracula Ethnica semper ambigua fuerunt, aut certe obscura: Dei contra sermo et veritas studet perspicuitati. Facit illud eo vt de admonitione nemo dubitare possit : relinquenda igitur Scripturae sacrae sua simplicitas et perspicuitas, nec est affectata introducenda obscuritas quam sua natura non habet.  \pend\pstart Tertio, quando id sit euenturum, VLTIMIS TEMPORIBVS, ύστέροις καιροῖς. Vltima tempora dicuntur ea quae ab ascensione Christi ad gloriosum eiusdem reditum decurrunt. Et in his quae fini semper propiora sunt, qualia sunt post Apostolorum tempora, nostra hodie, et quae deinceps supersunt. Monet igitur de periculo infuturum, quemadmodum cap.1. monuit de presenti tum corruptela, qualem inducebant tum νομοδιδάσκαλοι. Itaque diligenter est consi derandum, quibus temporibus semper hoc vaticinium sit completum.  \pend\pstart Quarto, quid euenturum sit, et qui sint hoc facturi, DEFICIENT QVIDAM A FIDE. Propositio est vaticinii. Deficere est a vero ad falsimm declinare : igitur defectio, ἀποςτασία erit, a vera Dei cognitione ad falsam declina tio, id dixit a fide deficere: fidem accipio pro simplicitare verae religionis, quod aliâs  \pend
\section*{COMMENTARII }
\marginpar{[ p.86 ]}\pstart dicit sanam doctrinam, et hic mox καλὴν δι- δασκαλίαν, bonam doctrinam. Vide igitur quae sit Apostolica doctrina, et contra quae sit apostatica doctrina. Apostolica est, quae retinet simplicitatem sanae doctrinae, vt est a Christo et Apostolis Ecclesiae relicta. Apostatica est, quae ab eadem deficit ad hominum commenta.  \pend\pstart Quinto, qui hoc facient? respondet, τίνες, quidam : hoc interpretor vt alibi loquitur Scriptura per πολλοὺς: quia multi vocati, pauci electi vt Christus ait, viam ad interitum esse amplam, et multos qui ambulent illam, et Matt.24. multos fore seductores, et apostasiam fore publicam, adeo vt electi quoque seducantur si fieri posset. Itaque ingens erit apostasia, qualis reuera fuit superioribus seculis et adhuc est, et erit deinceps forte adhuc maior.  \pend\pstart Sexto, quae causa illos ad apostasiam compellat, illa gemina est. Principalis et organica. Principalis est, πνεῦμα πλάνον, hoc est, spiritus errans, qui via dicitur: primum quia ipse in veritate non stetit, sed ad mendacium defecit: hinc mendacii pater dicitur, et alios tales efficit. Deinde errans dicitur spiritus, quia aliis quoque erroris est causa, vt enim i pse mendax est, sic alios tales efficit et men\pend
\section*{IN I. EPIST. AD TIMOTH. }
\marginpar{[ p.87 ]}\pstart daciorum, hoc est, errorum quaerit patronos. Is est spiritus mendax in ore omnium falsorum doctorum.  \pend\pstart Organica causa est, animi humani leuitas, qua natura corrupti vanitate delectantur, illam praeferunt veritati, quod est προσέ- χεα. Hinc nascitur communibus suffragiis do ctrina dae moniorum. Satan hanc ludit et hominum animis instillat ac dictat: hinc ambitiosi homines illam publice proponunt et propugnant.  \pend
\textit{2 In hypocrisi falsiloquorum, cauterio notatam habentium conscientiam. }\pstart Ortum descripsit apostaticae doctrinae per suas circunstantias, nunc eiusdem propagationem, quae fit per hypocrisin: pessimo ipsorum doctorum successu. Tria autem hic ponit quae diligenter expendi debent. Primum ipsi per se sunt dευ- δθλογοι, id est, falsae doctrinae defensores, quae iam dicta est apostatica doctrina. Fit autem Τευδολογία duplici modo. Primûm, quum per se falsum affertur, hoc est, quod Scripturis sacris non est expressum, sed his contrarium, vt Purgatorium, de inuocatione sanctorum, et similia πευδολογίαν habent, quia de his  \pend
\section*{COMMENTARII }
\marginpar{[ p.88 ]}\pstart scriptura nihil habet. Deinde est ψευδολογία, quando Scripturae alius assuitur sensus, quam ferat sana doctrina, qualis est interpretatio schismaticorum qui ad suas opiniones defendendas torquent loca Scripturae, vt illud prin cipium Ioannis, hodie quidam de principio patefactionis et praedicationis per Apostolos factae interpretantur, qui sensus Spiritus S. non est. Deinde hanc ψευδολογίαν proponunt in hypocrisi, hoc est, fingunt personam ad fallendum idoneam, assumunt zelum religionis, gloriae Dei, et similia iactitant, quasi vindices sint maiestatis Dei Patris, quae male tribuatur filio Christo, etc. Sic dae mon solet, non natiua prodit facie, hoc est, vt dici solet, non niger, sed specie Angelica in publicum prodit: Talis erat hypocrisis Pharisaeorum, hodie monastica vita absorpta est nefanda hypocrisi, sub qua enormia latent scelera: talis coelibatus, ieiuniorum, castitatis, votorum peregrinantium hypocrisis. Dicitur hypocrisis cum aliud latet, quam externa specie hominum oculis offeratur, vt in comoediis personati homines prodeunt, qui habitum, sermonem, gestus, facta aliorum repraesentant, quos illi nunquam viderunt: tales sunt isti personati sancti et Christiani.  \pend
\section*{IN I. EPIST. AD TIMOTH. }
\marginpar{[ p.89 ]}\pstart Iertio, HABENT CAVTERIATAM CONSCIENTIAM, quasi dicat, infeliciter illis cedit errorum defensio: foris splendidi sunt doctores, et Promethei: intus misere torquentur alastore conscientia ipsos accusante, et damnante. sunt istae furiae et manes a quibus nunquam tuti sunt. Καυτήριον, cauterium, chirurgicum est instrumentum, quo resecantur interdum vstione partes male sanae a corpore humano. Item instrumentum quo signabantur equi et mancipia, vt ex nota inusta cognosci possent. Mancipia aliquando sic etiam in fronte notabantur: hoc ad conscientias hicApostolus transfert. Cauterio notat a dicitur conscientia, quae damnat et reum agit hominem: est enim is domesticus testis qui falli non potest, conscius omnium dictorum, factorum, consiliorum, cuius iudicium etiam valet apud Dei tribunal.  \pend
\textit{3 Prohibentium nubere, etc. }\pstart Tertio loco describit tandem apostasiam a genere doctrinae. Ex qua duos primarios articulos producit, vt posteritas iudicium possit ferre inter apostaticam et Apostolicam doctrinam, vt mirum sit tam crassa ignorantia mundum haec duo capita neglexisse.  \pend\pstart 1. Prohibebunt γαμεῖν, hoc est, honestatem matrimonii tollent: ius eius in arctum redigent: quod factum est duplici modo. Pri\pend
\section*{COMMENTARII }
\marginpar{[ p.90 ]}\pstart mum, simpliciter quidam danarunt nuptias: alii secundum quid, hoc est, eos qui nati ad coelibatum non sunt, contra naturam voto castitatis constrinxerunt, quod adeo rigide seruant, vt potius admittant scortationem, quam legitimum coniugium. Quo in errore fuerunt olim Tatiani, Marcionitae, Manichaei, et nostri hodie Pontificii. De Manichaeorum rationibus vide August. de moribus Ecclesiae catholicae lib.1.cap.35. Epiphanium, Irenaeum, et alios. Nos de matrimonio et coelibatu egimus quoque in Locis nostris: vide etiam elegantem epistolam Cypriani lib.1.epist.11. de virginibus quae votum castitatis non seruabant.  \pend\pstart 11. De ciborum delectu, de quo vide Aug. epist. 86.ad Casulanum, et li.2. de morib. Eccle siae cap.13. vbi delicatam abstinentiam damnat, prae ferens ei suillae etiam carnisvsum. In Locis nostris explicata est mea sententia de ciborum delectu ex scripturae fontibus. Rom.14. Matth.15. Co loss.2.1.Cor.8.ett.quae vide.  \pend
\textit{Quos Deus condidit. }\pstart Hactenus depicta est apostatica doctrina:nunc breuiter illam confutat, nominatim articulum de ciborum delectu. Deus illos condidit, h.e de genere sunt bonorum sua natura:ergo non debent simpliciter damnari.  \pend\pstart Deinde fine bono conditi sunt, qui est  \pend
\section*{IN I. EPIST. AD TIMOTH. }
\marginpar{[ p.91 ]}\pstart μετάληψις : ergo ab vsu hoc non sunt excludendi. Μετάληψιν intelligo iudicium de cibis noxiis et innoxiis ex creaturis, quasi dicat, caro, fructus arborum, herbae, radices, etc. ideo conditae sunt vt hinc cibus homini depromatur. Peccant igitur qui carnis vsum prohibent. Iudicium hoc indidit Deus naturae, vt intelligat quid ad cibum vtile sit: alia enim condita sunt ad medicinam, alia ad artes, alia ad vestitum, a lia ad ornatum, alia ad cibum, etc. Et quae cibi causa condita sunt, non debent homini prohiberi. Praeterea fidelium hoc scire imprimis interest. Ideo addit, CVM GRATIARVM ACTIONE FIDELIBVS, ET HIS QVI NOVERVNT VERITATEM, q. d. impiis omnia immunda esse, mundis contra omnia esse munda. Et Eucharistiam accipio hic pro precibus quae solent recitari a sumpturis cibos, ante cibos et post cibos, quod proprie faciunt pii.  \pend
\textit{4 Quia omnis creatura Dei bona, etc }\pstart A toto ad speciem totius argumentatur. Omnis creatura est bona, ergo caro quoq. bona edenti. Καλὸν accipio pro νόμιμον, h.e. concessum et legitimum diuina authoritate. Huic opponit ἀπόβλητον, i. iudicio hominum reprobatum. Habes ex antithesi illustre argumentum. Cibi piis concessi sunt diuina authoritate. Hinc καλὸν dicitur: prohiberi autem hominum  \pend
\section*{COMMENTARII }
\marginpar{[ p.92 ]}\pstart doctrina, hinc ἀπόβλητον : igitur cauenda est haec doctrina, et sequenda prima illa Dei concessio. Addit tamen limitationem, quae prius iam fuit etiam expressa, SI CVM GRATIARVM ACTIONE SVMATVR. hoc est, piis salubres sunt cibi et a Deo conditi sunt. Et Eucharistia hic etiam sit oratio, qua consecratur mensa ante et post cibum.  \pend
\textit{5 Sanctificatur enim, etc. }\pstart Αἰτιολογία est superioris regulae, cur sanctis omnia pura sint: quia cibus consecratur eis verbo Dei, et precibus. Duo sunt modi consecrationis legitimae, verbum Dei, et oratio. Λόγος θεοῦ est verbum scriptum, videtur enim ad morem respicere piis consuetum, quo ad mensam recitare solebat locus scripturae, Psalmi aliqui aut simile: quae lectio vel recitatio formam habet consecrationis. vel λόγος sit τὸ ῤητὸν, quasi dicat, Dei pronunciatum est licitos esse omnes cibos. Deus dixit, Quod Deus sanctificauit, homo ne impurum dicat. Alter modus est ἐντεύξις, hoc est, preces conceptis verbis recitatae ad et post cibum : quales preces recitari solent ad piorum conuiuia pro depulsione omnis noxae, quae nobis debebatur ob peccatum in rebus per se ad vitam salutaribus. Hae preces igitur consecrant nobis cibos.  \pend
\section*{IN I. EPIST. AD TIMOTH. }
\marginpar{[ p.93 ]}\pstart Damnanda hic est consecrandi ratio impia, qua res in Papatu consecrantur ad vsus insolitos, quos natura non habet, nec a Deo his inditi sunt, vt dum consecrant herbas, radices, ramos arborum, plantarum ad fugandos daemones. Oua consecrant contra incendia, campanas contra tempestates, aquam, salem contra peccata:ignem, aërem, terram, aquam, omnia consecrant, hoc est, nouis vsibus dedicant magna certe impietate: id non consectare, sed magicis artibus potius est dementare et prophanare. Hinc consecrationibus talibus certa tempora sunt deputata, vt in die Marci, fructus terrae consecrant: in die ascensionis Mariae herbas, arbores, fructus, et radices: die Stephani pabula iumentorum: in die Ioannis vina: in die Paschatis lacticinia, mella, agnellos, carnes, butyrum, panes, placentas: in die Natali, gladios militares, vexilla militaria, altaria, campanas, salem, elementa, et similia, de quibus vide canoniale diuinorum, et similes libros.  \pend
\textit{6 Haec si subieceris fratribus, etc. }\pstart Epilogus est coeptae narrationis de futuris Ecclesiae corruptelis in doctrina de matrimo nio, et ciborum delectu : depromatur hinc propositio loci, Timotheus et omnis bonus Christi minister de his corruptelis quam fidelissime suos auditores monere debet:ratio  \pend
\section*{COMMENTARII }
\marginpar{[ p.94 ]}\pstart est, quia sic erit bonus minister. Ergo qui hoc negligit, non Christi est bonus minister, sed Antichristi, qui apostaticam sequitur doctrinam. Deinde sic declarabit, SE ENVTRITVM IN SERMONIBVS FIDEI, hoc est, peritiam habere sanae doctrinae. Λoγος πίςτεως, est certa et firma verae religionis formula, quam habere debet is qui alios docere vult. Hanc facultatem cum tribuat Timotheo, ostendit a puero in vera fide enutritum, primùm a Loide et Eunica in Prophetis, mox ab Apostolo Paulo in Christiana religione.  \pend\pstart Tertio, ET BONAE DOCTRINAE, καλὴ διδασκαλία est quae sana dicitur, et simplex fidei formula Propheticis et Apostolicis literis congruens. QVAM ASSECVTVS ES, addit, significans eum tenere veram mentem et sensum scripturae: significat enim παρακολουθεῖν, primùm mentem docentis assequi, vt discipuli cum audiunt praelegentem authoris mentem assequuntur ex interprete. Deinde debent παρακολουθεῖν, cum vita et morib. his ipsis actionibus intersunt, et respondent professioni, nam doctrina quaedam est theoria quae ad praxim refertur. Sic pii cum intersunt certaminibus persecutionum, debent assequi doctrinam, hoc est, percipere id quod Christus illis praedixerat: hic tamen pro do\pend
\section*{IN I. EPIST. AD TIMOTH. }
\marginpar{[ p.95 ]}\pstart cendi facultate loquitur proprie vt apparet.  \pend
\textit{7 Prophanas autem et aniles fabulas, etc. }\pstart cis, quae tamen ad propositionem, et illatioSecunda pars capitis, de vitandis aliis doctrinae sanae pestibus. Imprimis quoad dogmata disciplinae spectat. Nam fabulosam doctrinam et prophanam vocat eam quae in ritibus et disciplina vitae aliquid excogitat praeter Scripturae sacrae mentem. Huic opponit εὐσέβειαν, hoc est, veram rationem colendi Deum: et hanc esse mentem indicat mox consequens confirmatio, in qua rationem accommodat ad exercitia vitae corporalia, quae dum extenduntur ad maiora, hoc est, ad spiritualia, ius alienum inuadunt:igitur vt seruetur ἡ γνώμη τοῦ συγγράμματος, in propositione, profanum, fabulosum et anile de iisdem corporis exercitiis debet intelligi. Tria autem sunt huius loci membra: πρότασις, κατασκευη, et illatio: interseritur quoque grauis ἐκφώνη- nem pertinet.  \pend\pstart Propositio est, Timotheo et omni pio ministro curandum est diligenter ne in disciplina pietatis quicquam admittatur pugnans cum vero Dei cultu, vtut speciem sapientiae et religionis habeat. Haec praeponit Apostolus antithesi quadam: primum enim praecepti est oppositio. HOC AVERSARE, παραίοῦ: in illo vero, TEIPSVM γύμναζε Illud ἄρνησιν  \pend
\section*{COMMENTARII }
\marginpar{[ p.96 ]}\pstart habet, hoc θέσιν. Fugiendum est in disciplina morali μυθικὸν fabulosum, hoc est, quod speciem habet antiquitatis, sed cum vero Dei cultu non congruit: ideoque duo huic disciplinae addit epitheta. Primum quod sit βέβη- λον: deinde quod γραώδες. Βέβηλον dicitur impurum, nam Βήλον dicitur purum, qualis est puritas in caelo sereno, in astris, hinc Beβηλον impurum : nam βε syllaba significationem dictionis mutat: ideoque recte intelligimus, pollutum, coenosum, impurum, quod caeloet astris minime dignum est. Γραώδες vero dicitur, quod speciem habet sapientiae et antiquitatis, sed recedit a vera sapientia diuina, quae docet verum Dei cultum.  \pend\pstart Itaque damnantur hic duo pietatis verae extrema: defectus est Bεβηλοτὴς, contaminans pietatis exercitia profana vita. Excessus autem est γραώδες, hoc est, superstitio, quae alias diciturδυσιδαιμονία. Exemplo res declarari potest: tres simul vigilant diuersissimo fine et euentu. Apostolus Paulus aut Petrus vigilat pernox vt oret pro Ecclesiis Christi, aliis prosit, et munus sibi impositum melius obeat, docet Euangelium, et orat. Vigilat simul superstitiosus monachus aut Pharisaeus, qui vigiliis suis, opus supererogationis fabricatur, meritorium apud Deum pro sua salute. His addetur tertius profanus compotator qui pernox\pend
\section*{IN I. EPIST. AD TIMOTH. }
\marginpar{[ p.97 ]}
\marginpar{[ p.G ]}\pstart est in vino ab occasu Solis ad ortum eiusdem debacchatur. Hic quaeritur quis apud Deum bona vtatur disciplina, et cuius vigiliae Domi no approbentur. Hypocritae placere non pos sunt, quia δυσιδαιμονίᾳ, hoc est, superstitione peccat, et γραώδες est hoc opus, hoc est, peruë- tustus error est et anilis: lurconis quoque non placere possunt, quia in eius vigiliis est βεβη- λότης, hoc est, profanatio: itaque medium hîc quaerendum et sequendum est, ne in castigando corpore, aliam ve disciplina externa peccemus superstitione aut profanatione. Εὐσέ- βεια est idem quod θεοσέβεια, hoc est, recta Deum colendi ratio, quam non aliunde quam ex ipsius Dei voluntate petere oportet. Haec autem Sacris literis expressa est, ac refertur ad duo capita, fidem scilicet, et dilectionem leu studium bonorum operum.  \pend
\textit{8 Nam corporalis exercitatio, etc. }\pstart Confirmatio est propositionis: reddit enim rationem quare profanae et aniles fabuIae sint fugiendae. Ratio est, QVIA CORPORALIS EXERCITATIO PARVM PRODEST, PIETAS AVTEM AD OMNIA EST VTILIS, duae ratio siquid concludere debet, oportet fabulas aniles et profanas de exercitiis corporalibus accipere, alioquin nihil concludit oratio. Itaque nos de disciplina morum interpretati su\pend
\section*{COMMENTARII }
\marginpar{[ p.98 ]}\pstart mus. Est enim fabulosum in doctrina vel fidei vel morum ac disciplinae. sup. cap.1, vers, 4.taxauit quoque fabulas, sed ad doctrinam fidei, illae videbantur pertinere: nunc cum iterum damnet fabulas, et confirmet suum iudicium a ratione et fructu exercitiorum, consequens hoc genus fabularum ad disciplinam vitae spectare, Τυμνασία de omni dicitur, exercitio quo corpus, vel mens assiduo studio perpolitur: ideocum de disciplina morum loquatur, recte corporalem addidit: qualia sunt exercitia, nudipedalia, humicubia, ieiunia, vigiliae, preces longiores et saepe iteratae, peregrinationes, et similia quibus corpus proprie affligitur et castigatur. Tale exercitium fit aniIe deliramentum, si homines sibi tale deligant sine verbo Dei et studio promerendi sibi vel aliis, gradum ad vitam aeternam consequendam. Valet hoc sensu Apostoli argumentu, AD PAVCA VTILIS EST EXERCITATIO CORPORALIS, honc est, sua natura ad tanta non valet quanta superstitiosi ei conantur tribuëre: ergo vitium hoc cauendum est. Ὀλίγον illud quod sit, facile perspici potest ex operum natura: vt ieiunium res per se bona est, prodest etiam corpori, sed est tamem ὀλίγον, hie. res parua respectu eorum quae ei tribuunt superstitionsi, ac sanu reddit corpus sobrietas, vigiliae animum excitat ad orandum.  \pend
\section*{IN I. EPIST. AD TIMOTH. }
\marginpar{[ p.99 ]}\pstart Aliquid illa quidem sunt, imo magna, sed quid respectu abusus? hypocritae ei longe maiora assignant, merita erogationis et supererogatio nis, gradum in caelo, vitam aeternam, et similia.  \pend\pstart Deinde addit alterum argumentum a natura εὐσεβείας, QVAE AD OMNIA VTILIS EST: opponit gymnasiae veram pietatem, q.d.excessui veram rationem colendi Deum. Id quod plus prodest praeferri semper debet in vsu rerum: bonum enim maius praestantius est, sed pietatis ratio plus prodest caecis illis exercitiis. Ergo, etc.  \pend\pstart Πάντα accipio hic comprehensim generaliter, hoc est, ad vitae aeternae commoda, et vitam aeternam consequendam, et recte opponunt voces ὀλίγον et πάντα. Exercitatio corporis hic bis cadit: primum sua natura, deinde relatiue ad pietatem veram. Sed excipiunt hic supra, superstitiosi, quomodo pietas ad omnia possit? quomodo ad agriculturam? mercaturam? artes mechanicas discendas? Respondemus etiam ad has et in his plurimum prodesse, nam in omnibus actionibus aequitatem respicere iubet, Dei gloriam, proximi commodum, et iustitiae communem regulam: sic ad mercaturam prodest, ne fiat ex mercatore honesto turpis vsurarius, aut impostor, in mechanicis dictat honesta viuendi ratione, oportere esse contentum, nec defraudandum  \pend
\section*{COMMENTARII }
\marginpar{[ p.100 ]}\pstart in vendendo proximum: idem agricolae dictabit, et aliis in genere omnibus. Sed Apostoli explicatio audienda est. Propositum fuit, in disciplina vitae fugiendam superstitionem ac profanationem, sequendam autem ipsam veram colendi Deum rationem. Ratio erat, quia illa modicam habent vtilitatem:hoc autem ad omnia prodest. Nunc hoc etiam probat hac explicatione addita. HABET PROMISSIONEM VITAE PRAESENTIS ET FVTVRAE. promissionem accipio de defensione diuina, de auxilio Dei praesenter nos defen dente, eripiente ex periculis, fortunante conatus nostros, et similia. Nunc plana erit sententia: vera ratio colendi Deum quae dicitur εὐσέβεια, haec sola habet promissiones diuinas, de tutela, de defensione diuina quoad hanc vitam, et futuram: ergo ad omnia prodest. Si ad omnia prodest, ergo sola sequenda est. Plena autem est enumeratio praesentis et futurae vitae partitione proposita. Ac quod hanc vitam attinet, promissio manifesta est, quia vere Deum colentibus in veteri Testamento Deus promisit, quod ipsorum Deus et seminis illorum futurus sit, hoc est, protector et defensor: quae promissio ad Abrahami posteritatem pertinuit eatenus si manserit in statutis Domini, hoc est, Deum sic colerent, quemadmodum colendum sese proposuerat. Hinc  \pend
\section*{IN I. EPIST. AD TIMOTH. }
\marginpar{[ p.101 ]}\pstart sol pii recte Deum colentes sperare poterant illa bona externa, quae terrae Canaan nomine erant comprehensa, vt fertilitas, vbertas, sanitas, victoria contra hostes, et similia: et contra ab his exciderunt quoties veram rationem colendi Deum amiserunt, et cultum hominum iudicio excogitatum receperunt in mores. Porro sub his typis rerum externarum caelestia quoque illis erant promissa: vt habet promissio de regno Messiae in aeternum duraturo. Sic in nouo Testamento in hac vita auxilium praesens piis promittit Christus, Non relinquam vos orphanos, Ioan.14. et Matth.6. Quaerite primum regnum caelorum, et reliqua omnia adiicientur vobis: ex quibus efficitur veris Dei cultoribus proprie seruire promissiones etiam corporales, de vita aeterna, nemo ambigere potest, quin eius promissio ad solos εὐσε- βεῖς spectet. Recte igitur et vere dixit Apostolus, quod pietas huius et futurae vitae habeat promissionem, et ob id ad omnia sit vtilis, et hinc etiam quod sola sit sequenda in disciplina corporis.  \pend
\textit{9 Fidelis sermo, etc. }\pstart Ἐκφώνησις est, qua sententiam etiam confirmat, ac reuocat nos iterum ad ὁμολογούμενα Christiana, et αὐτόπιςτα, hoc est, quae per se et sine demonstratione debebant valere in Ecclesia Christi: qualia supra quoque recensuit  \pend
\section*{COMMENTARII }
\marginpar{[ p.102 ]}\pstart cap.1.ver.15. et initio cap.3.et ver.16, cap.3.tale hoc quoque est, quod in disciplina corporali sola vera Deum colendi ratio sit sequenda, quam vocauit εὔσέβειαν: et reuera ext ra controuersiam hoc debebat apud Christianos esse, pla ne apud omnes ὁμολογούμενον, et ἀναμφισβητη- τὸν: sed quis hic non scrupulos mouet? quis non bonam intentionem praefert? et c.  \pend
\textit{10 In hoc enim laboramus, etc }\pstart Pergit in confirmatione sua, addens superioribus suum exemplum, ac aliorum fideliũ, qui pro vera pietate sola aduersa quique ferebant, non autem pro disciplina corporis, vt exempli causa pro ieiuniis, vigiliis, precib.conceptis, non ferunt suas persecutiones, sed pro vera colendi Deum ratione, hoc est, cultu vero vnius Dei: ergo hic cultus solus respici debet. Talis non erat hypocritarum conditio, nihil illi pro vera pietate, sed omnia mouebant pro ceremoniis, pro ieiuniis, pro discriminib. ciborum et temporum, pro circuncisione, pro lotionibus, et similibus nugis anilibus disceptabant in religione Christiana: qualia hodie sunt ferme omnia certamina proPurgatorio, ieiunio, cereis imaginib. vigiliis, intercessione sanctorum, cucullis, vita monastica, et c. Obserua in hoc argumento, quomodo veri cultores Dei distingui possint ab hypocritis. Primum, crucis forma multum differunt,  \pend
\section*{IN I. EPIST. AD TIMOTH. }
\marginpar{[ p.103 ]}\pstart dum veri cultores duriter tractantur in hoc mundo, quod significat voce κοπιῶμεν. Deinde ignominiose quoque tractantur, quod exprimit vox ὀνειδιζόμεθα. Inde κόπος et ὄνειδος notae voces  \pend\pstart Quomodo Apostolus Paulus laborem et igno mimam subiuerit, ꝓ vera pietate, vide 2.Cor.11. et1Cor.4. ait, Persecutionem patimur et sustinemus. Rom.8. Quis nos separabit ab amore Dei? num persecutio, etc. Gal.5.Si adhuc circuncisionem doceo quid persecutionem patior? et de Mose scribit epistola ad Hebr.11. quod maluerit cum populo Dei ignominia affici, quam gloriose viuere apud Aegyptios: contra faciunt hypocritae.  \pend\pstart Deinde distinguütur etiam causa crucis, quia Paulus et similes affliguntur propter confessionem fidei in Deum viuentem, h.e. de Deo sentiunt et credunt quae filius Dei iussit nos credere. Hanc doctrinami deformant hypocritae in ritus et ceremnomias bona parte.  \pend\pstart Tertio, differunt etiam euentu, quia pii laeto rerum fine sperata bona obtinent: contra hypocritae spem irritam fouent, et ob id misere pereunt.  \pend\pstart DEVS DICITVR SERVATOR OMNIVM, h.e. benefactor, vt Matth.5. Qui sinit oriri Solem super bonos et malos, id est, communiter bonis et malis contulit et prospicit. IMPRIMIS AVTEM FIDELIVM, hoc  \pend
\section*{COMMENTARII }
\marginpar{[ p.104 ]}\pstart est, hos serio curat ipsorum bono fine, cum malis externa beneficia cedant ad damnationem. Oratio argumentatur a toto ad praecipuam totius partem, quemadmodum Psal. 135. dicitur, Dat escam omni carni. et Psal.111. Escam dedit timentibus ipsum, hoc est, dum omnia animalia prouida sua cura alit, imprimis tamen hac sua prouidentia complectitur electos. Sic Christus argumentatur a parte ad partem. Si curae sunt ei aues, quanto magis nos: si curat pilos vestri capitis, quanto magis animas nostras? simile est hoc, nisi quod a toto ducitur ad praecipuam totius partem.  \pend
\textit{11 Annuntia haec et doce, etc. }\pstart Illatio est totius loci, denuntiare refero ad errores reprehendendos in hac quaestione de exercitiis corporalibus, de quibus debet auditores diligenter et vigilanter monere  \pend\pstart Docerevero ad ipsam ευσέβειαν, hoc est, verum Dei cultum, qui in his sequendus est  \pend
\textit{I2 Nemo tuam iuuentutem, etc. }\pstart Tertia pars capitis huius de dotibus boui ministri quae etiam aetatem iuuenilem excusant: solent enim homines interdum materiam contemnendi doctrinam petere ab aetate docentis, contra quos hic disserit Apostolus. Et exempla declarant Deum in suis donis non excludere iuuentutem quin ei det  \pend
\section*{IN I. EPIST. AD TIMOTH. }
\marginpar{[ p.105 ]}
\textit{Nemo tuam iuuentutem despiciat, etc. }\pstart illustria dona: sic Daniel puer inter Prophetas habitus est, et Samuel. Dauid, iuuenis egregia dedit specimina futuri regis. Solomon iuuenis illustre tulit iudicium inter mulieres con tendentes de infante. Haec et similia ad laudem iuuenilis aetatis proferri possunt. Porro quia haec aetas enormia habet vitia, ideo non simpliciter aetatem excusat, sed in persona idonea, qualis erat Timothei, alioquin exempla satis testantur quantum iuuenes oratores Rebuspub.aliquando obfuerint: conqueritur Cicero alicubi inter alias causas dissipatae Reipubl. Romae illam fuisse, quod surrexerint oratores noui, homines adolescentuli. Et in religione idem saepe accidit.  \pend\pstart Verum ad rem. Apostolus docet hic quibus actibus docere possit etiam iuuenis cum authoritate et grauitate. Quatuor autem sunt loci partes. Prima habet materiam contemptus scil. aetatem: altera recitat ἀντιδοτα, hoc est, virtutes quae contemptum hunc tollere poterunt. Tertia, rem concludit. Quarta addit duo illustria argumenta a gemino fru ctu, habent illa vim excitandi ad diligentiam.  \pend\pstart Propositio sit, Aetas si caeterae adsint dotes, in ministro non vsque adeo respicienda est.  \pend
\section*{COMMENTARII }
\marginpar{[ p.106 ]}\pstart Praeoccupatione orditur locum: ac praeceptum dat auditoribus quid illis sit faciendum in persona Timothei: Non damnanda aut contemnenda doctrina, quia a iuuene tradatur.  \pend\pstart Ratio est, quia reliquae dotes illustres adsunt. Discimus hic Timotheum iuuenem fuisse, de quo Act.16. et sup.cap.3. et in illo illustria fuisse dona quae contemptum merito prohibere poterant apud pios: in quo valere debet illud, μᾶλλον τὰ ἔργα σκοπεῖν χρὴ, ἤ τὸν χρόνον, magis actiones quam tempus respicere oportet. Deinde magis etiam commendari senes si his virtutibus praediti sint: quia aetas illos magis liberabit a contemptu. Inde forte infra 5. duplici honore dignos facit.  \pend
\textit{Sed sis typus fidelium.. }\pstart Alter locus de caeteris dotibus quae adesse debent, si iuuentus in ministro respici non debeat. Facit autem illustrem ministrum quemadmodum Rhetores suum oratorem virum bonum; dicendi peritum. Addunt illi tria, πίςτιν, ἤθος καί πάθος. Πίςτις facultas dicendi vere illa in ministro est δύναμις docendi sanam doctrinam:  ἔθος , authoritatem conciliat, id fit in Ecclesia vitae innocentia: πάθος vero si graui oratione auditores ad sui imitationem possit propellere. Haec in Timo\pend
\section*{IN I. EPIST. AD TIMOTH. }
\marginpar{[ p.107 ]}\pstart theo videmus: primum enim sanam doctrinam illi tribuit iam prius, nunc addit virae grauitatem, et morum innocentiam, quo comparabit sibi etiam iuuenis authoritatem, et facultatem consolandi et exhortandi.  \pend\pstart Sed ordine videamus quas dotes iuuenis minister sequi debeat, vt doctrinae suae authoritatem faciat apud auditores. Primum, TYPVS, ait, SIS FIDELIVM IN SERMONE: typum esse vult, hoc est, exemplar gregis, quod tuto possint imitari auditores. Idem praecipitinfra Tito cap.2.vers.7. nam idiotae plus docentur exemplo quam oratione, adeo vt quibene docet, male autem viuat, plus destruat quam aedificet: sic Christus exemplar fuit, et dedit nobis humilitatis typum, Ioan. 13.tales Apostoli et Prophetae.  \pend\pstart Deinde, quos debeat respicere, pios scilicet, hi enim sunt quos non licet offendere. Impii semper habebunt quod carpant, quem madmodum momus, ideo illorum offensa non curanda est, nec indigne feret si his non ad omnia satisfecerit: sed piorum semper habenda est grauissima ratio.  \pend\pstart Tertio, sequitur virtutum catalogus, in quibus tanquam in lucidissimo speculo contemplanda est vita et doctrina.  \pend\pstart Primum, in sermone debet minister praelucere dotibus omnibus, quae in aliis me\pend
\section*{COMMENTARII }
\marginpar{[ p.108 ]}\pstart diocres ferri possunt, in ipsis excellentes debent esse. Hac vt ad mores principis totus componitur orbis: sic auditores facile arripient pastoris vitam, vt sit populus qualis est sacerdos. Vt igitur moratus sit, sacerdotem moratum esse oportet, et vere voμεν ἔμψυχον, vita debet esse lex. Id orditur a doctrina : λόγον enim accipio, primum pro doctrina sana quam profitetur publice ne obscuret ipsam aut sophistice implicet gryphis non necessariis. Deinde sermo quotidianus sit castus et pudicus.  \pend\pstart Deinde, SIT TYPVS ETIAM IN CONVERSATIONE: duo summa sunt capita, do ctrinae puritas, et vitae innocentia:illam dixit in genere λόγον. hanc dicit ἀναςτροφήν, qua conplectitur totam vitam, quae simpliciter debet esse inculpabilis, vt 1.Pet.2.ver.12. Apostolus Petrus quoque requirit a fidelibus om nibus: ideo hic in ministro excellentia accipiuntur: virtus enim in auditoribus quoque requiritur, sed excellentia in doctore.  \pend\pstart Tertio, IN DILECTIONE, tum qua Dei amnorem testabitur, tum erga proximum benefaciendi et viuendi studium, vt Deum supra omnia diligat: proximum autem vt seipsum, vel proxime ad hunc accedat gradum.  \pend\pstart Quarto, IN SPIRITV, hoc est,zelo pieta\pend
\section*{IN I. EPIST. AD TIMOTH. }
\marginpar{[ p.109 ]}\pstart tis, qui gestibus, oratione, habitu, vita, adeoque verbis ac factis declarari potest: is zelus quia singularis est motus spiritus Dei agentis in hominibus piis, ideo Spiritus dicitur simpliciter, qualis fuit in Mose, Helia, Ioanne Baptista, Stephano, aliisque, imprimis in filio Dei.  \pend\pstart Quinto, IN FIDE, hoc est, professione verx doctrinae, in asserenda et propugnanda veritate, in toleranda cruce pro nomine Dei : in his certaminibus praebeat eremplum suis auditoribus, quantum pre Chusto, et pro veritatis confessione facierdum sit.  \pend\pstart Sexto, IN CASTITATE, speciale est vitium iuuenum ardor venereus: hunc igitur in tota vita reprimendum esse docet, et optandum est imprimis adolescentiae, vt Deus illorum corpora casta et impolluta conseruet, ne se scortatione foedent accessuri munus docendi.  \pend
\textit{13 Donec venero attende lectioni, etc. }\pstart Transit ad occupationes literarias et priuatas. Ac commendat illi primum lectionem domesticam, hinc commoda eius subindicat. Sit igitur septima virtus quae iuuentutem excusabit in ministro, assiduum esse in lectione Scripturae. De lectione aliâs diimus. Commendatur Psalmo 1. nam inter  \pend
\section*{COMMENTARII }
\marginpar{[ p.110 ]}\pstart beati viri virtutes, ponitur die et noctu meditari in lege Domini: et Ioann.5. Christus iubet scrutari Scripturam. Sic neglectum eius taxat Oseas in populo Dei, cap.4. et alii Prophetae, quoties Iudaeis obiiciunt quod a Domino Deo suorecesserint, verbum illius neglexerint.  \pend\pstart Porre simul hic promittit aduentum suum, ex qua spe tamen videtur postea excidisse, nam Ephesinos seniores Miletum euocauit praeternauigans Ephesum. Et notanda est phrasis, praecipi lectionem donec venerit, hoc est, semper: non enim voluit hancnegligi postquam ad eum venisset, sed intentissime nunc hac occupari ad aduentum de aliis etiam eum moniturus. Facit contra errorem Heluidii, qui male interpretabatur Hanc partieulam Donec.  \pend\pstart Octauo, INTENDE ETIAM EXHORTATIONI, VEL CONSOLATIONI: pigri indigent stimulo, hi igitur excitandi sunt: languentes vero consolatione, ne desperent: recte currentes confirmatione, quae tria omnia παρά- κλησις nobis significare potest.  \pend\pstart Nono, ATTENDE QVOQVE DOCTRINAE, hoc est, vt opportune doceas, ne bonam causam male agendo deperdas: doctrina indigent rudes, vt iuuentus, cui rudimenta proponuntur pietatis perspicua breuitate. Deinde  \pend
\section*{IN I. EPIST. AD TIMOTH. }
\marginpar{[ p.111 ]}\pstart adulti haerentes in perplexis quaestionibus, vtlocis in speciem pugnantibus. Item contumaces, qui nostra oppugnant: nam et κατη- χητικόν est διδακτικόν, et ἐλεγχτικὸν, et καταςκε - υαστικὸν, hoc est, quo nostra probamus, quo errores confutamus, et quo rudimenta tradimus. Porro haec duo: doctrina inquam et consolatio, fluunt ex lectione priuata, quae dictabit modum in his, et suppeditabit semper idonea quaestionum omnium argumenta et valebit hic quod dici solet, lectio decies repetita placebit. Ideo diligenter et serio a iuuenibus est amplectenda.  \pend
\textit{14  Ne donum negligas, etc. }\pstart Transit ad generalia, nam hic eum hortatur in genere ad donorum suorum cultum, vt illa diligenter excolat, et auctiora reddat. Sit illa cura decima donorum priuatorum studium augendi illa et reddendi splendidiora. χαρίσματα intelligo tum naturae indolem, quod fuerit εὐφυὴς, quod quanquam naturae sit, non mpedit Spiritus sancti esse donum.  \pend\pstart Deinde dona piae educationis, vt quod sacras literas a puero didicerat. Denique singularia, quae illi forte miraculose collata erant in impositione manuum, qualia tum saepe continge bant piis, quale fuit maior dexteritas interpretandi Scripturam quam habuerit prius:  \pend
\section*{COMMENTARII }
\marginpar{[ p.112 ]}\pstart solertia regendi Ecclesiam, docendi Euangelium, et similia. Hoc donum non debet neghgere, hoc est, debet auctius reddere, negligunt qui contemptim de his iudicant, ne gligentia. sua reddunt deterius: quod Sct iptura dicit defodere suum talentum. Excolendum igitur est singulis suum χάρισμα, hoc est, talentum qualecunque illud sit: et nemo credit quantum assiduus vsus et conatus in hac re valeat.  \pend\pstart Deinde considera hic quomodo illi datum sit hoc donum, PER PROPHETIAM. Prophetiam hic etiam accipio pro piorum omine felici, et coniectura. Act.1 6.testimonium bonum habuisse scribitur. Inter illos fratres puto quosdam Prophetico spiritu praedixisse qualis hic iuuenis aliquando futurus esset doctor in Ecclesia Christi: et his testimoniis Paulus inductus est vt illum sibi adiungeret. Itaque hic ponitur tanquam caula occasionans.  \pend\pstart Tertio, quomodo in ipso sit reipsa confirmatum: IMPOSITIONE MANVVM SENIORVM, qui qualis fuerit ritus alibi diximus. Hinc certe constat cum impositione manuum Apostolorum, et aliorum fidelium, Deum certa edidisse testimonia, quod talis persona illi ad tale munus placeret. Illapsus est aliquando visibili specie Spiritus sanctns, aliâs  \pend
\section*{IN I. EPIST. AD TIMOTH. }
\marginpar{[ p.113 ]}\pstart linguis nouis loquebantur, aliosq. motus no uos in auditorib.edebat S.sanctus, et indubie siidem symbolum rite administraretur adhuc hodie in electionibus non minus quam olim esset efficax, si maxime externa et insolita ratione sese non exereret Spiritus S. Locus est de vocatione quoque ordinaria, de qua vide locos.  \pend
\textit{15 Haec exerce, in his esto, etc. }\pstart Illatio sequ itur post longam donorum comme morationem: et ταῦτα, refero ad omnes dotes enumeratas, q.d.in his omnibus tibi elaborandum est: et μελέτη est studium ardens aliquid reddendi auctius, id em repetit immoratione, IN HIS ESTO, hoc est, haec tua sit praxis, illa tua occupatio, cura et solicitudo. Addit praecepto argumentum a causafinali, VT PROFECTVS TVVS CONSPICVVS SIT OMNIBVS: duplex est sensus argumenti. Primum si neutraliter accipiamus eπᾶσιν, sensus erit, in omnibus iam commemoratis virtutibus illustris et notorius debet tuus esse profectus. Deinde si ad auditores re feras ἐν πᾶσιν, hoc est, apud omnes, coram om nibus hominibus notoria debet esse tua virtus, quae laus admodum praeclara est, si minister sic vixerit, et a virtutibus omnibus sit notus.  \pend
\textit{16 Attende tibi ipsi et doctrinae. }\pstart Quarto loco appendicem addit duplici argu\pend
\section*{COMMENTARII }
\marginpar{[ p.114 ]}\pstart mento munitam. Duo autem hic consideranda veniunt: praeceptum etcausa praecepti. Praeceptum est, ATTENDE TIBI ET DOCTRINAE: quod sibi ipsi iubet attendere, refero ad moderandos affectus inter docendum, vt iuuenilibus cogitationibus sese non permittat, sed audiat Spiritum S.optimum gubernatorem. Nos diceremus, halt dich ein zoum: ideo mox sequenti capite praescribit for mam quomodo cuiusque aetatis et conditionis homines debeat vtiliter admonere.  \pend\pstart Doctrinae aunt attendere, accipio pro sanae doctrinae cura, vt ad hanc quasi cynosuram omnia sua consilia referat, et doceat quam frequentissime sanam doctrinam, vt quam latissime propagetur, vt ipse Apostolus fecit. Hanc sententiam etiam immoratione illustrat, vt prius in illatione: non enim potest rei huius magnitudinem, et necessitatem satis commendare.  \pend\pstart Causa tanti praecepti est duplex, et vim habet ab egregio et specioso fructu. Hoc SI FECERIS, SERVABIS TEIPSVM, hoc est, consequeris coronam illam beatam quae debetur electis, inter quos ipsi doctores praecipui sunt. Prima igitur cura doctoris sit pro suae animae salute: dicitur autem seruare se organice, alioquin principaliter et absolute Deus seruat, vt priusdixit, qui est seruator  \pend
\section*{IN I. EPIST. AD TIMOTH. }
\marginpar{[ p.115 ]}\pstart omnium hominum, maxime fidelium. Hoc non euertit quin Timotheus sui quoque sit seruator: Deus enim in ipso seruando eodem vtebatur organo. Sic ad Philip.2. iussit nos vt cum timore et tremore operemur salutem nostram: seruat igitur Deus nos, sed non sine nobis, et coronat in appellationibus sua donain nobis, cumque ipse vnicus sit saluator et seruator, tamen nos nostri seruatores dici permittit.  \pend\pstart Altera causa est, diligentia in officio efficiet quoque vt alios seruet, hoc est, auditores suos, hos, inquit, seruabis vna tecum, hoc est, eris causa media et organum bonum, per quod illi saluabuntur. Sic Iacobus cap.5.ait, Qui peccat orem conuertit a peccato suo, animam liberat a morte. Et Dan.12.Splendebunt vt stellae caeli in aeternum qui alios ad iustitiam erudierint. Sic Apostolus dicit se ge nuisse Corinthios, quum solus Deus regeneret.  \pend\pstart Organis tribuit Scriptura nomine, quia coperarii sunt Dei et dispensatores mysteriorum Dei. Sed organa se meminerint, quae si ad Deum comparentur, recte dicent se nihil esse, etc.  \pend
\section*{COMMENTARII }
\marginpar{[ p.116 ]}
\textit{CAPVT V. }
\textbf{T }\pstart RADIT hoc cap. diuersas regulas ipsi Timotheo, quas in administratione officii sui sequatur etiam oportet, is, cui Ecclesiarum Christi cura est deputata. Initio, vt doctrina frugifera sit ostendit: quomodo cuiusque aetatis, et sexus homines officii admonere debeat, scilicet summa cum lenitate, ne quod odium priuatum sese in corripiendo, aut arrogantia iuuenilis prodat.  \pend\pstart Deinde transit ad curam viduarum, docens quae de publico ali debeant, quaeue non: item quales ad ministerium pauperum admittenda sint, vt conuersatio illarum scandalo careat.  \pend\pstart Tertio, agit de presbyterorum cura et prouisione.  \pend\pstart Quarto, de peccantibus, quomodo ferendum fit iudicium.  \pend\pstart Sexto, priuata adiicit, de curanda valetudine in tot laboribus. De his fere disputatur hoc capite quinto.  \pend\pstart Propositio in singulis locis quaerenda et con  \pend
\section*{IN I. EPIST. AD TIMOTH. }
\marginpar{[ p.117 ]}\pstart tituenda est specialiter, ne partes confundantur. Habent tamem hic ministri praesertim iuniores quod sequantur: De moderanda acerbitate inter docendum, ne tragicis clamorib. cathedram Christi onerent potius quam ornent. Seniores vero quibus aliorum inspectio et ordinatio est conmissa, ne cui manus imponant, omnes vero vt suorum in alendo curam habeant, ne praeter necessitatem onerent Ecclesiam: sed de singulis audiamus ipsum Apostolum.  \pend
\textit{Vers.I. Seniores ne increpes, sed, etc. }\pstart Locus primus est de moderatione adhibenda in corripiendis peccatoribus. Propositio est, Timotheo, et ob id omnibus ministris, imprimis vero iunioribus opera danda est, vt prudenter et cum lenitate quadam pia auditores corripiant. Prudenter id fiet si habeatur ratio aetatis etsexus: leniter autem si modestiae limites non excedat iuuenis mimiter.  \pend\pstart SENIOREM, inquit, NE INCREPES DVRIVS, SED HORTARE VT PATREM. Seniorem accipio simpliciter aetate, quanuis Ambrosius ad officii dignitatem referat: verum Apostolus respicit totum coetum Ecclesiae, qui constat peccatoribus aetate et sexu differentibus. His omnibus prae est Timotheus iuuenis: hunc igitur primum cum auditoribus committit, deinde cum aetate paribus aut infe\pend
\section*{COMMENTARII }
\marginpar{[ p.118 ]}\pstart rioribus, mox cum sexu differentibus, vt sciat quomodo se erga omnes gerere debeat. Seniorem non debet ἐπιπλήττειν, hoc est, durius tractare verbis et sermone, tragico terrere, quasi plagis castigare, nam hoc modo non seruaret decorum suae personae, quod ne salua quidem pietate negligi potest. Videant igitur hodie ministri quomodo temperent suas ἐπιπλήξεις, certe plus strepitus auditur quam rei, personant magis quaedam templa vociferationibus more bacchantium, quam vtili doctrina, id vitandum est Timotheo: faciendum autem illi, SED HORTAREVT PATREM. In genere respicit personam docentis Timothei quem vt iuuenem opponit senioribus in coetu, qui opus habent correptione, quia et ipsi delinquunt, et peccatores sunt: tamen vt cum fructu eos corripiat, prudenter admonet lenitatis: finis enim docentis est peccatorem senem etiam reddere meliorem, id fieri non potest si in sermone reprehensoris prodat se insultatio arrogans, mordax contemptus, iuuenilis petulantia, et similia. Igitur et oratio et gestus accommodandi sunt, vt summam in reprehendendo lenitatem intelligat, hoc est, animum filii prodere. Seniorum honori parcere, loco parentum colere. Hoc Timotheum iubet praestare: igitur omni\pend
\section*{IN I. EPIST. AD TIMOTH. }
\marginpar{[ p.119 ]}\pstart bus iunioribus ministris curandum est vt prudenter et leniter seniores castigent.  \pend
\textit{Iuniores vero vt fratres, }\pstart Et hic tenenda est prudentia et lenitas: prudentiae est aetati sese posse accommodare, lenitas fraternitatis vinculum retinere. Matt.23. Christus nos omnes fratres esse dicit.  \pend
\textit{2 Mulieres natu grandiores, etc. }\pstart Transit ad sexus discrimen: ac ponit iterum duo praecepta: matronae peccantes corripiendae sunt vt matres, iuniores vt sorores in omni castitate: ergo debet adesse ἐπίπληξις, adhibenda autem est παράκλησις: haec ministris curanda sunt. Nos qui de numero auditorum sumus, discamus omnes nos opus habere correptione, quia omnes peccatores sumus, ratio est in verbis Apostoli, QVIA SENES OPVS HABENT CORREPTIONE, IVVENES OPVS HABENT EADEM, MATRONAE ET IVVENCVLAE, quae plena est enumeratio. Quod si qui reprehendi nolint, caueant ne ei materiam praebeant: minister vero si hanc moderationem non seruet, ipse viderit:omnia debent ad aedificationem facere quae euentu ipso prodit, quam studiosi sint omnes moderationis et lenitatis.  \pend
\textit{3 Viduas honora quae vere, etc. }\pstart Secumdus locus de cura viduarum. Propositio  \pend
\section*{COMMENTARII }
\marginpar{[ p.120 ]}\pstart igitur hic specialis constituatur, Ministro diligenter curandum est vt viduae de publico alendae, aut ministerio Ecclesiastico deputandae, idoneae solum assumantur. Quod vt recte intelligamus sic accipi debet. Habebat vetus Ecclesia imprimis pauperum rationem, quibus connumerabat orphanos, viduas, exu les, peregrinos, valetudinarios, etc. In quibus curandis vtebatur ministerio quoque viduarum: nam inulta rectius per viduas administrantur quam per viros, vt adesse aegris puellis, matronis, puerperis, infantibus, pueris, et c. Ideo vt vitaretur passim ois scandali occasio, praefecit talibus viduas quae de publico simul alerentur: vnde postea illis inhaesit nomen vt dicerentur diaconissae, in nostris regionibus vmbra illius rei fuit in collegiis paeginarum, die Bteginen huser, in quibus tales aleban tur viduae, quae aegris adessent si res posceret.  \pend\pstart VIDVAS, inquit, HONORA: viduam a viduitate appellamus, hoc est, solitudine, sic Graeci χήραν vocant quasi ἔρημον, desertam a viro. χηρόω illis est idem quod ἐρημόω, vasto, destituo, et recte, quia vidua est viduata, hoc est, orbata viro: sic χήρά ἑςτι  μονοθεῖσα ἀπʹ ἀνδρὸς, vel vt Apostolus loquitur, μεμονωμένη, solitaria facta, et destituta viri patrocinio et  \pend
\section*{IN I. EPIST. AD TIMOTH. }
\marginpar{[ p.121 ]}\pstart necessario auxilio: tales honora, inquit. Verbum τιμᾶν accipio de victu etiam et amictu, vt in Decalogo dicitur, Honora patrem tuum et matrem, hoc est, prospice illis de victu: sic honorandas hic accipio viduas, hoc est, illis prospice etiam de victu vt alantur. Ad prius tamen praeceptum, de correptione moderamda, potest etiam accommodari, HONORA VIDVAS, hoc est, modeste quoque illas admone sui officii: sunt enim et hae peccato obnoxiae. Dum igitur peccant, minister iuuenis orationem ad illas habiturus seruet decorum, vt honorare videatur increpando, vt dixit de adolescentibus mulieribus, quas obiurgare debet seruata castitate. Ex quo et illud sequitur, ministrum iuuenem cum viduis iuuenilis periculose colludere familia eius: laeditur enim castitas, si non facto, certe verbis et moribus, quae vno saepe verbo proditur.  \pend\pstart Sed placet prior sensus de curandis viduis ex publico, quod recte dicitur honorare. Ac sciant viduae honorem virtutis praemium solum esse: si igitur honorari velint, virtutem Vt habeant curent, ne sola aetate se venditent: quo facit Apostoli forma loquendi, quod addit, quae vere sunt, ὄντως χείρας, non obscura est differentia inter viduas: aliae vere sunt, quod dicuntur re et nomine viduae: aliae solo nomine. Apostolus igitur praeceptum de ho\pend
\section*{COMMENTARII }
\marginpar{[ p.122 ]}\pstart norandis viduis applicat, ad solas illas quae ὄντως sunt quod dicuntur: id quale sit ex subiectis notis intelligatur.  \pend
\textit{4 Si vero aliqua vidua, etc. }\pstart Ἔκθεσις est plenior, quae sint vere viduae, et quae non vere. Vere est vidua, quae non solum viro orbata est, sed quae non habet Τέκνα, nec ἔκγονα, hoc est, nulla plane auxilia humanitus expectanda.  \pend\pstart Itaque illa quae habet filios, filias, aut nepotes, similesve personas, nondum est vere vidua, et ideo non idonea quae de publico alatur, nisi et isti plane omni subsidio careant. Hoc postea dicet μεμονωμένην. Colligamus igitur notas idoneae viduae alendae ex bonis Ecclesiasticis. Prima sit nota, DEBET ESSE VERE' VIDVA, hoc est, μεμονωμένν, orbata et marito et omni sobole: ratio est, quia dum tales supersunt, debent discere parem referre gratiam. Sed est ambigua sententia: discant primum propriam domum εύσεβεῖν, et reddere vicem maioribus: potest intelligi primum de viduis, vt ipsae malint cum suis habitare, et in pietate domesticos erudire, quod recte dicitur ἴδιον οἴκον εὐσεβεῖν, hoc est, pie colere suam familiam: est enim leuitatis argumentum si vidua nolit suis cohabitare, et ma gis delectetur aliorum conuersatione, quam domus suae et familiae. Et id etiam dicitur  \pend
\section*{IN I. EPIST. AD TIMOTH. }
\marginpar{[ p.123 ]}\pstart vicem rependere maioribus suis:: nam vidua suis maioribus debet si recte instituta est, hanc pietatem relinquat posteritati. Alter sensus est, qui ad filios et filias seu nepotes illa refert, ac discant accepit de filiis: illi discant viduam matrem pie alere, quo exemplo domum suam in pietate confirmabunt, et vicesreddere maioribus pium esse docebunt: quod Graeci dicunt ἁμοιβὰς et ἀντιθρεπτήρια αποδιδόναι, parem referre gratiam, ἀντιπε- λαργεῖν, ciconiarum more gratum esse parentibus. Vterque sensus recte quadrat ad institutum Apostoli, vt scilicet vidua non alatur de publico cum a suis ali possit  \pend\pstart Addit Apostolus rationes duas: prior est, QVIA HOC FACTVM EST καλόν scilicet natura: docet enim natura maioribus oportere gratiam referre, vide leges ciconiarum apud Aristophanem, quibus iubentur pelargides seniores ciconias alere, et exempla Ethnica illustria in hac gratitudinis specie Deus nobis proposuit, vt in Cimonis filia quae patrem vberibus in carcere alit: Aeneae, qui patrem senem ex incendio Troiae effert, et alia infinita, quibus ostendi potest καλὸν τῇ φύσει.  \pend\pstart Altera est, quia idem DOMINO EST GRATVM,h.e. officium hoc Deo acceptum est: nam multa videntur natura honesta, quae tamen  \pend
\section*{COMMENTARII }
\marginpar{[ p.124 ]}\pstart talia non sunt, nec pacatur his Deus: et vim vi repellere Christiano in loco et tempore non licet. Voluptas natura honesta videtur, quae piis tamen interdicta est: sed dum concurrunt, ista duo validissima sunt, naturae iudicium et Dei approbatio.  \pend
\textit{5 Quae vero reuera vidua est, etc. }\pstart Altera pars ἀντιθέσεως superior non vere vidua erat, quia habuit τέκνα vel ἔκγονα a quibus debuit et potuit ali. Haec veron nunc reuera est vidua: vides ὄντως χήραν describi quae sit μεμονωμην, hoc est, orbata et marito et sobole. Additur mox altera nota vere viduae, nec illud satis erat vt esset idonea quae de publico aleretur. Altera igitur nota est, pietatis constans meditatio ac studium: quod Apostolus dicit, SPERAT IN DEVM, scilicet quia omni ope humana est destituta, non habet maritum, non filium, nec filiam: superest igitur soli Deo nitatur. Nam si filii sint, spes aliqua est in vicino posita: talis igitur soli Deo fidens in orationibus et precibus perduret, vt sit idonea quae de publico alatur, et ad ministerium quoque admittatur.  \pend
\textit{6 Petulans autem, vivens mortua est. }\pstart Iterum contrarium addit vt fecit in prima nota: illic opposuit μεμονωμένην, et ἔχουσαν τέκνα, ἤ ἔκγονα.  Hîc προσμένουσαν προ-  \pend
\section*{IN I. EPIST. AD TIMOTH. }
\marginpar{[ p.125 ]}\pstart σευχαῖς opponit ei quae est σπαταλουσα, hoc est, lasciuae et petulanti. Vox desumpta est a spathale, quae significat oruamenta muliebria, qualia in brachiis et collo habent superbulae, Hinc σπαταλᾶν, luxum prodere eiusmodi ornamentis, id si faciant viduae, indignae sunt quae alantur de publico, quia mortuae sunt Christo. Viuens morrua dicitur, quia officio suonon respondet: et viuum dicitur cadauer. Christo mortua est et viuit mundo: hinc σπaτᾶν, σπαθᾶν, et σπαταλᾶν, idem pro ἀναλίσκειν ἀτάκτως, hoc est, sumptus facere inepte et lasciuire: quale in illo Aristophanis notatur. ὦ γύναι λίαν σπαθᾶς, ô mulier nimium lasciua es. Σπαθα̃͂ς, hocest, τρυφᾶς: hinc etiam σπα- τάλν cibus delicacior προφὴ, aut in quo mores sunt petulantiores, quale est in illo Graeco carmine, Οἱ δὲ μύες νῦν ὀρχοῦνται, δρασάμενοι τῆς σῆς σπατάλης, Mures nunc saltant, fruentes tua spathale, hoc est, carne felium.  \pend
\textit{7 Et haec indica, etc. }\pstart Repetitio est praecepti totum complectens. Duo autem exprimit: primo Timotheo non nisi idoneas admittendas esse: deinde hoc publice proponendum vt viduae tales sint re ipsa inter Christianos, quod vno verbo dicit ἀνεπίληπτοι, irreprehensibiles, hoc est, si suos habeant, ne petant de publico ali, piae sint non lasciuae.  \pend
\section*{COMMENTARII }
\marginpar{[ p.126 ]}
\textit{8 Si qua vero, etc. }\pstart Comminatio est quae habet vim argumenti, ne de publico ali velit, quae a suis ali possit. Ambiguus iterum est sermo, sed vtrinque seruit instituto. Nam si de viduis lasciuis accipias, quae suis neglectis, aliunde vitam quaeritant: recte dicitur de eo quod suos negligant, illis non consulant, nec prospiciant cum quib. manere nolunt: et saepe illarum prae sentia rem domesticam plurimum auget: TALES FIDEM NEGARVNT, h.e. Christianam professionem, cui non respondent, et ideo infidelibus sunt peiores, qui suis pie et sancte prospicere maluerunt, et apud filios tenuiter viuere, quam alibi laute. Itaq. nulla pia vidua hoc conari debet. Sin ad filias referas, illas quoque terrebit argu mentum vt curent suas matres viduas, quia domesticae sunt, prae terea propriae sunt, praeterea fidem non negasse videbuntur, et ab Ethnicis minime victae: certum enim est ingratas erga parentes damnari ab Ethnicis, cuius rei vide exempla apud Val.de gratis et ingratis.  \pend
\textit{9 Vidua allegatur, etc. }\pstart Tertia sequitur nota quae vidua digna possit esse ministerio publico, et alimento, que scil.attigerit annum 6o. Nam talis inepta est partui et ob id libidini minus subiecta. Posterius seculum mitigauit hunc numerum, quadragenariam admittendo: plus peccant hodie, quamlibet  \pend
\section*{IN I. EPIST. AD TIMOTH. }
\marginpar{[ p.127 ]}\pstart suscipiendo modo votum sanctae viduitatis promittat, licet id nunquam seruare cogitet.  \pend\pstart Quarta nota hic etiam additur, quae fuerit vnius mariti vxor, hoc est, vno viuente, non alteri se copularit, nec etiam suo aliquovitio ab vnius matrimonio repudiata ad alium transierit thorum, quod est δευτέροις γάμοις ὁμιλεῖν, nec etiam extra matrimonium prostituta pudicitia vixerit: sic vnius mariti vxor fuerit. Igitur non damnatur si vno mortuo ad alterius connubium transeat honeste, quia mors cum soluat matrimonium, efficit vt adhuc vnius sit mariti vxor.  \pend
\textit{10 In operibus honis, etc. }\pstart Quinta nota habeat etiam bonorum operum plenum testimonium: μαρτυρουμένην accipio vndique conuictam a bonis operibus, hoc est, cuius pietas publice sit notoria. Et quae sint bona opera hic accipienda vtiliter admonet ne alia intelligeremus. Sunt igitur erudire liberos, hospitalem esse, piorum lauare pedes: hi enim sunt sancti in Scriptura, afflictis adesse. Haec matronalia sunt ornamenta, quae viduam postea factam commendabunt vt de publico ali debeat, et ministerio adhiberi possit.  \pend
\textit{II Iuniores autem viduas auersare, etc. }\pstart Locus est de iuniorib. viduis diligenter i mimi steriopublico cauendis. Constat locꝰ propositione,  \pend
\section*{COMMENTARII }
\marginpar{[ p.128 ]}\pstart ratione, et illatione. Propositio est, IVNIORES VITATO, hoc est, indignas iudicato publico alimento et ministerio Ecclesiae, item priuato et solitario colloquio ne violetur castitas: periculosum enim est iuuenem ministrum cum iuuencula vidua secretius agere, et familiarius colloqui: proditur enim pudor, et proximus est lapsus et periculum scamdali. Ratio est a communiter consequentibus vtplurimum veris. Primum.n.SOLENT LASCIVIRE CONTRA CHRISTVM, ET NVPTVRIVNT MVTATO CON. quod quum in Ecclesia Christi sine infamia non possint, cogitant plenam defectionem a tota professione Christi, quod est lasciuire contra Christum.  \pend
\textit{12 Habentes iudicium, etc. }\pstart Alterum argumentum ex superiori natum: fidem primam abnegarunt, ergo cauemdae sunt. Item, habent iudicium, hoc est, damnabiles sunt : igitur cauendae. Κρῖμα habent, hoc est, κατακρίμα, idque apud Deum, quem fallere non possunt. Deinde apud piam Ecclesiam militantem, quomodo sua petulantia offendunt. Denique apud Satanam cui parent, qui accusator illarum erit apud Deum.  \pend\pstart Primam autem fidem accipio pro tota Christianismi professione, quae merito prima fides est, hoc est, prima fidei professio. Idem Graecis dicitur πίςτις, hoc est, fidei professio,  \pend
\section*{IN I. EPIST. AD TIMOTH. }
\marginpar{[ p.129 ]}\pstart quam adulti edebant in Baptismo post conuersionem, renunciantes Satanae et mundo, obstringentesq. se vni Christo. Hunc negant viduae lasciuae, cogitantes defectionem a toto Christianismo. Alii primam fidem accipiunt pro voto castae et sanctae viduitatis, quod votum postea frangunt nuptias meditando. Hinc πίςτις est συνθήκη et Neder, votum: mihi prior arridet sensus.  \pend
\textit{1 Simul autem et otiosae, etc. }\pstart Tertium malum quod hinc fluit: redduntur ἀργαί: quarto fiunt vagabundae: quinto φλύαροι, nugaces, garrulae: φλύαρος dicitur nugarum amatrix, et φλυαρολογία, studium nugandi, et φλυαρόγραφοι, illarum scriptores: deniq. περίεργοι, curiosae, inquirentes in res non necessarias, vel quae ipsarum scire nihil intersit: inquirunt enim libenter in proximi vicini mores et familiam. Haec catena malorum efficit vt sint cauendae. Argumentum omnium tale est, Quid quid est periculosum et plenum scamdalo, id ministro diligenter est cauendum, sed iunores viduas ministerio adhibere, et illarum vti consortio, plenum est periculo ac scandalo:ergo diligenter est cauendum.  \pend
\textit{14 Volo igitur, etc. }\pstart Illatio ex superiorib. vt plurimum consequentibus, et veris malis: constituit autem generalem regulam quae valere debet intota  \pend
\section*{COMMENTARII }
\marginpar{[ p.130 ]}\pstart illa aetate, scil. vt iuniores omnes nubant dum praedicta pericula cauere non possunt. Hinc ꝓbantur secundae nuptiae, quia Apostolus illaimperat iuniorib. viduis, adeoq. praefert voto sanctae viduitatis: igitur sanctas esse oportet nuptias secundas. Ratio est, quia his votum castitatis laqueus est, et occasio perditionis: ergo vitandum. Addit autem Apostolus bona opera, vt prius, coniugalia, quae vere Deo placent: vt est liberos gignere et educare, domum regere, cauere infamiae notas. Vide de hac re elegantem epist. apud Cyprianum, lib.1.epist.11.est.  \pend
\textit{15 Iam enim quaedam, etc. }\pstart Addit conclusioni nonum argumentum ab exemplis sumptum: vsu enim ipso edoctus Apostolus canonem hunc de nuptiis repetendis constituit. De certis exemplis loquens, ait quasdam euersas esse, retro Satanam abeundo, quod non potest nisi de plena apostasia a to ta Christianismi professione intelligi.  \pend
\textit{16 Quod si quis fidelis, etc. }\pstart Generalis epilogus secundi loci, de ido neis saltem admittendis, vt de publico alantur: repetit igitur generalem sententiam, vt quae habeat amicos et ab his alatur. Vides hic vi duam dici non veram quae habet πιςτόν, hoc est, filium, aut nepotem, vt dixit ver.4. vel quae habet πιςτὴν, hoc est, filiam, aut neptem: talis vidua debet ali a suis domesticis, et opponitut  \pend
\section*{IN I. EPIST. AD TIMOTH. }
\marginpar{[ p.131 ]}\pstart equae hic et supra versu tertio dicitur ὄντως χήρα, hoc est, re et nomine vidua: haec alenda est, prior non: ratio est, ne greuetur Ecclesia: gitur iuuenculam de publico alere, vel quae habeat suos per quos alenda sit, illud est grauare Ecclesiam.  \pend
\textit{17 Qui bene praesunt, etc. }\pstart Tertius est locus capitis de ministris recte alendis ex publico. Propositio est, Curandum est Episcopo vt ministris Ecclesiae benigne victus et amictus subministretur. Presbyteri nomen generale est, complectens semores Ecclesiae, qui duplices sunt.  \pend\pstart Alii cum ministris solum disciplinae prae sunt. et mores regunt: alii praeter illud eriam docent, et Sacramenta administrant, et fieri potest, vt hi posteriores re ipsa iuuenes sint, officio senes, qualis erat Timot. inter presbyteros iuuenis, HI DVPLICI HONORE DIGNI SVNT,h.e. sunt benigne alendi:fed meminerint hoc pertinere ad eos solum qui bene praesunt,h.e. qui officium suum serio praestant: id faciant singuli, hine quaerant de honore duplicato.  \pend\pstart Sunt qui duplicem honorem intelligant de stipendio geminato: primum fit stipendium pro se et familia: alterum pro peregrinis, vt possit quia supra dicebat Episcopum vnius vxoris, esse hospitalis: quod tamen mihi non placet, hoc est, vnius Heclesiae debere esse maritum,  \pend
\section*{COMMENTARII }
\marginpar{[ p.132 ]}\pstart h.c.praesidem. Non potest igitur ab vna Ecclesia duplum accipere stipendium. Altera est sententia inter pretantium duplicem honorem: primum de liberali victu, qui honor primus est: deinde de honestis sententiis de pio ministro, qui alter honor est: ac si dicat pium ministrum benigne non solum alendum, sed etiam honesto habendum loco, pie de ipso et honorifice loquendum, quae sententia vera est. Mihi tamem placet tertia: nam more Scripturae duplum dicitur abundams, et pleniore memsura traditum, vt Matt. 23.ver.15. De hypocritis dixit Christus quod suos proselytos faciant filios gehennae duplo ipsis maiores. et Apocal.18. De Babylone ait, Duplicate illi dupla secumdum opera eius. h.e. pleniore mensura puniatur vt promeruit. Ita duplicatum accipio honorem, hoc est, benigne et plenius praestitum, quod fit vno stipemdio hilariter soluendo, prompte et benigne alendo, ac vt par est colendo.  \pend
\textit{18 Dicit enim Scriptura, etc. }\pstart Propositio exposita est, in qua tria sunt: primum qui sint liberaliter alendi, scil. quibene praesunt: deinde quomodo liberaliter duplicato honore: tertio ad quos singulariter hoc spectat: ad eos qui in doctrina suam praestant operam vigilanter. Nunc propositionis additur confirmatio a pronunciatis: estq. argumentum geminum:primum a veteri testamento  \pend
\section*{IN I. EPIST. AD TIMOTH. }
\marginpar{[ p.133 ]}\pstart depromptum, alterum ex nouo. Mosis enim est sententia, Boui trituranti non obligabis os, de qua diximus, Cor.9.ver.9. Vnde hoc sequitur. Si boui benigne exhibendus est victus propter laborem, quanto magis homini docenti vigilanter et vtiliter? Christi vero est sententia, Dignus est operarius mercede sua, Matt.10.ver.10.Luc. 10.ver.7. In quib. sententiis obseruemus oportere esse non pigros boues, aut otiosos homines in foro, siquidem nobis velimus applicare id quod Apostolus hic docet. Deinde Scripturae nomine, iam tunc comprehensas fuisse Domini Cheist; sen tentias de religionis cura, etc.  \pend
\textit{19 Aduersus presbyterum, etc }\pstart Quartus locus, de iudicio contra peccantes, videtur hic locus pertinere ad primum de moderatione adhibenda in corripiendis peccatorib. cuiusuis aetatis et sexus homines: nisi quod hic magis tractat forensem inquisitionem in peccata: illic vero in genere obiurgationisformam quae leuis et prudenter accommodata esse debet personis auditorum.  \pend\pstart Propositio huius loci est, Episcopo danda est opera vt iudicia Ecclesiastica quam sanctissime administrentur.  \pend\pstart Partes a diuisione instituit. Nam peccatum aut latet, et ob id conuincendum est testibus: aut notorium est, nec habent opus peccator testibus, quia constat de peccato, aut etiam fatetur:  \pend
\section*{COMMENTARII }
\marginpar{[ p.134 ]}\pstart de vtroque hoc genere breuiter agit.  \pend\pstart Exorditur autem a peccato quod latet, vt solet in magnis viris, quales sunt praesides Reip, et Ecclesiae, de quorum peccato non facile constat propter authoritatem: isti si deferantur, inquirendum dicit, testibus, significans delatoribus non facile esse credendum. Deinde parcendum bonorum famae et nomini, ne ad quosuis rumusculos palam traducamtur. Primum hic presbyterum accipio officio et aetate, hoc est, ministros Ecclesiae non damnandos nisi diligenti inquisitione habita: nec solum ministros, sed omnes personas aetate graues: quib. enim Dominus dedit bonum nomen, et aetatis grauitatem, hos oportet parentum loco colere. Ideo diligenter est cauemdum, ne quid temere contra tales credatur delatoribus. Deinde, quia tales presbyteri erant iudices et Senatores Ecclesiastici qui alios ob peccata pro delicti ratione corrigebant, credibile est in multorum offensam incidisse, adeo vt non carerent obtrectatoribus:ne igitur temere quid contra ipsos credatur admonitione dignum fuit. Nec ideo sequitur, solum in his adhibendi sunt testes, in aliis sine testibus agendum est : imo quia in his grauitas requiritur in omni iudicio Ecclesiastico eam esse adhibendam docet. Sunt interim multa leuia et puerilia in quibus nec de famae nec nomi\pend
\section*{IN I. EPIST. AD TIMOTH. }
\marginpar{[ p.135. ]}\pstart nis integritate agitur, in quibus non est opus fimili rigore. Statuamus igitur hanc communem regulam. In omni iudicio Ecclefiastico curandum est praesidi vt inquisitio in peccata mortalium pie et sancte fiat. Ratio est, quia si contra seniores fiat inquisitio, aetatis et grauitatis ratio hoc postulat: sin contra iuniores, idem postulat fraternitatis ius et regula naturae, Quod tibi fieri nolis alteri ne feceris.  \pend\pstart Deinde quia Apostolus in tali censura iubet adhibere testes, eosque plures secundum legem Mosis, duos vel tres, in quorum ore consistat veritas.  \pend
\textit{20 Eos qui peccant, etc. }\pstart Altera est species peccantium, eorum scilicet qui notorii sunt, nec inquisitione opus Habent: in his quoque procedendum est pie et grauiter, sed non per testes, verum correptione: Ἐλέγχειν est peccatorem sic describere, vt limitato et agnito peccato certis argumentis ad poenitentiam ducatur, in peccato suo erubescat cognita eius magnitudine et spe ve niae ad Christum impellatur. Itaq. non loquitur de confusione publica, quae nimis est immoderata apud quosdam, qui satyrice magis debacchantur in auditorum peccata, quam vtiliter. Addit rationem Apostolus, VT RELIQVI TIMOREM HABEANT, hoc est, iram Domini  \pend
\section*{COMMENTARII }
\marginpar{[ p.136 ]}\pstart metuant, offensam eiusdem, quae fluit experpetrato peccato. Illustre igitur hinc ducitur argumentum, Timor in Ecclesia est confirmandus: ergo Ecclesiastica iudicia grauiter sunt administranda.  \pend
\textit{21 Obtestor in conspectu Dei, etc. }\pstart Grauissima obtestatione concludit locum de iudiciis Ecclesiasticis, quae argumenta aliquot suppeditat. Primum enim hoc genus dicendi non adhibet Apostolus nisi in rebus grauissimis, maximique momenti: cuius exempla supra aliquot vidimus, et infra cap. 6.ver.13. quoque habebimus: cum igitur de iudiciis Ecclesiasticis agens idem adhibeat, vtique pie et religiose sunt administranda. Deinde quia obtestatur per conspectum Dei, hoc est, prae sentiam eius: ex quo efficitur Dei esse iudicia, in quibus ipse velit esse praesens.  \pend\pstart Sciant igitur iudices Dei se esse vicarios, et inter iudicandum ad illius voluntatem respiciant, pronuncientque tamquam Deo praesente et inspiciente ipsorum corda. Tertio. PER CHRISTVM DEI FILIVM, scilicet iudicem superiorem ad cuius tribunal illis quoque olim sit subeundum iudicium. Quarto. PER ELECTOS ANGELOS, testes scilicet nostrorum iudiciorum, quae de aliis peccatoribus ferimus et consortes vitae aeternae, qui ad Christi Domini nostri tribunal spectateres  \pend
\section*{IN I. EPIST. AD TIMOTH. }
\marginpar{[ p.137 ]}\pstart erunt incorrupti iudicii Christi, quemadmo. dum spectatores fuerunt nostris iudiciis de fratribus peccantibus. Haec argumenta pondus addunt propositioni quam vno verbo refricat, VT HAEC SERVES.  \pend\pstart Ταῦτα primum refero ad iudicia Ecclesiastica, vt sit sensus, In his volo te quam sanctissime versari, nec solum te, sed vt alii idem faciant illis debes esse author et impulsor. Deinde ταῦτα potest quoque referri ad omnia praecepta hactenus tradita: sed prior sentent ia mihi magis arridet propter sequemda,  κωρὶς προκρίματος, nihil faciens κατά πρόκλη- σιν: duo enim haec commata pertinent ad iudiciorum formam. Πρόκριμα dicitur, quod iudex sumit sibi non ex actis et prolatis, sed domo secum affert aliunde ad ipsam causam: tale praeiudicium mirifice impedit veritatis cognitionem, hinc enim mox nascitur πρόκλησις, hoc est, inclinatio in vnam partem, causa etiam nondum cognita quum iudicia ante cognitionem debeant esse ἄκλιτα, hoc est, inuiolata et minime flexa: post cognitionem vero ἔγκλιτα, hoc est, ad veritatis fauorem accommodata: vitia igitur iudiciorum sunt ωρόκλησις, et πρόκριμα quae peccant anticipatione. Quare ταῦτα ad priorem sententiam refero.  \pend
\textit{22 Manus ne cui cito imponas, etc. }
\section*{COMMENTARII }
\marginpar{[ p.138 ]}\pstart Quintus locus sequitur de impositione manuum: habet autem locus iste duo, πρότασιν et αἴτιον. Propositio est, Episcopo danda est opera vt in eligendis ministris quoque prouidus et cautus sit: nam peccatur hic quoque celeritate et praecipitantia iudicii, vt idoneum iudices qui talis minime est. Manuum impositio illustre olim fuit symbolum veris ritibus adhibitum. In nouo Testamento ministerium Ecclesiasticum isto conferebatur, testabaturque is qui manus imponebat se hunc dignum et idoneum iudicare qui alios doceret: sic in Timothei electione et confirmatione fuit, supra 4 et infra 2.epistola cap.1. Aliâs isto symbolojdona spiritualia conferebantur et visibili aliquo signo declarabat se praesentem Spiritus S. vt Act.8.et 19. Alias sanationis erat argumentum verum et minime dubium, Matt.9. in primatis filia, Marci 5.in principis synagogae filia. In veteri Testamento, vnde hic mos fluxit, symbolum hoc adhibebatur in electionibus et confirmationibus, imperatorum, vt Num. 27.in electione Iosuae. In benedictionibus confirmandis et conferendis, vt Gen.48. In deputanda cer ta aliqua hostia, Exod.29. etc. De quibus suo loco a nobis est dictum.  \pend\pstart Ratio propositionis est gemina: prior, Ne ALIENIS COMMVNICES PECCATIS hoc est,  \pend
\section*{IN I. EPIST. AD TIMOTH. }
\marginpar{[ p.139 ]}\pstart ne reus fias corrupti ministerii. Cogitent hoc diligenter quibus electiones sunt concessae, ne alie no peccato sese maculent, dum eligunt, admittunt, et confirmant ebrios, scortatores, rudes et imperitos asinos, qui populum pietatem vel non docent, vel male, aut pessimo exemplo doctrinam bonam conmaculant: quorum vocationem Deus nulla ratione approbat.  \pend\pstart Altera est, TEIPSVM CASTVM RETINE, hoc est, si prouidus fueris vt praecipio, hunc thesaurum retinebis, vt maneas coram Domino iudice ἅγιος, hoc est, innocens, qui maximus est thesaurus, et omni laude et praemie humano superior.  \pend
\textit{23 Ne amplius aquam bibas, etc. }\pstart Sextus locus sequitur qui personale habet praeceptum: spectat enim ad personam Timothei, sed ita vt facile hinc constitui regula generalis possit. Sit igitur haec propositio hu ius loci, Timotheo et ob id omni pio ministro ieiunium, aquae potio, et omnis carnis castigatio sic temperanda est, vt valido corpore quam diutissime Christi Ecclesiae prodesse possint. Ratio est, quia Apostolus hoc fi ne aquae potionem hic interdicit Timotheo, et contra praecipit moderatum vini vsum. Aquae potio minus in nostris ministris habet periculi, quia saluberrima quum sit nostrae regioms aqua, tamen caute illa vtuntur,  \pend
\section*{COMMENTARII }
\marginpar{[ p.140 ]}\pstart vino magis dediti, imprimis delicatioribus. Olim peccabant magis in defectu boni pastores, hoc est, vt minus vinum curarent, magis essent ὑδροπόται: quae laus est Musarum apud Ethnicos, ideo fontes limpidos et fluminum tractus accolebant. Hodie res conuersa est, ideo obseruent nostri οἰνοπόται, concedi illis in persona Timothei paululum vini, vt pitissando magis et gustando illo vtantur quam replendo et ingurgitando: nec minus est hoc peccatum in excessu quam quod in Timot heo timebat Apostolus in defectu, aquae potione reddebat, sed languidum et ob id inutilem. Sic nostri vini potione reddunt se morbidos, podagricos, chiragricos, fraeneticos, et aliis morbis sese corrumpunt: in quo multiplex est peccatum. Peccant in vinum ipsum, nimis exquisito vtentes, quum vulgaria illis non sapiant, nisi bis peregrina. Deinde exemplo pessimo aliis praesunt, luxu et deliciis effoeminati. Tertio, in suamet corpora quae morbis corrumpunt. Demum in officium suum et Ecclesiam, qui redduntur inhabiles et inutiles, interim saginandi otiosi.  \pend\pstart Ratio Apostoli est, contra aquae potionem accommodata, PROPTER STOMACHVM, inquit, TVVM, quia scilicet is frigiditate aquae corruptus etiam concoctionis officium non praestabat: hinc varii morbi et ex mor\pend
\section*{IN I. EPIST. AD TIMOTH. }
\marginpar{[ p.141 ]}\pstart bis languidum et inhabile ad tantum mu nus corpusculum. Nostrae hodie rationes instruendae magis sunt contra vini potionem quae nimium immoderata et impudens est apud ministros Christi.  \pend
\textit{24 Quorundam hominum peccata ante manifesta sunt, etc. }\pstart Epilogus est continens instructionem ac consolationem: videnturque haec commode superioribus cohaerere, si versum 23.includas parenthesi, eruntque haec verba antipophora: poterat enim dicere Timotheus, vix potero cauere hy pocritas quo minus sese insinuent et ingerant vtcunque sim prouidus et seuerus examinator: malum enim occultum est saepe, vt quid sequendum sit seire non possit. Respondet igitur Apostolus concessione. Verum est, impietas in multis occulta latet, attamen semper latere non potest: aut enim manifesta sunt peccata quae κρίσιν et iam facilem efficiunt.  \pend\pstart Haec peccata praecedunt iudicium et informant nos quid sequendum sit et inter examinandum notoria saepe fit ruditas, inhabilitas, stupor et inscitia Scripturae. Id totum deducit te ad certam crisin. In aliis latent, attamen post crisin produntur, vt superbia et arrogantia, quae se prodit, si hypocrita fuerit admissus, nam vestis virum faciet, et magistratus  \pend
\section*{COMMENTARII }
\marginpar{[ p.142 ]}\pstart illum arguet, hoc est, mores mutabit iam euectus ad aliquam dignitatem. Illa licet sequantur crisin factam, tamen correctionem admittant, vt talis iterum deponatur commode, matureque adhibeatur morbo remedium. Sic dum censura peccantium agitur, quidam adeo callide sua peccata possunt tegere, vt nullis argumentis aut testibus satis conuinci poss int. Hi impediunt actibus iudicia sana, vt statui certo non possit: attamem deinceps sequetur, inquit, vt notoria fiant. Dominus enim non fert ista monstra vt semper lateant sub ouilla veste. Interdum per ipsam crisin declarantur, vt in Actis, Ananias et Saphira produntur ipsa crisi, quam SpiritusS. per Petrum fecit. Itaque consolatio non habet locum, constanter exploranda omnia in censuris, in examinibus seu peccantium seu eligendorum ministrorum fore vt iudicia nostra non semper fallantur, si non in actione produntur, certe paulo post manifesti fient.  \pend
\textit{25 Similiter et bona opera, etc. }\pstart Contrarium membrum de bonorum innocentia, quae itidem vel notoria fit, per ipsam in quisitionem vt certa sit crisis: vel latens in actione post crisin suo tempore patefiet. Consolatio et in hisest, Veritatem semper latere non posse quin patefiat: temporis enim est filia.  \pend
\section*{IN I. EPIST. AD TIMOTH. }
\marginpar{[ p.143 ]}
\textit{CAPVT VI. }
\textbf{V }\pstart ARIOS persequitur locos hoc capite quos ordine recensebimus. quo facilior sit illorum deinceps explicatio. Primum, disserit de officio seruorum in suos ordinarios heros. Deinde, describit schismaticos, vt eo commodiùs vitari possint. Tertio, agit de vera animi sufficientia qua sua sorte homines contenti sunt, nihil insuper appetentes. Quarto, contra auaritiam disputat propositis incommodis, quae ipsos auaros solent comitari. Quinto, sermonem ad Timotheum conuertit, et hortatur iterum ad officium in Ecclesia quam vigilantissime exequendum. Sexto, diuitibus suum praescribit decorum quond ab ipsis requirit pietatis professio. Septimo, de mum breui epilogo totam epistolam concludit. De his ordine audiemus ipsum ApostoIum, fingulis locis suas quoque constituentes speciales propositiones.  \pend
\textit{1 Quicunque sunt sub iugo serui, ett }\pstart Locus est primus de officio seruorum erga suos heros. Hunc locum etiam habuimus ad Eph.6.v.5.ad Col.3.V.22.et habebimus infra ad Tit.2.V.9.1.ep. Pet.2.V.18, quibus consonant illa quae ad Rom.13. habentur. Hoc loco Apo\pend
\section*{COMMENTARII }
\marginpar{[ p.144 ]}\pstart stolus hunc ordinem seruat. Initio dat praece ptum quod seruis est sequendum: mox addit praecepti rationem: tertio respondet impatientiae seruorum, de fidelibus dominis excipientium : quarto, hoc ipsum firma ratione probat: quinto demum concludit locum. Sunt illa, πρότασις, αἴτιον, ἀνθυποφορὰ, κατασκευὴ, et συμπέρασμα.  \pend\pstart Propositio est in illis verbis, QVICVNQVE SVB IVGO SVNT SERVI, PROPRIOS HEROS OMNI HONORE DIGNOS IVDICENT. Praeceptum est, cuius sententia sit, serui dominis suis officium suum omni reuerentia quam promptissime debent praestare.  \pend\pstart Qui serui sint describit, οἱ ὑπὸ ζυγὸν, eos humilitatis admonens a qualitate conditionis: sub iugo sunt, igitur patientia opus est. Item iugum hoc excuti propria authoritate non debet, quemadmodum iumenta subiugata non laudantur si excutiant illud aut recalcitrent. Hoc agnoscant serui, praesertim pietatem profitentes. Deinde addita est clausuIa vniuersalis ὅσι, admonens etiam eos qui sunt quacunque ratione superiori potestati subiecti, debere obedientiam praestare his quibus subiecti sunt, quales sunt subditi erga magistratum, famuli, ancillae, clientes, discipuli et similes personae. Obedientiam debent his quibus Dominus eos subiecit, quos  \pend
\section*{IN I. EPIST. AD TIMOTH. }
\marginpar{[ p.145 ]}\pstart notanter hic exprimit Apostolus, ἱδίους δεσπό- τας vocando. Dicuntur autem ἴδιοι iure diuino et humano: Deus enim in populo suo seruitutem approbauit, imo concessit etiam piis, vt in casu necessitatis libertatem propriam vendere possint et assumere heros sibi quorum imperio regerentur: quanquam sem per velle seruum esse in suo populo Deus ignominiosum iudicarit. Deinde iure Gentium iustum est vt hi qui vel bello seruati essent, vel empti pecunia, vel nati in seruitute, vel suo aliquo vitio libertatem amisissent, fideliter seruiant suis dominis. Itaque seruus perfidus et impatiens peccat in rem propriam. Omni honore dignos iudicare, est simplici animo pie sentire de hero suo, prompte res ipsius et commoda procurare, et constanter illi bene velle: laudantur in seruo, candor, animi promptitudo, et voluntatis constantia, etc. Vide locum de seruitute.  \pend\pstart Ratio Propositionis, NE NOMEN DEI, ET DOCTRINA BLASPHEMET VR:grauissima est illa ratio apud pias mentes. Omnes enim ope ram dare debemus, vt nomen Domini per nos fanctificetur, quemadmodum petimus in oratione Dominica. Discimus deinde hinc per inobe dientiam subditorum praesertim sernilem impatientiam blasphemari nomem Dei:fic ad to tam vitam argumentum facit, quae si flagitiosa sit  \pend
\section*{COMMENTARII }
\marginpar{[ p.146 ]}\pstart et professioni minime respondeat, nomem Dei blasphematur: sic argumentatur Apostolus ad Rom.2.V.24. Iudaeorum moribus effectum vt nomem Domini male audiat apud Gentes, quemadmodum et D. Petrus scribit 2. epistola 2.V.2.de falsis doctoribus. Hac ratione in pri mitiua Ecclesia magnam et indignam inurebant veritati notam serui, qui praetextui libertatis Christianae sese iugo seruitutis cupiebant exi mere. Hinc. n. Ethnici clamabant, Ecce haec il lorum est doctrina, reddit nobis seruos seditiosos, etmancipia ad arma incitat:quae calumnia, doctrina fuit diluenda: deinde seruorum constantia in seruitute. Ideo grauissimo argu mento eos vrget, ne Dei nomen ipsorum re bellione calumniam patiatur. Altera ratio est, candem calumniam ad doctrinam Euamgelii spectare, quo genere vehementer peccarunt nostro seculo reflorescente vera Euangelii doctrina Catabaptistae, qui miseram rusticam turbam ad arma incitarunt libertatis Christianae praetextu. Nam cursum Euang elio praecluserunt apud multos qui clamabant, Ecce hoc illorum est Euangelium, seditionibus omnia complere. Contra hic docet Apostolus, hoc ipso nomine imprimis cauem das seditiones, et seruilem rebellionem, ne doctrina Christi calum niam patiatur apud infideles.  \pend
\textit{2 Qui vero habent fideles dominos, etc. }
\section*{IN I. EPIST. AD TIMOTH. }
\marginpar{[ p.147 ]}\pstart Occurrit Apostolus seruili impatientiae, ac eripit ei firmamentum defensionis suae. Primum igitur hic obseruetur superiora pro prie loqui de obedientia, quam debemus etiam infidelibus dominis: alioquin praeoccupatio haec non commode haberet locum. Valebat igitur argumentum a maiori, si enim etiam impiis est par endum, ergo multo magis fidelibus dominis. Verum Apostolus respexit hic ad defensionis firmamentum: impatientes enim seruitutis dicebant: Ecce Christianus sum factus, et herus meus quoque: cur ioitur me non manumittit? sum illius frater, et in Christo, neque liber neque seruus debet esse, sed noua creatura: id non perpendit herus meus, volo igitur ipse sumere quod ingratus herus Christo non vult dare: sic serui videntur fraternitatis Christianae nomem arri puisse ad praetextum rebellionis. Quo docemur apud hy pocritas, honoris et sortis Christianae ae qualitatem contemptum parere: et quia in fide est honoris quaedam aequatio omnium fidelium, in Baptismo vero sortis, ideo mox prorumpunt in has cogitationes, quas hic damnat in seruis Apostolus. Propositio est, fideles heri non contemnendi sunt seruis suis, h. e. fideliter etiam his est seruiem dum non minus quam plane impiis, qui conuersi  \pend\pstart nondum sunt. Firmamentum ipsorum seruorum eratifratres sunt, ergo fraterne nos debebant  \pend
\section*{COMMENTARII }
\marginpar{[ p.148 ]}\pstart dimittere, id non faciunt, causam igitur praebent nobis ipsos contemnendi. Respondet Apostolus, non recte hoc effici: imo ob hoc ipsum magis diligendos. Fidelis tibi contigit dominus: ergo magis debes illi seruire propter communem professionem. Approbatur hoc loco manifeste seruitus. Deinde noncon cedendum seruis excutere iugum: et impium esse praetextum qui ex Euangelio petitur ad defensionem seditionis. Deinde sunt ἀγαπητοὶ, id est, dilectione: ergo non vr gendum illis officium seruitutis: vel αγαπη- τoὶ sunt Deo et Christo: ideo abs te non sunt contemnendi. Sed malo ad seruos referre, sic enim magis illos stringit argumentum in sequentibus, qui beneficium recipiant, hoc est, dominos fideles ob hoc ipsum dignos esse qui a seruis beneficio afficiantur: nam fideles humanius sua mancipia tractant, ob quam humanitatem digni sunt vt serui fideliter administrent. Alii ad Deum ista referunt. Sequatur quisque quod sibi videbitur confor me esse pietati.  \pend
\textit{Haec doce et hortare. }\pstart Conclusio est loci prioris. Docere est ex seripturis ostendere credenda et facienda. Nam causas reddimus doctrinae ex scriptura. Hortari aunt proposito conmodo aut inconmodo ad eadem auditores incitare. Vtrunque faciendum est  \pend
\section*{IN I. EPIST. AD TIMOTH. }
\marginpar{[ p.149 ]}\pstart ministro: docendi sunt serui, ne propter fidem conditionem suam deserant, idque ex semptura: deinde incitandi sunt ad obedientiam proposito praemio, honestate, vtilitate rei, etc  \pend
\textit{3 Siquis aliud docet, etc. }\pstart Secundus locus de cognoscendis et cauendis schismaticis. Locus iam partim tractatus est sup.c.1. et ad Rom.16. Phil.3. et 2. ad Tim.3. vers.6.ad Tit.3. In presentia colligantur notae quibus schismatici cognosci possunt.  \pend\pstart Prima est ἔτεροδιδασκαλεῖν: id quid sit diximus sup.cap.1.ver.3. ἕτερον dicitur quidquid Sacris literis conforme non est, hoc est, nec τὸ ῥητόν, nec διάνοιαν inde habet. Ideo Apostolus hoc vitium opponit hic sanae doctrinae Domini nostri Iesu Christi, hoc est, verae interpretationi Propheticae scripturae, a qua quidquid ἕτερον est, hoc est, alienum reperitur, id totum schismaticum est, et ob id cauendum. Porro schismaticos libenter declarari suis opinionibus certum est : volunt nouae sententiae videri repertores, ideo figmentis delectantur, et siquid ex officina rationis subtile possint depromere, vt videre est in Eutyche, Arrio, Nestorio, aliisque, et hodie qui veteres haereses renouant hanc ipsam notam manifeste in se produnt.  \pend
\textit{4 Inflatus est nihil sciens, etc. }
\section*{COMMENTARII }
\marginpar{[ p.150 ]}\pstart Altera est nota, iidem etiam superbi et arrogantes sunt, τύφωσις et τύφος dicitur  μετα- βατικῶς, inflatio, arrogantia, fastus, sumpta metaphora a typhonico vento, quem veteres crediderunt hominibus mentem ac rationem adimere: hinc δεινὸς καὶ ὑβριςτὴς ἄνεμος dictus fuit, quia hoc vento correpti insanire ac delirare credebantur. Et est reuera arrogantia, delirium intolerabile piis: mag nifice enim de se sentiunt, et plus quam oportet sibi tribuunt, aliis facile detrahunt, praesertim in his qui de ipsis non magnihct videntur sentire. Hic morbus hodie nimium frequens est, non solum inter schismaticos, sed eos etiam qui longe abesse videri volunt a schismate.  \pend\pstart Tertia nota est, NIHIL SCIENS: ignorantia igitur schismaticos etiam comitatur, non est illa docta et pia ignorantia, quam Augustinus etiam fateri non erubuit: sed veritatis inscitia turpis, quae sibi plurimum sapere videtur: ideo effrons et impudens est, temere enim iudicat et pronunciat: ideo dici solet, Inscitia nihil esse impudentius. Ἐπιςτήμη rerum est verarum, quae cum Dei iudicio congruunt: sic in fide quae analoga sunt religioni Christianae, et sanae doctrinae respondent, sunt ἑπιστημονικά. Schismatici igitur quum ab analogia fidei  \pend
\section*{IN I. EPIST. AD TIMOTH. }
\marginpar{[ p.151 ]}\pstart ertent, recte dicuntur nihil scire, vtvt docti videantur, quia ad finem veritatis non perueniunt.  \pend\pstart Quarta est, νοσῶν περί ζητήσεις καὶ λογομα- χίας, id est, morbidus est circa quaestiones et verborum pugnas: supra quoque capite primo damnauit ζητητικόν καὶ μυθικὸν: infra ad Titum capite primo, versu Io.ματαιολογία dicitur, quia his quaestionibus frustra omnis cura impenditur. Hoc loco grauissime deterret ab eo vitio subiecta turpitudine. Νόσος  enim est quod lae dit naturalem vitae actionem : in religione autem νόσος est studium litigandi, et curiositas quaestionum nouarum, quod ipsam quoque laedit actionem vitae renati hominis: ideo quam diligentissime haec vitia cauenda sunt. Habes euidentem notam schismatici hominis, morbidum esse quodammodo, et ad insaniam vsque se quaestionibus nouis macerare, Considera hic quaestionum diuersitatem: in Theologia alia est vtilis quaestio et pia ζήτηπς εὐσεβηε, quae in Sacris literis fundamentum habet, et ad cognitionem veritatis, aut ad mores renati hominis confert: vt quaestio de fide, de bonis operibus, de via ad salutem, etc. Alia est ζήτησις νοσώδης, morbida, quae Christi meritum obscurat, bonis obest moribus, aut ad cognitionem veritatis  \pend
\section*{COMMENTARII }
\marginpar{[ p.152 ]}\pstart nihil confert, qualis est quaestio de purgatorio, de inuocatione sanctorum, de operibus condignis, et congruis, de intercessione sanctorum pro nobis apud Deum, etc.  \pend\pstart Λογομαχία disputationum est vitium, quam do de rebus ipsis constat, lis saltem de vocibus est ac phrasi: oportet autem in vsu vocum faciles esse quando de rebus constat, quod facile agnoscunt moderata ingenia quibus dissidia displicent: sed schismatici, vt in aliis, ita hic quoque difficiles sunt. Ideo Apostolus morbum esse dicit, hoc est, animi vitium: morbus etiam est, quia causa mala suscipitur. Vt enim simplex est veritatis orantio, sic falsi callida sit defensio oportet: sic Eurip.eleganter dixit, ἄ- δικος λόγος, νοσῶν ἐν ἑαυτῷ φαρμάκων δεῖται σοφῶν  \pend\pstart Quinta nota est, Ex QVIBVS EFFICITVR INVIDIA,CONTENTIO, BLASPHEMIAE, SVS HICIONES MALAE, IMPROBAE EXERCI TATIONES. Ab effectis describit schismaticam doctrinam, quorum turpitudine deterret nos a schisinatis studio. Effecta turpia et perniciosa non possunt non ab impia deduci causa, huc referuntur effecta perniciosa schismatis ergo ipsum schisma perniciosum esse oportet. φθόνος est rei bonae, hoc est, mali dum bonos virtutibus pellere vident, inuident, non quod tales esle vellent, sed quod ipsis similes in malitia omnes non sint.  \pend
\section*{IN I. EPIST. AD TIMOTH. }
\marginpar{[ p.153 ]}\pstart Ἔριν, contentio ex rationum collisione nascitur, et fere praecedit inuidiam: nam virtus demum cognita patitur φθόνος.  \pend\pstart Blasphemia accipiatur vel in viros bonos: sic Athanasius infinitas calumnias, et maledicta ab Arrianis, vel in ipsum Deum; fiunt enim schismatici etiam in Deum tandem maledici, et dubitant de eius clementia, praesciemtia, cura pro piis, de Christi diuinitate, merito ipsius, et similia infinita rapiunt tandem in dubium.  \pend\pstart Suspiciones malae sunt, quibus bonos viros immerenter obruere conantur, quod vitium hodie multos latet: nascitur illa suspicio ex praeiudicio quodam prauo quod malae mentes concipiunt de aliis, hinc mox in vnam partem inclinans nascitur suspicio, quam effundunt et spargunt in vulgus non cognita veritate. In quo cum schismaticis multi hodie communicant.  \pend\pstart Παραδιατριβαί, et hic fructus est schismaticorum: non quiescunt sua spargere, ideo exercitia adhibent. Διατριβὴ est idem quod γuμ- νασία, estque vox media: bonorum enim, et malorum est. Malorum hoc loco exercitiorum habes exemplum:nam παρὰ praepositio significationem primam limitat, malamque facit. Exempla peruersarum exercitationum, in scholastica Theologia obuia sunt, vt apud mona\pend
\section*{COMMENTARII }
\marginpar{[ p.154 ]}\pstart nachos, peregrinationes, humicubationes, vigiliae, inediae, cantus, aliaq. satis testari possunt.  \pend
\textit{Hominum mente corruptorum, etc. }\pstart Ostendit ad quos spectent notae positae: nimirum testimonia esse reprobi hominis, quantumuis simulet pietatem. Corruptos mente intelligo schismaticos, qui diuersam ab Apostolica tradunt doctrinam. Corrupti autem sunt primum animi malitia, deinde studio corrumpendi etiam alios: vide infra ad Tit. 1.dicuntur φρεναπάται, deceptores mentium.  \pend\pstart Est autem οθορὰ in Scriptura sacra quidquid cursum ad salutem impedit, et morum integritatem vitiat. Sit igitur haec sexta ꝓprietas schismaticorum, quod corruptas habeant mentes.  \pend\pstart Septima est, PRIVATI VERITATE, veritas Christus est, ipso dicente Ioan.4. Item cap. 17. Sermo tuus veritas est. Huius doctor et explicator est Spiritus S. Itaque et Christo et Spiritu S. et vera Scripturae notitia carent schismatici, dum priuati sunt veritate.  \pend\pstart Octaua est, PIETATEM PVTANT QVAESTVM ESSE. Exemplum habes in Simone Mago, qui donum Spiritus Sancti pecunia emere voluit vt illi lucrosa esset facultas edendorum miraculorum: hi dicuntur ad Rom.16.ver. 18.non Christo, sed ventri suo inseruire, et ad Tit.1.vers.11.docentes quae non oportet turpis lucri gratia.  \pend
\section*{IN I. EPIST. AD TIMOTH. }
\marginpar{[ p.155 ]}\pstart HOS DECLINA. Conclusio est loci de vitandis schismaticis: sensus est, nihil cum his conmune tibi sit, nec in doctrina, nec moribus.  \pend
\textit{6 Est autem quaestus magnus, etc. }\pstart Tertius locus, vbi quaerenda sit vera αὐ- τάρκεια, nimirum in Christo, qui veras opes animi largitur. Cohaerent autem superioribus ratione praeoccupationis: negauit prius pietatem conuertendam in quaestum. Nunc suo quodam sensu permittit quaestum in illa, siscilicet adsit animi moderatio boni consulentis qualemcunque prae sentem statum. Vide locum περί τῆς αὐταρκείας.  \pend
\textit{7 Nihil enim intulimus, etc, }\pstart Moderationem suadet ab ipsa natura quae nudos nos genuit, et nudos etiam recipit: simile habes apud Iobum cap.1.omnia in morte nos destituunt, opes, aurum et argentum, amici, gloria, honor: ad quid igitur immoderatum harum rerum desiderium?  \pend
\textit{8 Habentes autem alimenta, etc. }\pstart Alterum argumentum contra quaestus desiderium, quia necessariis tantum fruimur, reliqua vt superflua relinquimus: necessaria autem, sunt secundum naturam, victus et amictus. His igitur contentos esse oportet.  \pend
\textit{9 Qui autem volunt ditescere, etc. }\pstart Locus quartus contra auaritiam, quam in  \pend
\section*{COMMENTARII }
\marginpar{[ p.156 ]}\pstart schismaticis taxauit : rationibus autem firmis remouit a pio Christi seruo. Iam expresse morbum ipsum describit a suis fructibus. Πλουτεῖν opponit autarciae, et animo qui victu et amictu qualicunque est contentus. Qui aliud etiam ad ornatum vult habere, et ad voluptatem, hunc quae maneant pericula, ostendit.  \pend\pstart Primum, INCIDVNT IN TENTATIONEM, scilicet Satanae, qui varia consilia ditescendi clam suggerit, omnibi sque modis animum inflammatum incitat. Deinde has tentationes sequitur laqueus, hoc est, illae angustiae ex quibus non possunt emergere, quemadmodum auibus ponitur esca, quasi tentatio ad laqueum, ad quem si ingrediantur, egredi non datur. Tertio varns cumpiditatibus agitamtur, insomnes saepe ducunt noctes, contabescunt curis, amore, odio passim laborant, quibus submerguntur in corporis perniciem, quae est ὄλεθρος, et animarum interitum, quae est ἀπώλεια.  \pend
\textit{10 Radix enim omnium malorum, etc. }\pstart Effectorum est confirmatio ab ipsa natura petita auaritiae, cui mox annectit noua effecta: confirmatio habet definitionem: dicit enim quid sit auaritia, scilicet. Est radix omnium malorum. Radix dicitur μεταβατικῶς: est enim radix proprie plantarum, sed transfertur vox illa ad ea quae principii vicem et  \pend
\section*{IN I. EPIST. AD TIMOTH. }
\marginpar{[ p.157 ]}\pstart originem signifieant: vt apud Pindarum de Therone, quia Thaesandro genus duxit, scribitur, ὅτεν σπέρματος ἔχοντι ῥίζαν, vnde generis habet radicem. Auaritia quoque aliorum vitiorum respectu, origo et principium est:quemadmodum enim radix oris vicem habet, vnde in reliquas plantarum partes pet stipitem succus nat uralis transfertur, et inde in minores partes seu foliorum seu fructuum distribuitur: sic auaritia materiam et alimentum suppeditat malis omnibus. Nam auarus etiam facile fit periurus, facile iniustus, impostor, ma Iedicus, homicida, hypocrita, religionis verae abnegator, et nihil non faciet lucri causa. Hinc ad Coloss.3. dicta est auaritia idololatria, et ipse auarus idololatra ad Eph. 4. quia opibus tanquam Deo seruit. Et hic Apostolus inquit, QVOD AVARITIAE STVDIOSI QVIDAM ABERRARINT A FIDE, quia nemo potest duobus dominis seruire, Deo et Mammonae, qui et ipse Deus haberi vult. Referatur igitur et hoc ad effecta, auaritia euertit fidem, et est causa vera infidelitatis: ergo studiose est cauenda omni quidem pio, imprimis vero ministro Christi. Porro caute loquitur Apostolus, τινὲς, inquit, non omnes igitur, at certe maior pars, quia Christus dixit difficile esse talem ingredi in regnum caelorum, quod mox respectu Dei mitigat, apud quem omne  \pend
\section*{COMMENTARII }
\marginpar{[ p.158 ]}\pstart verbum sit possibile: potest igitur auaros seruare, vt Zachaeum forte et Matthaeum: Publicanos enim auaros fuisse probabile est. Verum exempla talia admodum rara sunt, et mali metus maior quam vt vitium hoc veniam mereatur.  \pend\pstart Alterum effectum est, quod non solum fidem euertat, sed doloribus multis conficiat homines: mala enim conscientia eos vrget, insomnes ducunt noctes, cibum et potum nunquam quiete sumunt, sursum ac deorsum cursitant, animum curis, corpus laboribus conficiunt. Quod emphatice Apostolus expressit περιέτειραν, quot enim sunt cogitationes et conatus, tot sunt tela et aculei in animo auari, quibus vulneratur et transfoditur.  \pend
\textit{11 Tu vero homo Dei, etc. }\pstart Quintus locus huius capitis, in quo grauiter hortatur Timotheum ad officium in Ecclesia quam vigilantissime praestandum. Conuertit orationem iterum ad personam Timothei, vt fit in transitionibus, ac complectitur duas diuersas propositiones: vnam fugiendorum, alteram sequendorum, quas sic complectamur: Timotheo et omni pio ministro est curandum vt quiquid vel in dogmatibus fidei sanam laedit doctrinam, vel in moribus vitae innocentiam euertit, omnibus modis vitetur. Contra virtutes quae vel in docente  \pend
\section*{IN I. EPIST. AD TIMOTH. }
\marginpar{[ p.159 ]}\pstart requiruntur, vel in genere innocentiam vitae concernunt, quam studiosissime exprimat. Priorem partem vno commate expressit Apostolus, haec fuge, ταῦτα φεῦγε. Ταῦτα accipio de auaritia et iplius fructibus, quae immediate praecesserunt: deinde ad reliquavita transfero quae in hac epistola hactenus prohibuit, qualis est negligentia in docendo, temeritas in eligendo, importunitas in corripiendo, oscitantia in orando. Ταῦτα, HAEC OMNIA FVGE, quia res periculosae sunt in ministerio. plusque nocent quam prosint et docenti et auditoribus: sed Apostolus nouo hic vtitur argumento a persona ipsius Timothei petito. HOMO DEI ES: ergo ista tibi sunt fugienda. Dicitur homo Dei renatus et qui Spiritu Dei regitur, ac viuit ad voluntatem Dei.  \pend\pstart Videant igitur ministri vt sint homines Dei ex vita et propriis moribus, ne sint homines Satanae  \pend\pstart Alteram partem, seu ἀντιπρότασιν copiose persequitur, quia sub finem epistolae omnes animi affectus cupit effundere, ac variis modis Timotheum ad officium animosum reddere: ex ordine videamus.  \pend\pstart Primum proponit illi virtutes persequendas, ex quibus facile colligi potest quid in contrario membro ταῦτα φεῦγε, comprehenderit.  \pend
\section*{COMMENTARII }
\marginpar{[ p.160 ]}\pstart Enumerat hic sex ordine virtutes. 1. IVSTI TIAM, quae vinculum est humanae societatis, ac imprimis auertitur ab auaris, qui iusti et aequi nullam rationem habent. 2. PIETATEM, εὐσέβειαν, quae est verus Dei cultus, quo Deus vt vult adoratur et colitur, estque iustitia quaedam in Deum. Ideo ab auaris quoque violatur, et ab idololatria, superstitione, impietate, etc. 3. Tertia virtus est, FIDES, hoc est, totius professionis Christianae cura et assiduum studium. 4. Quarta, DILECTIO PROXIMI : fidei fructus est, quo proximum amplectimur Dei et veritatis amore, nihil nostri quaerentes, ad externa concernit. 5.Quinta est, TOLERANTIA, quae est for titudinis species, aduersa quaeuis Christi nomine aequo tolerans animo, etiam cum vitae periculo. Inde Graecis dicta est ὑπομονή, quod perduret reb. aduersis, h.e. minime frangatur. 6. Sexta virtus est, MANSVETVDO, quae est lenitas in ferendis et corrigendis fratrum infirmitatibus. Hae virtutes persequendae sunt. Ex quo discimus eas ad officium renati hominis pertinere, imprimis vero ad ministros: vnde falsum est dogma vulgi, nihil referre quomodo viuamus, modo recte credamus.  \pend\pstart Ratio est, quia vt credimus, sic viuimus, et vt viuitur, sic creditur. Deinde admonemur  \pend
\section*{IN I. EPIST. AD TIMOTH. }
\marginpar{[ p.161 ]}\pstart has virtutes nunquam plane a nobis in hac vita possideri: sed ex parte, semperque fugaces in nobis esse, vt perpetua venatione et insecutione sit opus.  \pend
\textit{12 Decerta bonum certamen, etc. }\pstart Secundo loco proponit illi agonistica. quae ad officium Ecclesiasticum proprie etiam spectant. Agonem vocat fidei, officium ipsius, quo fides imprimis ornatur et promouetur, et recte quidem:vita enim hominis pii per se dicitur et militia, magis igitur ministri vita talis videri debet, qui no simpliciter Christo militat, sed inter antesignanos positus est. Hoc igitur vult Apostolus, vt Timotheus suae vocationi diligenter inseruiat. Vocat autem καλόν ἀγῶνα, vt rei honestate et simul difficultate ad vigilantiam excitet: sunt enim καλὰ honesta et simul difficilia.  \pend\pstart Tertio loco proponit ei apprehendenda, ad quae tanquam vltimum finem respiciat, quo pertinet vita aeterna: nam ista laborum omnium praemium et finis tandem erit. Dicitur vita aeterna status iste, in quo ad imaginem Dei restituemur, quae hic incipit per regenerationem, perficitur post mortem. Haec restitutio et reparatio naturae secundum primam conditionem nobis ardenter est affectanda. Ratio est, quia ad hanc vocati sumus, ideo toties dicitur vt digne Deo et vocatione no\pend
\section*{COMMENTARII }
\marginpar{[ p.162 ]}\pstart stra ambulemus. Indignus igitur vita aeterna censendus est, quia innocentiae non studiosus, qui in officio negligens est.  \pend\pstart Altera ratio est propria Timothei et confessorum, QVIA CONFESSVS ES CORAM MVLTIS TESTIBVS. Vis est argumenti, Quae semel cum laude sumus confessi, in his constanter debemus persistere:sed tu hoc iam fecisti, et quidem coram multis testibus. Ergo, etc. Vbi et quando hoc fecerit Timotheus, non satis liquet. Primum hoc fecit in conuersione factus ex professo Christi sectator, quam professiouem Baptismo admilso obsignauit: deinde in electione ad ministerium, vbi non solum Christianum sese est professus, sed inter hos fore doctorem et ministrum frugi. Tertio, idem confessus est forte coram tyranno aliquo vel in vinculis cum vitae periculo, quo inclino, quia καλὴν vocat confessionem, hoc est, speciosam et cum periculo coniunctam. Deinde quia addit factam esse coram multis testibus, h.e. intrepide, contempto omni vitae periculo: quod qui praestitissent, postea singulari titulo dicti sunt confessores, martyribus proximi. Praeterea hoc sensu postea fecit mentionem confessionie Christi, quae vt exemplum profertur constamtiae in aduersis pro veritatis testimonio. Itaque pathetice vrget Timotheum prius cum peri\pend
\section*{IN I. EPIST. AD TIMOTH. }
\marginpar{[ p.163 ]}\pstart culo vitae praeclare te gessisti, perge igitur magno et forti animo in coeptocursu.  \pend
\textit{13 Denuntio tibi coram Deo, etc. }\pstart Quarto loco seruanda illi proponit ea quae hactenus tradidit, magna et sanctissima contestatione: ac vt res maiori pondere et affectu a Timotheo audiretur, contestat ionem geminam praemittit. Ducitur argumentum a Dei prae sentia et potestate.  \pend\pstart Coram Deo enim geri dicitur quod singulari illius timore agitur, cumque inspectore et animaduersore Deo. Omnia quidem illi praesentia: sed illa imprimis coram ei dicuntur, quae curat et promouet, aut quorum neglectum singulariter punit. Sic a potentia quoque, VIVIFICAT OMNIA:ergo oportet magno illius timore in officio versari, habet enim vitae et necis potestatem Et a Christi exemplo, QVI SVB PONTIO PILATO CONFESSIONEM ILLAM SPECIOSAM STATVIT. Confessionem hanc explico ex cap.18.Ioan. Deinde refero ad totam historiam Passionis ac mortis. Vnde efficitur, ministros exemplo Christi debere in confessione veritatis contra omnia vitae pericula quam constantissime perdurare, et ob id in officio etiam extra periculum esse fortes.  \pend
\textit{14 Vt serues hoc mandatum, etc, }\pstart Redditio est, continens instituti propositionem, significat se obtestationes illas attuliffe,  \pend
\section*{COMMENTARII }
\marginpar{[ p.164 ]}\pstart vt in officio sit quam diligentissimus. Ἐντολὴν accipio pro officii praecepto: scopus enim epistolae huius est, vt quam vigilantissime exequatur officium suum in Ecclesia Ephesina. Itaque ad hoc praeceptum pertinet quidquid hactenus dixit de sana doctrina retinenda et et propaganda: de sectariis vitandis, de electione ministrorum grauiter obeunda, de mo deratione et lenitate corrigendi peccatores, de precibus assidue habendis, de vitando omni vitae scandalo. Quo facit quod addit, τηρῆ- σαί σε ἄπηλον καὶ ἀνεπίληπτον, nam inculpatus esse et labe carere aliter non potest omnis minister nisi illa seruet Apostoli praecepta, cùm ἐντολὴ tamen maneat ἄπηλος et ἀνεπίληπτος. Quod addit ad finem de apparitione Do mini, habet argumentum a spe praemiorum et metu poenarum: nam iudex ille erit et reddet singulis pro factis suis condigna. Absoluit argumentum ab apparitione Domini sumptum. Ac cauet tergiuersatione impiorum, qui solent opinione longae morae insolentius peccare: hos reprimit dicendo, Ve niet tandem quod praedico. Sunt autem ἵδιοι καιροί, ea tempora, quae iudicio nouissimo de putata sunt. Ideo nihil agunt peccatores sese consolando diutina mora, quia quod Christu  \pend
\textit{15 Quem temporibus suis, etc. }\pstart iudex nondum venit, non probat illum nun\pend
\section*{IN I. EPIST. AD TIMOTH. }
\marginpar{[ p.165 ]}\pstart quain venturuin, sed tempora nondum esse completa, quae vbi cursum absoluerint, indubitanter aderit iudex. Iudicium hic commemditur a maiestate authoris. Explicantur igitur hic encomia diuinae maiestatis. Primum μα- κάριος dicitur natura sua, quia hanc in sua habet potestate: deinde effectiue, quia nos infelices beatos reddit: denique quia a nobis debitis laudib.ornatur propter beneficium redemptionis. Secundo loco dicitur SOLVS POΤEης, δυνάςτης dicitur respectu impiorum et Satanae, quos refrenat et sua potentia superat. Δύνω enim est ὁπλίζω: dicitur igitur δυνάςτης, quia vitae et necis plenum habet ius, et quidem solus: nam quidquid vitae et imperii est, id totum a Deo descendit. Tertio, REX REGVM. Regis nomen Deo recte tribuitur propter patrocinium quod suis praebet: ad hunc enim refugium habent, et in eo quaerunt et reperiunt fundamentum: quo facit etiam nominis ratio: Βασιλεὺς enim dicitur quasi τοῦ λαοῦ βάσις, id certe agnoscunt pii.  \pend\pstart Quarto, DOMINVS DOMINANTIVM: Κύ- ριος oecconomicum est vocabulum, nam vxor et liberi proprie patremfamiliâs κύριον vocant: serui vero δεσπότην. Agnoscunt igitur sancti Deum κύριον, hoc est, rectorem Ecclesiae suae, qui alias dominationes omnes superat.  \pend\pstart Quinto, QUI SOLVS HABET IM\pend
\section*{COMMENTARII }
\marginpar{[ p.166 ]}\pstart MORTALITATEM, et solus potens est, non excludens alias potentias, sic solus est immortalis, hoc est, natura sua, eamque sic possidet vt eam largiatur quibus velit: sic Angelicas naturas immortales fecit: hominis animam: tandem etiam corpora post resurrectionem eadem ornabit.  \pend\pstart Sexto, LVCEM INHABITANS INACCES SAM: lucem dicit illam ineffabilem vitae aeternae beatitudinem, quam opponit tenebris, hoc est, terrori et cruciatibus impiorum: Deus igitur lux est, hoc est, ipsamet maiestas felicitatis, hanc possidet vt proprium bonum. Ἀπρόσιτον autem dicitur, quia ratio hominis et vis eloquendi hic deficit: melius et facilius dicimus quid non sit, quam quid sit. sic 1.Corin.2.vers.9. in cor hominis non ascendisse, nec oculum vidisse, nec aurem audiuisse illam maiestatem diuinorum bonorum. Hinc est quod in rebus diuinis caeci sumus, quam infirmitatem hic agnoscere debemus, inuia sunt nobis omnia diuina et quid quid nobis hic contingit cognoscendo, id totum gratiae Dei est. Agnouit hoc Aristoteles cum diceret oculos nostros caecutire in rebus diuinis, quemadmodum noctuae oculi in luce meridiana, sed causam ignorauit, hanc nos docent sacrae literae, peccatum scilicet, quod nos imagine Dei priuauit.  \pend
\section*{IN I. EPIST. AD TIMOTH. }
\marginpar{[ p.167 ]}\pstart Septimo, QVAM NEMO VIDIT HOMINVM, NEC VIDERE POTEST: et haec laus est diuinae maiestatis, quia adeo sincera est vt effugiat hominis oculos. Hinc dictum est Mosi, Exod.33. Nonvidebit me homo et viuet, ratio est in peccato, quod Dei prae sentia ferre non potest. Inde fuga et trepidatio primum coepit in primis hominibus cum peccassent. Hinc crassus est redditus visus oculorum et caecus, vt Deum nec possit, nec velit videre. Vident tamen Deum pii ac renati, sed in speculo et aenigmate, 1.Cor.13.ver.9.12. Vident eum in filio suo, qui est icon et character substantiae eius: sic in Scripturavidetur dum attenditur ipsius voluntas, sed omnes isti modi imperfecti sunt: ad plenum videbimus illum in vita aeterna, 1.Ioan.3. vers.2. Iustinus Martyr eleganter hoc dixit, Θεὸν νοήσαι χαλεπόν, φράσαι δὲ ἀδυνα- τον, τῷ καὶ νοήσαι δύνατον. Ista encomia concludit clausula pietati conformi: CVI SIT HONOR ET ROBVR AETERNVM. Amen. Δοξολογία est pii pectoris, non solum confitentis, sed humiliter Dei maiestatem adorantis: daemones enim quoque fatentur, sed contremiscunt: aliter pii.  \pend
\textit{17 Diuitibus in hoc seculo, etc. }\pstart Sextus locus de diuitum officio, ac pertinet ad quartum, qui iam expositus est. Duo  \pend
\section*{COMMENTARII }
\marginpar{[ p.168 ]}\pstart autem hic explicantur: primum vitia cauenda exprimit: mox his opponit virtutes sequemdas diuitibus. Vitia sunt primum ὑψηλοφρο- νεῖν, altum sapere, quia enim variis ornati sunt instrumentis elabendi et propulsandi casus communes, ideo inferiores libenter contemnunt, et se efferunt quasi omnibus meliores.  \pend\pstart Hinc recte de eis dicitur ab Euripide, ἔμ- φυτον τοῖς πλουτοῦσι σκαιοῖσιν εἶναι, hoc est, innatum illis esse petulanter agere. Alterum vitium est, SPEM PONERE IN INCERTITVDINE. OPVM. Ἀδηλότης  habet vim argumenti:incertis enim fidere stultum est:opes sunt incertae. Igitur hic fidere stultum est.  \pend\pstart Virtutes autem sunt quas diuites sequi debent. Primum, DEO FIDENDVM EST, qui est vita, hoc est, res firma per se et ratione agens. Deinde, QVIA NOBIS EXHIBET OMNIA: igitur largitur etiam opes: ideo non haerendum in effectu, sed redeundum est ad causam largientem. Tertio, ABVNDE LARGITVR, non parce, vide Psalm.147. Quarto, dat ad απόλαυσιν, hoc est, χρήσιν ad vsum: non vt recondamus, sed vtamur, χρήσις igitur non κτάσις respici debet.  \pend
\textit{18 Benefacere, etc }\pstart Reliquas persequitur virtutes, vt est ἀγαθοεργεῖν: benefacere opponitur otio diuitum: deinde malis illorum operibus, vt est  \pend
\section*{IN I. EPIST. AD TIMOTH. }
\marginpar{[ p.169 ]}\pstart vsurari, libidinari, scortari, et similia. Tertia virtus est abundare bonis operibus, hoc est, beneficum esse in multos. Quarta εὐμεταδό- τους εἶναι, ad erogandum faciles, opponitur illorum morositati, quia rogari volunt, et diu, nec tandem etiam faciles sunt, sed tristes et tetrici.2.Cor.9. Hilarem datorem amat Deus. Rom.12. Qui conmunicat, in hilaritate faciat. Quinta est κοινωνικους εἶναι, hoc est, in omnibus faciles etiam vt non rogati dent: quasi sua communia faciant omnibus communicatiui, quae boni est natura, vt se aliis libenter communicet ad conseruationem.  \pend
\textit{19 Recondentes sibi ipsis, etc. }\pstart Addit argumentum quo ad liberalitatem incitat, a spe magni praemii sumptum. Opum vsus verus est qui in futurum prodest, hoc est, ad vitam aeternam gradum struit: ergo, ad hunc vsum opes sunt impendendae. Comparatur illa per se Christi solius merito. Interim scriptura liberalitati idem tribuit vt excitet ad beneficentiam. Deus per se dat gratis, tamen nostra benefacta ornat hac laude quasi illorum respectu det.Vide Matt.6.Luc.12.  \pend
\textit{20 O Timothee, etc. }\pstart Epilogus sequitur continens principalem Propositionem repetitam. Apostrophe pathos habet. Depositum vocat officium Ecclesiasticum quod debet seruare, hoc est, quam  \pend
\section*{COMMENTARII }
\marginpar{[ p.170 ]}\pstart vigilantissime obire. Dicitur depositum alieni non nostri iuris, et debet summa fide custodiri et reddi, sed officium docendi Euangelium est depositum: igitur summa cura est conseruandum et sincera fide reddendum. Haec prior est loci huius pars. Altera est de vitiis euertentibus officii integritatem, quale est κενοφωνία, hoc est, inanis vocum strepitus, qualis erat illo seculo de Lege, Circuncisione, quaestionibus Iudaicis:talem κενοφω- νίαν habet Philosophia Ethnica, et hodie inducitur in literas S.vt de essentiali iustitia, et substantia peccati, et similibus rixis. Alterum vitium est ἀντιθεσις, hoc est, studium opponemdi sese veritati:Item pugnantem facere Scripturam: item nouum Testamentum cum veteri conmit tere, quae vitia sunt interpretationis. Falso di cta cognitio est, quam homines magni faciunt. Deus autem aspernatur et odit. Vide Colos.2. Ab exemplo periculoso deterret a curiositate. Quidquid periculo plenum est, id debet diligentia vitari a pio ministro:sed κενοφωνία et rixandi studium plenum est periculo: Ergo, etc. Haec considerent homines certaminum et contentionum hodie auidi. Dominus Ie\pend
\textit{21 Quam quidam, etc }\pstart sus Christus conseruet Esclesiam suam in tot certaminibus adeo periculosis, Amen.  \pend
\end{pages}
\end{document}
        