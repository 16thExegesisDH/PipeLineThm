
%%%%%%%%%%%%%%%%%%%%%%%%%%%%%%%% SCRIPT FOR E-RARA AND MDZ FILES     %%%%%%%%%%%%%%%%%%%%%%%%%%%%%%%%%%%%%%%%%%%%%%%%
%%%%%%%%%%%%%%%%%%%%%%%%%%% fini le 30.04.2025 par F. GOY            %%%%%%%%%%%%%%%%%%%%%%%%%%%%%%%%%%%%%%%%%%%%%%%%
% !TeX TS-program = lualatex
\documentclass{article}
\usepackage[T1]{fontenc}
\usepackage{microtype}
\usepackage[pdfusetitle,hidelinks]{hyperref}

\usepackage{polyglossia}
\setmainlanguage{english}
\setotherlanguages{latin,greek}
\usepackage[series={},nocritical,noend,noeledsec,nofamiliar,noledgroup]{reledmac}
\usepackage{reledpar}

\usepackage{fontspec}
\setmainfont{TeX Gyre Termes}

\usepackage{sectsty}
\usepackage{xcolor}

\usepackage{fancyhdr}
\pagestyle{fancy}
\fancyhf{}
\fancyhead[LE,RO]{\nouppercase{\leftmark}}  
\cfoot{\thepage}
\renewcommand{\headrulewidth}{0.4pt}

% Redefine \section to remove numbering
\usepackage{titlesec}
\titleformat{\section}[block]{\normalfont\scshape\color{gray}}{}{0pt}{} % no number in heading
\titleformat{\subsection}[hang]{\normalfont}{}{0pt}{} % also remove subsection number
\titleformat{\subsubsection}[hang]{\normalfont\footnotesize\color{black}}{}{0pt}{}

% Modify how section marks are stored to exclude numbers
\makeatletter
\renewcommand{\sectionmark}[1]{%
	\markboth{#1}{}} % Only store the section title, without number
\renewcommand{\subsectionmark}[1]{%
	\markright{#1}} % Only store the subsection title, without number
\renewcommand{\numberline}[1]{} % Hide the section number in TOC
\makeatother

\begin{document}

\date{}
        \title{Commentarii in epistolas D. Pauli ad Timotheum: [Hyperius Andreae],[Ed. Mylius, Johannes], [1582]}
\maketitle
\tableofcontents
\clearpage
\begin{pages} 
\beginnumbering
        
\section*{AD TIMOTHEVM CAP. I. }
\marginpar{[ p.63 ]}\pstart re ecclesiam recusarat, diuexaret in corpore, morbum videlicet, vel aliud malum (exe plo Hiob, quem Deus voluit eo pacto probars) quo caro, quae petulantior fuerat, domaretur, immittendo. Alii vero per traditionem satanae intelligunt simpliciter declarationem separationis a Christo et a coetu fidelium, ad tempus, ob quam peccator vsque  adeo angeretur vt videns se contemni et vitari ab omnibus, absorberetur propemodum prae moerore: quo pacto caro illius humillata afflictaquequodammodo interiret. Vtraque  autem interpretatio, si primae ecclesiae statum consideremus, locum merito habebit. Etenim consentaneum vero est, quod quo tempore expediebat veritatem doctrinae Euangelicae et ministerii ecclesiastici dignitatem comprobari miraculis, nomnunquam corpore visibiliter sint diuexati, quos ecclesia Dei propter peccata satanae tradidit. Haec conueniebat sane tunc multas ob caussas ita fieri, Caeterùm ex quo tempore tam doctrina, quam ministerlum ecclesiae satis est confirmatum, nequaquam pereque vt ante, necessarium est, eos qui excommunicantur, in corpore a satana cruciari. Magnum est, si pudore, vt ad Thessalonicenses dicitur, suffundantur, atque ita emendentur:ideoque erit excommunicatio legitime facta, iuxta metuenda et efficax, atque si externa afflictio in corpore accideret. Rata vsque  manet Christi sententia et promissio dicta ad ecelesiam, sontes iustis de caussis et seruato ordine excommunicantem, Matt.18. Amen dico vobis, quaecunque  alligaueritis super terram, erunt ligata in coelo: et quaecunque  solueritis super terram, erunt soluta in coelo. Neque  vero eos, qui excommunicati ac traditi satanae sunt, existimabimus sic penitus exclusos a regno Christi, quasi iam simpliciter damnati, nulla spe salutis relicta, essent: sed statuemus esse admodum aegrotos, quibus tamen medicamem adhuc porriga, et quos seruari posse in vita confidamus. Ne habeatis, inquit Apostolus, velut inimicum, sed admonete vt fratrem: 2. Thessal. Quapropter his, qui aliquamdiu excommunieati fuerunt, si modo egerunt poenitentiam, reditus patet ad ecclesiam: atque  sano corpori membra sanata rursus copulantur.ltaqueApostolus 2. ad Corinth.2. Scripsi vobis, inquit, non vt moerore afficeremini, mox addit : Sufficit istiusmodi homini increpatio haec, quae facta est a pluribus  adeo, vt e diuerso magis condonare debeatis, et consolars, ne quomodo fiat, vt immodico dolore absorbea tur huiusmodi. Quapropter obsecro vos efficite, vt valeat in illum charitas. Quando igitur in homine lapso dolor, seria poenitentia, vitae emendatio, apparent, suadet charitas noxam ei condonari, atque  ecclessaeiterum adiungi. Porro apud diuersas ecclesias diuerlum fuilse olim morem lapsos et poenitentes in gratiam recipiendi, ex patrum scriptis licet cognoscere. Vide Tertullian. in liber  de poenitentia, Cyprianum liber  3. Epistol. Epistola 14. et 16. Augustinum in Enchiridio ad Laurentium capite 65. Chrysostomum homil. 4. in 2. Epist. ad Corinthios. Et haec de altera parte doctrinae episcoporum quae est defide.  \pend
\phantomsection
\addcontentsline{toc}{subsection}{\textit{Sequitur de doctrina εὐταξίας, seu quarundim rerum pertinentium ad bonum ordinem in ecclesia. }}
\subsection*{\textit{Sequitur de doctrina εὐταξίας, seu quarundim rerum pertinentium ad bonum ordinem in ecclesia. }}
\section{CAPVT II ❧}
\phantomsection
\addcontentsline{toc}{subsection}{\textit{\huge\textbf{A}\normalsize Dhortor igitur, vt ante omnia fant deprecationes, obsecrationes, interpellationes, gratiarum actiones pro omnibus hominibus, pro regibus, et omnibus in eminentia constitutis, etc vsque  ad cap.3. }}
\subsection*{\textit{\huge\textbf{A}\normalsize Dhortor igitur, vt ante omnia fant deprecationes, obsecrationes, interpellationes, gratiarum actiones pro omnibus hominibus, pro regibus, et omnibus in eminentia constitutis, etc vsque  ad cap.3. }}\pstart \huge\textbf{S}\normalsize Equitur de doctrina εὐταξίας, seu quarundam rerum pertinentium ad bonum ordinem in ecclesia: quae doctrina semper necessaria est post doctrinam fidei. Etsi enim non sit ad salutem necessaria, quippe apud perfectos ea nequaquam opus est, vtpote qui omnia huiusmodi agunt sponte, neque oportet eos vllis ceremoniis regi aut gubemnari: tamen quia hinc accedit non\pend
\vspace{0.5cm}\noindent
\vspace{0.2cm}\rule{1cm}{0.2pt}\\ 
\hspace{0.2cm}\textit{mg}
\footnotesize De traditione satanae spirituali. 
\normalsize| 
\hspace{0.2cm}\textit{mg}
\footnotesize An corporaliter uexati sint excommunicati. 
\normalsize| 
\hspace{0.2cm}\textit{mg}
\footnotesize Excommunicatio legitina aeque  efficax, ac corporalis affuictio. 
\normalsize| 
\hspace{0.2cm}\textit{mg}
\footnotesize Ande exconmunicatorum salute prorsus sit desperandum. Lapsis et excommunicatis reditus patet ad ecclesuam per patentiam. 
\normalsize| 
\hspace{0.2cm}\textit{mg}
\footnotesize Εὐταξία necessaria est in ecclesia. 
\normalsize| 
\section*{IN I. EPIST. D. PAVLI }
\marginpar{[ p.64 ]}\pstart nihil adiumenti rudibus, et per hanc doctrinam opera quaedam Charitatis promouentur, et seruatur in conuentibus publicis bonus ordo, ideo non est temere contennenda. Ceremoniae et huiusmodi ordinationes habendae sunt in vita, sicut fabro habentur praeparamenta ad aedificandum, quae non in hoc parantur, vt aliquid sint, aut aliquid maneant, sed quod sine iis aedificari non possit. Nam perfecto aedificio tolluntur, et stultus esset, qui semper illa magnifaceret et ampliaret, neglecto interim principali aedificio, id est, omissa doctrina fidei. Vnde eatenus adhibendae sunt ceremoniae, quatenus sunt vtiles, id est, quatenus significant aliquid et figurant, et quatenus sunt necessariae, id est, non possunt commode negligi. Alias enim periculum foret, ne doctrina fidei per doctrinam vrαξιας obscuraretur et laederetur: Sicut pueris opus est et vtile, vt foueantur mulierum et puellarum gremiis. quod tamen periculosum foret paulon prouectioribus: Ita iudicandum de ceremoniis. Et sicut paupertas spiritualis periclitatur in diuitiis, fidelitas in negociis, humilitas in honoribus, abstinentia in conuiuiis, castitas in delitiis: lta doctrina fidei periclitatur interdum in doctrina εὐταξίας. Haec de doctrina εὐταξίας in genere.  \pend\pstart Caeterùm non hic praescribit Apostolus omnia, quae ad bonam ordinationem pertinent, quae in coetu ecclesiastico agi conueniat: sed vnam tantumodo rem vrget, vel ante non satis commendatam, vel parum diligenter obseruatam. Ea est de orando pro omnibus hominibus, potissimumque pro magistratibus. Quae quidem ἐυταξία siue bona constitutio adeo est necessaria in ecclesia, vt nulla magis: Et doctrinae fidei sic adiuncta, vt diuelli aegre queat: et ad charitatem declarandam nulla magis idonea, sic itaque  ait Apostolus: Adhortor, vt ante omnia fiant deprecationes:) Adhortor, inquit, Dignissima res de qua inprimis episcopi admone antur, et cuius prima cura omnibus Christianis esse debet, nimirsi vt oretur. In genere autem oratio dicitur constans et ex intimo affectu alicuius rei necessariae a Deo petitio: Etiamsi alsj malunt, esse mentis ad Deum eleuationem. Dico petitionem, quia oportet in ea agnoscamus aliquid nobis deesse, quod cupimus a Deo impetrare: vt videlicet nostram fragilitatem vel miseriam intelligentes humiliemur contra Pharisaeorum superbiam, et cum semper egeamus, semper ad orandum excitemur. A Deo autem, quia ab illo solo omne bonum descendit, et a nullo alio debemus petere vel expectare. Necessaria res petenda est, quia stulta vota vel inutilia Deus non audit, et petita iuste negat. Constans praeterea esse debet oratio: quia Deus vult saepe rogari, et fidem probare, et saepe rogatus tandem concedit, oportet semper orare et nunquam defatigari. Ex intimo affectu: quia exanimo, non fucate, et ardenti affectu praecipuem fide, loann.4. Veri adoratores in spiritu et veritate orant. Iacobus sine haesitatione et cum fide vult nos orare. Haec vero requiruntur in oratione, quam vnusquisque  pri uatim facit quotidie et vbiubisit locorum. Sed multo magis requiruntur in oratione publica, quae fit ab omnibus in publico coetu et congregatione: quam credi par est semper etiam efficaciorem fore, quando a quamplurimis et concordibus animis fit, Vult autem fieri orationes ante omnia. Id dupliciter accipi potest: vel vt significet, nihil sic necessarium, atque est oratio: ac propterea eam prae quibuscunque  alujs charitatis officiis exercendam: vel vt significet circumstantiam temporis, id est, prima statim diei parte conueniendum ad orationem, priusquam aliquid operis faciamus. Id quod etiam conuenientissimum est, vt primum opus et prima diei portio rebus diuinis dicetur: et quia nihil prius Deo habere debemus, non est quod putemus bene successurum in relsquis rebus, nisi Deum prius orauerimus: quam opinionem commodissimum est teneris et rudibus animis inculcare. Atque hunc morem olim seruatum scriptores testantur: inprimis Tertullianus, apud quem saepe antelucanorum coetuum mentio est. Et Plinius ipse Epistol. libro decimo, ad Traianum Imperatorem idem refert. Testantur eundem morem certi hymni, vt ille Prudentii, Alius diei nuncius et alter, iam lucis orto sydere, etc. Atque ita faciendo nihil detrahebantur diurnis operis. Etanimus sobrius, ieiunus, a quiete corporis mane maxime compositus est ad res diuinas peragendas, magis quam alia diei parte. Tametsi vtnec eleemosyna, nec ieiu\pend
\vspace{0.5cm}\noindent
\vspace{0.2cm}\rule{1cm}{0.2pt}\\ 
\hspace{0.2cm}\textit{mg}
\footnotesize A fimili. 
\normalsize| 
\hspace{0.2cm}\textit{mg}
\footnotesize Ceremoniae quatenus in ecclefia retirendae et coIendae. A ratione aetatis. 
\normalsize| 
\hspace{0.2cm}\textit{mg}
\footnotesize Ordtionis de finitio. 
\normalsize| 
\hspace{0.2cm}\textit{mg}
\footnotesize 1. In oratione homo aguoscit, et confitetur suam ipsius fragilitatem. 
\normalsize| 
\hspace{0.2cm}\textit{mg}
\footnotesize 2.A Deo solo res bonae patendae. 
\normalsize| 
\hspace{0.2cm}\textit{mg}
\footnotesize 3. Res necessariae petendae, uota inutilia praeter mittenda. 
\normalsize| 
\hspace{0.2cm}\textit{mg}
\footnotesize 4. Oratio debet esse constans et ardua. 
\normalsize| 
\hspace{0.2cm}\textit{mg}
\footnotesize 5. Oratio proficisci debet ex animo et corde. lacobus  1. Preces labores et studia hominum debent praecedere. Coetus antelucani. Preces non impediunt labores et operas homium. 
\normalsize| 
\section*{AD TIMOTHEVM CA P. II. }
\marginpar{[ p.55 ]}\pstart nium: ita nec oratio temporibus aut diebus  est alliganda: sicut nec tunc tantum modo, sed praeterea quacunque alia hora orare, oportunum est. Porro, quo magis excitet omnes ad orandum, vtitur congerie, plures orationum species coaceruans: quo nimirum declaretrem esse maxime necessariam et plurimi faciendam, vt pro eadem signifcatione liceat omnia illa accipere, et saepe contunduntur, sicut in exordio Philippensium et alibi Apostolus precationibus includit gratiarum actionem. Nonnullis tamem magis placet distinguere, vt Charitatis officia in orando plenius intelligantur, sic autem legimus: Fiant deprecationes, obsecrationes interpellationes, gratiarum actiones Deprecatio quae Graece δὲησις, significat orationem, qua petimus liberari a malis vrgentibus vel imminentibus: vnde dicimus quotidie, libera nos a malo. Obsecratio, Graece προσευχή, est qua precamur nobis dari bona: quo pertinet illud, Panem nostrum quotidianum da nobis. Interpellatio, ἔντευξις, qua querimur de iniuria nobis illata: quo in genere sunt multi Plalmi, quibus petimus tales in viam redire et resipisscere: quo videtur pertinere nllud. Sanctificetur nomen tuum. Gratiarum actio ευχα- ριστία, cum pro bonis accepris, imo et pro malis bona voluntate Det nobis inflictis, Deo agimus gratias: quo illud, Fiat voluntas tua. Et haec omnia non solum nostra causa facienda sunt, verum etiam propter alios. Deprecandum enim, vt alii quoque a malis liberentur:obsecrandum vt aliis quoque bona contingant:interpellandum pro aliorum quoque aduersariis, vt Deus eos emolliat: gratiae agendae, tùm etiam, cum aliis bene fuerit. Ideo additur, pro omnibus hominibus: Charitas siquidem Christiana nullam personam debet ex cludere. Nec solum pro nobis orare debemus, verùm pro aliis quoque. Nec pro illis tantûm, qui amici sunt, qui bene volunt, quiqueconiuncti sunt religione: verum etiam pro inimicis et pro aduersariis nostrae fidei. Pro omnibus, inquit, ἐμφατικῶς. Deus facit solem suum oriri super malos perinde ac bonos. Et si non nisi amicos diligimus, nihilo meliores sumus gentibus incredulis. Christus orauit pro erucifigentibus, et Stephanus pro lapidantibus: et quondie oramus, Remitte nobis debita sicut nos remittimus debitoribus, et iis qui nos offendunt. Et pro bonis ac coniunctis nobis in fide orandum est, vt tales perseuerent, et quantum fieri potest, in melius proficiant: pro malis orandum, vt conuertantur, resipiscant. et tandem Deus per eos et in eis glorificetur. Pro regibus et omnibus in eminentia constitutis, Cum pro omnibus orandum esset constet, at multo iustiisime orandum pro magistra tibus. Atque id eo tempore voluit Apostolus fieri quam dillgentilsime in ecclesia cûm magistratus passim erant ethnici, religionis Christianae hostes: sed ideo multon maxime erat pro omnibus orandum, quia illorum nutu atque arbitrio multa fieri poterant per quae Euangelii propagatio velimpediretur, vel propagaretur. Vt autem animi eorum immutarentur ad fauendum potius Euangelio, quam ad persequendum indurarentur, aequum erat orationibus id a Deo flagitare. Quae doctrina etiam a Sanctissimis viris olim diligenter fuit tradita. Sic enim Hieremias capite 29. iubet orari pro salute regis et regni Babylonis, vnde apparet Pauli verba quaedam huc translata. Quaerite, ait ille, pacem ciuitatis, in quam migrare vos feci, et orate pro ea Dominum, quia in pace eius erit pax vestra. Similiter Esdras pro Cyro rege Persarum, Baruch. cap.1. pro salute Nabuchodonosor orari voluit. Tertullianus Apologetici sus capite 30. hanc Christianorum pietatem celebrat. Nos oramus, inquit, pro omnibus Imperatoribus, vitam illis prolixam, imperium securum, domum tutam, exercitus fortes, senatum fidelem, populum probum, orbem quietum, et quaecunque hominis et Caesaris vota sunt. Sunt autem multae caussae, quae nos moueant ad orandum pro magistratibus, quas exposuimus in libello nostro de honorandis magistratibus. Apostolus etiam hic argumenta quaedam suasoria attexuit. Et placidam ac quietam vitam agamus cum omnt pietate et honestate.) A caussa finali siue ab vtili. Ideo orandum est pro magistratibus, quia per eos potest nobis contingere,ut placide ac quiete agamus, aclibere Deum laudemus sine ullis molestiis. Certe si nunquam datur piis quiete agere, necessario omittunt multa bona opera:imo  \pend
\vspace{0.5cm}\noindent
\vspace{0.2cm}\rule{1cm}{0.2pt}\\ 
\hspace{0.2cm}\textit{mg}
\footnotesize semper et omni loco orandum est. Congeries συναθροισμός. 
\normalsize| 
\hspace{0.2cm}\textit{mg}
\footnotesize Deprecatio dέησις. Objecratio προσευχή. Interpellatio ἔντευξις. 
\normalsize| 
\hspace{0.2cm}\textit{mg}
\footnotesize Gratiaram action ευχαριστία. 
\normalsize| 
\hspace{0.2cm}\textit{mg}
\footnotesize Emphasis. 
\normalsize| 
\hspace{0.2cm}\textit{mg}
\footnotesize Pro inimicis etiam orandum. Math.5. Luc. 6. Matth.27. Actor.7. 
\normalsize| 
\hspace{0.2cm}\textit{mg}
\footnotesize Pro magistratu orandum. 
\normalsize| 
\hspace{0.2cm}\textit{mg}
\footnotesize Vota Christianorum pro Imperatoribus. 
\normalsize| 
\hspace{0.2cm}\textit{mg}
\footnotesize ab utili. 
\normalsize| 
\section*{IN I. EPIST. D. PAVLI }
\marginpar{[ p.66 ]}\pstart quidam etiam malorum magnitudine perculsi, Euangelio malunt renunciare, quam diutius propter veritatem affllgi. Qua ex re cùm obscuratur gloria Dei, merito est orandum pro magistratibus. Et credibile est etiam ethnicorum principum iras remissas, imminuisseque  nonnihil suam truculentiam, vbi resciscerent Christianos pro ipsorum salute orasse.lta nimirum tum oraruat Christiani pro impüs principibus, et quietem sibi a Deo impetrarunt, et beneuolentiam saltem principum conciliarunt aliquam. Proinde non eo line optanda est placida et quieta vita, vt luxu et licentia perdamur, et quasi eiecto freno excurramus in quaeuis vitia, sed vt agamus cum omni pietate et honestate, id est, ex nostra libertate incrementum accipiat vera pietas et promoueatur itidem uitae honestas. Emphasis in nota uniuersali. Nomine pietatis intellige quicquid ad cultum Dei, ad ueram religionem pertinet, siue per dogmata, siue per externa opera:ut liberi conuentus ad orandum, libera praedicatio uerbi, libera exercitatio operum Charitatis, quae proprie pertinent ad propagationem Euangelii et ad incrementum gloriae Dei. Nomine honestatis comprehenduntur quae commeudant nos etiam ciuiliter apud homines, et per quae homines aedificantur, instruunturquein bonis exemplis, et occasionem habent bene et sentiendi et loquendi de nobis. Ad hunc ergo finem, ut uidelicet liberem perficiamus ea, quae pertinent ad gloriam Dei et aedificationem proximorum, optanda est placida ac quieta uita. Quod si hic finis desit, certe metuendum, ne quies et orium potius caussa sint malorum, quam bonorum. Certe uix unquam ecclesia fuit in meliori statu, nunquam credentes fuerunt tam puri, simplices, integri, incorrupti, quam tùm, atque  cum ryranni saeuirent. Contra ubi ecclesiae coeperunt otiari, quiescere, ditari, dominari: diuitiae et quies extinxerunt penitus fidem et charitatem:filia deuorauit matrem, inquit Bernardus.  \pend
\phantomsection
\addcontentsline{toc}{subsection}{\textit{Nam hoc bonum est et acceptum coram seruatore nostro Deo, qui cunctos homines vult saluos fieri et ad agnitionem veritatis venire. }}
\subsection*{\textit{Nam hoc bonum est et acceptum coram seruatore nostro Deo, qui cunctos homines vult saluos fieri et ad agnitionem veritatis venire. }}\pstart Aliud argumentum a Pio. Ait orare pro omnibus, inprimisque ́, pro magistratibus, bonum esse et acceptum coram Deo. Ac videtur subesse occupatio. Nam vbi fortassia quorundam sragilitas aut coecitas iudicasset absurdum orare pro iis, qui persequuntur pios, et impediunt gloriam Dei promoueri : respondet, imo id pium esse et Deo in primis gratum. Quando igitur audimus id per se esse bonum, et acceptum Deo, nequaquam debemus hic esse cessatores. Et vere id bonum ac pium dici potest, quia inde sequuntur plerunque  multa bona, et ea inprimis per quae syncera pietas promouetur. Cum addit, Qui cunctos homines vult saluos sieri, et ad cognitionem veritatis venire, Latet argumentum ab exemplo: quasi dicat: Deus vult omnes homines saluos fieri, quicunque illi sint: igitur et tu debes idem velle:ac proinde pro omnibus, etiam impiis, orare. Est ergo hic exemplum insignis bonitatis diuinae, quam conuenit et nos tùm affectu tum etiam diligentia omni imitari. Possumus etiam hic discere, quid maxime debeamus impiis magistratibus comprecari, videlicet vt ad cognitionem veritatis perueniant. Veritas autem hic significat Christum, qui est via, vita, veritas. Sed hic quaestio oritur. Si Deus velit omnes homines saluos fieri, qui fit, quod non omnes saluantur ? Respondet paucis verbis Ambrosius. In omni locutione sensus est, conditio latet: Vnde secundae Petri primo dicitur, omnis scriptura indiget interpretatione. Vult Deus omnes saluos fieri, sed si accedant ad eum. Non enim sic vult, vt nolentes saluentur, sed vult illos saluari, si et ipsi velint. Nam vtique qui legem omnibus dedit, nullum excipit a salute. Numquid item medicus idcirco proponit in pubhco, vt omnes ostendat se velle seruare: si tamen ab aegris requiratur ? Non est enim vera salus, si nolenti tribuatur, nec gaudere potest in percepta salute, qui inuitus, si  \pend
\vspace{0.5cm}\noindent
\vspace{0.2cm}\rule{1cm}{0.2pt}\\ 
\hspace{0.2cm}\textit{mg}
\footnotesize A pio. Occupatio. 
\normalsize| 
\hspace{0.2cm}\textit{mg}
\footnotesize Ab exemploA maiori ad minus. 
\normalsize| 
\hspace{0.2cm}\textit{mg}
\footnotesize loan. 14 Questio. Cur non omnes saluentur. Responsio ex Ambrosio. 
\normalsize| 
\hspace{0.2cm}\textit{mg}
\footnotesize a simili. 
\normalsize| 
\section*{AD TIMOTHEVM CAP. II. }
\marginpar{[ p.67 ]}\pstart E tamen fieri potest, accepit medicinam, ut non dicam, quod medicinae effectum habere non potest, nisi ad illam aeger animum accommodauerit, quia haec medicina non est corporalis sed spiritualis, quae neque dubiis proficit, neque inuidis. Fides est enim quae dat salutem, quam nisi mens tora susceperit uoluntate, non solum nihil proderit, sed et oberit. Fidei etenim gratia hanc habet potestatem, ut deuotis sibi diuinam infundat medelam, indeuotis uero conferat morbum, per quem totus homo intaereat. lta Ambrosius. Nec obest, quod Roman.9. dicitur: Deus cuius uult miseretur, et quem uult indurat. Sicut saluatio hominis non est humani meriti. Ita perditio siue induratio non est simpliciter diuinae uoluntatis. Deus peccatorum non est autor. Sed indurare sic quodammodo dicitur: quia eos ea facta perpetraturos praeuidit, per quae et ob quae merito indurari eos permittit. Quare sic statuendum est in tota hac causa de hoc loco et similibus: quod, ut nostris uiribus nequaquam tribuendum est, quod debetur soli gratiae et potentiae ipsius Dei: lra e diuerso diuinae bonitati nequaquam est imputandum, quod nostra contingit propria malitia. A nobis igitur excaecatio est et indusratio, a Deo gratia et salus. Excaecatio contingit nobis per Dei iustitiam, quae merito et iure punit nostra peccata: saluatio contingit per Dei bonitatem et misericordiam, quam nullis nostris meritis promeremur: Sicque  Deus excaecare et illuminare, misereri et indurare, sed respectu diuerso dicitur. Nonnulli clarius uolentes explicare et simplicius, dicunt, in hac propositione, Deus uult cunctos homines saluos fieri, in nota illa uniuersali esse Hebraismum, quo, Omnes, accipitur, pro Omnis generis:ut sit sensus, Deum uelle omnis generis homines saluos fieri, siue gentiles, fiue ludaeos, siue magistratus, siue subditos. Sic quoque accipiendum, ubi mox dicitur, quod Christus dedit semetipsum redemptionem pro omnibus. Et facit haec ratio non parum ad commendationem diuinae bonitatis, utpote cùm ostenditur, Deum non esle acceptorem personarum, neque agnoser uelle tantummodo a ludaeis, aut in sola ludaea, uerùm ab omnibus et ubique terrarum.  \pend
\phantomsection
\addcontentsline{toc}{subsection}{\textit{Vnus enim Deus, unus etiam reconciliator Dei et hominum, homo Christus lesus, qui dedit semetipsum pretium redemptionis pro omnibus, vt esset testimonium temporibus suis. }}
\subsection*{\textit{Vnus enim Deus, unus etiam reconciliator Dei et hominum, homo Christus lesus, qui dedit semetipsum pretium redemptionis pro omnibus, vt esset testimonium temporibus suis. }}\pstart Plenior probatio adiecta, quod Deus uelit omnes homines saluos sieri : et adhuc persequitur argumentum ab Exemplo, Vnus est, inquit, Deus omnium, qui non respicit personam. Ergo uos quoque respicere personam non debetis, sed pro omnibns, etiam aduersariis orabitis. Praeterea unus est reconciliator Dei et hominum, uidelicet Christus, qui dignatus est semetipsum dare in prerium redemptionis pro omnis generis hominibus. Ergo uos quoque saltem orando intercedere pro quibuscunque aequum est. Proinde dicit, Vnus Deus ommum, etiamsi ab omnibus nequaquam agnosceretur. Verùm cum unus Deus omnes condiderit, idemque  omnes uoluerit in gratiam recipere, et ad agnitionem sui nominis uenire, nullos spernetis uel abiicietis: merito omnium et unus Deus dicitur. Vnde Roman.3.An ludaeorum Deus tantùm? Annón et gentium? Certe et gentium. Quandoquidem unus Deus, qui iustificabst circumcisionem ex fide, et praeputium per fidem. Ergo quando unus Deus omnium praedicatur, mirifice hic nobis commendatur charitas: ac iubemur inuicem, imo et inimicos et tyrannos diligere, ac pro eorum salute orare, utpote quorum idem est nobiscum Deus, idem conditor et communis quasi pater, adeoque  ipsi quasi communes sunt nostri fratres. Quos Deus ipse non despicit, iniquum est a te contemni.  \pend\pstart Vnus etiam reconciliator Dei et hominum, homo Christus Iesus.) Cum diceret, unum esse Deum, non exclusit spiritum sanctum uel filium: quippe substantia unum sunt, qui personis distinguntur. Sicut itaque unus est Deus atque omnium, eo mouert debemus, ut neminem aspernemur. lta unus quoque est omnium reconcliator, unus  \pend
\vspace{0.5cm}\noindent
\vspace{0.2cm}\rule{1cm}{0.2pt}\\ 
\hspace{0.2cm}\textit{mg}
\footnotesize Deus quomodo indurare dicatur. 
\normalsize| 
\hspace{0.2cm}\textit{mg}
\footnotesize A contrario. A nobis est excecatio et a Deo gratia et salus.. 
\normalsize| 
\hspace{0.2cm}\textit{mg}
\footnotesize Senentia fimplicior et dilucidior. 
\normalsize| 
\hspace{0.2cm}\textit{mg}
\footnotesize Actor.16. Coloso.3. Galat.3. 
\normalsize| 
\hspace{0.2cm}\textit{mg}
\footnotesize Vuus Deus 
\normalsize| 
\hspace{0.2cm}\textit{mg}
\footnotesize Vnus reconciliator. 
\normalsize| 
\hspace{0.2cm}\textit{mg}
\footnotesize Dei dilectio nos ad mutuam hortatur dilectionem. 
\normalsize| 
\hspace{0.2cm}\textit{mg}
\footnotesize Tres personae, una substantia, unus Deus. 
\normalsize| 
\section*{IN I. EPIST. D. PAVLI }
\marginpar{[ p.68 ]}\pstart est qui omnes cum Deo patre reducit in gratiam, acpropterea non est quod quilquam caeteros vilipendat, quasi non habeant eundem reconciliatorem omni tempore ad subue niendum ipsis paratum. Proprie autem reconciliator iste seu mediator (ita enim graece μεσίτης) Christus est, qui solus ad munus hoc fuit idoneus, et se paratum obtulit. Dissidium erat inter duas partes, videlicet inter Deum patrem, et homines, qui patrem peccando legesqueeius transgrediendo offenderant: ita ut iam Deus pater per suam iustitiam merieo vniuersum genus humanum erat perditurus. Nec interim homines vlla ratione iratum patrem potuissent placare. Christus itaque  solus reconciliandae multitudinis patrocinium ad se recepit, solus uoluit et potuit. Proinde ut commodius id faceret, sicut diuersa erat partium conditio: ita placuit ei, ut se vtrique  parti accommodaret, vtriusque  partis personam induere: sicque  mediator inter Deum et hominem dum vult fieri, sicut ipse semper fuit Deus, ita homo quoq in tempore factus est, aptissimeque  hominem cum Deo reconciliauit, quando ipse ut Deus erat, apud Deum patrem quiduris potuit efficere, et ut homo potuit hominum necessitatem et miseriam agnoscere: et sic utrique  parti visus est aequus arbiter, quando vtriusque  partis naturam in se habuit. Ille ergo homo Christus lesus, idemqueDeus dependit patri Deo quicquid pro hominum salute erat dependendum. Vnde sequitur,  \pend\pstart Dedit semetipsum pretium remdemptionis pro omnibus.) Caussa dissidii erat peceatum. Peccatum autem quoniam non nsi sanguine potest expiari, et quidem non nisi preriosissimo sanguine, non taurorum aut hircorum, dignatus est Christus suum ipsius sanguinem fundere, imo totum sese dare in pretium redemptionis, et quidem pro omnibus, ita ut id pretium esset sufficiens ad satisfaciendum irato patri, pro peccatis non nostris tantum, vt dicitur 1. Iohan.2. verum etiam totius mundi, et nullum uoluerit a satisfactionis illius participatione excludi: nisi qui semetipsum suis peccatis et perfidia excludit. Singula autem uerba habent suam Emphasin, illa praesertim semetipsum et pro omnibus. EstqueMetaphora in uoce, pretium redemptionis. Solent enim captiui redimi: nam et graece legitur ἀντιλυτρον, ubi nos legimus pretium redemptionis. Significat proprie id uocabulum pretium, quo redimuntur captiui ab hostibus, eamque  commutationem, qua capite caput, et uita redimitur uita. Proinde est egregia laus Christi, quaque  insignis illius charitas erga humanum genus quod perierat, celebratur, cum ipse uoluerit agere reconciliatorem, et quidem impendendo semetipsum pretium redemptionis, et pro cuiuscunque  conditionis hominibus. lohan.1o. Maiorem charitatem nemo potest exhibere, quam ut animam suam ponat quis pro amicis suis. Roman.5. Commendat suam charitatem erga nos Deus, quod cum adhuc essemus peccatores, Christus pro nobis mortuus est. Ac proinde nos hoc exemplo Christi ad charitatem moueri debemus. Praeterea ingens consolatio conscientiis nostris esse debet, quando nobis sic in scripturis proponitur Christus homo aut reconciliator: ut nunquam profecto sit desperandum, cum hominem habeamus Christum, qui uelit intelligere nostras miserias, et quem audeas compellare: (nam propterea Apostolus eum hic hominem diserte appellat:) et eundem Deum, qui possit peccata nostra    tollere. Huc pertinent illa Roman.5. Iustificati perfidem pacem habemus ad Deum,    per Dominum nostrum lesum Christum. Roman. 8. Christus est qui mortuus est, et suscitatus est, et sedet ad dexteram Dei, et intercedit pro nobis. 1.Iohan.2. Filioli mei    haec scribo uobis ne peccetis: et si quis peccauerit, aduocatum habemus apud patrem, lesum Christum. Et ipse est propitiatio pro peccatis nostris. Hebr.9. Ob id    noui testamenti conciliator est, ut morte intercedente ad redemptionem earum praeuaricationum, quae fuerant sub priori testamento, si qui uocati sunt, promissionem accipiant aeternae haereditatis. Hebr.4. Christus dicitur thronus gratiae. Romanor.3. Hunc proposuit Deus reconciliatorem per fidem, interueniente ipsius sanguine. In hos ergo locos et similes passim obuios decet perturbatas conscientias intueri. Postremo discimus unum duntaxat reconciliatorem agnoscere Christum, ne plures nobis reconeiliatores statuamus. Verum quidem est, orant sancti in terris positi pro se mutuo et exaudiuntur, sed propter illum vnum reconcillatorem Christum, qui  \pend
\vspace{0.5cm}\noindent
\vspace{0.2cm}\rule{1cm}{0.2pt}\\ 
\hspace{0.2cm}\textit{mg}
\footnotesize Vnus media tor Dei et hominum Christus Iesus. Dissidium inter Deum et homines. 
\normalsize| 
\hspace{0.2cm}\textit{mg}
\footnotesize Christus utriusque , partis personam suscipit. Christus Deus. Christus Homo. Peccatum caussa dissidii mter Deum et homi nem. 
\normalsize| 
\hspace{0.2cm}\textit{mg}
\footnotesize Hebr.9. 
\normalsize| 
\hspace{0.2cm}\textit{mg}
\footnotesize Peccata hominum non nisi sanguine Christi possunt expiari. Metaphora a captiuis. ἀντίλυτρον quid. 
\normalsize| 
\hspace{0.2cm}\textit{mg}
\footnotesize Christi charitas erga homines ineffebilis. 
\normalsize| 
\hspace{0.2cm}\textit{mg}
\footnotesize Ioan.10. Rom.5. 
\normalsize| 
\hspace{0.2cm}\textit{mg}
\footnotesize Vsus charitatis Christi in nobis. 
\normalsize| 
\hspace{0.2cm}\textit{mg}
\footnotesize Ron.8. 
\normalsize| 
\section*{AD TIMOTHEVM CAP. II. }
\marginpar{[ p.69 ]}\pstart ipssemet intercedit pro nobis, exaudiuntur. Iniuriam facit Christo, qui alium sibi mediatorem fingit. Sequitur:  \pend\pstart Vt esset testimonium temporibus suis.) q' d. Quod Christus dederit se in pretium redemptionis pro omnibus, non pro ludaeis tantùm, sed pro cuiuscunque  conditionis hominibus: id Deus testatum fieri uoluit toti mundo, omnibusqueretro temporibus, vt homines intuentes in tantam charitatem Dei, similiter ad Charitatis officia inter sese praestanda excitarentur. Addit, sibi iniunctum id munus, testandi uidelicet mundo, quod ita esset.  \pend
\phantomsection
\addcontentsline{toc}{subsection}{\textit{In quo positus sum ego praeco et Apostolus, veritatem dico in Christo, non memtior, doctor gentium cum fide et veritate. }}
\subsection*{\textit{In quo positus sum ego praeco et Apostolus, veritatem dico in Christo, non memtior, doctor gentium cum fide et veritate. }}\pstart Confirmatio praecedentis assertionis de vno reconciliatore Christo, a ratione officii sui, et sura ipsius fide integritateque  quam iuramento probat. Ostendit proprium suum officium esse testari et palam facere eximiam Dei erga nos charitatem, praecipueque  quod Christus sit vnus reconciliator Dei et hominum, qui semetipsum obtulerit in pretium redemptionis pro omnibus. Pronide hic notatur, hanc propriam esse doctrinam Apostolicam de uno reconciliatore Christo. Quare grauiseime peccant, et a doctrina apostolica longissime discedunt, qui alios reconciliatores, uel alias reconciliandi formas meditantur.  \pend\pstart Veritatem dico in Christo, non mentior.) Confirmatio est ueluti iuramento adhibito et interpretatio per contrarium. Doctorem gentium se uocat, simulut hanc doctrinam de reconciliatore Christo gentibus magis commendet, simul ut insinuet plenius Christum pro omnibus satisfecisse, non tantùm pro ludaeis, sed etiam pro gentibus: neque fastu ullo uel arrogantia, sed ministerium suum significat.  \pend\pstart Cum siue et vertate ) A sua side et integritate. Cum fide, inquit, non cum Syllogismis aut uana eloquentia, non ut fallam aut decipiam. Discant proinde hinc ministri uerbi, quibus rationibus doctrinam suam confirmare debeant, dentqueoperam, ut uere his uerbis uti queant, uidelicet, se ueritatem dicere in Christo, se docere in fide et ueritate. Hactenus, quod sit orandum pro omnibus.  \pend
\phantomsection
\addcontentsline{toc}{subsection}{\textit{Volo igitur orare viros, in omni loco, sustollentes puras manus absq ira et disceptatione. }}
\subsection*{\textit{Volo igitur orare viros, in omni loco, sustollentes puras manus absq ira et disceptatione. }}\pstart Addit nonnulla de orandi modo. Orare autem uult uiros omni loco. Nam orandum quidem tunc maxime, cum in coetum conueniunt, ubi orari, docert uerbum solet: sed non ibi tantùm, uerumetiam ubique  et quocunque  tempore. Itaque  excitat uiros ad orandi diligentiam et frequentiam: et praeterea superstitiones omnes de loco aut tempore orandi resecat. Orandum est omni loco, non solùm ubi omnes conueniunt publice, sed etiam domi. Est domus Dei domus orationis: sed nihilominus bene etiam erat, qui, ut habetur Matth.6. Petit secretum cubiculi sui, et ibi orat, humanam deuitans gloriam. Est etiam orandum foris, quia, ut Matth. 5. lucere debet lux nostra coram hominibus, ut uideant opera nostra bona, et glorificent patrem qui est in coelis: Sed sic orandum in propatulo, ut absit Pharisaica ostentatio: sie in cubieulo, ut animus foris non uagetur, sed ad solum Deum feratur. Superstitiosum autem erat, quod ut habet loan, 4. ludaei dicebant, orandum praecipue in Templo Hierosolymitano, Samaritani uero in montibus. Christus ubique  uult orari, modo in spiritu et ueritate oretur. Adsit spiritus et ueritas, nec multum refert, quo sis loco. Non locus orationem commendat, sed affectus fidei, id quod etiam uerba sequentia declarant.  \pend\pstart Sustollentes puras manus, abque  ira et disceptatione.) Metaphoris sumptis et a sacris deorum, in quibus omnia exterius munda esse debebant, et ab habitu corporis ueteribus inter orandum usitato, ostendit quomodo animos nostros ad orandum componi deceat. Vult sustolli manus: sic enim ueteres manibus eleuatis in altum orabant, id quod et picturae quaedam testantur. Et Exod.17. manus leuante et orante Mose uincit lsrael Amalechitas. Significatur ergo per manuum eleuationem eleuatio mentis: sicut et manibus  \pend
\vspace{0.5cm}\noindent
\vspace{0.2cm}\rule{1cm}{0.2pt}\\ 
\hspace{0.2cm}\textit{mg}
\footnotesize A ratione officii. 
\normalsize| 
\hspace{0.2cm}\textit{mg}
\footnotesize A contrario. Paulus doctor gentium. 
\normalsize| 
\hspace{0.2cm}\textit{mg}
\footnotesize Vbi orandum sit. 
\normalsize| 
\hspace{0.2cm}\textit{mg}
\footnotesize Onni loco orandum. 
\normalsize| 
\hspace{0.2cm}\textit{mg}
\footnotesize Quomode et domi et foris sit orandum. 
\normalsize| 
\hspace{0.2cm}\textit{mg}
\footnotesize In spiritu et uertate orandum. Metaphorae. 
\normalsize| 
\section*{IN I. EPIS T. D. PAVLI }
\marginpar{[ p.70 ]}\pstart sursum sublatis coelum demonstramus, Deum significamus. ltaqueut totus in Deum feratur animus per fidem, est necessarium, si exaudiri volumus. Sic Thren3. Leuemus corda nostra cum manibus in coelum. Debent praeterea manus esse purae, id est, nullis sordibus vitiorum inquinatae. Nam cum manibus multa peccata perpetrantur, vt furtum, caedes, rapina, vis illata, pugnae, et c. Cum exiguntur manus purae, exigitur ab omnibus sceleribus immunitas, ita vt Synecdochice per manuum puritatem intelligatur puritas totius hominis, praesertim animi siue cordis. Fide purificantur corda. Sic vero et impuritas manuum damnatur Esa.1.Cum extenderitis manus vestras, auertam oculos meos a vobis, et cum multiplicaueritis orationem, non exaudiam. Manus enim    vestrae sanguine plenae sunt. Sed vt plenius exprimat Apostolus, quid per manuum impuritatem intelligi queat, quasi per interpretationem adiicit, Absque ira et discgtatione. Ira praecipue manus polluit, et per eam omnes affectus vitiantur. Qui irato est animo, plerunque erumpit ad perpetranda scelera. Sed et ideo ira abesse debetab oraturo, quia oratio debet fieri tranquilla et quieta mente. Quies autem non contingit, vbi ira vel odium, vel appetitus vindictae habet locum. Huc pertinet illud Eccles.28. Homo homini seruat iram et a Deo quaerit medelam? Matth. 6. Si non remiseritis errata sua, nec pater vester remittet errata vestra. Item quod quotidie oramus, Remitte nobis debita nostra, sicut et nos remittimus debitoribus nostris. Disceptationem autem interpretantur nonnulli haesitantiam, qua videlicet quis secum disceptat, dubitans, num sit exaudiendus. Oportet ergo hominem oraturum non disceptare secum, non fluctuare: sicut lacobus docuit. Nam qui haesitat, is similis est fluctui maris, qui ventis agitur, et impetu rapitur. Neque existimet homo ille se quicquam accepturum a Domino. Habemus ergo hic non obscura documenta, quomodo nos ad orandum componere debeamus.  \pend
\phantomsection
\addcontentsline{toc}{subsection}{\textit{Consimiliter et mulieres in amictu modesto cum verecundia et castitate ornare semetipsas, non tortis crinibus, aut auro, aut margaritis, aut vestitu sumptuoso, sed quod decet mulieres profitentes pietatem per bona opera. }}
\subsection*{\textit{Consimiliter et mulieres in amictu modesto cum verecundia et castitate ornare semetipsas, non tortis crinibus, aut auro, aut margaritis, aut vestitu sumptuoso, sed quod decet mulieres profitentes pietatem per bona opera. }}\pstart Eadem sine dubio puritas affectuum requiritur a mulieribus quae a viris: sed est praeterea, cuius mulieres sunt admonendae: quomodo videlicet etiam in vestitu sese gerent, vbi ad ecclesiasticum coetum conuenerint. Consimiliter et mulieres, id est, orent, sustollentes puras manus, absque  ira et disceptatione.  \pend\pstart Sed praeterea volo mulieres ornare semetipsas in amictu modesto, cum verecundia et castitate.) Solent mulieres huic vitio esse obnoxiae, quod magnificentius comptae ambiunt prodire in publicum: qua ex re dum arroganter captant gloriam, et quemadmodum ille ait, videre cupiunt ac videri: sine dubio verecundiam deponunt, quae praecipuum est mulierum castarum ornamentum, et periclitatur earundem castitas. Quae enim uerecundiam exuerit, quaeque  uidere et uideri ambit, certe seipsam quodanmodo exponit aliis concupiscentibus expetendam. Vnde Ecclesiastici26. Fornicatio mulieris excellentia oculorum, et in palpebris illius agnoscetur. Et prouerbus 6. mulier ornatu meretricio describitur, garrula, uaga, quietis impatiens, quae apprehendens iuuenem deosculatur, ac procaci uultu blanditur, decipiens animam illius. Ecclesiastici 19. de uestitu. Amictus corporis et risus dentium, et ingressus hominis annunciant de illo. Itaque  non indecorum modo, sed saepem periculosum est, quod mulieres se uel supra morem patriae, uel supra facultates suas, uel secus omnino, quam decet matronas, praesertim profitentes pietatem, comunt exornantue. Debent ornatae esse amictu modesto, qui satis est, ut placeant suis uiris. Graece est, ἐν καταστολῇ κοσμίῳ. Significat autem καταστολὴ stolam quandam et amictum simplicem, qui undiquaque tegit ac uelat, ne quid sit conspicuum ceu fallax, ut Theophylactus annotauit. Cum humili animo sit orandum, nequaquam decet ueste ea uti, quae excitet fastum et superbiam. Et quae uenit in coetum, ut agat cum solo Deo, nequaquam debet in ueste prae se ferre, quo uidetur uelle agere cum quibusuis  \pend
\vspace{0.5cm}\noindent
\vspace{0.2cm}\rule{1cm}{0.2pt}\\ 
\hspace{0.2cm}\textit{mg}
\footnotesize Esa.1. 
\normalsize| 
\hspace{0.2cm}\textit{mg}
\footnotesize Irae effecta mala. 
\normalsize| 
\hspace{0.2cm}\textit{mg}
\footnotesize Iacobus 1. 
\normalsize| 
\hspace{0.2cm}\textit{mg}
\footnotesize Omamenti in muliebris sexus uerecundia est et castitas. 
\normalsize| 
\hspace{0.2cm}\textit{mg}
\footnotesize Qualis esse debeat mundicies muliebris. 
\normalsize| 
\section*{AD TIMOTHEVM CAP. II. }
\marginpar{[ p.71 ]}\pstart Hiris. Certem quae ueste operosa excedit reliquas, et suam ipsius conscientiam grauat superbia, et aliorum animos irritat ad ministerium libidinis. In summa, deformat matronam quicquid laedit verecundiam aut suspectam reddit castitatem. Nam etiam suspicionem omnem euitare consultum est. Sed quid sit amictus modestus, Apostolus per contrarium voluit explicare, dicens:  \pend\pstart Non tortis crinibus, aut auro, aut margaritis, aut vestitu sumptuoso.) Cum orandi gratia adeatur locus orationis, nequaquam se sic component, quasi adirent nuptias aur theatrum, aut lupanar, caussa saltandi aut libidinandi. Vestitum magno sumptu paratum merito damnat Apostolus, quem etiam damnarunt leges Vestiariae quorundam ethnicorum principum. Nam et Suetonius in Caesare scribit, qu conchiliatam vestem et margaritas, nisi certis personis et aetatibus, perquecertos dies, permisit. Sed multo iustissime damnantur etiam illa superuacanea et operosa, cuiusmodi est torquere crines, et id genus quae non nisi ad lasciuiam sunt excogitata. ldeoque  Petrus i. cap.3. Vxorum ornatus sit non externus, qui situs est in plicatura capillorum et additione auri, aut in palliorum amictu, verum oecultus, qui est in corde homo, si is caret omni corruptela, ita vt spiritus pla cidus sit ac quietus, qui spiritus in oculis Dei magnifica et sumptuosa res est. Nam ad eum modum olim et sanctae illae mulieres sperantes in Deo, comebant sese et subditae erant suis uiris. Sed de omni hac re copiosissime Tertullianus, qui unum librum conseripsit de habitu mulierum, in quo potissimum ornamentorum et monilium usum reprehendit: alterum, de cultu foeminarum in quo fucum, lenocinium et nitorem operosum reprobat: EundemqueCyprianus imiratus est libro de habitu uirginum.  \pend\pstart Sequuntur ouo racita argumenta, quibus ad modestiam uestitus hortatur. Sed quod decet mulieres: Argumentum a Decoro. Ideo contentas esse decet uestitu simpliciore, quia id decer, maxime eas quae conueniunt ad orandum. Non conuenit matronis euntibus ad locum orationis idem habitus, qui uix conueniat adeuntibus nuptias aut theatrum: sed omnium minime decebit Profitentes pietatem per bona opera: Argumentum ex circumstantia a conditione religionis. Vt talis sumptuosus habitus alias mulieres fortasse deceat, at nequaquam decet eas, quae profitentur religionem Christianam, quae pietatem suam testari debent per bona opera, id est, in omnibus moribus externis, non. solum dictis et factis, sed etiam ipso uestitus genere: Ita namque  religionem suam atque  innocentiam testari debent Christiani, in omni uita, in dictis, factis, uestitu, incessu, denique  in omnibus rebus, quae ad aedificationem seu bonum exemplum aliorum queant fieri.  \pend
\phantomsection
\addcontentsline{toc}{subsection}{\textit{Mulier in silentio discat cum omni subiectione. Ceterùm mulieri docere non permitto, neq autoritatem vsurpare in viros, sed esse in silentio. }}
\subsection*{\textit{Mulier in silentio discat cum omni subiectione. Ceterùm mulieri docere non permitto, neq autoritatem vsurpare in viros, sed esse in silentio. }}\pstart Quoniam hic praescripsit, quomodo in coetu ecclessastico mulier sese geret in uestitu, inter orandum, adiicit per occasionem quomodo se habebunt in coetu ecclesiastico in discendo:atque haec ordinatio ualde quoque necessaria est et utilis. Proinde uult mulieres in coetu ecclesiastico, cum docctur, nihil aliud quam humiliter silere et benignis auribus amplecti quod proponitur: non aliqua a doctoribus inquirere, etiam si fortassis sit de quo dubitent, multo minus ferendum ut ipsae doceant. Clarius explicat̃ haec ordinatio siuae εὐταξία 1. ad Cor.14. ubi uiris quidem permittit singulatim omnibus prophetare, ut omnes discant, et omnes consolationem accipiant: Sed seruato interim ordine, et ne plures simul loquantur: At mulieres uult prorsus silere: Mulieres vestrae in ecclesits sileant: inquit: nec enim permissum est illis ut loquantur, sed ut subditae sint quemadmodum et lex dicit. Quond si quid discere uolunt, domi suos uiros interrogent. Nam turpe est mulieribus in coetu loqui. En quod, etiam si quid discere uellent, ipsarum tamen non est in ecclesia interrogare doctores. Caussam addit, indecorum esse mulierem quoquo pacto loqui in coetu, Est enim superbum animal mulier, et sicut ex uestitu gloriam inanem captat, ita multo magis captaret gloriam ex quaestionibus. Et praeterquam quod fortassis parum conuenientia interdum proponerent, tum doctorum contemptus inde nasceretur, praesertim si absurdis quaestionibus muliercularum pro palato omniu non satisfaceret. Sed potuisset eadem res ridicula uideri ethnicis, si permissum  \pend
\vspace{0.5cm}\noindent
\vspace{0.2cm}\rule{1cm}{0.2pt}\\ 
\hspace{0.2cm}\textit{mg}
\footnotesize A decoro. 
\normalsize| 
\hspace{0.2cm}\textit{mg}
\footnotesize A circumstantia et condinone religionis. 
\normalsize| 
\hspace{0.2cm}\textit{mg}
\footnotesize Quomodo mulieres sese gerent in coetu ecclesiastico in docendo. 
\normalsize| 
\hspace{0.2cm}\textit{mg}
\footnotesize Mulierin in ecclesia silaet. 
\normalsize| 
\hspace{0.2cm}\textit{mg}
\footnotesize A turpi siue in decoro. Mulier animal seperbum. 
\normalsize| 
\section*{IN I. EPIST. D. PAVLI }
\marginpar{[ p.72 ]}\pstart fuisset mulieribus Christianorum in coetu loqui aut disputare. Minimem itaque  decet mulierem aliquid curiose aut fastuose inquirere, cum deceat ipsam in omnibus humilitatem prae se ferre. Ideo addidit Apostolus, discat, cum omni subiectione: Nequaquam se geret vt subiectam, quae in publico conuentu cupit videri vel audiri loquens vel disputans. lam si ne interrogare quidem licet mulieri aliquid in coetu, iuxta Apostoli praescriptum, sed potius domi debet discere ex marito:at multo minus permittendum est, vt in ecclesia doceat: Vnde sequitur,  \pend\pstart Caeterùm mulieri docere non permitto, neg autoritatem vsurpare in viros, sed esse in silentio.) Vnum inconueniens ex altero sequeretur. Si mulier permittatur docere. et praescribere etiam viris credendi et viuendi normam: lam plane sequitur, quod mulier arroganter vsurpabit autoritatem in viros: quod valde est absurdum et prorsus contra nasturae ordinem. Nam mulieri dictum est a Deo Genes.3. Sub viri porestate eris, et ipse dominabitur tibi. Quod praeceptum cùm de obedientia mulieris etiam in priuatis rebus sit intelligendum: nequaquam ferendum erit, vt illa aliquid pro suo arbitrio constituat in rebus publicis siue in coetu ecclesiastico. Si non potest suo nutu aliquid agere domi in causis priuatis, multon minus id licebit ei in ecclesia in negociis religionis. Est aunt turpissimum si qui viri tam degenerant, qui sibi permittant vxores licenter imperare, et videntur in naturam ac Dei ordinationem peccare. Eccles.25. Mulier si primatum habeat, contrario est viro suo. Tametsi non abutentur viri sua hac potestate sibi diuinitus concessa. Aliquid etiam concedent mulieri intra parietes. Nam et docere domi, mulieri non est negatum, siquidem Priscilla Actor.18. vna cum marito Aquila, docuit quaedam Apollon Alexandrinum. Et alibi fidelis mulier iubetur virum infidelem commonefacere, et si possit, Christo lucrifacere. Et exigitur ab omnibus matribus  familias, vt prima elementa seu Catechismum doceant suos liberos. Imo aliquando permittit̃ in aliqua causa mulieri etiam publicem adferre sanum, si quod habeat, consilium, tametsi non temere. ludic.5. Debora erudiuit populum lsrael. Non autem quond publice doceret, sed dedit quibusdam in rebus consilium, quoniam afflata erat Spiritu sancto, qui interdum et per sexum fragilem dignatur res magnas efficere, et occulta vel admiranda indicare. Vnde et Ioel.2. legitur: Et prophetabunt filii vestri et siliae vestrae. Et Maria virgo cecinit spiritu Sancto afflata canticum, Magnificat:Item Actor.21. Philippi quatuor filiae uirgines prophetabant: Rursus 1.Corinth.11. Mulier orans aut prophetans iubeturvelare caput. Sed tamen ex his non sequitur, quod publice docere permittentur, cur autem non sit istud ferendum, sequitur:  \pend
\phantomsection
\addcontentsline{toc}{subsection}{\textit{Adam enim prior formatus est, deinde Eua: et Adam non fuit deceptus sed mulier seducta fuit per praeuaricationem, }}
\subsection*{\textit{Adam enim prior formatus est, deinde Eua: et Adam non fuit deceptus sed mulier seducta fuit per praeuaricationem, }}\pstart Duo argumenta, cur mulieri non sit permittendum, vt doceat uel usurpet autoritatem in uirum. Prius est ex ratione creationis. Aequum est, ei praeeminentiam concedere in docendo et imperando, qui primùm est a Deo conditus. Sed Adam formatus est prior Eua. lgitur debetur ei praeeminentia et imperium. Intelligendum autem est, idem obseruari debere ius dignitatis in posteris, quod contigit primis parentibus. Et per Adamum conuenienter intelliguntur omnes uiri, per Euam omnes mulieres. Quod autem prior conditus fuerit Adam, Geneseos liber clare docet. Alterum argumentum est a ratione lapsus, negatiue. Illi non est permittendum munus docendi, qui ruinae caussa extiterit. Sed mulier extitit caussa ruinae. Igitur ei non est permittendum docendi munus. Significanter dicit mulserem seductam, non Euam, quia id malum in uniuersum genus mulierum debet redundare. Nequaquam autem tutum, ei credere, quae semel in praeeundo sic decepit. Nec est, quod ambiat docendi aut imperandi munus, quae in ipso exordio, ad leuissimam persuasionem impegit et in errorem induxit. Quomodo autem mulier prius assenserit fallaci serpenti, et prius fuerit praeuaricata gustando fructum ueti» tum, uide Genes.3. Sic ibi legitur: Vidit mulier, quod bonum esset lignum ad uescen» dum et pulchrum oculis aspectuquedelectabile, et tulit de fructu illius et comedit: de„ ditqueuiro suo, qui comedit. Et post pauca, Adam dicit, Mulier qua dedisti mihi sociam.  \pend
\vspace{0.5cm}\noindent
\vspace{0.2cm}\rule{1cm}{0.2pt}\\ 
\hspace{0.2cm}\textit{mg}
\footnotesize Ab inconue nienti. 
\normalsize| 
\hspace{0.2cm}\textit{mg}
\footnotesize A minori. 
\normalsize| 
\hspace{0.2cm}\textit{mg}
\footnotesize Ab exemplo. 1.Corin.7. 
\normalsize| 
\hspace{0.2cm}\textit{mg}
\footnotesize 1.Arg. A raione creationis siue ordine naturae. Synecdoche. 2.arg.a ratone lapsus. 
\normalsize| 
\section*{AD TIMOTHEVM CAP. II. }
\marginpar{[ p.73 ]}\pstart dedit mihi de ligno, et comedi. Et mulier. Serpens decepit me, et comedi. En ibi se priorem et seductam mulier confitetur. Ratio ergo est euidens, cur mulieri non sit permit tenda facultas docendi. Nunc viderint qui committunt mulieribus gubernandas respublicas et amplas ditiones: Sine dubio ira Dei est manifesta, cum mulieres alicubi admitrant. Amazones quidem maudito exeipio tenuerut regnu, ieu tantum inter mulieres, in regno viris vacuo. Neque  exemplum id est probatum, neque  diuturnum fuit earum regnum.  \pend
\phantomsection
\addcontentsline{toc}{subsection}{\textit{Salua tamen fiet per generationem liberorum, si manserint in fide ac dilectione, et sanctificatione, cum castitate. }}
\subsection*{\textit{Salua tamen fiet per generationem liberorum, si manserint in fide ac dilectione, et sanctificatione, cum castitate. }}\pstart Emollit, si quid fortasse durius dixisse uideatur: et ostendit aliud mulierum esse officium, nempe quod educare et docere debeant suos liberos et sic si dicantur caussa ruinae extitisse, saluae esse queant etiam: quasi dicat: Non sit vobis molestum, ô mulieres, si vobis cum sitis fortassis ingenio pollentes, negatur tamen munus docendi. Potestis alia ratione vos exercere in pietatis operibus ad salutem vestram, et si ita vultis, etiam docere. Etucate et docete diligenter, tum rudimenta religionis, tum caetera quae pertinent ad morum honestatem, vestros liberos, in his docendis operam impendite, et haec opera erit vobis salutaris, maxime sivos ipsae semper in fide et charitate perstantes sancte et caste vitam transigatis. Haec conditio vere honorifica est, et satis habens in se dignitatis, etiamsi publice docendi in ecclesia ius vobis adimatur. Et recte adiecit ista Apostolus, post inustam muliebri sexui notam, de caussa ruinae humani genenis. Quamuis mulier fuerit occasio damnationis, tamen potest illa quoque  esse occasio conseruationis. Ipsa inprimis potest salua fieri, si manserit in fide et dilectione, et aliquo modo dici caussa salutis filiorum, quos ipsa bene instruxerit, et instrus ad gloriam Dei rite curauerit. Sicut sine dubio multum iuuit Timotheum, quod recte ab infantia, procurante matre, in sacris literis fuerat institutus. Itaque hic cum ait, per generationem liberorum, intelligi conuenit, quicquid ad puerorum generationem et educationem pertinet, nempe partus dolores, labores in educando, diligentiam et curam in formanda infantia et pueritia, quae sine dubio laboriosa sunt, et Deo inprimis grata officia, maxime si ex animo et longanimiter a matribus perficiantur. Hinc discimus, non satis esse, si parentes curent, vrliberi viuant, sed oportet eos curare, vt bene viuant, vt in fide et timore Domini etiam a teneris vnguiculis instituantur. Eccles.7. filii tibi sunt, erudi illos, et incurua illos a pueritia illorum.  \pend\pstart St manserint in fide ac dilectione et sanctificatione cum castitate.) Graece legitur, ἔαν μένωσιν, plurali numero, si manserint. ἑτέρωσις numeri per hebraismum, qui est Paulo frequens, vt hic numerus respiciat ad id, quod principio habebatur. Consimiliter et mulieres. Quidam vitantes hebraismum malunt legere in singulari numero. Perspicuum autem est, verbum istud referendum ad mulieres, non ad liberos. Quis enim sensus fuerit: Mulieres saluas fore, si liben manserint in fide et dilectione: quasi vero in potestate mulieris esset, efficere, vt tales maneant, aut ipsa alioquin seruari non possit, nisi illi integri in fide perstent? Significatur itaque mulieres ipsas saluas fore, non quidem ratione educationis prolium, quasi ea re salutem promererentur, sed (quod per Correctionem uidetur addi) si quoque  diligentiam omnem expenderint in educandis prolibus, manserint in fide et dilectione, id est, si a fide et Charitate non excidant Et si manserint in sanctificatione, id est, ea praestent, per quae eas sanctas esse agnosci queat: denique  cum castitate, id est, si uerbis, factis, denique  uniuersa uita sobrietatem et castitatem, et modestiam prae se ferant. Proinde hic locus est admodum consolatorius, praesertim iis mulieribus, quae arbitrantur, se prolium turba grauari. Scire debent, suam conditionem Deo probari, maxime si diligentes sint in liberorum suorum honesta educatione. Et matronarum singulare decus esse liberos, constat ex Corneliae Gracchorum iudicio. Apud hanc enim cum Campana quedam matrona diuertisset, quae ostendit ei monilia sua, qualia illo seculo maximi fiebant, illa remorata est Campanam sermone, donec liberi e schola redirent. Tum, haec, inquit, ornamenta mea sunt: sentiens matronae nihil esse pretiosius, quam liberos recte educatos. Item com  \pend
\vspace{0.5cm}\noindent
\vspace{0.2cm}\rule{1cm}{0.2pt}\\ 
\hspace{0.2cm}\textit{mg}
\footnotesize vbi  et quos docere debeuut mulieres. 
\normalsize| 
\hspace{0.2cm}\textit{mg}
\footnotesize Mulier quomodo saluari queat. 
\normalsize| 
\hspace{0.2cm}\textit{mg}
\footnotesize ἑτέρωσις numeri per Hebraismum. 
\normalsize| 
\hspace{0.2cm}\textit{mg}
\footnotesize Correctio. 
\normalsize| 
\hspace{0.2cm}\textit{mg}
\footnotesize 1. Thess. 4. Haec est uoluntas Dei, Sanctificatio uestra, etc. 
\normalsize| 
\hspace{0.2cm}\textit{mg}
\footnotesize Liberis bene educatis mulieri nihil pretiosius, nihilquelaudabilius. 
\normalsize| 
\section*{IN I. EPIST. D. PAVLI }
\marginpar{[ p.74 ]}\pstart mendantur hic praecipua ornamenta matronarum, per quae salutem queant consequi. fides, inquam, et charitas, quae in omnibus requiruntur: Deinde sanctificatio et castitas, quae tanquam monilia sunt et decus praecipuum omnium matronarum. Ethaec quidem de doctrina τῆς εὐταξιας  breuiter delibata.  \pend
\phantomsection
\addcontentsline{toc}{subsection}{\textit{Posterior pars Epistolae, in qua ostendit qualis episcoporum vita esse debeat, ac primo quomodo se gerent priuatim in domo sua et erga familiares. }}
\subsection*{\textit{Posterior pars Epistolae, in qua ostendit qualis episcoporum vita esse debeat, ac primo quomodo se gerent priuatim in domo sua et erga familiares. }}
\endnumbering\beginnumbering\section{CAPVT III ❧}
\phantomsection
\addcontentsline{toc}{subsection}{\textit{\huge\textbf{I}\normalsize NDVBITATVS sermo. Si quis episcopi munus appetit, honestum opus desiderat. Oportet igitur episcopum irreprehensibilem esse, vnius vxoris maritum, vigilantem, sobrium, etc. usque ad cap.4. }}
\subsection*{\textit{\huge\textbf{I}\normalsize NDVBITATVS sermo. Si quis episcopi munus appetit, honestum opus desiderat. Oportet igitur episcopum irreprehensibilem esse, vnius vxoris maritum, vigilantem, sobrium, etc. usque ad cap.4. }}\pstart \huge\textbf{Q}\normalsize Vemadmodum in argumento diximus, distribuitur haec epistola in duas partes principales: quarum prior de doctrina episcoporum agit, posterior de vita. De hac incipit apostolus in praesenti disserere. Cum autem in vita episcopi spectandum sit, primo quomodo se gerat priuatim in domo sua et erga familiares: Deinde vero quomodo publice et in ecclesia erga cuiuscunque  ordinis homines: hoc loco depingit, qualem episcopum domi suae esse deceat, et quibus virtutibus ornatus conspiei debeat.  \pend\pstart Indubitatus sermo.) Statim autem, et velut lege Attica ἄνευ προοιμίου καὶ πάθος rem aggreditur, vti saepe alias. Indubitatus sermo: quasi dicat: Quae nunc dicam, certa acrata habeantur, fide digna sunt. Sic enim graece πιστός ὅ λόγος. Vno itaque  uerbo praefatur et attentos reddit. Nonnulli existimant ueluti per occupationem praefixum esse. Nam cum diuersi munus episcopi appeterent, atque  alii quidem spe quaestus aut gloriae, alii sine dubio synceriter, et ad promouendam pietatem, atque  illi nihilominus praetexerent honestatem perinde atque  isti : videtur Apostolus se velut arbitrum voluisse huic quaestioni interponere, dicens: Indubitatus sermo: quasi diceret: Dicam quodres est: Si quis episcopi munus appetit, honestum opus desiderat: Si tamen is talis esse velit, qualem in ea functione esse oportet. Theophylactus autem secutus suum Chrysostomum, haec verba, epilogi vice adiungit praecedentibus quae dicta sunt de officio mulieris, vt sit sensus: Ea quae dicta sunt de mulieribus per generationem et educationem liberorum saluandis, uera esse, indubitata, fide digna, ac minime de illis dubitandum. Magis consentaneum videtur, connectenda iis, quae sequuntur de vita et moribus episcoporum, et ad renouandam attentionem exordii loco praefixa. Proinde sic definit Apostolus.  \pend\pstart Si quis episcopi munus appetit, honeslum opus desiderat.) Quoniam episcopi munus recte administratum, per se bonum est et gratum Deo, definit Apostolus eum bene agere, qui id munus appetit, ea videlicet intentione vt recte administret. Obseruanda autem vocis proprietas ἐπισκοπὴ, ita enim graecem vna vox est, vbi nos habemus, episcopi munus. Significat igitur ἐπισκοπὰ diligentem inspectionem, speculationem, obseruationem, visitationem: a verbo ἐπισκοπὲω, quod est considero, superintendo, recenseo, recognosco, etiam uiso aegrotum, uel, ut uulgo dicunt, uisito. Vnde ἐπίσκοπος obseruator, speculator, explorator, custos, uisitator. Et erant episcopi dicti quidam magistratus apud Athenienses et Lacedaemonios, quos et αομοςάς uocabant. Praeerant ii reficiendis seruandisquearcibus et publicis aedificiis. Etiam iudices in publicis certaminibus dicti sunt episcopi. Hinc nomen episcoporum accommodatum est iis, qui praesunt ecelesiis, quorum interest iuxra nominis significationem, considerare, superintendere, uigilare, uisitare, denique  omnibus modis prospicere et cauere, ne quid gregi credentium incommodi in doctrina fidei uel morum accidat. Est episcopus magis officii.  \pend
\vspace{0.5cm}\noindent
\vspace{0.2cm}\rule{1cm}{0.2pt}\\ 
\hspace{0.2cm}\textit{mg}
\footnotesize Deuita episcopi. 
\normalsize| 
\hspace{0.2cm}\textit{mg}
\footnotesize Auentio. 
\normalsize| 
\hspace{0.2cm}\textit{mg}
\footnotesize ἐπισκοπή quid. 
\normalsize| 
\hspace{0.2cm}\textit{mg}
\footnotesize ἐπίσκοπος. 
\normalsize| 
\hspace{0.2cm}\textit{mg}
\footnotesize Episcopis Atheniensium et Lacedaemoniorum. Episcopi munus. 
\normalsize| 
\endnumbering
\end{pages}
\end{document}
        
