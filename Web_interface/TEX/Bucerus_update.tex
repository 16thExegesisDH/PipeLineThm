% !TeX TS-program = lualatex
\documentclass{article}
\usepackage[T1]{fontenc}
\usepackage{microtype}% Pour l'ajustement de la mise en page
\usepackage[pdfusetitle,hidelinks]{hyperref}
\usepackage[english]{french} % Pour les règles typographiques du français
\usepackage{polyglossia}
\setotherlanguage{greek}
\usepackage[series={},nocritical,noend,noeledsec,nofamiliar,noledgroup]{reledmac}
\usepackage{reledpar} % Package pour l'édition

\usepackage{fontspec} % 
\setmainfont{Liberation Serif}

\usepackage{sectsty}
\usepackage{xcolor}

% Redefine \section font
\sectionfont{\normalfont\scshape\color{gray}}


\begin{document}

\date{}
        \title{ Epistola ad Ephesios : [Bucerus, Martinus], [1527]}
\maketitle

\begin{pages} 
\beginnumbering
        ETISTOLA D. PAVLI AD EPHESIOS, LVA rationem Christianismi breuiter iuxta et locuplete, ut nulla breuius simul et locu pletius explicat, uersa paulo liberius, ne peregrint idiotismi rudiores scripturarum offenderent, bona tamen fide, sententiis Apostoli appensis. In tundei comuenarloPER MARTIRUH DVCERVMI, 
\section*{ARGVMENTVM.  }
\textit{ILLVSTRIS SIMOET PIENTISSIMO PRINcipi.  Fridericho, Schlesiorum et Ligniciae Duci etc. Martinus Bucerus salutem Christi precatur. }
\textbf{P }\pstart Rouidentia Dei comparatum est, clarissime et religiosissime Princeps, ut quam docunque hactenus sincera Dei scientia in publicum prodierit, ilico et errorum uenti, undique ingruerint, qui si deturbare e medio illam nequirent, tantum pulueris tamen opinionum terrena resipientium, atque inde dissidiorum excitarint, ut per paucis, hoc est, electis duntaxat, atque ex his multis nequaquam ἀνιδρωτι, illa rite conspecta fuerit. Visum namque Deo soli bono ac sapienti est, bonitatis ac uirtutis suae gloriam, contra luctantibus potestatibus aduersariis, et ex infirmitate suorum illustrem reddere. Hinc sanctis non cum sanguine et carne, humano scilicet hoste, sed ad uersus caelestes uersutias, mundi dominos, praeter bellum, quod caro ipsorum, aduersus spiritum citra inducias gerit, non minus saeuum et atrox, quamanceps et periculosum, perpetuo depugnandum est.   \pendProdeunte ueritate in furgunt et errores. 
\section*{EPISTOLA. }\pstart Quapropter, si idem hodie, reuecta nobis Euangelii lampade, usu ueniat, ut immissis multiplicium haereseon, atque imposturarum turbinibus, mendacii author, qui dum atrium suum humanis traditionibus satis munitum tranquille possedit, egit quietius, dum pellere rursus mundo illam nequit, obscurare tamen adeo conetur, ut a paucis plane agnosci, a nonnullis ne uideri quidem possit, par erit, ut memores, sic bonam dei uoluntatem decreuisse, atque quemad modum antea, ita modo quoque id ita temperaturam, ut inde cum uirtus tum bonitas sua clarior reddatur, boni consili um patris nostri super nos, ubique quae saluti nobis sint pro spicientis, consulamus, et ex animo, etiam pro hac re grati as ipsi agamus. Et enim si recte rationem subduxerimus, haud parum, animos nostros confirmabit etiam, quod quam Eu angelii ueritatem profitemur, princeps tenebrarum ad eun dem modum excipit, quo olim ab ipso seruatore, ciusque Apo stolis praedicatam excepit.  \pend\pstart Quam mox eni praeter grauissimas psecutionum procellas, innumeris insanorum dogmatum fluctibus, quorum ab surditatem nescias, an uarietatem atque inter se pugnantiam magis admireris, cymbam Euangelii ille quatiebat iactabatque. Idque haudquaquam sine successu, quippe breui a Pau lo, quo nemo unquam maiore uel sedulitate, uel dexterita te, Christum praedicauit, neque placere ad bonum omnibus quisquam magis studuit, ut qui omnibus   omnia fieret, ut uel aliquos lucraretur, omnes Asianos abalienauerat. Timotheo siquidem suo sic quaestum legimus. Nosti hoc, quod auersati fuerint me omnes, qui sunt in Asia, id est, indubie plurimi.   \pendQuo successu contra Pauli Euam gel. Satan saeuierit.  2 Thess Confirma re, non deii cere debet, quod Euangelion nostrum ua riis errori bus oppugnatur. 
\section*{MARTINI BVCERI }\pstart Vt autem obturbantibus   uariis errorum sectis, animos nequaquam despondere, sed assumere potius debemus, quod cum ita Domino uisum sit, oporteat, id non minus ad promouendam nostram salutem, quam ipsius gloriam illu strandam facere, qui probatos suos, inter huiusmodi turbines, et utiliter exercet, et magnifice prodit, ita conuenit nihilominus illis pro uiribus spiritus, cuique diuinitus dona ti, obuiamire, dareque operam, ut luce scripturarum, figmento rum humanae rationis tenebras, dispellamus, quod nos qui Christi sumus deceat, haudquaquam minore studio, cum hoc nostrocapite colligere, quam aduersarius soleat disper gere. Cumque ueterator iste, a contemplatione solis diuinae bonitatis, unde omnis cum iustitia, tum salus certo gignitur, modis omnibus, ad suspiciendum nihili, ac euanidum humanae industriae uirtutisq́ splendorem, unde nihil nisi fastidiosa hypocrisis, et certa pernities indubitato nascitur, animos mortalium detorquere conatur, nobis praecipue in hoc incumbendum est, ut fide Euangelii hominibus oculi corroborentur, quouiuificos solis illis diuinae beneuolentiae radios, rite percipere queant.  \pend\pstart Quemadmodum enim Paulo benignitatem Dei omni studio attollenti, Satanae satellites praedicando circumcisionem, Sabbatha et idgenus externa, ad quae hominumuires possunt pertingere, negocium facessebant, sic postea adserentibus candem, sanctis Dei confessoribus, et liberae electioni eius, bonitatiqué omnia accepta ferenti bus, hostis ille ueritatis, similibus instructos armis suos ohiecit, qui libero arbitrio, nostrisisqué operibus nihil non  \pendQuo studio obuiai dum errdri bus. Quib-armis super. orib saecu lis oppugnata sit ueritas. 
\section*{EPISTOLA}\pstart tribuerent. Donec Mahometh sua figmenta, operibus mixta legis, et nostri, sua iuncta in speciem Euangelicis, orbi obtruscrunt, eoque rem perduxerunt, ut nihil fere fidu tiae in bonitatem Dei, omnis uero spes, in conficta ab hominibus opera, eaque nullam frugem proximis adferentia, cuiusmodi bona opera Deus requirit, tantum non ab uniuerso mundo collocaretur.  \pend\pstart  Podemconjmo, reddito dudut nobls Ellangello, ex quo iam orbis discere incipit, nos natura filios irae, et nihil bo ni a nobis uelcogitare posse, Dei autem, qui ante conditum mundum nos sibi in filios elegit, gratuita beneuolentia spi£ ritum sanguine CHRISTI promeritum donari, quo de ipsius bonitate, ut bene speramus, ita et studio ipsius, ad honesta sanctaque rapimur, praecipue autem, ut proximis commodemus, neque dubitamus, hac uelut arra obsignati in redemptionem, olim nos Christi beneficio, omnis peccati puros, et aeternae felicitatis plene compotes fore, salutis nostrae inimicus, rursum excitat, qui ad externa opera, ea que non solum infrugifera, sedet noxia, a fidutia benignitatis diuinae, acriore, quam ulli antea studio, reuocare laborant. Baptismum siquidem hi, imo rebaptismum supra quam dici queat, urgent, sed quo te ab omnibus, qui illum reiiciunt, quamlibet sanctis Christi membris separes, atque ne magistratui, quem et negant posse Christianumesse, iurando, et arma pro re publica sumendo obtempares, adstrin gas. Hi praedestinationis et electionis Dei certitudinem rident, et in impletione legis, sed ad ipsoruminterpretationem, saepe absurdam, ad quam uires quoque adesse  \pendQuiu doceat Euan gelionErrores quidam Catabaptista rum. 
\section*{MARTINI SVCERI }\pstart omnibus hominibus adseruerant, spem figere docent Hinc beneuolentiae Dei praedicationem, hoc est, Euangelion, efficere blasphemant, ut homines a bonis operibus cessent, salutem a merito Christi frustra expectantes, cum tamen, nullum prorsus a nobis bonum opus sieri possit, nisi id sensus amorq bonitatis Dei gignat. Sed non hic tantum nobis isti impingunt, quod ipsi peccant.  Pecc. tum ipsi falsam et euanidam opinionem faciunt, quia non sit creatura D E I coque olim etiam prorsus euant turum, tumque saluos fore, et daemones et impios uni uersos, quem hoc figmentum, qui id receperit, non reddat socordem, bonorumque operum negligentem, cum etiam impio sibi, aditum ad beatem uitam fore, et si pau serius putet?  \pend\pstart Istis profecto u̶su uenit, quod et Iudaeis, mundique sapi entibus olim. Suam quaerunt istitiam statuere, igitur DE iustitiae quae per fidem est, subdi nequeunt, et indulgentes fidentesque ingeniis suis, sapientes esse uolunt, ideo plus nimio stulti fiunt. Verum dum insolentem prae se graut tatem, rerumque uitae praesentis contemptum, praeser tim apud cos, qui fermentum eorum nondum recaepe runt, prae se ferunt, minime paucis, quibus zelus aliquis est, sed scientia carens, imponunt. Nam ingeniun numanum praecipue admirari solet, quicquid a uulgan ratione uiuendi abhorret, etiamsi ne pilum quidem bo nae frugis adferat, qua certe occasione, Romanenses ab dicantes connubium, et Mahomethani uinum, et Me nachi usum carnium, ac rerum quarundam aliarum, it  \pendQuibus   a tibus mui do solcat impont.  
\section*{EPISTOLA. }\pstart quarum tamen usu ut nulla est impuritas, ita nulla in absti nentia earum sanctitas, totum fere mundum dementarunt. Porro ut istorum impostura serpat, non paruum mo mentum adfert, eorumqui CHRIS T Iesseuolunt infirmitas ne dicam, an socordia et detestanda leuitas. Dum enim simpliciores, nondum scilicet a spiritu intus satis roborati, uident tam imparem doctrina Euangelica uitam apud eos, qui illam profitentur, minimo nego tio ab iis, qui speciem aliquam eius prae se ferunt, persuadentur, non esse eam uere Euangelicam doctrinam, quam illi praedicant.  Vae igitur nobis, si blasphemandi doctrinam CHRISTI, et offensionis pusillorumoc casionem dederimus.  \pend\pstart Praeter hos, sunt et ex iis, qui Euangelii purissimiuin dices sibi uideri uolunt, neq sane uulgariter in prouchen do illo, sudarunt, qui nescio qua animis ipsorumobfusa, contentionis et φιλαυτίας incertum, an stuporis et ignorantiae nubecula, Sacramentis, quae symbola sum Christianorum, et externo uerbo, ea tribuunt, unde di uinae beneuolentiae, uirtutisqué claritas non parum obscuratur. Vehicula enim illa faciunt, spiritus sancti, et fidei, cum haec donet sola bonitas Dei, impetrata morte CHRI STI. Hi ex eadem cognitionis gratiae Dei angustia, et sanctorum salutem aeternam negant, dum hic, qui Christouere crediderint, rursum posse excidere, non tam inconsiderate, quam periculose affirmant. Quae enimfides, quae se dubitet habere repositam sibi penes Deumuitam aetcrnam, hoc est, esse filium atque haeredem Dei ?am  \pendSacramen is et exter io uerbo quidam ni nium tribuunt
\section*{MARTINI BVCERI }\pstart DE V Ssuis, pro fide ipsorum facit.  \pend\pstart Proinde cum tam multis errorum nubibus Christi 4 uerfarius, contra restitutam nobis illius lucem Euange licam, perstrenue adhuc pugnet, nouosque cottidie excitet, abusus et sanctorum ignorantia et infirmitate, sit denique eius in omnibus idem conatus, nimirum, ut opera manuum suarum homines adorent, et non ut deploratos peccatores, diuinae bencuolentiae, sanguiniqué CHRISTI curandos plena fide sese contradant, debent profecto, quibus Christi gloria, electorumque salus chara existit, nihil uicissimomittere, quoregniC HRI ST I Euangelion, et bonitatis D E I praedicatio, quam in primis uniucrsa scripturaubiq in culcat, orbis magis ac magis agnoscat.  Vere enim sordebunt nobis nostra, ubi diuinae bonitatis gustum recte percaeperimus, neque cessabimus tum a bonis operibus, sed gnauiter demum illis incumbemus, toti scilicet eo rapti, ut gratificaripatri nostro studeamus.  \pend\pstart Cum ucro Paulus in praedicanda D E I praedestinatione, electione, amplissimaque in nos bonitate et efficacia sanguinis CHRISTI, hoc est, in annunciando syncero Euamgelio, docendaqué doctrina certae salutis, ita excelluerit, ut non paruo interuallo, alios sacros scriptores post se reliquerit, praesertim si lucem, et copiam spectes, optandum est, ut illius Epistolae quam familiarissimae Christianis omnibus reddantur. Qui sane eas recte teneret, cum sibi tum aliis, quoslibet facile errores auerteret.  \pendPaulus cla rius et copiosius Eu amgelion prae dicauit. 
\section*{EPISTOLA. }\pstart Equidem igitur ut huc fratribus scripturarum rudioribus   adiutor essem, quibus et V. Fabus  Capito paratis nuper in Hoseam haudquaquam uulgaribus conmentariis, et MCellarius opere, de Operibus Dei in quo diuina bonitas magnifice praedicatur, plurimum consuluerunt, Epistolam illus ad Ephesios, hisce diebus enarrandam mihi desumad Ephespsi* Visumenim mihi est, doctrinam Christi, breui-iotius scri ter iuxta et luculenter, atque copiose adeo complexa esse, ut pturae conhis nominibus, nullam aliam ei praeferendam existimem, totiusque pendium. sacrae scientiae, compendium recte habendum censeam.  \pend\pstart Docet quidem fusius etscripturis conmunitius, ea quae Romams scripta extat, omnem carnem peccato obnoxiam, a lege non esse nisi peccati cognitionem, non iustitiam, a gra tia Dei, et merito Christi hanc prouenire, omnia ab electione libera Dei pendere, abnegatione nostri, et dilcctio ne proximi Deo gratificandum, magistratui parendum, offendicula fratrum cauenda, eandem esse Gentium et Iudaeorum redemptionem, sedhaec nostra, de his omnibus, et breuius, et ut uidetur luce maiore disserit, tum maiestatem Christi Paulo magnificentius praedicat, et de uitae Christianae officiis disputat distinctius, et admodum locuplete.  \pend\pstart Hanc ergo quam uel solam, si recte intelligatur, satis esse puto, et ad gratiam Christi conmendandam, et uitam quam pientissime instituendam, tum errores quosque conPraestiteuincendos, uerti primum, paulo liberius, ne idiotismi par rat sacra li£ tim Ebraei, partim peculiariter Paulini, rudioribus neberius uer bulam aliquam offunderent. Optarim enimut dum extat tere\pendEpistola ad Ephes totius scri pturae conpendium.  Quae Epi stola ad Romcon tincat. Praestiterat sacrali breius uer tere.  
\section*{M. BVCERI }\pstart Atrunque instiumentum sua lingua, et gralla Deo, multi sunt hodie, qui eas linguas calleant, ut ad fontes semper redire liceat, si quid interpretes, parum assequerentur, ea libertas in uertendis sacris adhiberetur, quam sibi fidi in terpretes permittunt in prophanis. Qui enim dices in lin guam aliquamuersum, quod in ea non dum propter alterius linguae ex qua uersum est idiotismos, potest intelligis? Scio religionem hanc interpretum, maiestati scripturarum tributam, sed ut dixi autogropha spiritus extant, ad quae semper potest corrigi, sicubi interpretes dormitent, quarc mihi praestare uidetur, dum uersionibus iis consuli debet, qui sacras linguas Ebraeam et Graecam ignorant, neque sacrorum scriptorum schematis assueti sunt, ut quam familiarissime et planissime etiam sacra tranfferrentur. Id igitur pro mea tenuitate in hac Epistola dedi operam, quid assecutus sim alii iudicent. Si cui uidear alicubi magis paraphrasten egisse quam interpretem, is si libet ita quoque me nominet, uideor mihi tamem nihil a mente Pauli alienum assuisse. Adieci deinde et commentarium, quo singula Epistolae huius pro uirili explicare iis, qui in Pau Linis literis nondum exerciti sunt, et quidem omnia simplisime sum annisus, ut successerit fratrũ esto iuditium.   \pend\pstart Quicquid hoc operis est, tibi clarissime princeps dedi care decreui, qui tanta sedulitate, regnum Christi, synceramque ueritatis cognitionem plantare apud tuos studes, con uocatis undique uiris pie doctis, comptempto et nequaquam uulgari in quod tuam dignitatem et opes adducis pericu lo, et sumptu, quem huius caussa haudquaquam medio\pendQuare liberius uer sa sit EpistolaInstitutum imitabile principis Schlesiorum.  
\section*{EPISTOLA. }\pstart crem facere incoepisti. Cui siquidem principum rectius sa crorum conmentarii inscribantur, quam ei qui sacrorum scientiam suo locohabet, digneque colit? Inter quos admodum paucos tua celsitudo praeclare fulget, quae cum syn ceriore pietatis doctrina, et linguarum studia, sine quibus nulla solida eruditio constat, suis coniungi curat. Neque pa rum uero huc audaciae me prouocauit, quod sciam, te cam Christi cognitionem nactum, quam haec Pauli Epistola, ut et reliquae omnes, tradit, cui nihiladhaereat adhuc, fermenti conmentorum rationis humanae, quae dixerit uale mundi elementis, confiteatur plantantem ct rigantem nihil esse, ne quid nostris operibus arrogemus, sacrum tamen habeat externum quoque Euangelii, exhortationisque in Domino, usum, et si uehiculum spiritus inde nullum faciat, de quare Caspar Schuenftfeldius tuus, pro eximia sua re uelatione, praeclare scripsit, neque sacramenta contemnat, sed suoloco, pro symbolis scilicet externae conmunionis inter Christianos et instituendae et alendae, habeat.  \pend\pstart Hanc nostramitaque qualemcunq operam celsitudo tua sibi incupatam, eo animo excipiat, quo quemlibet ad pro mouendam Christi gloriam in quouis conatum solet, hoc est, iucundo et fauenti, quodque coepit, post doctrinam Christi synceram, et nullis humanis figmentis fermentatam, pergat curare, ut linguas et aliae frugi artes, iuuentus eius doceatur, neque audiat non tam regiliosos ut uideri uo lunt, quam fastidiosos quosdam, qui haec sancta Dei dona contemnunt, a spiritu uolentes omnia discere, cum quo haud perinde magnam familiaritatem habent, qui profecto di\pendPura prin cipis huius rtligioIn contem ptores bo narum ar tium et linguarum. 
\section*{M. BVCERI }\pstart gni essent, ut et uentri suocibum, item ab illo per miracu lumcogerentur expectare, donec discerent, non minus ad sanctameruditionem, nostram operam adhibendam, quan quam omnia bona ueraque doceat solus spiritus Dei, quam ut corpori parentur alimenta, quod et ipsum sola uirtus Dei sustentat. Etenim dum rectis iudiciis plurimi pollebunt, et minus loci inueniet impostura, et amplior in rebus singulis bonitatis Dei admiratio atque cognitio, plurimum ad pietatem promouebit.  \pend\pstart dapuanne dujpimeiauenapaeuta,taujair ctum et laudatum Cel.tuae institutum, uelint aemulari, sic declaraturi agnoscere se, a Domino ipsius plebi. praefectos, quorum primum studium esse debeat, ut illa bonitatem Dei sui agnoscat, et aduoluntatem ipsius sua uniuersa comparet. Nam haud uulgaris Dei ira existit, ut quos dam omnium cura plus quam pietatis, apud suos excolen dae solicitet, et nullorum sumptuum plus pigeat, quam qui in optima studia fiunt. Dominus qui omniumm principi um corda in sua manuhabet, dirigat ea, ut rite ipsum agnoscant et colant, quo placidamubique et quietamuitam electi degant, cum pietate et honestate. Is et sanctos C.T. conatus secundet, eamque cum liberis, et omni ditione seruet, bonisq́ omnibus exornet.  Illi sit gloria in saecula omnia, AmenArgentorati pridie Calend. Septembus  Anno Christi M. D. XXVII.  \pend
\textit{IN EPISTOLAM D. PAVLI AD EPHE.  sios Argumentum.  }
\textbf{P }\pstart phesi, quae Asiae minoris metropolis fuit, et Dianae multimammiae studiosissima cultrix, triennio Paulus Christum praedicauerat, et non contemnendum populum cum ex urbe, tum tota Asia fuerat lucratus. Ita ut quidam qui curiosas artes sectati fuerant, recepto Euangelio, comportatos libros suos, coram omnibus ad testandam resipiscentiam exusserint, quorum precia tamem fuere ar genti myriades quinque , quae summa supputatore Budaeo quinque milia Coronatorum ualet.  Indicibili siquidem stu dio, atque constantia, in hoc incubuerat, ut annunciaret quae in rem illorum essent, et omne Dei consilium ipsis aperiret, doceretque; et publice et per singulas domos poe\pendQuae Pau lus apud Ephesum effeceris 
\section*{M. BVCERI }\pstart nitentiam ac fidem, quae est erga Dominum nostrum IESVM Christum, denique et moneret cum lachrymis unum quemque die et nocte. Ad haec ne sua uideretur quaerere, cum suis, tum corum qui secum erant necessitatibus, suae ipsius manus suppeditauerant. Dominus quoque sermonem cius cum miris uirtutibus confirmauerat, nam et su dariis eius, et semicinctis, super infirmos delatis, morbi recedebant, malique spiritus egrediebantur, tum incompa rabili spiritus fortitudine reddiderat illustrem, supra mo dum enim in hac urbe adflictus fuerat, et Domini tamen consolatione superior semper euaserat.   \pend\pstart Cum itaque eximie charos Ephesios haberet, ut apud quos non uulgari successu regnum Christi instituisset, pri mum Timotheum praecipuum collegam remancre apud eos, ut a se plantata rigaret, uoluit, deinde Hierosolyma petens, cum sciret in eas regiones se non rediturum, euocatos e Mileto Ephesiorum praesbyteros, persancte monuit, ut sibi, et cuncto gregi attenderent, ne quid incommodi a subingressuris lupis, quos graues futuros illis prae dicebat, Ecclesia eorum perciperet. Hinc postremo et ex uinculis, quibus Romae detinebatur, hanc Epistolam ad eos scripsit, qua uelut in compendium, omni Christianismi ratione contracta, eos in instituto confirmaret, atque proueheret.  \pend\pstart Proinde mira et luce et breuitate de singulis, quae Christianos scire refert, in ea monuit.  Vt omnis mortaliu tam iustitia, quam salus a libera Dei praedestinatioue et  \pendQuam fuerit pro Ephesiis solicitus 
\section*{EPISTOLA }\pstart electione pendeat, et per Christum, per quem omnia instauranda sunt, in caelo et in terra, quique super omnia ex altatus, omnium potestate pollet, in electis tam ex Gentibus quam Iudaeis perficiatur, ad hoc ut Deus in nobis glo rificetur .  Vt natura perditi uniuersi sumus, aeque Iudaei ac Gentes, et gratuita Dei beneuolentia seruemur, quot quot seruamur, reconditi scilicet in Christo, ad bona opera, qui ex Adam nati, tantum mala operari possumus.  Vt denique Christus morte sua legem cerimoniarum, ob quam Iudaei gentes aspernabantur, abrogarit, et donatas sui cognitione gentes, cum Iudaeis in se, in unum hominem coagmentarit, qui cottidie incrementa pietatis accipiens, in templum sanctum Domino cresceret.   \pend\pstart Horum miris affectibus, et magna perspicuitate, per duo priora capita admonuit, simul et gratias pro illis aegit, precarique se pro ipsis memorauit, ut in cognitione horum proficerent, quo agnoscerent, ad quantam spem uocati essent, quanta sit gloria haereditatis quam expecta rent, quamꝗ̃; praecellens uirtus qua niterentur, qua utique Christus a mortuis est excitatus, et super omnia ad dexte ram patris euectus.  In tertio Capite eadem illis precatur, et simul meminit muneris sui, qui peculiariter fuerit ad Euangelizandum gentibus delegatus, utque aperiret, quod mysterium a saeculis etiam ipsis Angelis absconditum fuerat.  Denique hortatur, ne ob suas afflictiones, quas ipsorum caussa ferat, frangantur, sed magis roborato interno homine, bonitatem Dei quam amplissime cognoscant.   \pendCap. primum et secundum Cap.  tertium.  
\section*{ARGVMENTVM.  }\pstart Deinde quarto Capite, et bona parte quinti, ad studium dilectionis ueramque animorum unitatem, tum et depositionem ueteris hominis cum coperibus suis, multis argumentis hortatur, memorans, in hoc Christum imtio regni sui alios dedisse Apostolos, alios Prophetas etc. ut in cognitione Christi singuli confirmati, in unum hominem summa dilectione coalescerent, atque abdicatis carnis operibus, in omni iustitia et sanctitate Deum glorificarent. reliquam partem quinti, et bonam sexti, de peculiaribus officiis quibusdam, uxorum scilicet, maritorum, filiorum, parentum, seruorum et dominorum, sancta prae cepta tradit.  Postremo armat eos contra uersutias caelestes, nempe potestates malorum spirituum, et ad instantem hortatur precationem, cum pro sanctis omnibus, tum et pro se ipso, quo digne Euangelico munere fungatur.  Inde precatus pacem et beneuolentiam Dei, et ipsis et omnibus diligentibus Christum, Epistolam finit. Quae uero hoc argumem to de Ephesiis memoraui, legun tur 1.Cor.15. 2.Cor.1. 1. Timoth.1.Act. 19. et 20. ARGVMENTI finis.   \pendCap.quar tum et quintum. Cap.  sex.  
\section*{AD I. PAVL. AD TIM. }
\marginpar{[ p.9 ]}
\marginpar{[ p.10 ]}
\marginpar{[ p.11 ]}
\marginpar{[ p.B ]}
\textit{D. PAVLI AD EPHESIOS Epistola}
\textbf{P }\pstart Aulus ex uoluntate Dei legatus IESV Christi, sanctis qui sunt Ephesi, nempe fidem habentibus Chri sto IESV, beneuolentiam uobis et pacem, Dei patris et Domini nostri lesu Chriftiprecor.  \pend\pstart Gratia sit et laus Deo patri Domini nostri IESV Christi, qui omni nos genere spiritalis beneficentiae, bonis nimirum caelestibus, affecit per Christum, quemadmodum per illum nos ante conditum mundum, in hocelegit, ut simus sancti, et ipsius quoq iuditio inculpati, dilectioni scilicet proximorumdediti.  \pend
\section*{D. PAVLI AD EPHESIOS }
\marginpar{[ p.3.  ]}
\marginpar{[ p.4.  ]}
\marginpar{[ p.5.  ]}
\marginpar{[ p.6.  ]}
\marginpar{[ p.7.  ]}\pstart Qui praedestinauit pridem nos, ut in filios sibi, per IESVM Christum, ex dignatione uoluntatis suae adoptaret, quo gloria beneuolentiae eius celebris in nobis esset, qua nos pro pter illum dilectum, dignatus est.  \pend\pstart Per  quem, satisiacieste pro nobis sanguine illius, habemus redemptionem, remissionem nimirum peccatorum.  Idqueex amplissima beneuolentia eius, quam ubertim nobis impartiuit, donatis omni sapientia et prudentia, postquam uidelicet ex singulari dignatione notam fecit arcanam uoluntatem suam.  \pend\pstart Qua proposuerat, dum ex certa rerum omnium et temporum dispemsatione, statutum huius tempus appetiisset, instaurare per illum, nempe Christum, uniuersa, per hunc inquam tam quae in caelo sunt, quam quae in terra.  \pend\pstart Per quem et in sortem sanctorum asciti su mus, praeordinati iuxta propositum eius, qui cuncta ubique efficit, et beneuolum arbitrium eius, ut celebris innobis reddatur gloria eius, qui in Christum spem nostram priores collocauimus, inquem et uestram collocastis, postquam sermonem ueritatis, Euange lion salutis uestrae, audiuistis.  \pend\pstart Cui ut fidem habuistis, obsignati quoque estis spiritu lancto, qui promissus nobis fuerat, qui est arrabo haereditatis nostrae, quo fre  \pend
\section*{EPISTOLA. }
\marginpar{[ p.90 ]}
\marginpar{[ p.91 ]}
\marginpar{[ p.92 ]}\pstart ti, redemptionem, qua continget certa uitae possessio, indubitato expectemus. Idqueut glo ria eius in nobis celebris reddatur.  \pend\pstart Ea propter cum et ego audiuissem, qua uos in Dominum Iesum fide, quaque dilectione in sanctos omnes estis praediti, non desino gratias pro uobis agere, memor uestri in precibus meis, ut Deus Domini noftri Iesu Christi, pater ille gloriosus, aspiret uobis sapientiam et reuelationem, ut ipsum cognoscatis, illuminatisqueoculis mentis uestrae sciatis, quid ex uocatione uestra uobis sperandum sit, et quae opulentia sit gloriae haereditatis, quam donauit sanctis, quae denique praecellens uirtutis eius in nos credentes magnitudo.  \pend\pstart Quae in nobis declarabitur, iuxta efficaciam fortis roboris eius, quam efficaciam in Christo exhibuit, cum excitauit illum a mortuis, et ad dexteram suam collocauit in caelestibus, super omnem principatum et potesta tem, et uirtutem, et dominationem, et quic quid omnino eximium siue in praesenti, siue in futuro saeculo celebratur, subiecitque omnia pedibus eius, ipsumque dedit caput Ecclesiae super omnia, quae est corpus eius, et in qua cumulate exuberantem bonitatem suam declarat, qui illa perficit omnia in omnibus.  \pend
\section*{D. PAVLI AD EPHESIOS }
\marginpar{[ p.1. ]}
\marginpar{[ p.2. ]}
\marginpar{[ p.3. ]}
\marginpar{[ p.4. ]}\pstart Etuos, cum mortui eratis delictis et pecca tis, in quibus uixistis iuxta rationem uiuendi mundi huius, secumdum instinctum principis potestatis aerae, spiritum, qui agit modo incredulos, inter quos et nos omnes aliquando uersabamur, indulgentes cupiditatibus na turae nostrae, et perficientes quae ferebat arbi trium carnis nostrae, et cogitationes, qui nimirum aeque acalii, natura irae Dei, obnoxii eramus.  \pend\pstart Deus autem diues misericordia, propter multam dilectionem qua nos dilexit, etiam mortuos delictis, conuiuificauit, una cum Chri sto (beneuolentia gratuita seruati estis) et simul excitauit, atque simul collocauit in caelestibus, insitos Christo Iesu. Vt ostenderet saeculis futuris, excellentem beneuolentiae illius opulentiam, benignitate erga nos in Christo Iesu.  \pend\pstart Beneuolentia gratuita seruati estis perfidem, idque non ex uobis, Dei donum est, non ex operibus ne quis glorietur. Ipsius enim sumus figmentum, conditi in Christo lesu, ad bona opera, quae Deus praeparauit, ut in eis uersaremur.  \pend\pstart Memores igitur estote, quod olim natura Gentes eratis, praeputio impuri uocati ab iis, qui circuncifi, circuncisione manu facta dicebantur, quod, inquam, tum sine Christo era\pend
\section*{EPISTOLA. }
\marginpar{[ p.11 ]}
\marginpar{[ p.50 ]}
\marginpar{[ p.60 ]}
\marginpar{[ p.70 ]}\pstart tis, et abalienati a R. Pubus  Israelis, atque extra nei a foederibus, |promissionem in se continenti bus, nullam spem habentes, et sine Deo ac plane impii habiti in mundo.  \pend\pstart Nunc uero insiti Christo Iesu, qui longe eratis, facti estis prope, per sanguinem Christi. Ipse enim author est pacis nostrae, qui fecit ex utrisqueunum̃, sublato per sacrificium corporis sui interstitio maceriae, inimicitia, abolita indelege, mandata et decreta continente, ut duos sibiipsi conderet in unum nouum hominem, pacequefacta conglutinaret duos, in unum corpus Deo per crucem, extincta in semetipso inimicitia.  \pend\pstart Et adueniens Euangelizauit pacem uobis qui eratis longe, ut et iis qui prope fuerunt. Per ipsum enim utrique aditum habemus ad patrem, afflati eodem spiritu.  \pend\pstart Proinde iam non estis hospites et aduene, sed conciues sanctis, atque domestici Dei, inae dificati fundamento, quo et Apostoli atque Prophetae inaedificati sunt, existente summo angulari lapide Iesu Christo, per quem omne aedificium coagmentatum, exurgit in templum san ctum Domino, perquem et uos coaedificamini in domicilium Dei, agente in uobis spiritu.  \pend
\section*{sti intelligam. }
\marginpar{[ p.CAPVT III ]}
\marginpar{[ p.Dei, datum mihi ex efficaci uirtute eius. ]}
\marginpar{[ p.per Iesum Christum creauit, absconditi. ]}
\marginpar{[ p.1. ]}
\textit{D. PAVLI AD EPHESIOS }\pstart Huius gratia ego Paulus uinctus sum Chri sti Iesu, pro uobis gentibus.  Audistis siquidem dispensationem beneuolentiae Dei, quae mihi commissa est erga uos. Per reuelationem enim notum mihi fecit arcanum, quemadmodum paucis ante scripsi, ex quibus  cognoscere potestis legentes, quid de arcano Chrissti intelligam.  \pend\pstart Quod utiquealiis saeculis, non fuit hominibus retectum, uti nunc sanctis Apostolis eius bus retectum, uti nunc sanctis Apostolis eiu et Prophetis per spiritum retectum est, nempe quod gentes sint cohaeredes, et eiusdem corporis, atque participes promissionis eius, in Christo exhibitae per Euangelion, cuius factus sum minister, iuxta donum beneuolentiae  \pend\pstart Mihi inquam minori quolibet minimo Mihi inquam minori quolibet minimo omnium sanctorum, haecbeneuolentia con tigit, ut inter gentes Euangelizarem, in perue stigabilem opulentiam Christi. Et illato lumine monstrarem omnibus, quae sit commu nio arcani a saeculis penes Deum, qui omnia  \pend\pstart Vt nota modo fieret per Ecclesiam, prinVt nota modo fieret per Eccclesiam, principatibus et potesfatibus caeleltibus, multifa\pend
\section*{EPISTOLA.  }
\marginpar{[ p.12 ]}
\marginpar{[ p.5.  ]}
\marginpar{[ p.6.  ]}
\marginpar{[ p.7. ]}
\marginpar{[ p.8. ]}
\marginpar{[ p.9. ]}\pstart aeternum, quam exhibuit, in Christo lesu Do mino nostro.  \pend\pstart Perquem confidentel agimus, et adituin habemus ex fiducia, quam suppeditat, qua er ga ipsum sumus fide . Quareprecor, ne fran gant uos adflictiones meae, quas pro uobis su stineo, quando quidem haec u estra gloria est.  \pend\pstart Huius caussa flecto genua mea ad patrem Domini nostri Iesu Christi, ex quo omnis cognatio in caelis et in terra constat, ut det uobis secundum opulentam gloriam suam, uir tute corroborari per spiritum, ut bene habeat homo interior, et habitet Christus per fidem in cordibus uestris, sitque in uobis confirmata, probequefundata dilectio.  \pend\pstart Vt ualcatis cum sanctis omnibus comprehende re, quae sit latitudo, et longitudo, profunditas et celsitudo, noscerequepraeeminentem cognitioni dilectionem Christi. Vt bonitate Dei ad summum repleamini.  \pend\pstart Potenti uero facere cumulate, secundum uir tutem in nobis efficacem, ultra omnia quae uel petimus, uel cognoscimus, illi sit gloria in Ecdlesia per Christum lesum, in omnes aetates saeculi saeculorum amen.  \pend
\section*{D. PAVLI AD EPHESIOS }
\marginpar{[ p.2. ]}
\marginpar{[ p.3.  ]}
\marginpar{[ p.4.  ]}\pstart Adhortor igitur uos ego uinctus Domino, ut digne ambuletis uocatione, qua uocati estis, cum omni submissione et mansuetudine, cum longanimitate, tolerantes uos inuicem per dilectionem, studentes seruare unitatem spiritus, uinculo pacis.  \pend\pstart Vnum corpus entis, uno ipiritu uiuitis, quem admodum una spes est uocationis uestrae, qua uocati estis. Vnus dominus, una fides, unum baptisma, unus Deus et pater omnium, qui omnibus praesidet, per omnia uirtutem suam exerit, in omnibus uobis agit.  \pend\pstart Vnicuiqueuero uestrum contigit beneuolentia, secundum modum, quo illam Christus impartit. Quare ille dicit: Cum ascendisset in al tum, dedit dona hominibus. Hoc uero, ascem dit, quid indicat, nisi quod descendit prius, in haecinferiora, in terram? Qui descendit, ipse est, qui et ascendit super omnes caelos, ut perficeret omnia.  \pend\pstart Et hic dedit hos quidem Apostolos, alios ue ro Prophetas, alios uero Euamgelistas, alios au tem Pastores et Doctores, ad hoc ut sancti in staurentur, et opus ministerii procedat, aedificiumquecorporis Christi incrementum accipi at, donec uniuersi occurrerimus inuicem eadem fide et cognitione filii Dei praediti, similes ui  \pend
\section*{EPISTOLA. }
\marginpar{[ p.1 ]}
\marginpar{[ p.r. ]}
\marginpar{[ p.„ ]}
\marginpar{[ p.7.  ]}\pstart ro iam aucti roboris, assecuti modum integrae aetatis CHRISTI, ut non simus iam amplius pueri, quos uertat, et circum ferat, quilibet doctrinae uentus, ex impostura hominum, et uersutia, quibus er rorem obtrudunt.  \pend\pstart Siliceta againus dilectione, et dugeica mus, relaturi per omnia eum, qui caput est Christum. Exquolomne corpus coagmentatum, et compactum, per omnem commissu ram, qua fit subministratio, secundum effica tiam pro portione cuiusque partis, incrementum accipit, ad instaurationem sui ipsius per dilectionem.  \pend\pstart i cergo dico, et obtestor per dominum, ne post hac conuersemini, ut reliquae gentes conuersantur, iuxta uanitatem mentis earum, obtenebrati sensu, abalienatiq a uita Dei, propter ignorantiam quae in eis est, et cae citatem cordis eorum. Qui postquam ad indolentiam peruenerunt, tradiderunt semetipsos lasciuiae, ut quamlibet immunditiam patrarent, simul addicti auar itiae.  \pend\pstart Vos autem non sic didicistis Christum. Si quidem audiuistis eum, et docti estis, quemadmodum ueritas Iesu habet, quod oporteat uos deponere iuxta priorem conuersatio\pend
\section*{D. PAVLI AD EPHESIOS }
\marginpar{[ p.8. ]}
\marginpar{[ p.9.  ]}
\marginpar{[ p.10.  ]}
\marginpar{[ p.11.  ]}
\marginpar{[ p.12.  ]}
\marginpar{[ p.13.  ]}\pstart nem ueterem hominem, qui corrumpitur iuxta desyderia erroris, renouari autem per spiritum menteuestra, et induere nouum ho minem, qui conditus sit secundum DEVM, ut iustitia et solida sanctitate polleat.  \pend\pstart Pionde ablegalites melldacium, loquimini ueritatem, quisque proximo suo, eo quod inuicem membra sumus.  \pend\pstart iaicanmunopeccetb, Sol, durattena uestra, non occidat, neque detis locum Diabolo.  \pend\pstart emuiruauenue, Minoidetar, miagiadero laboret manibus, utili honestaeque arti incum bens, ut possit impartire ei, qui opus habuerit.  \pend\pstart Nullus sermo ipurrareacreucihoprouat, sed si quis frugi sit, et idoneus quo utiliter alii instaurentur, ut frugem audientibus adferat, et ne contristetis spiritum sanctum dei, quo, ut redemptionem olim consequamini obsignati estis.  \pend\pstart Omnis amarulentia, et commotio,  ira, et tumultuans uociferatio, et maledicentia reiiciatur a uobis, cum omni malitia.  \pend\pstart Sitis autem inuicem benigni, propenso affectu, gratificantes inuicem uobis, quemadmodum et DEVS uobis gratificatus  \pend
\section*{EPISTOLA.  }
\section*{est, per CHRISTVM.  }
\marginpar{[ p.u ]}
\marginpar{[ p.1.  ]}
\marginpar{[ p.2.  ]}
\marginpar{[ p.3.  ]}
\marginpar{[ p.4.  ]}
\marginpar{[ p.5.  ]}
\textit{CAPVT V.  }\pstart Estote igitur imitatores D E I, ut dilecti filii, et conuersamini secundum dilectionem, quemadmodum et CHRISTVS dile xit nos, tradiditquesemetipsum pro nobis hostiam et sacrificium Deo, cuius ille nidore plurimum quoquefuit oblectatus.  \pend\pstart Scortatio uero et omnis immunditia, aut auaritia, nihil loci apud uos habeat, sicut decet sanctos, sed neque turpitudo, et stultiloquium, aut facetiae, quae omnia non decent, sed magis gratiarum actio.  \pend\pstart Hoc siquidem agnoscitis quod omnis scor tator, aut immundus, aut auarus, qui est idolorum cultor, non habet haereditatem in regno Christi et Dei. Nemo uobis imponat inanibus sermonibus, propter haec enim quae dixi, uenit ultio Dei in incredulos.  \pend\pstart Ne igitur participetis cum illis. Eratis aliquando tenebrae, nunc autem lux, postquam Domino consecrati estis, ut filii luce praediti igitur conuersamini.  \pend\pstart Fructus enim spiritus, situs est in omni  \pend
\section*{D. PAVLI AD EPHESIOS }
\marginpar{[ p.6.  ]}
\marginpar{[ p.7.  ]}
\marginpar{[ p.8.  ]}
\marginpar{[ p.9.  ]}\pstart bonitate, et iustitia et ueritate, cum exploratis quid Domino beneplaceat.  \pend\pstart Et ne communicetis infrugiferis tenebra rum operibus, magis uero illa arguite. Quae namque in abscondito ab illis fiunt, foedum est dicere. Omnia uero quae arguuntur, ceu alumine manifestantur. Quicquid enim manifestatur, iam luminis aliquid habet. Quapropter ait: Expergiscere qui dor mis, et surge a mortuis, et illuminabite te Christus.  \pend\pstart Videte igitur quomodo circumspecte con uersemini, non ut insipientes, sedut sapientes, redimentes occasionem, quod dies mali sint. Proinde ne sitis imprudentes, sed intelli gentes quae sit uoluntas Domini.  \pend\pstart Et ne inebrieminiuino, quod uitam reddi. intemperantiorem, sed implemini spiritu, lo quentes uobis inuicem, Psalmis et hymnis, atque spiritalibus cantilenis, canentes et psallentes, in corde uestro Domino, gratias agen tes pro omnibus, per Dominum nostrum lesum Christum, Deo et patri.  \pend\pstart Subditi uicissim alius alii estote in timore Dei-Vxores propriis uiris subiectae sint, ut do mino, quod uir caput sit mulieris, sicut Christus caput est Ecclesiae, et idem saluator sui  \pend
\section*{EPISTOLA. }
\marginpar{[ p.2 ]}
\marginpar{[ p.10.  ]}
\marginpar{[ p.11.  ]}\pstart corporis. Vt ergo Ecclesia subdita Christo est, ita uxores propriis uiris in omnibus.  \pend\pstart Viri diligite uxores uestras, quemadmodum Christus dilexit Ecclesiam, et tradidit semetipsum pro ea, ut illam sanctificaret, pur gatam lauacro aquae per uerbum, ut exhibeat sibiipsi gloriosam Ecclesiam, non habentem maculam aut rugam, aut quid huiusmodis sed ut sancta sit, et inculpata. Sic debent diligere uiri uxores suas, ut sua ipsorum corpora. Diligens uxorem suam, se ipsum diligit. Nemo enim unquam carnem suam odio ha buit, sed nutrit et fouet eam, sicut Dominus Ecclesiam.  \pend\pstart Quoniam membra sumus corporis eius, ex carne eius et ossibus eius, pro quo relinquet homo patrem et matrem et adhaerebit uxori suae, eruntque; duo, ut uiuant unum homi nem. Hoc arcanum magnum est, ego autem dicoquoad CHRISTVM, et Ecclesiam. Veruntamen et uos singillatim, suam quisque uxorem diligat perinde atque seipsum, et uxor uolout reuereatur maritum.  \pendCAPVT VI. \pstart Filii morigeri estote parentibus uestris, ut  \pend
\section*{D. PAVLI AD EPHESIOS }
\marginpar{[ p.]}
\marginpar{[ p.]}
\marginpar{[ p.]}
\marginpar{[ p.]}\pstart dignum Domino est. Hoc enim iustum est. Honora patrem tuum, et matrem, quod pri mum quidem praeceptum est, habens promis sionem, hanc nimirum, ut bene tibi sit, sisque longaeuus in terra.  \pend\pstart Patres, ne prouocetis ad iram, filios uestros, sed educate eos, institutione et correpti one Domini.  \pend\pstart Seruo auscultate dominis carnalibus, cum timore et tremore, in simplicitate cordis ue stri, tanquam CHRISTO, non adoculum seruientes, uelut qui placere hominibus student, sedut serui CHRISTI, perficientes uoluntatem D E I ex animo et beneuole, seruientes uidelicet DOMINO et non hominibus, scientes quod unusquisque, quod fecerit boni, id referet a Domino repen sum, siue seruus sit, siue liber.  \pend\pstart Et domini eadem facite erga illos, remittentes minas, scientes, quod et uester dominus est in caelis, et personarum respectus apud eum nullus.  \pend\pstart Quod reliquum est, fratres mei, corroboramini, per Dominum, et robur fortitudinis eius, induimini totam armaturam DE I, ut possitis stare aduersus insidias Diaboli. Ete nim non est nobis lucta aduersus sanguinem et carnem, sed aduersus principatus, et pote\pend
\section*{EPISTOLA.  }
\marginpar{[ p.10.  ]}
\marginpar{[ p.]}
\marginpar{[ p.]}
\marginpar{[ p.]}\pstart tes, aduersus Dominos mundi, tenebrarum saeculi huius, aduersus spiritales uersutias, quae sunt in caelestibus  \pend\pstart Quapropter anumste Iotain afssaturani Dei, ut possitis resistere in die malo, et omnibus peractis stare. State igitur circumcinctis lumbis uestris, ueritate, et induti thoracem iustitiae, et calceati pedes, promptitudinean nunciandi Euangelii pacis. Super omnia assumentes scutumfidei, quo possitis cuncta iacula mali illius ignita extinguere, et galeam salutis accipite, et gladium spiritus, quod est uerbum Dei.  \pend\pstart Omni oratione et deprecatione, orantes in omni tempore, et spiritu, in idipsum uigilantes, cum omni instantia et deprecatione pro sanctis omnibus, et pro me, ut detur mihi sermo, dum os meum aperiendum est, cum omni fidutia, ad notum faciendum arcanum Euangelii, cuius legatione fungor, constrictus catena, ut confidenter illud annunciem, sicut oportet me loqui.  \pend\pstart Vt uero sciatis et uos, quomodo res meae habeant, quid agam, omnia uobis nota faciet Tychicus dilectus frater, et fidus in negotio DOMINI minister, quem in hoc ad uos misi, ut cognosceretis de  \pend
\section*{D. PAVLI AD EPHESIOS }
\marginpar{[ p.9.  ]}\pstart rebus nostris, et consolaretur, corda uestra.  Pax sit fratribus, et dilectio cum fide a Deo patre et Domino nostro Iesu Christo. Beneuolentia Dei, cumomnibus diligentibus Dominum nostrum lesum Christum cum integritate. Amen Missa fuit Roma per Tychicum. Finis Epistolae D. Pauli ad Ephesios.  \pend
\section*{M. BVCERI COMMENTARIVS }
\marginpar{[ p.17 ]}
\textit{Paulus. }\pstart De ratione nomimis mullis disserere non est consilium, tum quod certi hic nihil queat referri, tum quod quicquid afferatur, haud magnum ad pietatem momentum attulerit. In Actis 13. scribit Lucas: Saulus uero qui et Paulus plenus spiritu sancto etc. Vnde coniscio binominem fuisse, Iudaeis tamen fere Saulum appellatum, nomine scilicet illis familiari, Gentibus autem Paulum, quod nomen ob imperium Romanorum apud illas frequens esse coepe rat. Quod igitur sibi nato ciui Romano forsan inditum fuerat, scribens gentibus, libenter epistolis suis praefixit, ut ethnicis hactenus ethnicus fieret et ad Christum eos pel liceret-Quo exemplo docemur, ut mediis rebus in gratiam ho minum, si quid inde commodi ad illos queat redire, libenter utamur, quandoquidem uere Christiani sit, studere ut pla ceat ad bonum omnibus Rom.14.Proinde quouis alio quam quo Paulus actus fuit, spiritu aguntur, qui superstitiose adeo non religiose a bonis Del creaturis plerisque, eo quod illis mali abutantur, abhorrent, non sustinentes uel aliquantisper hominibus, e quibus forte Christoaliquos usu illarum lucrifacerent, gratificari. שאול, id est, Schaul indubie, cum circumcideretur fuerat Apostolus hic appellatus, hactenus tamen ethnicis ethnicus fieri uoluit, ut Romanum: nomen Ebraeo in omnibus epistolis suis prae ferret. Cum hodic uitio, et quidem non leuiculo, uertant cum Anabaptistis, et doctores, suo iudicio, egregii  \pendDe nomine Pauli
\section*{IN I. EPIST. AD TIMOTH. }
\marginpar{[ p.compa]}\pstart quidam, quod cognomina sua aliqui fratrum, dum latine scri bunt, uel in latinam linguam aut graecam uertunt, uel ad modum barum inflectunt. Adeo non minus rara quam eximia res est, recto de rebus istis externis iudicio esse praeditum.   \pend
\textit{Ex uoluntate Dei legatus Iesu Christi. }\pstart Eplithetis semper Paulus ad institutum suum facientibus uti solet, describere Ephesiis sinceriorem rationem Christianismi hac Epistola uoluit, et falutem per Dominum nostrum Iesum Christum partam pectoribus   corum penitius inserere, se igitur legatum Iesu Christi laudat, uo lens significare se non aliud quam salutis illorum negocium, ex mandato Seruatoris et uoluntate patris apud cos hac Epistola acturum.  Atque ita uel breui hoc titulo Ephesiorum animos beneuolos et attentos sibi reddidit.  Quis enim non ex animo uelit et studiose quidem audire legatum Iesu Christi, id est, Seruatoris et uncti regis, qui suos spiritu sancto donatos, iustos et beatos reddit, eoque non nisi salutaria mandat? Ad idem facit, quod se ex decreto pa tris Dei hac salutis legatione fungi testatur.   \pend\pstart Docemur uero hic primum, ut his praecipue nominibus nos fratribus commendemus, quae proxime fecerint ad promouendam apud illos, ut gloriam Dei, ita et ipsorum salutem, hoc est, quibus illis simus usui, non quibus admirationi, multo minus quibus terrori.  Deinde, ut dignationem erganos Dei ingenue prae nobis feramus.  In\pend\pstart  \pendQuibus Epithetis Paulus utatur.  
\section*{M. BVCERI COMMENTARIVS }
\marginpar{[ p.18 ]}\pstart comparabilis siquidem dignitas erat esse legatum IESV Christi, idque ex uoluntate Dei, eam tamen Paulus quamli bet modestus, nequaquam dissimulauit, sed in gloriam dei et fratrum salutem, ingenue professus est. Certa sunt do na Dei, eoque ut uere adesse sentiuntur, ita debent indissimulanter, sed in gloriam donatoris ac fratrum salutem, proferri atque suo modo ostentari.  Postremo monemur, neminem posse legatum Christi esse gloriaeque eius preconem, nisi ex uoluntate missionequeDei.  Quomodo enim praedicabunt nisi mittantur? scribit Apostolus Romanis: Legatum nanque Christi is agit, qui annunciat ipsum unicum esse saluatorem, idque agit quecumque a spiritu dei extrusus fuerit, ut quam plurimos huic caelesti regi adducat.  Huiusmodi aunt ꝗduel cogitare posset ex sese mortalis? Cor. Paul.  scribit, ubi de hac sua legatione loqtur, nos haudquaq idoneos esse, qui ex nobisipsis cogitemus qequam, tanq ex nobisipsis etc.  Nemo ergo legatum Iesu Christi aget, hoc est, Euan gelion sinceriter annunciabit, nisi ad hoc opus, ex uoluntate Dei extrusus:  \pend\pstart Tantum abest ab eo, ut tanquam non missus, non sit audi endus, qui Christi doctrinam pure quidem tradiderit, non autem simul uita expresserit. Vtcunq uita habet, quicumquerecte Christum praedicat, is id ex sese cogitare non potuit, eoque a deo ad hoc missus est, acuae ei, ꝗ quae praedicat contempserit, multum enim interest, inter ultrocurrentes, qui sua aut alio rum figmenta praedicant, et qui ipsissimam Christi ueritatem, quam nemo docebit nisi impulsus missusque diuinitus.   \pendQuis Christi Apostolus.  
\section*{IN I. EPIST. AD TIMOTH. }\pstart Impelliautem mittique ad praedicandum hanc, possunt etiam qui non per omnia illi uita respondent, neque pure gloriam Dei, et nihil sui quaerunt. Christum enim, licet ex mala occasione praedicabant, et humana commenta nequaquam, quorum Paulus Philpp. 2. meminit, id tamen serio Apostolus gaudebat, nequaquam gauisurus sihumana commenta praedicassent, quia alienorum uocem oues audire nequaquam sustinent, tam abest ut ad illam gaude reut. Hypocritae quoque illi, quibus Dominus Matth.7. testimonium perhibet, quod in nomine suo prophetarunt, utique id ex spiritu cius atque uoluntate patris fecerunt. Humanum nanque ingenium nihil tale uel conminisci potest.  Missi igitur diuinitus et hi fuerunt, et Christum ipsum contempsit, quicumque illorum contempsit prophetiam.  Vere igitur et si plane impii ad praedicandum missi fuerant, au diendique timentibus Deum erant, utcunque uita illorum cum doctrina pugnaret.   \pend\pstart Orandus igitur deus ut in messem suam extrudat, qui non tam uerbis quam uita doceant, et danda opera est, ut tales ad docendi munus deligantur. At nequaquam contemnenda interim est, et eorum praedicatio, qui uita sunt insinceriore, de Christo tamen uera testantur.  Missio ita que ad praedicandum Christum, nihil sane aliud fuerit, quam uoluntas praedicandi, quam nobis ille salutem parauit, con iunctam habens huius facultatem, quam utramque solus Deus suppeditat, de qua in Matthaeum dixi plura.  Sanctis qui sunt Ephesi.  \pendQui currentes et qui missiQuid mis sio.  
\section*{M. BVCERI COMMENTARIVS.  }
\marginpar{[ p.19 ]}\pstart Sanctum scriptura uocat, quod est Deo consecratum, eoque augustius iam et diuinae quodammodo reuerentiae particeps.  Sic populus Israel, praecipue sacerdotes, templum et instrumentum eius, sanctitatis nomine passim in scripturis celebrantur.  Iam qui Christo uera fide nomen dederunt, filii dei sunt, ac ita gloriae dei consecrati atque de diti, ut ad cam sint penitus demum reformandi, futuri con formes imagini filii Dei, plane iusti et beati.  Vel soli igitur hi sunt, qui sanctitatis nomine uere insigniuntur, quip pe cum nihil possit uel cogitari augustius et diuinius, quam esse filium Dei et cohaeredem Christi, cuius potestas fit omnibus qui in nomine Christi credunt Ioan. 1.  Vnde hic mox Paul.  subiicit και' πισοῖς ἐν χριςσῷ, quod reddidi, nempe fidem habentibus etc. Nam omnino haec interpre tatio est eius, quod praemisit. Sanctis, eo quod sancti, id est, Deo dediti, eoque uere augusti et diuini, non sint nisi qui per fidem in sortem siliorum Dei ad sciti sunt.  \pend\pstart Hoc autem ωι σοῖσέμ χριοῷ, uerti, fidem habentibus Christo.  Nam quod scriptura suo idiotismo dicit credere in aliquo uel aliquem, nos dicimus credere alicui, fidemque habere.  In 2.  namqueMosch.  cap. 14. legimus, et uidit Israel manum Domini magnam, quam contra Aegyptios exercuit, et timuit populus dominum, et crediderunt in dominum, et in Moscheh seruum eius, quid hoc crediderunt et in Moscheh seruum eius fuit aliud, quam uerbis Moscheh habu erunt fidem? Item in eodem libro cap. 19. dixit dominus ad Moscheh: Ecce egouenio ad te in caligine nubis, quo  \pendSancti. Fideles in Christo. Credere in domino. 
\section*{IN I. EPIST. AD TIMOTH. }\pstart dudiat populus uerba medadte, et etiam in te credant perpetuo. Hoc sane, et etiam credant in te, perinde fait, atque si dixisset, ut tibi credant et uerbis tuis, agnoscentes te ex mandato meoloqui, fidem habeant. Sic ergo et credere in Christo, perinde est atque credere Christo, uerbisque eius fidem habere\pend\pstart Hoc uero ex se mox gignit fiduciam in eum, efficitque ut ipsum suspiciamus et adoremus seruatorem. Nam de re, suscitandumque a se in die nouissimo. Haec qui uera agnoscit, quod nemo nisi a spiritu sancto persuasus poterit, quomodo non illico omnem fiduciam suam in illo collocet, atque felicitatis suae autorem illum colat? Paulo itaque fides proprie persuasio est, qua mens de uerbis D O M IN I nihil addubitat, persuasa scilicet a spiritu sancto Hinc liquet, Vallam illum inimitabilem uirum, non linguarum tan tum, sed et scripturarum D E I proprictatem in hoc recte considerasse, qui, quam graeci τaμπιSίμ uocant, latinis censuit, persuasionem probe dici.  \pend\pstart Certe nihil aliud fuit laudata illa Abrahae fides, ob quam et iustus Deo babitus, et credentis populi pater fuit con stitus.  Deus siquidem illi promiserat filium, et posteritatem ptura suo idiotismo habet: Et credidit in Domino, Hebrai ce דחאיו בי חוח, quod, quid fuit aliud, quam pro mittenti sobolem Deo, habuit fidem, persuasus utique a spiritu dei? Non aliter se habet et fides, qua iustos uiuere  \pendFidutia ex fide.  Fides Abrabae.  
\section*{M. BVCERI COMMENTARIVS }
\marginpar{[ p.20 ]}\pstart Habalcuk cecinit.  Nam promissionem de liberatione populi illic Vates praemisit, testatus perturbato et adflicto animo fore, qui ei fidem habere non sustinerent. Contra autem iustum, fide sua uicturum, id est, bene habiturum.  Persuasi enim a spiritu sancto electi, liberationem a malis omnibus per Christum certo futuram, ut hinc bene de uoluntate Dei fidunt, ita et spe uiuunt, omnia boni consu lentes, utcunque ex praesenti calamitatc uideantur morti addicti.   \pend\pstart Proinde quam Apostolus τϰμπιςίμ, et uulgo fidem latine dicunt, si germanam uocis huius, ut ea Paulus usus est, significantiam recipimus, persuasionem Euangelii per spiritum sanctum electis et uocatis factam, per eam intelligemus. Ea nonnunquam quidem pro fidutia in bonitatem DE I accipitur, sed per Metalepsim.  Ex persuasione enim de promissis Domini, fidutia in bonitatem eius gignitur.  \pend\pstart Quod si noui illi hostes fidei, uirtutis, quae omnis picta tis et salutis unicum fundamentum est, quidam Anabaptistarum cognouissent, non ita contra nos fremerent, qđ fidei in iustificationc primas tribuimus, et agnoscerent friuolum esse, quod contra nos afferunt.  Neminem ei posse fidere, quem non antea amauerit.  Eonque amori DEI non fidei, primas tribuendas.  Et enim ut nemo fidutiam suam in illo fixam habebit, quem non dilexerit, ita neminem quisquam diliget de cuius uirtute  \pendIustus ex fi de uiuit.  Quid fi des.  Cauillum Anabapti starum.  
\section*{IN I. EPIST. AD TIMOTH. }\pstart et bonitate, non fuerit antea persuasus. Quis namque amet eum, quem bonus an malus sit, ignorat? Vnde autem deum bonum et nobis bonum, ut amare eum, confidereque in eum possimus, cognoscemus? Certe quae DEI sunt, et de deo scriptura praedicat, homo secundum animam uiuens, non percipit.   \pend\pstart Ergo ad flatu spiritus sancti hic opus est, qui Euangelion cordibus electorum persuadeat, hoc demum pacto uo cantur, et a patre ad filium pertrahuntur, quotquot ad uitam praeordinati sunt.  Atque hic auditus ille est, unde fides prouenit Rom. 10.  Ad quem et si ministret Euange lii externus praeco, attamen quia plantans et rigans nihil sunt, inefficax erit omnis praedicatio, donec cordi Euam gelion praedicauerit et persuaserit spiritus sanctus.  Apud quem uero ea persuasio obtinuerit, is etiam si ex parte quam bonus sit Deus cognoscat, fieri tamen non potest, quin illum et amet et reueretur, ac optima quaeque de ipso sibi polliceatur, fidem scilicet habens Euangelio, quod sic illum praedicat, quae et certo consequetur, cum spes in deum non pudefaciat.  Itaque post electionem sanctorum, proximum est, ut DEVS suorum se cordibus per uerbum, quod tamen suo spiritu illis persuadeat, insinuet, ex hac mox persuasione certamque fide, quamuerbis eius iam habent, consequitur fiducia cius et amor, propensaque ad gratificandum ipsi in omnibus uoluntas.   \pend\pstart Tria ergo haec fides, spes, charitas sunt, quibus prae\pendVnde fides et amor Dei.  
\end{pages}
\end{document}
        