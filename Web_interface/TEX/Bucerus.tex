% !TeX TS-program = lualatex
\documentclass{article}
\usepackage[T1]{fontenc}
\usepackage{microtype}% Pour l'ajustement de la mise en page
\usepackage[pdfusetitle,hidelinks]{hyperref}
\usepackage[english]{french} % Pour les règles typographiques du français
\usepackage{polyglossia}
\setotherlanguage{greek}
\usepackage[series={},nocritical,noend,noeledsec,nofamiliar,noledgroup]{reledmac}
\usepackage{reledpar} % Package pour l'édition

\usepackage{fontspec} % 
\setmainfont{Liberation Serif}

\usepackage{sectsty}
\usepackage{xcolor}

% Redefine \section font
\sectionfont{\normalfont\scshape\color{gray}}


\begin{document}

\date{}
        \title{ Epistola ad Ephesios : [Bucerus, Martinus], [1527]}
\maketitle

\begin{pages} 
\beginnumbering
        ETISTOLA D. PAVLI AD EPHESIOS, LVA rationem Christianismi breuiter iuxta et locuplete, ut nulla breuius simul et locu pletius explicat, uersa paulo libe- rius, ne peregrint idiotismi rudiores scripturarum of- fenderent, bona tamen fi- de, sententiis Aposto- li appensis. In tundei comuenarlo- PER MARTIRUH DVCERVMI, 
\section*{ARGVMENTVM.  }
\textit{ILLVSTRIS SIMOET PIENTISSIMO PRIN- cipi.  Fridericho, Schlesiorum et Ligniciae Duci etc. Martinus Bucerus salu- tem Christi precatur. }
\textbf{P }\pstart Rouidentia Dei compara- tum est, clarissime et reli- giosissime Princeps, ut quam docunque hactenus sincera Dei scientia in publicum prodierit, ilico et errorum uenti, undique ingruerint, qui si deturbare e medio il- lam nequirent, tantum pulueris tamen opinionum terre- na resipientium, atque inde dissidiorum excitarint, ut per paucis, hoc est, electis duntaxat, atque ex his multis ne- quaquam ἀνιδρωτι, illa rite conspecta fuerit. Visum namque Deo soli bono ac sapienti est, bonitatis ac uirtutis suae gloriam, contra luctantibus potestatibus aduersariis, et ex infirmitate suorum illustrem reddere. Hinc sanctis non cum sanguine et carne, humano scilicet hoste, sed ad uersus caelestes uersutias, mundi dominos, praeter bellum, quod caro ipsorum, aduersus spiritum citra inducias gerit, non minus saeuum et atrox, quamanceps et periculosum, perpetuo depugnandum est.   \pendProdeunte ueritate in furgunt et errores. 
\section*{EPISTOLA. }\pstart Quapropter, si idem hodie, reuecta nobis Euangelii lampade, usu ueniat, ut immissis multiplicium haereseon, at- que imposturarum turbinibus, mendacii author, qui dum atrium suum humanis traditionibus satis munitum tranquil- le possedit, egit quietius, dum pellere rursus mundo illam nequit, obscurare tamen adeo conetur, ut a paucis plane agnosci, a nonnullis ne uideri quidem possit, par erit, ut me- mores, sic bonam dei uoluntatem decreuisse, atque quemad modum antea, ita modo quoque id ita temperaturam, ut inde cum uirtus tum bonitas sua clarior reddatur, boni consili um patris nostri super nos, ubique quae saluti nobis sint pro spicientis, consulamus, et ex animo, etiam pro hac re grati as ipsi agamus. Et enim si recte rationem subduxerimus, haud parum, animos nostros confirmabit etiam, quod quam Eu angelii ueritatem profitemur, princeps tenebrarum ad eun dem modum excipit, quo olim ab ipso seruatore, ciusque Apo stolis praedicatam excepit.  \pend\pstart Quam mox eni praeter grauissimas psecutionum pro- cellas, innumeris insanorum dogmatum fluctibus, quorum ab surditatem nescias, an uarietatem atque inter se pugnantiam magis admireris, cymbam Euangelii ille quatiebat iacta- batque. Idque haudquaquam sine successu, quippe breui a Pau lo, quo nemo unquam maiore uel sedulitate, uel dexterita te, Christum praedicauit, neque placere ad bonum omnibus quisquam magis studuit, ut qui omnibus   omnia fieret, ut uel aliquos lucraretur, omnes Asianos abalienauerat. Timo- theo siquidem suo sic quaestum legimus. Nosti hoc, quod auersati fuerint me omnes, qui sunt in Asia, id est, indu- bie plurimi.   \pendQuo suc- cessu contra Pauli Euam gel. Satan saeuierit.  2 Thess Confirma re, non deii cere debet, quod Eu- angelion nostrum ua riis errori bus oppu- gnatur. 
\section*{MARTINI BVCERI }\pstart Vt autem obturbantibus   uariis errorum sectis, animos nequaquam despondere, sed assumere potius debemus, quod cum ita Domino uisum sit, oporteat, id non minus ad promouendam nostram salutem, quam ipsius gloriam illu strandam facere, qui probatos suos, inter huiusmodi turbi- nes, et utiliter exercet, et magnifice prodit, ita conuenit nihilominus illis pro uiribus spiritus, cuique diuinitus dona ti, obuiamire, dareque operam, ut luce scripturarum, figmento rum humanae rationis tenebras, dispellamus, quod nos qui Christi sumus deceat, haudquaquam minore studio, cum hoc nostrocapite colligere, quam aduersarius soleat disper gere. Cumque ueterator iste, a contemplatione solis diuinae bonitatis, unde omnis cum iustitia, tum salus certo gigni- tur, modis omnibus, ad suspiciendum nihili, ac euanidum humanae industriae uirtutisq́ splendorem, unde nihil nisi fastidiosa hypocrisis, et certa pernities indubitato nasci- tur, animos mortalium detorquere conatur, nobis praeci- pue in hoc incumbendum est, ut fide Euangelii hominibus oculi corroborentur, quouiuificos solis illis diuinae bene- uolentiae radios, rite percipere queant.  \pend\pstart Quemadmodum enim Paulo benignitatem Dei omni studio attollenti, Satanae satellites praedicando circum- cisionem, Sabbatha et idgenus externa, ad quae homi- numuires possunt pertingere, negocium facessebant, sic postea adserentibus candem, sanctis Dei confessoribus, et liberae electioni eius, bonitatiqué omnia accepta ferenti bus, hostis ille ueritatis, similibus instructos armis suos ohiecit, qui libero arbitrio, nostrisisqué operibus nihil non  \pendQuo stu- dio obuiai dum errdri bus. Quib-ar- mis super. orib saecu lis oppu- gnata sit ueritas. 
\section*{EPISTOLA- }\pstart tribuerent. Donec Mahometh sua figmenta, operibus mixta legis, et nostri, sua iuncta in speciem Euangelicis, orbi obtruscrunt, eoque rem perduxerunt, ut nihil fere fidu tiae in bonitatem Dei, omnis uero spes, in conficta ab ho- minibus opera, eaque nullam frugem proximis adferen- tia, cuiusmodi bona opera Deus requirit, tantum non ab uniuerso mundo collocaretur.  \pend\pstart  Podemconjmo, reddito dudut nobls Ellangello, ex quo iam orbis discere incipit, nos natura filios irae, et nihil bo ni a nobis uelcogitare posse, Dei autem, qui ante conditum mundum nos sibi in filios elegit, gratuita beneuolentia spi£ ritum sanguine CHRISTI promeritum donari, quo de ipsius bonitate, ut bene speramus, ita et studio ipsius, ad honesta sanctaque rapimur, praecipue autem, ut proximis commodemus, neque dubitamus, hac uelut arra obsignati in redemptionem, olim nos Christi beneficio, omnis peccati puros, et aeternae felicitatis plene compotes fore, salutis nostrae inimicus, rursum excitat, qui ad externa opera, ea que non solum infrugifera, sedet noxia, a fidutia benignitatis diuinae, acriore, quam ulli antea studio, reuocare laborant. Baptismum siquidem hi, imo rebaptismum supra quam dici queat, urgent, sed quo te ab omnibus, qui illum reiici- unt, quamlibet sanctis Christi membris separes, atque ne magistratui, quem et negant posse Christianumesse, iu- rando, et arma pro re publica sumendo obtempares, adstrin gas. Hi praedestinationis et electionis Dei certitudinem rident, et in impletione legis, sed ad ipsoruminterpre- tationem, saepe absurdam, ad quam uires quoque adesse  \pendQuiu do- ceat Euan gelion- Errores quidam Ca- tabaptista rum. 
\section*{MARTINI SVCERI }\pstart omnibus hominibus adseruerant, spem figere docent Hinc beneuolentiae Dei praedicationem, hoc est, Euange- lion, efficere blasphemant, ut homines a bonis operibus cessent, salutem a merito Christi frustra expectantes, cum tamen, nullum prorsus a nobis bonum opus sieri possit, nisi id sensus amorq bonitatis Dei gignat. Sed non hic tantum nobis isti impingunt, quod ipsi peccant.  Pecc. tum ipsi falsam et euanidam opinionem faciunt, quia non sit creatura D E I coque olim etiam prorsus euant turum, tumque saluos fore, et daemones et impios uni uersos, quem hoc figmentum, qui id receperit, non red- dat socordem, bonorumque operum negligentem, cum eti- am impio sibi, aditum ad beatem uitam fore, et si pau serius putet?  \pend\pstart Istis profecto u̶su uenit, quod et Iudaeis, mundique sapi entibus olim. Suam quaerunt istitiam statuere, igitur DE iustitiae quae per fidem est, subdi nequeunt, et indulgen- tes fidentesque ingeniis suis, sapientes esse uolunt, ideo plus nimio stulti fiunt. Verum dum insolentem prae se graut tatem, rerumque uitae praesentis contemptum, praeser tim apud cos, qui fermentum eorum nondum recaepe runt, prae se ferunt, minime paucis, quibus zelus aliquis est, sed scientia carens, imponunt. Nam ingeniun numanum praecipue admirari solet, quicquid a uulgan ratione uiuendi abhorret, etiamsi ne pilum quidem bo nae frugis adferat, qua certe occasione, Romanenses ab dicantes connubium, et Mahomethani uinum, et Me nachi usum carnium, ac rerum quarundam aliarum, it  \pendQuibus   a tibus mui do solcat impont.  
\section*{EPISTOLA. }\pstart quarum tamen usu ut nulla est impuritas, ita nulla in absti nentia earum sanctitas, totum fere mundum dementa- runt. Porro ut istorum impostura serpat, non paruum mo mentum adfert, eorumqui CHRIS T Iesseuo- lunt infirmitas ne dicam, an socordia et detestanda le- uitas. Dum enim simpliciores, nondum scilicet a spiritu intus satis roborati, uident tam imparem doctrina Euan- gelica uitam apud eos, qui illam profitentur, minimo nego tio ab iis, qui speciem aliquam eius prae se ferunt, persua- dentur, non esse eam uere Euangelicam doctrinam, quam illi praedicant.  Vae igitur nobis, si blasphemandi doctri- nam CHRISTI, et offensionis pusillorumoc casionem dederimus.  \pend\pstart Praeter hos, sunt et ex iis, qui Euangelii purissimiuin dices sibi uideri uolunt, neq sane uulgariter in prouchen do illo, sudarunt, qui nescio qua animis ipsorumobfusa, contentionis et φιλαυτίας incertum, an stuporis et ignorantiae nubecula, Sacramentis, quae symbola sum Christianorum, et externo uerbo, ea tribuunt, unde di uinae beneuolentiae, uirtutisqué claritas non parum obscu- ratur. Vehicula enim illa faciunt, spiritus sancti, et fidei, cum haec donet sola bonitas Dei, impetrata morte CHRI STI. Hi ex eadem cognitionis gratiae Dei angustia, et sanctorum salutem aeternam negant, dum hic, qui Chri- stouere crediderint, rursum posse excidere, non tam in- considerate, quam periculose affirmant. Quae enimfi- des, quae se dubitet habere repositam sibi penes Deumui- tam aetcrnam, hoc est, esse filium atque haeredem Dei ?am  \pendSacramen is et exter io uerbo quidam ni nium tri- buunt- 
\section*{MARTINI BVCERI }\pstart DE V Ssuis, pro fide ipsorum facit.  \pend\pstart Proinde cum tam multis errorum nubibus Christi 4 uerfarius, contra restitutam nobis illius lucem Euange licam, perstrenue adhuc pugnet, nouosque cottidie excitet, abusus et sanctorum ignorantia et infirmitate, sit denique eius in omnibus idem conatus, nimirum, ut opera manuum suarum homines adorent, et non ut deploratos pecca- tores, diuinae bencuolentiae, sanguiniqué CHRISTI cu- randos plena fide sese contradant, debent profecto, qui- bus Christi gloria, electorumque salus chara existit, nihil uicissimomittere, quoregniC HRI ST I Euan- gelion, et bonitatis D E I praedicatio, quam in primis uniucrsa scripturaubiq in culcat, orbis magis ac magis agnoscat.  Vere enim sordebunt nobis nostra, ubi diuinae bonitatis gustum recte percaeperimus, neque cessabimus tum a bonis operibus, sed gnauiter demum illis incum- bemus, toti scilicet eo rapti, ut gratificaripatri nostro studeamus.  \pend\pstart Cum ucro Paulus in praedicanda D E I praedestina- tione, electione, amplissimaque in nos bonitate et efficacia sanguinis CHRISTI, hoc est, in annunciando syncero Euamgelio, docendaqué doctrina certae salutis, ita excelluerit, ut non paruo interuallo, alios sacros scriptores post se reliquerit, praesertim si lucem, et copiam spectes, optan- dum est, ut illius Epistolae quam familiarissimae Chri- stianis omnibus reddantur. Qui sane eas recte teneret, cum sibi tum aliis, quoslibet facile errores auerteret.  \pendPaulus cla rius et co- piosius Eu amgelion prae dicauit. 
\section*{EPISTOLA. }\pstart Equidem igitur ut huc fratribus scripturarum rudioribus   adiutor essem, quibus et V. Fabus  Capito paratis nuper in Hoseam haudquaquam uulgaribus conmentariis, et M- Cellarius opere, de Operibus Dei in quo diuina bonitas magnifice praedicatur, plurimum consuluerunt, Epistolam illus ad Ephesios, hisce diebus enarrandam mihi desum- ad Ephes- psi* Visumenim mihi est, doctrinam Christi, breui-iotius scri ter iuxta et luculenter, atque copiose adeo complexa esse, ut pturae con- his nominibus, nullam aliam ei praeferendam existimem, totiusque pendium. sacrae scientiae, compendium recte habendum censeam.  \pend\pstart Docet quidem fusius etscripturis conmunitius, ea quae Romams scripta extat, omnem carnem peccato obnoxiam, a lege non esse nisi peccati cognitionem, non iustitiam, a gra tia Dei, et merito Christi hanc prouenire, omnia ab ele- ctione libera Dei pendere, abnegatione nostri, et dilcctio ne proximi Deo gratificandum, magistratui parendum, offendicula fratrum cauenda, eandem esse Gentium et Iu- daeorum redemptionem, sedhaec nostra, de his omnibus, et breuius, et ut uidetur luce maiore disserit, tum maie- statem Christi Paulo magnificentius praedicat, et de ui- tae Christianae officiis disputat distinctius, et admodum locuplete.  \pend\pstart Hanc ergo quam uel solam, si recte intelligatur, satis esse puto, et ad gratiam Christi conmendandam, et uitam quam pientissime instituendam, tum errores quosque con- Praestite- uincendos, uerti primum, paulo liberius, ne idiotismi par rat sacra li£ tim Ebraei, partim peculiariter Paulini, rudioribus ne- berius uer bulam aliquam offunderent. Optarim enimut dum extat tere-  \pendEpistola ad Ephes totius scri pturae con- pendium.  Quae Epi stola ad Rom- con tincat. Praestite- rat sacrali breius uer tere.  
\section*{M. BVCERI }\pstart Atrunque instiumentum sua lingua, et gralla Deo, multi sunt hodie, qui eas linguas calleant, ut ad fontes semper redire liceat, si quid interpretes, parum assequerentur, ea libertas in uertendis sacris adhiberetur, quam sibi fidi in terpretes permittunt in prophanis. Qui enim dices in lin guam aliquamuersum, quod in ea non dum propter alte- rius linguae ex qua uersum est idiotismos, potest intelligis? Scio religionem hanc interpretum, maiestati scriptura- rum tributam, sed ut dixi autogropha spiritus extant, ad quae semper potest corrigi, sicubi interpretes dormitent, quarc mihi praestare uidetur, dum uersionibus iis con- suli debet, qui sacras linguas Ebraeam et Graecam igno- rant, neque sacrorum scriptorum schematis assueti sunt, ut quam familiarissime et planissime etiam sacra tranfferrentur. Id igitur pro mea tenuitate in hac Epistola dedi operam, quid assecutus sim alii iudicent. Si cui uidear alicubi ma- gis paraphrasten egisse quam interpretem, is si libet ita quoque me nominet, uideor mihi tamem nihil a mente Pauli alienum assuisse. Adieci deinde et commentarium, quo singula Epistolae huius pro uirili explicare iis, qui in Pau Linis literis nondum exerciti sunt, et quidem omnia sim- plisime sum annisus, ut successerit fratrũ esto iuditium.   \pend\pstart Quicquid hoc operis est, tibi clarissime princeps dedi care decreui, qui tanta sedulitate, regnum Christi, synce- ramque ueritatis cognitionem plantare apud tuos studes, con uocatis undique uiris pie doctis, comptempto et nequaquam uulgari in quod tuam dignitatem et opes adducis pericu lo, et sumptu, quem huius caussa haudquaquam medio-  \pendQuare li- berius uer sa sit Epi- stola- Institutum imitabile principis Schlesio- rum.  
\section*{EPISTOLA. }\pstart crem facere incoepisti. Cui siquidem principum rectius sa crorum conmentarii inscribantur, quam ei qui sacrorum scientiam suo locohabet, digneque colit? Inter quos admo- dum paucos tua celsitudo praeclare fulget, quae cum syn ceriore pietatis doctrina, et linguarum studia, sine quibus nulla solida eruditio constat, suis coniungi curat. Neque pa rum uero huc audaciae me prouocauit, quod sciam, te cam Christi cognitionem nactum, quam haec Pauli Epistola, ut et reliquae omnes, tradit, cui nihiladhaereat adhuc, fermenti conmentorum rationis humanae, quae dixerit uale mundi elementis, confiteatur plantantem ct rigantem nihil esse, ne quid nostris operibus arrogemus, sacrum tamen habeat externum quoque Euangelii, exhortationisque in Domino, usum, et si uehiculum spiritus inde nullum faciat, de quare Caspar Schuenftfeldius tuus, pro eximia sua re uelatione, praeclare scripsit, neque sacramenta contemnat, sed suoloco, pro symbolis scilicet externae conmunionis inter Christianos et instituendae et alendae, habeat.  \pend\pstart Hanc nostramitaque qualemcunq operam celsitudo tua sibi incupatam, eo animo excipiat, quo quemlibet ad pro mouendam Christi gloriam in quouis conatum solet, hoc est, iucundo et fauenti, quodque coepit, post doctrinam Chri- sti synceram, et nullis humanis figmentis fermentatam, pergat curare, ut linguas et aliae frugi artes, iuuentus eius doceatur, neque audiat non tam regiliosos ut uideri uo lunt, quam fastidiosos quosdam, qui haec sancta Dei dona contemnunt, a spiritu uolentes omnia discere, cum quo haud perinde magnam familiaritatem habent, qui profecto di-  \pendPura prin cipis huius rtligio- In contem ptores bo narum ar tium et linguarum. 
\section*{M. BVCERI }\pstart gni essent, ut et uentri suocibum, item ab illo per miracu lumcogerentur expectare, donec discerent, non minus ad sanctameruditionem, nostram operam adhibendam, quan quam omnia bona ueraque doceat solus spiritus Dei, quam ut corpori parentur alimenta, quod et ipsum sola uir- tus Dei sustentat. Etenim dum rectis iudiciis plurimi pollebunt, et minus loci inueniet impostura, et amplior in rebus singulis bonitatis Dei admiratio atque cognitio, plurimum ad pietatem promouebit.  \pend\pstart dapuanne dujpimeiauenapaeuta,taujair ctum et laudatum Cel.tuae institutum, uelint aemulari, sic declaraturi agnoscere se, a Domino ipsius plebi. praefe- ctos, quorum primum studium esse debeat, ut illa bonita- tem Dei sui agnoscat, et aduoluntatem ipsius sua uni- uersa comparet. Nam haud uulgaris Dei ira existit, ut quos dam omnium cura plus quam pietatis, apud suos excolen dae solicitet, et nullorum sumptuum plus pigeat, quam qui in optima studia fiunt. Dominus qui omniumm principi um corda in sua manuhabet, dirigat ea, ut rite ipsum agnoscant et colant, quo placidamubique et quietamui- tam electi degant, cum pietate et honestate. Is et sanctos C.T. conatus secundet, eamque cum liberis, et omni ditione seruet, bonisq́ omnibus exornet.  Illi sit gloria in saecula omnia, Amen- Argentorati pridie Calend. Septembus  Anno Christi M. D. XXVII.  \pend
\textit{IN EPISTO- LAM D. PAVLI AD EPHE.  sios Argumentum.  }
\textbf{P }\pstart phesi, quae Asiae minoris metropolis fuit, et Dianae multimammiae studiosissima cultrix, triennio Paulus Christum praedicauerat, et non contemnendum popu- lum cum ex urbe, tum tota Asia fuerat lucratus. Ita ut qui- dam qui curiosas artes sectati fuerant, recepto Euange- lio, comportatos libros suos, coram omnibus ad testandam resipiscentiam exusserint, quorum precia tamem fuere ar genti myriades quinque , quae summa supputatore Budaeo quinque milia Coronatorum ualet.  Indicibili siquidem stu dio, atque constantia, in hoc incubuerat, ut annunciaret quae in rem illorum essent, et omne Dei consilium ipsis ape- riret, doceretque; et publice et per singulas domos poe-  \pendQuae Pau lus apud Ephesum effeceris 
\section*{M. BVCERI }\pstart nitentiam ac fidem, quae est erga Dominum nostrum IE- SVM Christum, denique et moneret cum lachrymis unum quemque die et nocte. Ad haec ne sua uideretur quaerere, cum suis, tum corum qui secum erant necessitatibus, suae ipsius manus suppeditauerant. Dominus quoque sermo- nem cius cum miris uirtutibus confirmauerat, nam et su dariis eius, et semicinctis, super infirmos delatis, morbi recedebant, malique spiritus egrediebantur, tum incompa rabili spiritus fortitudine reddiderat illustrem, supra mo dum enim in hac urbe adflictus fuerat, et Domini tamen consolatione superior semper euaserat.   \pend\pstart Cum itaque eximie charos Ephesios haberet, ut apud quos non uulgari successu regnum Christi instituisset, pri mum Timotheum praecipuum collegam remancre apud eos, ut a se plantata rigaret, uoluit, deinde Hierosolyma petens, cum sciret in eas regiones se non rediturum, euo- catos e Mileto Ephesiorum praesbyteros, persancte mo- nuit, ut sibi, et cuncto gregi attenderent, ne quid incom- modi a subingressuris lupis, quos graues futuros illis prae dicebat, Ecclesia eorum perciperet. Hinc postremo et ex uinculis, quibus Romae detinebatur, hanc Epistolam ad eos scripsit, qua uelut in compendium, omni Christia- nismi ratione contracta, eos in instituto confirmaret, atque proueheret.  \pend\pstart Proinde mira et luce et breuitate de singulis, quae Christianos scire refert, in ea monuit.  Vt omnis mortaliu tam iustitia, quam salus a libera Dei praedestinatioue et  \pendQuam fuerit pro Ephesiis solicitus 
\section*{EPISTOLA }\pstart electione pendeat, et per Christum, per quem omnia in- stauranda sunt, in caelo et in terra, quique super omnia ex altatus, omnium potestate pollet, in electis tam ex Genti- bus quam Iudaeis perficiatur, ad hoc ut Deus in nobis glo rificetur .  Vt natura perditi uniuersi sumus, aeque Iudaei ac Gentes, et gratuita Dei beneuolentia seruemur, quot quot seruamur, reconditi scilicet in Christo, ad bona ope- ra, qui ex Adam nati, tantum mala operari possumus.  Vt denique Christus morte sua legem cerimoniarum, ob quam Iudaei gentes aspernabantur, abrogarit, et donatas sui cognitione gentes, cum Iudaeis in se, in unum hominem coagmentarit, qui cottidie incrementa pietatis accipiens, in templum sanctum Domino cresceret.   \pend\pstart Horum miris affectibus, et magna perspicuitate, per duo priora capita admonuit, simul et gratias pro illis aegit, precarique se pro ipsis memorauit, ut in cognitio- ne horum proficerent, quo agnoscerent, ad quantam spem uocati essent, quanta sit gloria haereditatis quam expecta rent, quamꝗ̃; praecellens uirtus qua niterentur, qua utique Christus a mortuis est excitatus, et super omnia ad dexte ram patris euectus.  In tertio Capite eadem illis precatur, et simul meminit muneris sui, qui peculiariter fuerit ad Euangelizandum gentibus delegatus, utque aperiret, quod mysterium a saeculis etiam ipsis Angelis absconditum fuerat.  Denique hortatur, ne ob suas afflictiones, quas ipsorum caussa ferat, frangantur, sed magis roborato interno ho- mine, bonitatem Dei quam amplissime cognoscant.   \pendCap. pri- mum et secundum Cap.  ter- tium.  
\section*{ARGVMENTVM.  }\pstart Deinde quarto Capite, et bona parte quinti, ad stu- dium dilectionis ueramque animorum unitatem, tum et de- positionem ueteris hominis cum coperibus suis, multis argumentis hortatur, memorans, in hoc Christum imtio regni sui alios dedisse Apostolos, alios Prophetas etc. ut in cognitione Christi singuli confirmati, in unum homi- nem summa dilectione coalescerent, atque abdicatis carnis operibus, in omni iustitia et sanctitate Deum glorifica- rent. reliquam partem quinti, et bonam sexti, de pe- culiaribus officiis quibusdam, uxorum scilicet, maritorum, filiorum, parentum, seruorum et dominorum, sancta prae cepta tradit.  Postremo armat eos contra uersutias caele- stes, nempe potestates malorum spirituum, et ad instan- tem hortatur precationem, cum pro sanctis omnibus, tum et pro se ipso, quo digne Euangelico munere fungatur.  Inde precatus pacem et beneuolentiam Dei, et ipsis et omnibus diligentibus Christum, Epistolam finit. Quae uero hoc argumem to de Ephesiis memoraui, legun tur 1.Cor.15. 2.Cor.1. 1. Timoth.1.Act. 19. et 20. ARGVMENTI finis.   \pendCap.quar tum et quintum. Cap.  sex.  
\section*{AD I. PAVL. AD TIM. }
\marginpar{[ p.9 ]}
\marginpar{[ p.10 ]}
\marginpar{[ p.11 ]}
\marginpar{[ p.B ]}
\textit{D. PAVLI AD EPHESIOS Epistola- }
\textbf{P }\pstart Aulus ex uoluntate Dei legatus IESV Christi, sanctis qui sunt Ephesi, nempe fidem habentibus Chri sto IESV, beneuolen- tiam uobis et pacem, Dei patris et Domi- ni nostri lesu Chriftiprecor.  \pend\pstart Gratia sit et laus Deo patri Domini no- stri IESV Christi, qui omni nos genere spiri- talis beneficentiae, bonis nimirum caelesti- bus, affecit per Christum, quemadmodum per illum nos ante conditum mundum, in hocelegit, ut simus sancti, et ipsius quoq iu- ditio inculpati, dilectioni scilicet proximo- rumdediti.  \pend
\section*{D. PAVLI AD EPHESIOS }
\marginpar{[ p.3.  ]}
\marginpar{[ p.4.  ]}
\marginpar{[ p.5.  ]}
\marginpar{[ p.6.  ]}
\marginpar{[ p.7.  ]}\pstart Qui praedestinauit pridem nos, ut in filios sibi, per IESVM Christum, ex dignatione uo- luntatis suae adoptaret, quo gloria beneuo- lentiae eius celebris in nobis esset, qua nos pro pter illum dilectum, dignatus est.  \pend\pstart Per  quem, satisiacieste pro nobis sangui- ne illius, habemus redemptionem, remissio- nem nimirum peccatorum.  Idqueex amplissi- ma beneuolentia eius, quam ubertim nobis impartiuit, donatis omni sapientia et pru- dentia, postquam uidelicet ex singulari digna- tione notam fecit arcanam uoluntatem suam.  \pend\pstart Qua proposuerat, dum ex certa rerum o- mnium et temporum dispemsatione, statutum huius tempus appetiisset, instaurare per illum, nempe Christum, uniuersa, per hunc inquam tam quae in caelo sunt, quam quae in terra.  \pend\pstart Per quem et in sortem sanctorum asciti su mus, praeordinati iuxta propositum eius, qui cuncta ubique efficit, et beneuolum arbitri- um eius, ut celebris innobis reddatur gloria eius, qui in Christum spem nostram prio- res collocauimus, inquem et uestram collo- castis, postquam sermonem ueritatis, Euange lion salutis uestrae, audiuistis.  \pend\pstart Cui ut fidem habuistis, obsignati quoque estis spiritu lancto, qui promissus nobis fue- rat, qui est arrabo haereditatis nostrae, quo fre  \pend
\section*{EPISTOLA. }
\marginpar{[ p.90 ]}
\marginpar{[ p.91 ]}
\marginpar{[ p.92 ]}\pstart ti, redemptionem, qua continget certa uitae possessio, indubitato expectemus. Idqueut glo ria eius in nobis celebris reddatur.  \pend\pstart Ea propter cum et ego audiuissem, qua uos in Dominum Iesum fide, quaque dilectione in sanctos omnes estis praediti, non desino gra- tias pro uobis agere, memor uestri in preci- bus meis, ut Deus Domini noftri Iesu Chri- sti, pater ille gloriosus, aspiret uobis sapientiam et reuelationem, ut ipsum cognoscatis, illumi- natisqueoculis mentis uestrae sciatis, quid ex uocatione uestra uobis sperandum sit, et quae opulentia sit gloriae haereditatis, quam donauit sanctis, quae denique praecellens uirtutis eius in nos credentes magnitudo.  \pend\pstart Quae in nobis declarabitur, iuxta efficaci- am fortis roboris eius, quam efficaciam in Christo exhibuit, cum excitauit illum a mor- tuis, et ad dexteram suam collocauit in caele- stibus, super omnem principatum et potesta tem, et uirtutem, et dominationem, et quic quid omnino eximium siue in praesenti, siue in futuro saeculo celebratur, subiecitque o- mnia pedibus eius, ipsumque dedit caput Ec- clesiae super omnia, quae est corpus eius, et in qua cumulate exuberantem bonitatem suam declarat, qui illa perficit omnia in omnibus.  \pend
\section*{D. PAVLI AD EPHESIOS }
\marginpar{[ p.1. ]}
\marginpar{[ p.2. ]}
\marginpar{[ p.3. ]}
\marginpar{[ p.4. ]}\pstart Etuos, cum mortui eratis delictis et pecca tis, in quibus uixistis iuxta rationem uiuendi mundi huius, secumdum instinctum principis potestatis aerae, spiritum, qui agit modo in- credulos, inter quos et nos omnes aliquan- do uersabamur, indulgentes cupiditatibus na turae nostrae, et perficientes quae ferebat arbi trium carnis nostrae, et cogitationes, qui ni- mirum aeque acalii, natura irae Dei, obnoxii eramus.  \pend\pstart Deus autem diues misericordia, propter multam dilectionem qua nos dilexit, etiam mortuos delictis, conuiuificauit, una cum Chri sto (beneuolentia gratuita seruati estis) et si- mul excitauit, atque simul collocauit in caelesti- bus, insitos Christo Iesu. Vt ostenderet saeculis futuris, excellentem beneuolentiae illius opu- lentiam, benignitate erga nos in Christo Iesu.  \pend\pstart Beneuolentia gratuita seruati estis perfi- dem, idque non ex uobis, Dei donum est, non ex operibus ne quis glorietur. Ipsius enim su- mus figmentum, conditi in Christo lesu, ad bona opera, quae Deus praeparauit, ut in eis uersaremur.  \pend\pstart Memores igitur estote, quod olim natura Gentes eratis, praeputio impuri uocati ab iis, qui circuncifi, circuncisione manu facta dice- bantur, quod, inquam, tum sine Christo era-  \pend
\end{pages}
\end{document}
        