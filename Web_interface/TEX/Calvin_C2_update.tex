
%%%%%%%%%%%%%%%%%%%%%%%%%%%%%%%% SCRIPT FOR E-RARA AND MDZ FILES     %%%%%%%%%%%%%%%%%%%%%%%%%%%%%%%%%%%%%%%%%%%%%%%%
%%%%%%%%%%%%%%%%%%%%%%%%%%% fini le 30.04.2025 par F. GOY            %%%%%%%%%%%%%%%%%%%%%%%%%%%%%%%%%%%%%%%%%%%%%%%%
% !TeX TS-program = lualatex
\documentclass{article}
\usepackage[T1]{fontenc}
\usepackage{microtype}
\usepackage[pdfusetitle,hidelinks]{hyperref}

\usepackage{polyglossia}
\setmainlanguage{english}
\setotherlanguages{latin,greek}
\usepackage[series={},nocritical,noend,noeledsec,nofamiliar,noledgroup]{reledmac}
\usepackage{reledpar}

\usepackage{fontspec}
\setmainfont{TeX Gyre Termes}

\usepackage{sectsty}
\usepackage{xcolor}

\usepackage{fancyhdr}
\pagestyle{fancy}
\fancyhf{}
\fancyhead[LE,RO]{\nouppercase{\leftmark}}  
\cfoot{\thepage}
\renewcommand{\headrulewidth}{0.4pt}

% Redefine \section to remove numbering
\usepackage{titlesec}
\titleformat{\section}[block]{\normalfont\scshape\color{gray}}{}{0pt}{} % no number in heading
\titleformat{\subsection}[hang]{\normalfont}{}{0pt}{} % also remove subsection number
\titleformat{\subsubsection}[hang]{\normalfont\footnotesize\color{black}}{}{0pt}{}

% Modify how section marks are stored to exclude numbers
\makeatletter
\renewcommand{\sectionmark}[1]{%
	\markboth{#1}{}} % Only store the section title, without number
\renewcommand{\subsectionmark}[1]{%
	\markright{#1}} % Only store the subsection title, without number
\renewcommand{\numberline}[1]{} % Hide the section number in TOC
\makeatother

\begin{document}

\date{}
        \title{Commentarii in utranque Pauli epistolam ad Timotheum: [Calvin Jean], [1548]}
\maketitle
\tableofcontents
\clearpage
\begin{pages} 
\beginnumbering
        
\section*{l. TIMOTH. }
\marginpar{[ p.16 ]}\pstart Audientes colamus bona conscientia fidei nostrae possessionem, vt salua nobis ad extremum maneat.  \pend\pstart Quos tradidi Sathanae.) Quemadmodum diximus in quintum caput prioris Epistolae ad Corinthios, sunt qui interpretantur extraordinariis flagellis fuisse coercitos: idque referunt ad δυναμίν, cuius meminit Paulus eiusdem Epistolae capite decimo. Nam vt dono sanationis praediti erant Apostoli ad testandam erga pios Dei gratiam et beneficentiam, ita aduersus impios et rebelles armati erant potentia: vel vt vexandos Diabolo traderent: vel vt aliis flagellis in eos animaduerterent. Cuius potentiae specimen edidit Petrus in Anania et Sapphira Paulus vero in Bar-Iesu mago. At mihi de excommunicatione exponere magis placet. Nam vt credamus aliter, quam excommunicatione castigatum fuisse Corinthium illum incestum, nulla probabilis coniectura affertur. Si autem illum excommunican do Sathanae tradidit Paulus: cur non eadem locutio idem hic valebit? Adde quod vim excommunicationis optime explieat. Nam cum in Ecclesia sedem regni sui habeat Christus, extra non est nisi Sathanae dominium. Proinde qui ab Ecclesia exterminatur, eum tantis per sub tyrannide Sathanae degere necesse est, dum Ecclesiae reconciliatus, ad Christum redeat. Vnum excipio, quod, pro mali atrocitate. perpetuum istis anathema denunciare potuit. De quo nihil tamen ausim pro certo asserere. Sed quid sibi vult proxima particula: Vt discant non maledicere? Nam qui ab Ecclesia eiectus est, plus licentiae sibi vsurpat: quia tanquam communis disciplinae iugo solutus, petulantius insurgit. Respondeo, vtcunque exultet eoruni improbitas, aditum tamen praecludi,ne sua contagione gregem inficiant. Nam eo maxime nocent improbi, dum eiusdem fidei praetextu se insinuant. Nocendi ergo facultas illis eripitur, dum notantur publica infamia. vt nemo simplicium ignoret, profanos esse homines et execrandos : ideóque omnes ab eorum communicatione abhorreant. Ipsos etiam interdum contingit irrogata sibi talis ignominiae nota confusos. a proteruia desistere. Ergo etiam si improbiores reddantur aliquando: tamen non semper inutile est ad domandam ferociam temedium.  \pend
\section{CAPVT II.}
\phantomsection
\addcontentsline{toc}{subsection}{\textit{\huge\textbf{A}\normalsize Dhortor igitur, vt ante omnia fiant deprecationes, obsecrationes, interpellationes, gratiarum actiones pro omnibus hominibus, pro regibus et omnibus in eminentia constitutis, vt placidam et quietam vitam degamus cum omni pietate et honestate. }}
\subsection*{\textit{\huge\textbf{A}\normalsize Dhortor igitur, vt ante omnia fiant deprecationes, obsecrationes, interpellationes, gratiarum actiones pro omnibus hominibus, pro regibus et omnibus in eminentia constitutis, vt placidam et quietam vitam degamus cum omni pietate et honestate. }}
\section*{CAP. II. }
\marginpar{[ p.17 ]}
\phantomsection
\addcontentsline{toc}{subsection}{\textit{ Hoc enim bonum et acceptum coram saluatore nostro Deo, qui omnes homines vult saluos fieri, et ad agnitionem veritatis venire. }}
\subsection*{\textit{ Hoc enim bonum et acceptum coram saluatore nostro Deo, qui omnes homines vult saluos fieri, et ad agnitionem veritatis venire. }}\pstart Adhortor igitur.) Haec pietatis exercitia exercent nos in sincero cultu Dei ac timore fouentque bonam conscientiam de qua dixerat. Quare non abs re, illatiua particula vtitur: quia ex superiore praecepto exhortationes istae pendent. Ac initio quidem de publicis orationibus disserit: quas iubet non pro fidelibus modo concipi, sed pro vniuerso genere humano. Poterant enim nonnulli ita secum reputare: Cur de infidelium salute essemus soliciti, quibuscum nihil est nobis necessitudinis? Nónne satis est, sifratres pro fratribus mutuo oremus, ac commendemus Deo totam suam Ecclesiam? Nihil enim ad nos extranei. Huic sinistrae opinioni occurrit Paulus, ac iubet Ephesios suis precibus complecti omnes mortales, nec eas ad corpus Ecclesiae restringere. Porrd quid inter se differant quatuor species quas enumerat, fateor me non penitus tenere. Puerile quidem est, quod Pauli verba Augustinus ad ritus suo tempore vsitatos detorquet. Sim plicior est eorum expositio, qui deprecationes esse putant, quibus petimus liberari a malis: obsecrationes, quibus vtilia nobis poscimus: interpellationes, quibus apud Deum iniurias nobis illatas deploramus. Quanquam ego non ita subtiliter distinguo, vel saltem aliam distinctionem magis probo. *ροσωυχας sane Graeci vocant omne genus orationis: δεή σας autem eas precationum formas, ybi certum aliquid petitur. Hoc modo, inter se conuenirent haec duo nomina, tanquam genus et species. ιν rιύfας vocare solet Paulus quas alii pro aliis suscipimus preces. Pro quo Latina transsatio habet intercessiones. Quanquam Plato in Alcibiade secundo aliter accipit, nenpe, pro seipso certam precationem concipere. In ipsa autem libri inscriptione et compluribus locis satis quod dixi ostendit, προσευχὴν nomen esse generale. Sed ne plus aequo laboremus in re non necessaria: Paulus, meo iudicio, simpliciter iubet, quoties publicae orationes habentur, supplicare et deprecari pro omnibus: etiam qui in presentia nihil nobiscum habent coniunctionis. In gratiarum actione nihil est obscurum. Nam quemadmodum vult infidelium salutem commendari Deo, ita et propter laetos eorum ac prospe ros successus gratias agi. Ista enim admirabilis Dei bonitas, quam quoridie ostendit, cum Solem suum oriri facit super bonos et malos, digna est quam laude prosequamur. Et charitas nostra vsque ad indignos extendere se debet.  \pend
\section*{I. TIMOTH. }
\marginpar{[ p.18 ]}\pstart Pro Regibus.) Nominatim de Regibus meminit et magistratibus reliquis, quia prae aliis exosi esse poterant apud Christianos. Quotquot enim erant illo tempore magistratus, totidem erant quasi iurati Christi hostes. Poterat igitur obrepere ista cogitatio, non esse pro illis orandum, qui totas vires opesque suas conferrent ad oppugnanduni Christi Regnum, cuius propagatio imprimis optanda est. Occurrit autem Apostolus, ac diserte, iubet pro illis etiam precari. Et certe, non efficit hominum prauitas quominus amanda sit Dei institutio. Proinde cum magistratus ac principes Deus ad conseruationem humani generis creauerit: vtcunque multi degenerent a diuina ordinatione: non tamen cessare proprerea debemus, quin et amemus quod Dei est, et saluum cupiamus. Haec causa est, cur debeant fideles, in quacunque regione degant, non modo legibus et magistratuum imperio parêre:sed in suis etiam precibus eorum salutem commenda re Deo. Dicebat Israelitis Ieremias: Orate pro pace Babylonis: quoniam in pace eius pax vestra. Haec vniuersalis est doctrina, vt ordinatas a Deo potestates cupiamus saluas et pacatas stare.  \pend\pstart Vt tranquillam.) Proposita vtilitate, stimulum nobis addit. Fructus enim enumerat, qui ex principatu rite composito nobis proueniunt. Primus est, tranquilla vita. Gladio enim sunt armati magistratus, vt nos in pace contineant. Nisi compescerent improborum audaciam, latrociniis et caedibus omnia essent plena. Haec igitur pacis tuendae ratio, cum vnicuique redditur quod suum est, nec impune grassatur potentiorum violentia. Secundus est, pietatis conseruatio: dum scilicet incumbunt magistratus, ad fouendam religionem, ad asserendum Dei cultum, ad sacrorum reuerentiam exigendam. Tertius est, cura publicae honestatis. Nam hoc etiam fit magistratuum beneficio, ne ad beluinas foeditates se homines prostituant, aut indecore lasciuiant, sed porius vigeat modestia et moderatio. Haec tria si tollantur qualis erit humanae vitae status? Si qua ergo vel publicae tranquillitatis, vel pietatis, vel honestatis cura nos tangit: meminerimus eorum quoque habendam esse curam, quorum ministerio tam praeclara bona ad nos perueniunt. Vnde colligimus, omnis humanitatis expertes essefanaticos homines, nec aliud quam feram barbariem spirare, qui magistratus e medio sublatos cupiunt. Quantûm enim haec discrepant: vt vigeat ius et honestum, vt floreat religio, orandum esse pro Regibus: et non modo regni nomen, sed totam politiam religioni esse aduersam? Atqui prioris sententiae authorem habemus Spiritum Dei. Secundam ergo a Diabolo esse oportet, Si roget quispiam, an etiam pro Regibus orandum sit, a quibus nibil tale percipimus? Respondeo, huc tendere vota nostra, vt Spiritu Dei gubernati, bonorum, quibus antea nos priuabant, mini\pend
\section*{CAP. II. }
\marginpar{[ p.19 ]}\pstart stri nobis esse incipiant. Non ergo pro iis tantd̀m, qui iam digni sunt, orare nos decet: sed rogandus Deus, vt ex malis bonos faciat. Tenendum enim semper istud principium: tam ad religionis, quam ad tranquillitatis et honestatis publicae custodiam destinatos esse a Deo magistratus: non aliter quam terra procreandis alimentis ordinata est. Quemadmodum itaque pro pane quotidiano orantes, Deum rogamus, vt terram sua benedictione foecundet: fic in illis prioribus beneficiis ordinarium medium, quod sua prouidentia constituit, respicere debemus. Huc accedit, quod si beneficiis istis, quorum dispensationem magistratibus Paulus assignat, destituimur: id contingit nostra culpa. Inutiles enim magistratus nobis reddit ira Dei, non secus, ac terram sterilem. Quare nostrûm est, eiusmodi flagella deprecari, quae peccatis nostris irrogantur. Praeterea hic officii sui vicissim admonentur principes, et quicunque magistratum gerunt. Neque enim satis est, si ius cuique suum reddendo, iniurias omnes coerceant pacemque foueant:nisi et religionem promouere, et honesta disciplina mores componere studeant. Neque enim frustra hortatur Dauid, vt Filium osculentur nec frustra Iesaias denunciat, fore Ecclesiae nutricios. Quare non est quod sibi blandiantur, si ad cultum Dei asserendum adiutores se praebere neglexerint.  \pend\pstart Hoc enim bonum est.) Postquam vtile esse docuit quod praecipiebat, iam validius argumentum proponit: Deo ita placere. Nam vbi constat de eius voluntate, instar omnium rationum esse nobis debet. Bonum accipit, pro recto et legitimo. Quia autem Dei voluntas regula est, ad quam exigenda sunt officia nostra omnia: rectum inde esse probat quia Deo acceptum est. Sequitur deinde huius etiam secundi membri confirmatio quia velit Deus omnes homines saluos facere. Quid autem magis aequum, quam vt huic Dei decreto vota nostra subseruiant? Postremd autem, Deo cordi esse omnium salutem demonstrat, quia omnes ad veritatis suae agnitionem vocet. Argumentum est a posteriori. Nam si Dei potentia est Euangelium in salutem omni credenti: certum est inuitari omnes ad spem vitae aeternae, quibus Euangelium offertur. Denique sieut vocatio documentum est arcanae electionis: ita quos facit Deus Euangelii sui panticipes, eos ad salutis possessionem admittit. Quia Euangelium iustitiam Dei nobis reuelat: quae certus est in vitam ingressus. Hinc apparet, quam pueriliter hallucinentur, qui locum hunc praedestinationi opponunt. Si Deus, inquiunt, omnes indifferenter vult saluos fieri: falsum est, alios ad salutem, alios ad exitium aeterno eius consilio esse praedestinatos. Aliquid forte dicerent, Si Paulus hic de singulis homi nibus ageret. Quanquam tunc quoque non deesset solutio. Nam etsi Dei voluntas non ex occultis ipsius iudiciis aestimanda est, vbi exter  \pend
\section*{I. TIMOTH. }
\marginpar{[ p.80 ]}\pstart nis signis eam nobis patefacit: non tamen propterea sequitur, quin constitutum intus habeat, quid de singulis hominibus fieri velit. Caeterùm id, quia nihil ad praesentem locum facit, praetereo. Nam Apostolus simpliciter intelligit, nullum mundi ordinem a salute excludi: quia omnibus sine exceptione Euangelium proponi Deus velit. Est autem Euangelii praedicatio viuifica. Merito itaque colligit, Deum omnes pariter salutis participatione dignari. At de hominum generibus, non singulis personis sermo est. Nihil enim aliud intendit. quam principes in hoc numero includere. Quod autem his quoque Euangelii doctrinam communem esse velit Deus, constat ex testimoniis iam citatis, et similibus. Neque enim de nihilo dictum est: Nunc Reges intelligite. In summa, indicare Paulus voluit, non esse considerandum, quales tunc essent principes: sed quales esse Deus vellet. Est autem officium charitatis, quoscunque vocatione sua complectitur Deus, eorum salutis curam studiumque suscipere: idque piis votis testari. Atque huc pertinet, quod Deum vocat nostrum saluatorem. Vnde enim nobis salus, nisi ex gratuita Dei beneficentia? Atqui Deus idem qui iam nos salutis fecit compotes, ad eos quoque gratiam suam extendere aliquando poterit. Qui iam nos ad se traxit, poterit illos nobiscum adducere. Fadurumi autem pro confesso sumit: quia Prophetarum vaticiniis ita praedictum erat, de vniuersis tam ordinibus, quam nationibus.  \pend
\phantomsection
\addcontentsline{toc}{subsection}{\textit{Vnus enim Deus, vnus et mediator Dei et hominum, homo Christus Iesus, qui dedit semetipsum precium redemptionis pro omnibus, vt esset testimonium temporibus suis, in quod positus sum praeco et Apostolus. Veritatem dico in Christo, non mentior, Doctor Gentium in fide et veritate. }}
\subsection*{\textit{Vnus enim Deus, vnus et mediator Dei et hominum, homo Christus Iesus, qui dedit semetipsum precium redemptionis pro omnibus, vt esset testimonium temporibus suis, in quod positus sum praeco et Apostolus. Veritatem dico in Christo, non mentior, Doctor Gentium in fide et veritate. }}\pstart Vnus Deus.) Parum valida in speciem foret haec ratio: Deum ideo velle omnes saluos esse, quia vnus est. Chrysostomus, et post eum alii sic accipiunt: non esse plures Deos, sicuti idololatrae com miniscuntur. Sed aliud fuisse Pauli consilium arbitror: vt hic sit tacita comparatio vnius Dei cum toto mundo diuersisque nationibus. Quemadmodum et tertio capite ad Romanos: Nunquid Iudaeorum Deus tantùm? An non et Gentium? Imo vnus Deus, qui iustificat Circuncisionem ex fide, et praeputium per fidem. Ergo vtcunque in hominibus tunc esses diuersitas, quia multi ordines alieni erant a fide, multae que gentes: ad vnitatem Dei fideles reuocat Paulus: vt  \pend
\section*{CAP. II. }
\marginpar{[ p.21 ]}\pstart sciant cum omnibus sibi esse coniunctionem, quia vnus sit omnium Deus: vt sciant non in perpetuum a spe salutis exclusos, qui sub eius dem Dei sunt potestate. Idem sibi vult quod de vno mediatore continuo subiicit. Nam sicuti vnus est Deus omnium creator et pater: ita vnicum mediatorem esse dicit, per quem accessus nobis ad Deum patet. Atque hunc mediatorem non vni tantùm genti vel paucis hominibus certae conditionis esse datum, sed omnibus. Nam sacrificii, quo peccata expiauit, fructum ad omnes pertinere. Particula vniuersalis semper ad hominum genera referri debet non ad personas. Acsidixisset: Non solos Iudaeos, sed Gentiles quoque: non solos plebeios, sed etiam principes redemptos esse morte Christi. Cum itaque commune mortis suae beneficium omnibus esse velit: iniuriam illi faciunt, qui opinione sua quempiam arcent a spe salutis.  \pend\pstart Homo Iesus.) Hominem dum praedicat, non negat esse Deum: sed cum vinculum notare vellet nostrae cum Deo coniunctionis, humanae potius naturae meminit, quam diuinae. Quod sedulo obseruandum est. Hinc enim factum est ab initio, vt homines sibi hos vel illos fingendo mediatores, longius a Deo recesserint, quia hoc errore praeoccupati, Deum procul abesse, quo se verterent nesciebant. Huic malo Paulus medetur, cum nobis Deum quasi praesentem sistit: quia ad nos vsque descendit, ne supra nubes quaerendus esset. Idem ergo hic dicitur, quod quarto ad Hebraeos capite: Non habemus Pontificeni, qui compati infirmitatibus nostris nequeat: cum tentatus fuerit per omnia. Et sane si id omnium animis insideret: porrigi nobis fraternam manum a filio Dei, et naturae societate nobis coniunctum, vt nos ex hac nostra tam abiecta conditione in coelum vsque attollat, quis non recta hanc viam tenere mallet, quam in deuiis salebris vagari? Proinde quoties orandus est Deus, si in mentem venit sublimis illa et inaccessa maiestas: ne eius formidine absterreamur, simul occurrat etiam homo Christus qui comiter nos inuitat, nósque veluti manu prehensat, qui Patrem ex formidabili ac tremendo propitium et facilem nobis reddat. Haec sola clauis est, qua nobis ianua coelestis Regni reseratur, vt appareamus cum fiducia in Dei conspectum. Proinde etiam videmus Sathanam omnibus seculis hunc lapidem voluisse, vt inde homines auerteret. Omitto quam variis artibus, ante Christi aduentum, distraxerit hominum mentes, ad media comminiscenda, quibus ad Deum peruenirent. Iam ab Ecclesiae Christianae exordio, cum nuper apparuisset Christus cum tam praeclaro pignore: et adhuc in terris ex eius ore suauissima illa vox propemodum resonaret: Venite ad me omnes, et c.erant tamen quidam fallendi artifices, qui Angelos pro mediatoribus obtruderent ipsius loco: quemadmodum ex Epistola ad Colossenses colligere promptum est.  \pend
\section*{I. TIMOTH. }
\marginpar{[ p.22 ]}\pstart Sed quod tunc clanculum moliebatur Sathan, sub Papatu ed̀ vsque perduxit, vt vix millesimus quisque mediatorem Christum, vel titulo tenus agnosceret. Atque ita sepultum erat nomen, vt res magis esset ignota. Nunc postquam sanos, piosque Doctores excitauit Deus, qui tanquam postliminio reuocare in hominum memoriam studuerunt, quod vnum ex notissimis fidei nostrae principiis esse debuerat: omnia excogitant Romanenses sophistae, quibus rem clarissimam obscurent. Primùm ita illis nomen est inuisum, vt si quispiam Christi mediatoris faciat mentionem praeteritis sanctis, mox suspitione grauerur haereseos. Quia autem repudiare in totum non audent quod hic a Paulo docetur, insulso comniento eludunt: vnum vocari, non solum. Quasi vnum ex magna turba Deum nominauerit. Cohaerent enim haec duo membra: Vnum esse Deum, et vnum mediatorem. Quare qui Christum vnum aliquem ex multis faciunt, idem ad Deum quoque trahant oportet. An eo prorumperent impudentiae, nisi caecus furor ad opprimendam Christi gloriam eos raptaret? Alii sibi videntur acutiores, dum Christum vnicum statuunt mediatorem redemptionis, sanctos autem intercessionis mediatores nominant. Atqui horum quoque insulsitatem coarguit loci circunstantia quandoquidem hic de precibus ex profe sso tractatur. Pro omnibus, inquam, orare Spiritus praecipit, quia vnicus noster mediator omnes ad se admittat: sicuti morte sua omnes reconciliauit Patri. Et adhuc Christiani censeri volunt, qui tam sacrilega audacia Christum spoliant suo honore. Obiicitur tamen quod speciem repugnantiae habeat. Nam hoc ipso loco Paulus iubet nos pro aliis intercedere: quod octauo ad Romanos, Christo quasi proprium assignat. Respondeo, sanctorum intercessores, quibus se mutuo iuuant apud Deum, minime obstare, quominus tamen vnicum omnes intercessorem habeant. Nemo enim vel pro se, vel pro alio, nisi Christi parrocinio fretus exauditur. Quod itaque alii pro aliis intercedimus, adeo non tollit vnicam Christi intercessionem, vt inde maxime pendeat, côque referatur. Putabit quispiam, nos igitur posse facile conciliari cum Papistis, si vnicae Christi intercessioni subiiciant, quascunque ipsi fanctis tribuunt. Non ita est. Nam ideo tranfferunt intercedendi officium ad sanctos, quoniam alias destitui nos aduocato fingunt. Hoc est vulgare inter ipsos principium: nos patronis opus habere, quia indigni sumus, qui per nos appareamus in Dei conspectum. Ita loquendo, Christum honore suo priuant. Deinde execranda est blasphemia, affingere sanctis dignitatem, quae nobis apud Deum gratiam conciliet. A quo tani longe absunt, tam Prophetae omnes quam Apostoli et Martyres, vsque ad ipsos Angelos, vt patrono ipsi quoque eodem nobiscum opus habeant. Deinde merum est figmentum.  \pend
\section*{CAP. II. }
\marginpar{[ p.23 ]}\pstart natum in eorum cerebris, mortuos pro nobis intercedere. Ergo in eo fundare preces nostras, est fidem penitus a Dei inuocatione auellere. Atqui Paulus regulam Dei rite inuocandi praescribit, fidem ex verbo Dei conceptam Ro.1o. Meritd itaque repudiamus, quod curiosi homines extra Dei verbum imaginantur. Sed ne prolixior sim, quam loci expositio flagitat: summa sit, Christo vno fore contentos, qui vere didicerint eius officium: mediatores sibi propria libidine fabricare eorum esse, qui nec Deum nec Christum nouerunt. Vnde consti ruo, Papistarum doctrinam, quae et Christi patrocinium obscurat, imo fere sepelit, et fictitios inducit Patronos absque vsso scripturae testimonio, tum impiae diffidentiae, tum etiam impiae temetitatis plenam esse.  \pend\pstart Qui dedit semetipsum.) Non est superuacua hoc loco redemptionis commemoratio. Sunt enim res necessario coniunctae, sacrificium mortis Christi, et perpetua intercessio: suntque duae sacerdotii partes. Nam hac significatione sacerdos noster Christus vocatur, quod semel peccata nostra morte sua expiauit, vt Deum nobis propitium redderet, et nunc sanctuarium coeli ingressus, ad impetrandam nobis gratiam coram Patre apparet, vt eius nomine exaudiamur. Quo melius detegitur sacrilega Papistarum impietas, qui dum in hoc officio sanctos Christo socios adiungunt, simul ad eos transferunt sacerdotii gloriam. Lege quartum caput ad Hebraeos circa finem et initium quinti. Reperies quod dico, intercessionem, qua propitiatur nobis Deus in sacrificio fundatam esse. Quinetiam tota veteris sacerdotii ratio hoc demonstrat. Sequitur ergo nihil ex officio intercedendi a Christo transeribi posse ad alios, quin sacerdotii titulo nudetur. Praeterea cum αντιλυτρον appellat, alias omnes satisfactiones eueritit. Quanquam non me latet Papistarum argutia: fingunt enini redemptionis preciumi, quod morte sua persoluit Christus, in Baptismo nobis applicari, vt deleatur originale peccatum: postea sarisfactionibus nos reconciliari Deo. Hoc modo, quod erat vniuersale et perpetuum beneficium, ad exiguum tempus et speciem vnam restringunt. Sed plenam huius rei tractationem ex Institutione pete.  \pend\pstart Vt esset testimonium.) Hoc est, vt haec gratia patefieret tempore constituto. Dixerat, pro omnibus. Quae parcicula quaestionem mouere poterat:cur ergo peculiarem vnum populum elegisset Deus, si omnibus communiter se propitium Patrem exhiberet, ac vna omnibus communis esset redempcio in Christo. Huic quaestioni ansam praecidit, cum reuelandae gratiae opportunitatem refert ad Dei consilium. Nam si hyeme non miramur emarcidas arbores, agros niue coopertos, prata gelu rigentia: vere aucem reuirescere.  \pend
\section*{I. TIMOTH. }
\marginpar{[ p.24 ]}\pstart quae ad tempus quodammodo emortua fuerant, quia ordinatae sunt a Deo temporum vices: cur non ipsius prouidentiae in aliis idem iuris concedemus? An propterea Deum inconstantiae arguemus, quia quod semper apud se decretumfixum est habuit, suo tempore profert in me dium? Quare et si hoc repentinum mundo accidit et minime expectatum, quod Christus Iudaeis ac Gentibus promiscue redemptor apparuit: ne tamen putemus subitum hoc fuisse Deo: sed potius admirabili eius prouidentiae discamus subiicere sensus omnes nostros. Ita fiet, vt nihil quod ab ipso prodierit nobis non sit maxime tempestiuum. Ideo frequenter apud Paulum occurrit haec admonitio: praesertim eum de Gentium vocatione agitur, quae sua nouitate multos tunc percellebat, et quasi reddebat attonitos. Quibus hac solutione non satisfit, Deum occulta sua sapientia distribuisse temporum vices: hi aliquando experientur, quo tempore ociosum fuisse putant, fabricasse inferos curiosis.  \pend\pstart In quod positus sum.) Ne temere, vt multi solent, de re sibi parum comperta asserere videatur: se ad hoc praedicat diuinitus ordinatum, vt Gentes prius a regno Dei alienas ad Euangelii participationem adducat. Nam Apostolatus eius ad Gentes, certum erat diuinae vocationis fundamentum. Et ideo tantopere in ipso asserendo laborat: vt certe apud multos non absque difficultate recipiebatur. Ius iurandum vel obtestationem adhibet, vt in re maxime seria, doctorem se esse Gentium: idque in fide et veritate. Quae duo bonam conscientiam significant. Sed eam quoque fultam esse oportet certitudine diuinae voluntatis. Proinde non tantùm sincero affectu se Gentibus Euangelium praedicare intelligit: sed etiam sana et intrepida conscientia: eo quod nihil agat nisi Dei mandato.  \pend
\phantomsection
\addcontentsline{toc}{subsection}{\textit{Volo igitur orare viros in omni loco, sustollentes puras manus, absque ira, et disceptatione. Consimiliter et mulieres in amictu modesto cum verecundia et temperantia ornare semetipsas, non tortis crinibus, aut auro, aut Margaritis, aut vestitu sumptuoso: sed quod decet mulieres profitentes pietatem per bona opera. }}
\subsection*{\textit{Volo igitur orare viros in omni loco, sustollentes puras manus, absque ira, et disceptatione. Consimiliter et mulieres in amictu modesto cum verecundia et temperantia ornare semetipsas, non tortis crinibus, aut auro, aut Margaritis, aut vestitu sumptuoso: sed quod decet mulieres profitentes pietatem per bona opera. }}\pstart Volo igitur.) Pendet haecillatio ex proxima sententia. Nam, vt in Epistola ad Galatas vidimus, Spiritu adoptionis praeditos esse nos oportet, vt Deum rite inuocemus. Postquam ergo Christi gratiam omnibus exposuit, ideôque se Gentibus datum esse  \pend
\section*{CAP. II. }
\marginpar{[ p.25 ]}\pstart Apostolum meminit, vt eodem redemptionis beneficio promiscue cum Iudaeis fruerentur: omnes pariter ad orandum inuitat. Fides enim inuocationem parit. Vnde et ad Romanos 15. Gentium vocationem probat istis testimoniis: Exultent Gentes cum populo eius. Item: Omnes Gentes laudate Dominum. Item Confitebor tibi inter Gentes. Valet enim reciprocum argumentum a fide ad inuocationem, et ab inuocatione ad fidem: tanquam a causa ad effectum, et ab effectu ad causam. Quod obseruatione dignum est: quia inde admonemur, Deum se nobis patefacere verbo suo, vt a nobis in uocetur: et hoc praecipuum esse exercitium fidei. Quare particula, in omni loco, peraeque hic valet, atque initio prioris ad Corinthios: vt nullum sit iam discrimen inter Gentilem et Iudaeum, inter Graecum et Barbarum: quia Deus communis est omnium Pater.  \pend\pstart sustollentes puras manus.) Acsi diceret: modo adsit bona conscientia, nihil obstare, quominus vbique inuocent Deum omnes Gentes. Sed rei loco signum posuit. Nam purae manus sunt puri cordis indices. Quemadmodum econuerso Iesaias exprobrat Iudaeis, quod manus ad Deum sanguinolentas extollant, dum inuehitur in eorum crudelitatem. Porrô ceremonia haec vsitata fuit omnibus seculis: quia natura hic nobis ingenitus est sensus, cum Deum quaerimus, sursum aspicere. Adeóque semper valuit, vt ipsi quoque idololatrae, cum alioqui Deum affigerent ligneis et lapideis simulacris, morem tamen in coelum tollendi manus retinerent. Discamus ergo, ceremoniam esse verae pietati congruentem, modo respondeat, quae per eam figuratur veritas. Nempe, vt admoniti Dumin caelo quaerendum esse, primùm de ipso terrenum vel carnale nihil imaginemur deinde exuamus aftectus carnales, ne quid impediat, quominus animi nostri supra mundum assurgant. Idololatrae vero et hypocritae, manus inter oranduni attollendo, simiae sunt: quia dum externo symbolo profitentur, se mentes sursum habere erectas, priores in ligno et lapide haerent, quasi illic inclusus esset Deus: secundi aut inanibus curis, aut prauis cogitationibus impliciti in terra iacent. Proinde contrario gestu testimonium aduersum se ferunt.  \pend\pstart Absque ira.) Nonnulli fremitum indignationis exponunt, dum secum tumultuatur conscientia, et quasi cum Deo expostulat. Quod acidere solet, cum res aduersae grauius nos premunt. Tunc enim aegreferimus, non statim Deum nobis succurrere, etimpatientia turbamur. Concutitur etiam fides variis insultibus. Nam quia non apparet eius auxilium, obrepunt nobis dubitationes, an curam nostri habeat, an saluos velit nec ne: et similes. Talem consternationem haesitantis animi disceptationis voce notari putant. Ita secundùm eos sensus esset: pacata conscientia et intrepida fiducia  \pend
\section*{I. TIMOTH. }
\marginpar{[ p.26 ]}\pstart orandum esse. Alii, vt Chrysostomus, requiri putant, non tam ergo Deum quam homines placidos animos et omni perturbatione vacuos. Quia nihil est quod puram Dei inuocationem magis impediat, quam rixae et contentiones. Vnde etiam iubet Christus si quis dissideat cum fratre, prius reconciliari, quam munus ad altare offerat. Horum vtrunque verissimum esse fateor. Sed dum praesentis loci cireunstatiam expendo, non dubito quin respexerit Paulus ad disceptationes, quae inde oriebantur, quod Iudaei Gentes sibi aequari indignabantur, et ideo controuersiam mouebant de earum vocatione, imo eas repudiabant, arcebantque a gratiae confortio. Vult ergo Paulus, sedatis eiusmodi contentionibus, omnes vbiuis gentium ac terrarum Dei filios vnanimes orare. Quanquam nihil prohibet, quin ex particulari sententia eliciamus generalem doctrinam.  \pend\pstart Consimiliter et mulieres.) Quemadmodum viros iussit tollere puras manus, ita nunc praescribit, qualiter mulieres ad rite orandum compararas esse deceat. Ac videtur inter virtutes quas commendat, et externani Iudaeorum sanctificationem tacita subesse antithesis. Nam innuit, nullum esse profanum locum, et unde non ad Deum accessus pateat, modo suis vitiis non arceantur, tam viri quam mulieres. Porrd̀ sumpta occasione, vitium, quo fere laborare solent mulieres, corrigere voluit. Fieri quoque potest, vt Ephesi, in vrbe opulenta et celebri emporio, magis grassantum sit. Est autem nimia ornatus cura etcupiditas. Vult autem vt ad pudorem ettemperantiam compositus sit earum ornatus. Nam inde luxuria et immodicus sumptus, vel quod superbiae, vel quod lasciuiae causa, ostentare se appetunt. Atque inde petenda est regula mediocritatis. Nam quia res est indifferens, vestitus vt sunt res omnes externae. diffieile est praescribere certum modum, quousque liceat. Magistratus quidem leges sumptuarias ferre poterit, quibus libidinem vtcunque coerceat. Sed pii doctores, quorum est regere conscientias, semper debent vsus legitimi finem respicere. Ita hoc saltem erit extra controuersiam quidquid non est cum pudore et temperantia consentaneum in vestitu, esse improbandum. Quanquam ab affectu semper est incipiendum. Nam vbi intus regnat proteruia, illic nullus pudor. Vbi regnat intus ambitio, illic nulla modestia in externo vestitu extabit. Sed quia hypocritae vitiosos suos affectus quibuscunque possunt coloribus fucare folent, necessario protrahendi sunt ad id quod apparet. Magnae improbitatis erit negare, honestis et castis mulieribus pudorem conuenire, tanquam proprium et pe rpetuum ornamentum. Similiter, moderationem ab omnibus seruandam esse. Quid quid ergo erit his contrarium, frustra excusabunt. Certas deinde species excessus nominatim reprehendit: vt sunt capilli complexi,  \pend
\section*{CAP. II. }
\marginpar{[ p.27 ]}\pstart Margaritae, et aurea monilia: non quod vetitus sit in totum auri vel Margaritarum vsus, sed quod haec vbicumque refulgent secum, fere trahunt alia mala quae dixi, et ex ambitionis fonte nascuntur aut impudiciuae.  \pend\pstart Quod decet mulieres.) Nam certe cultum honestae et piae mulie ris a meretricio differre oportet. Illae autem sunt discretionis notae, quas posuit. Et si operibus testanda est pietas, in vestitu etiam castoapparere haec professio debet.  \pend
\phantomsection
\addcontentsline{toc}{subsection}{\textit{Mulier in quiete discat, cum omni subiectione. Docere autem mulieri non permitto, neque imperiosam esse in virum, sed quietam esse. Adam enim creatus fuit prior, deinde Eua. Et Adam non fuit deceptus, sed mulier decepta, transgressionis rea fuit. Seruabitur autem per generationem, si manserint in fide, et charitate, et sanctificatione, cum temperantia. }}
\subsection*{\textit{Mulier in quiete discat, cum omni subiectione. Docere autem mulieri non permitto, neque imperiosam esse in virum, sed quietam esse. Adam enim creatus fuit prior, deinde Eua. Et Adam non fuit deceptus, sed mulier decepta, transgressionis rea fuit. Seruabitur autem per generationem, si manserint in fide, et charitate, et sanctificatione, cum temperantia. }}\pstart Postquam de vestitu locutus est, nunc addit, qua se modestia gerere in sacro coetu debeant mulieres. Ac primum iubet eas placide discere. Nam quies silentium significat, ne loquendi vices sibi vsurpent in publico. Quod statim clarius exponit, dum eas vetat docere. Non vt illis auferat munus instituendae familiae: sed tantùm vt a munere docendi arceat, quod solis viris Deus mandauit, Qua de re tractauimus in epistola ad Corinthios priore. Si quis Debotam et similes obiiciat, quas aliquando gubernando populo Dei mandato praefectas legimus: responsio est facilis: extraordinariis Dei factis non euerti communem politiam, cui nos voluit alligatos . Ergo si aliquando prophetandi et docendi locum tenuerunt mulieres, idque Dei Spiritu excitatae: potuit hoc, qui ab omni lege immunis est. Sed quia hoc singulare est, non pugnat cum perpetua et vsitata politia. Addit quod proximum est officio docendi: ne sibi authoritatem sumant in viros. Nam ista est ratio,eur docere prohibeantur: quia hoc non patitur earum conditio. Subiectae enim sunt docere autem est potestatis, vel superioris gradus. Quanquam videtur ratio non satis firma:cum Prophetae quoque et doctores regibus sint subiecti, aliisque magistratibus. Respondeo, nihil esse absurdi, quin praesit aliquis, et simul pareat, secundùm diuersos repectus. Sed in muliere id non valet, quae natura, hoc est, ordinaria Dei lege, ad parendum nata est. Nam γυναικοκρατιαν omnes prudentes semper instar portenti repudiarunt. Quare, coelum quodammodo terrae miscebitur, si docendi ius arripiant mulieres. Iubet ergo eas quiescere, hoc est,se continere intra suum ordinem.  \pend\pstart Adam enim creatus.) Videtur haec parum valida esse subie\pend
\section*{I. TIMOTH. }
\marginpar{[ p.28 ]}\pstart ctionis ratio. Nam et Ioannes Baptista tempore Christum praecessit, dignitate tamen longe inferior. Sed Paulus circunstantias omnes, quae a Moise referuntur, tametsi non exprimat, voluit tamen a lectoribus expendi. Moises autem docet, ita posteriore loco creatam esse mulierem, vt sit quasi viri accessio: et hac lege fuisse viro adiunctam, vt praesto adsit, ad exhibenda obsequia. Cum ergo non duo capita Deus aequa potestate creauerit: sed viro addiderit adiumentum inferius: merito ad illum creationis ordinem nos Apostolus reuocat, in quo relucet aeterna et inuiolabilis Dei institutio.  \pend\pstart Et Adam non fuit deceptus.) Respicit ad poenam mulieri inflictam: Quia obedisti voci serpentis: sub potestate viri eris, et ad illum appetitus tuus. Nam quia exitiale consilium dederat, digna fuit, quae ab alieno iure et arbitrio pendêre disceret: quia virum abduxerat a Dei imperio, digna fuit quae omni libertate priuata, sub iugum redigeretur. Caeterdm non simplicem aut nudam transgressionis causam hic Apostolus obtendit: sed pronunciata a Deo sententia nititur. Sed videntur haec duo nonnihil pugnare: quod mulieris sub iectio poena sit transgressionis: et tamen illi fuerit a creatione imposita. Nam inde sequitur, prius addictam fuisse seruituti, quam delinqueret. Respondeo, nihil impedire, quominus et naturalis ab initio fuerit conditio seruiendi: et deinde propter peccatum accidentalis esse coeperit: vt iam minus liberalis sit subiectio, quam prius fuisset. Caeterùm locus hic quibusdam occasionem praebuit, vt negarent Adam errore lapsum esse:sed vxoris tantùm illecebris fuisse victum. Putant igitur mulierem tantûm astu diaboli deceptam, vt se et virum diis similes fore crederet. Adae vero nihil fuisse persuasum: sed frustum gustasse, vt morem vxori gereret. Atqui hanc opinionem promptum est refellere. Nisi enini fidem habuisset sathanae mendacio: non illi exprobraret Deus: Ecce Adam quasi vnus ex nobis. Sunt aliae rationes quas taceo: quia non longa refutatione indiget error, nulla probabili coniectura fultus. Paulus enim suis verbis non significat Adam non eadem implicitum fuisse Diaboli fallacia: sed tantum causam vel originem transgressionis ab Eua profectam esse.  \pend\pstart Seruabitur autem.) Quia imbecillitas sexus mulieres reddit magis suspitiosas et timidas: et talis erat superior sententia, quae vehementer percellere et consternare vel maxime viriles animos posset: ideo adhibita consolatione, temperat quod dixit. Neque enim nos ideo accusat vel coarguit Spiritus Dei, vt pudore confusis insultet: sed prostratos mox erigit. Poterat hoc, vt iam dixi, exanimare mulieres: eum sibi imputari audirent totius humani generis exitium. Qualis enim est hic reatus? Prasertim cum in subiectione  \pend
\section*{CAP. II. }
\marginpar{[ p.29 ]}\pstart ira Dei assidue versetur ob oculos. Ergo Paulus, vt illas soletur,ac suam illis conditionem tolerabilem reddat, spem salutis relictam esse admonet, vteunque sustineant poenam temporalem. Duplex iste consolationis fructus notandus est: quod proposita salutis spe retinentur: ne reatus sui commemoratione perterritae, in desperationem ruant: deinde quod assuefiant ad tolerandam aequis et tranquillis animis seruiendi necessitatem, vt libenter se viris submittant, cum admonentur hoc genus obsequii etsibi esse salutare et Deo acceptum. Si ad operum adstruendam iustitiam hic locus torqueatur, vt faciunt Papistae:simplex est solutio: quia hic non disputat Apostolus de causa salutis, non posse nec debere ex eius verbis colligi, quid mereantut opera: sed tantùm ostendi, qua nos via perducat Deus ad salutem cui nos sua gratia destinauit.  \pend\pstart Per generationem.) Ridiculum hoc foret hominibus nasutis, Christi Apostolum non modo hortari mulieres ad dandam operam procreandae soboli: sed hoc opus tanquam pium et sanctum ita vrgere, vt rationem esse dicat obtinendae salutis. Quinetiam videmus, quibus probris tlorum coniugalem infamauerint hypocritae, qui sanctiores videri prae aliis omnibus volebant. Sed aduersus sannas impiorum facilis est defensio. Primùm enim non tantùm de gignenda prole hic tractat Apostolus: sed de perferendis molestiis omnibus, quae et durae sunt etmultiplices tam in partu quam in educatione liberorum. Deinde quidquid hypocritae iudicent vel mundi sapientes, pluris hanc obedientiam facit Deus, cum mulier ad quid vocata sit reputans, impositae sibi a Deo conditioni se subiicit, nec taedium gestationis, nec ciborum fastidium, nec morbos, nec difficultatem pariendi, imo potius dirum cruciatum, nec solicitudinem pro foetu, nec alia quae sunt officii sui detrectat, quam si haeroicis alioqui virtutibus se ostentaret, Dei interim vocationi obsequi detrectans. Adde, quod non potuit aptior afferri consolatio, nec efficacior, quam si rationem et medium, vt ita loquar, salutis consequendae in ipsa poena ostenderet.  \pend\pstart Si manserint in fide.) Quia vetus interpres generationem liberorum posuerat: vulgo receptum fuit, vt ad liberos referrent hoc membrum. Atqui vnica vox est apud Paulum, τεκνογονία. Proinde ad mulieres referri necessarium est. Quod autem plurale verbum est, nomen vero singulare, nihil habet incommodi. Si quidem nomen indefinitum, vbi scilicet de omnibus comimunis est sermo, vim collectiui habet: ideóque mutationem numeri facile patitur. Porro, ne totam mulierum virtutem in coniugalibus officiis includeret:ideo maiora adducit. Imo tunc demum generatio gratum est Deo obsequium,cum ex fide et charitate procedit. His duo\pend
\section*{I. TIMOTH. }
\marginpar{[ p.30 ]}\pstart bus adiungit sandificationem, quae totam vitae puritatem continet, dignam Christianis mulieribus. Postremo sequitur temperantia, cuius pauloante meminerat, de vestitu loquens. Sed nunc eam latius ad alias vitae partes extendit.  \pend
\endnumbering\beginnumbering\section{CAPVT. III.}
\phantomsection
\addcontentsline{toc}{subsection}{\textit{\huge\textbf{C}\normalsize ERTVS sermo: Si quis episcopatum appetit, praeclarum opus desiderat. Oportet ergo episcopum irreprehensibilem esse, vnius vxoris maritum, sobrium, temperantem, compositum, hospitalem, aptum ad docendum, non vinolentum, non percussorem, non turpiter lucri cupidum, sed aequum, alienum a pugnis, alienum ab auaritia, qui domui suae bene praesit, qui filios habeat in subiectione, cum omni reuerentia. Quod si quis propriae domui praeesse non nouit: EcCicsiam Dei quomodo curabit? Non nouitium, ne inflatus in condemnationem incidat Diaboli. Oportet autem illum et bonum testimonium habere ab extraneis:ne in probrum incidat et laqueum Diaboli. }}
\subsection*{\textit{\huge\textbf{C}\normalsize ERTVS sermo: Si quis episcopatum appetit, praeclarum opus desiderat. Oportet ergo episcopum irreprehensibilem esse, vnius vxoris maritum, sobrium, temperantem, compositum, hospitalem, aptum ad docendum, non vinolentum, non percussorem, non turpiter lucri cupidum, sed aequum, alienum a pugnis, alienum ab auaritia, qui domui suae bene praesit, qui filios habeat in subiectione, cum omni reuerentia. Quod si quis propriae domui praeesse non nouit: EcCicsiam Dei quomodo curabit? Non nouitium, ne inflatus in condemnationem incidat Diaboli. Oportet autem illum et bonum testimonium habere ab extraneis:ne in probrum incidat et laqueum Diaboli. }}\pstart Certus sermo.) Quod Chrysostomus clausulam esse putat superioris doctrinae, mihi nomplacet. Solet enim Paulus hanc formulam praefando magis vsurpare. Deinde non erat illic tantae affirmationis vsus. Quod autem nunc dicturus est, plus habet aliquanto ponderis. Sit igitur praefatio, ad notandam rei grauitatem  \pend\pstart habita. Si quis episcopatum.) Quoniam mulieribus interdixerat officium docendi sumpta inde occasione, nunc de episcopatu ipso disserit. Primo, quo melius appareat, quam non sine causa mulieres admittere noluerit ad ipsum exercendum: deinde, ne solas mulieres arcendo viros omnes promiscue admittere videatur: tertio quia expediebat, Timotheum et alios admoneri, quanta in eligendis episcopis religio seruanda esset. Talis ergo estcontextus, mea sententia, acsi diceret Paulus, adeo non esse idontas mulieres obeundae tantae functioni, vt ne viris quidem ipsis indiscriminatim patefieri aditum  \pend
\endnumbering
\end{pages}
\end{document}
        
