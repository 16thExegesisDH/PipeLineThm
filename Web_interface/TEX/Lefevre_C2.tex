
%%%%%%%%%%%%%%%%%%%%%%%%%%%%%%%% SCRIPT FOR E-RARA AND MDZ FILES     %%%%%%%%%%%%%%%%%%%%%%%%%%%%%%%%%%%%%%%%%%%%%%%%
%%%%%%%%%%%%%%%%%%%%%%%%%%% fini le 30.04.2025 par F. GOY            %%%%%%%%%%%%%%%%%%%%%%%%%%%%%%%%%%%%%%%%%%%%%%%%
% !TeX TS-program = lualatex
\documentclass{article}
\usepackage[T1]{fontenc}
\usepackage{microtype}
\usepackage[pdfusetitle,hidelinks]{hyperref}

\usepackage{polyglossia}
\setmainlanguage{english}
\setotherlanguages{latin,greek}
\usepackage[series={},nocritical,noend,noeledsec,nofamiliar,noledgroup]{reledmac}
\usepackage{reledpar}

\usepackage{fontspec}
\setmainfont{TeX Gyre Termes}

\usepackage{sectsty}
\usepackage{xcolor}

\usepackage{fancyhdr}
\pagestyle{fancy}
\fancyhf{}
\fancyhead[LE,RO]{\nouppercase{\leftmark}}  
\cfoot{\thepage}
\renewcommand{\headrulewidth}{0.4pt}

% Redefine \section to remove numbering
\usepackage{titlesec}
\titleformat{\section}[block]{\normalfont\scshape\color{gray}}{}{0pt}{} % no number in heading
\titleformat{\subsection}[hang]{\normalfont}{}{0pt}{} % also remove subsection number
\titleformat{\subsubsection}[hang]{\normalfont\footnotesize\color{black}}{}{0pt}{}

% Modify how section marks are stored to exclude numbers
\makeatletter
\renewcommand{\sectionmark}[1]{%
	\markboth{#1}{}} % Only store the section title, without number
\renewcommand{\subsectionmark}[1]{%
	\markright{#1}} % Only store the subsection title, without number
\renewcommand{\numberline}[1]{} % Hide the section number in TOC
\makeatother

\begin{document}

\date{}
        \title{Commentarii in epistolas d. Pauli: [Lefèvre d'Etaples, Jacques ], [1512]}
\maketitle
\tableofcontents
\clearpage
\begin{pages} 
\beginnumbering
        
\section*{1 Timo. }
\section*{x }
\marginpar{[ p.203 ]}\pstart esse cum ei. est enim is cui Paulus gratias agit:et qui eum inualescere fecit qui et ei vires for- titudinemque  suggessit. Paulus. καί χάριν ἔχω τῶ ἐνδυμώσαντί με χριστῶ ἰησοῦ   κυρίω ἡμῶν. ¶LVulgata aeditio. Sed misericordiam dei consecutus sum. ¶ Dei: abundati   etsi nichil intelligentiae officit. Paulus. ἀλλὰ ἠλεήθην. ¶ Vulgata aeditio. Venit in   hunc mundum. ¶Satis est dicere:in mundum. Paulus. ἦλθεν εἰς τὸν κόσμον. ¶ Vulga-   ta aeditio. Ostenderet CHRISTVS IHESVS. ¶Mutatus vocabulorum ordo. dicit enim   Ihesus Christus. Paulus. ἐνδείξηται ἰησοῦς χριστός . ¶ Vulgata aeditio. Regi autem   saeculorunt imortalit inuisibilit soli deo. ¶ Dicendum: soli sapienti deo. Deest eni sapienti. Paulus.  τῶ δὲ βασιλεῖ τῶν αἰώνων, ἀφθάρτω, ἀοράτω, μόνῶ θεῶ. Vt solus sapiens: quia solus nichil permixtum ignorantiae habens. sic solus rex saeculorum: quia solus omnium dominatur. solus immortalis: quia solus absolute interire non potest caetera autem possunt. solus inuisibilis: quia caetera compraehendi possunt (compraehendi enim est quodam modo videri) ipse vero solus compraehendi non potest. Et vt sacer ait Cyprianus. Videri non potest: visu clarior est. nec compraehendi: tactu purior est. nec aestimari: sensu maior est. Et ideo sic eum digne aestimamus: dum inaestimabilem ducimus. Ergo cum Iacob cum Moses dicum- tur deum vidisse: id non designat deum appraehendisse sed aliquod vestigium et certum dei indicium attigisse. non eam ipsam incomprae hensibilem quae incompraehensibilis omnia appraehendit dei naturam atque  substantiam.  \pend
\endnumbering\beginnumbering\section{CAP. II.}\pstart \huge\textbf{N}\normalsize Vnc quo modo ecclesiae ad publicam pietatem sint instituendae: informat Ti- motheum apostolus ac ait. ¶ Obsecro igitur primum omnium: vt fiant preces orationes supplicationesgratiarum actiones pro omnibus homnibus pro regibus et omnibus qui in excellentia sunti vt tramquillam et tacitam vitam degamus in omni pietate et honestate. Nam hoc bonum et acceptabile in conspectu saluatoris nostri dei: qui omnes homines vult saluos fieril et ad agnitionem venire ve- ritatis. ¶Ante omnia vult publicam pietatem praecedere: et id deo gratum esse et acceptabile qui vult omnes homines saluos fieri et ad agnitionem veni- re veritatis. et si venirent: omnes hoies in magna pace et tramquillitate vitam degere possent. si venirent inquam: effectu ipso ad cognitionem veritatis. Nam sunt qui verbo fatentur et facto veritatem negamr: et illi nondum ad perfectam cognitionem sufficientemque  venerunt ad tramquillam et tacitam degendam vitam ineque  in omni pietate et honestate viuunt. Sunt alii qui neque  verbo neque  facto veritatem cognoscunt: veritatem dico quae necessaria est ad salutem. et haec non est dei voluntas: sed propriae voluntatis eorum a veritatis luce incuruatio. Nam qui vnum non co- gnoscunt deum et vnum Christum dominm nostrum pro omni salute mortem oppetiisse non fatentur et mediatorem inter deum et homines vt qui medium sit per quod  omnes homines quotquot salua- buntur saluari oportet: ii nondum ad agnitionem venerunt veritatis etsi omnem scientiam et phi- losophorum et magorum et gymnosophistarum haberent exploratam. Haec eni omnia scire: si ad illud conferas nichil est scire. haec omnia sciens: ad immensas ignorantiae deuoluitur tene- bras efficiturque  tandem perpetuus ignorans. illud vero sciens: ad imensum sapientiae lumem ex- tollitur efficiturque  omnia sciens! et sciens quidem cum vitae et sapientiae thesauris, cuius diui- tiarum nullus est finis. illi vero: cum ignorantia  cumulus erit indeficiens omnis in morte pau- pertatis. Hanc itaque  diuitem sapientiam omni mundi sapientia maiorem: his paucis verbis ex- plicat Paulus dicens. ¶ Vnus eni est deus. vnus et mediator dei et hominum homo CHRI- STVS IHESVS: qui dedit semetipsum redemptionem pro omnibus. ad qd̃ testimonium propriis   temporibus positus sum ego praeco et apostolus (veritatem dico in CHRISTO haud mem-   tior) doctor gentium in fide et veritate. ¶ Ad testimonium illius ditissimae sapientiae qua vitam et sa- lutem praebet animabus in mansionibus lucis aeternae  in plenitudine temporum a deo definita: positus est Paulus praedicator et legatus non a seipso (nam nullus apostolus! nullus lega- tus: a seipso) sed ab eo cuius est legatus. itaque  a deo et Christo: a quo et doctor gentium fa- ctus esti non ad vllam disciplinam mundil quae nichil facit ad salutent sed ad doctrinam coelestem  ad doctrinam spiritus  quae sola saluat et quam angeli magnifaciunt  et docerent si quam docerent. doctor (inquit) gentium in fide et veritate. Quid hoc in fide et veritate: nisi in fide vera  in fide veritatis in fide filii dei  qui est via veritas et vita? Non igitur se effert! non gloriatur cum dicit veritatem dico haud mentior: positus sum doctor gentium in fide et veritate  sed testimonium reddit veritati ei gloriam dans qui fecit eum praeconem suum  apostolum  et doctorem gentium. Cum itaque  sit constitutus a deo doctor gentium et doctoris officium sit doce- re : quod  nunc eum viros? tum mulieres doceat audiamus. Docet itaque  primum viros hocpacto    dicens. ⁋ Volo igitur vt viri in omni loco orent: leuantes sanctas manus sine ira et contentione. ⁋Quid est in omni loco: nisi in omni ecclesia vbicumque  locorum in terra cognoscitur Christus? Sine ira et contentionc. Nam et dominus mandat: si offers munus tuum ad altare et ibi  \pend
\section*{COM. }
\marginpar{[ p.2 ]}\pstart recordatus fueris quia frater tuus habet aliquid aduersum tetrelinque munus tuum ante altare et vade prius reconciliari fratri tuo. et tunc veniens: offeres munus tuum. Quo mo- do enim audes te praesentare domino pacis: non habens signum pacis in mente tua sed oppo- situm signum quod est discordiae? Ergo nullo irato concitatoque  animos neque  qui dissentio- nem cum altero habet oratione maxime publica orare audeat. priuata autem vt deus adiuuet iraque  sedetur et dissentio cesset pacemque  suam restituat: orare nichil prohibet. is enim facit: quod aeger ad medicum suum languorem detegens et opem implorans. et haec:non publica oratio est quae intercessio dicitur quam sine ira et contentione deo offerre opor- tet. sed priuata: et quaedam (si sic dicere mauis) confessio suique  accusatio. Docet et mulie-   res dicens. ¶ Identidem et mulieres in amictu mundolcum verecundia et modestia ornan-   tes seipsas (non i tortis crinibus aut aurotaut margaritis aut veste sumptuosa: sed quae   decet mulieres profitentes diuinam pietatem per bona opera. Mulier in silentio discat:   cum omni subiectione. Mulieri autem docere non concedo:neque  viri authoritatem habere    sed esse in silentio. Nam Adam primus plasmatus est: deinde Eua. Et Adam non seductus   est: sed mulier seducta  in praeuaricatione fuit. Saluabitur autem per filiorum generationens   si manserint in fide et dilectione! et sanctitate cum modestia. ¶ Haec sancti apostoli do- ctrina pro mulieribus. Mulierum habitus mundus sit:non sumptuosus. ornamentum mulie- rum sit verecundia   et modestiat siue castitas. talis enim mulieres spirituales decet or- natus: adiunctis bonis operibus per quae diuinam pietatem profiteri videantur. Cunctis mulie- ribus ex inobedientia primae mulieris ingenita esse debet verecundia quae eas retrahat a per- petratione malorum vetitorum. ad quae ex imitatione primae parentis secundum carnem pro- nae feruntur. Et ex virgine  culpae parentis Euae reparatrice: debet omnibus mulieribus virginitatis amor et appetitus inesse. quam si cum foecunditate habere non possunt:stu- dere tamen debent habere foecundam castitatem quae virginitatis proxime aemula est- Ergo ex prima natiuitate verecundiam: ex secunda virginitatem aut saltem castitatem ha- bere debent. hanc: theosebias id est erga deum pietatis  et purus ac syncaerus dei cultus sequitur. illam vero: cautio fugaque  malorum. Mulier non putet ornamentum in crinibus tortis aut aurosaut margaritis aut veste sumptuosa:qualis luxus nostra tempestate multas oc- cupat cum alibil tum in vrbibus insubriae si forte contingit illuc iter facere. quae manifeste filiae Belial potius iudicari possunt: quam filiae Christil nisi quis oculos praestigiatos. et fasci- natos habeat. Mulier si ad virum conferatur qui caput est: haec vt appetitus estrille vero vt ratio. at appetitui non conceditur vt doceat: sed rationi. ita neque  mulieri: sed virorett maxi- me in publico. appetitus: a ratione formatur et discit. ita et mulier: a viro discere debet. mulier dico: quae vxor sit. quamquam et omnes in publico orationis loco:a viro qui communem omnium habet informationis vicem audire possunt in silentio. non qu temerarie veniant etiam vir- gunculae ad os publici praeceptoris: aliqua quesiturae. interrogent enim aut patres aut ma- tres aut sapientiores matronas: quae melius quam ipsae potuerunt publicam vocem intelligere. Mu- lier: nunquam ausit authoritatem supra virum caperet id est dominari proprio viro. Nam hoc diuinae ordinationi reluctatur  atque  voluntati eius qui eam posuit sub viri potestate et virum fecit dominum, sub viri (inquit) potestate eris: et ipse dominabitur tui. O generis insoten- tia et opulentia dotis: quot mulieres fecit stultas quae authoritatem virorum sibi vendicanti seipsasi sexumque  suum  et dei voluntatem ignorantes  superbae  stultae  insanae  quantuncumque  subli- mes se existiment non hominibus sed deo rebelles? qui enim diuinae ordinationi resistit: deo resistit. estque  deo rebellis: similis angelo tenebrarum qui hac de causa e coelis ruit in abyssum. Et mulierem viro deferre debere: vel ex ipsa creatione constat. Nam vir primo creatus est- tanquam primogeniturae sibi vendicans honorem. mulier vero secundo loco. Et mulier seducta fuit: non vir. persuasa enim fuit a serpente: non vir. absente viro: deserpsit fructum vetitum, man- dit: et viro eius rei ignaro porrexit mandendum. mandit porrectum: vnde generi nostro mise- ra irrepsit mortis contagio. Attamem cum magis obnoxia culpae q̃ vir teneret: voluit nichilo- minus deus tam viro q̃ mulieri lalutem esse communem. mulieri quidem ob liberorum procreationem (ad id eni necessaria viro est) et id maxime dum vir et mulier in fide manserint et dilectio- ne etsanctitate cum modestia atque  castitate. castum enim tandiu thorum habent: quamdiu carnem suam per fornicationem neque  hicineque  illa diuidunt. Alioqui si fides non assit non dilectiot non san- ctimonial vitaeque  puritas et thori castitas: procreatio filiorum mulierem non saluabit. Nam preciosa ad saluandum pignora fides dilectio  sanctitas castitas: magis quam carnalis pro- les. Id tribuente domino nostro omnium saluatore: qui erat  et est et erit benedictus sine fine et in omnia saeculorum saecula. AMEN.  \pend\pstart   ¶ EXAMINATIO nonnullorum circa literam. ⁋ Vulgata aeditio. In omni sanctitate et   castitate. ⁋ Vocabulum quod  hic vetus interpres castitatem dicit:et castimoniam et grauitate  \pend
\section*{1 Timo. }
\section*{x }
\marginpar{[ p.204 ]}\pstart et honestatem pro loco interpretari solent. Paulus.ἐν πᾶσι ἔυσεβεία, καὶ σεμνό-   τητι.  ¶ Vulgata aeditio. Cuius testimonium temporibus suis confirmatum est: in quo   positus sum ego. ⁋ Potius dicendum est: ad quod testimonium. Et abundat confirmatum est. Paulus. το μαρτύριον καίροις ἰδίοις ἐις ὃ ἐτέθην ἐγω. illud enim ἐις ὃ   reuocaui ad orationis caput. ¶ Vulgata aeditio. Veritatem dico  non mentior. ¶Codi- ces Graeci habent: veritatem dico in Christo. Paulus. αλήθειαν λέγω ἐν χριστῶ , ὀυ ψέυδομαι. ¶Vulgata aeditio. Leuantes puras manus sine ira et disceptatione, ⁋ Sunt qui volunt hic positum vocabulum dubitationem et haesitationem significare. verum si a διαλέγομαι venit: non male disceptationem  disputationem dissentionem et contentionem quandam significat. Et cum iram et contentionem inuicem habemus: qu tunc manus ad deum non sint leuandae praeceptum est saluatoris. Si offers (inquit) munus tuum ad altare et ibi recordatus fueris quia frater tuus habet aliquid aduersum te  relinque ibi munus tuum ante altare et vade prius reconciliari fratri tuo: et veniens of- feres munus tuum. Cum enim puros conspectui diuino nos praesentamus: exaudit nos pro clementia sua qui audire exaudire  et superexaudire vult  vt par est et oportet pa- ratos. Paulus. ἐπάιροντας ὁσίους χεῖρας, χωρὶς ὀργῆς καὶ διαλογισμοῦ. ⁋ Vulgata aeditio. Cum verecundia et sobrietate ornantes se. ⁋ Quod hic vetus inter- pres dicit sobrietate: vsitatius modestia dici potest. Vult etiam sacer Hieronymus casti- tatem dici ac modestiam. Paulus. μετὰ ἀιδοῦς καὶ σωφροσύνης κοσμεῖν ἐαυ-   τάσ. ¶ Vulgata aeditio. Sed quod decet mulieres promittentes pietatem per bona ope- ra. ¶ Dicendum esset quae  foeminine:non quod  neutro genere. Nam hoc relatiuum re- fert veste. quod ex graeco cognoscitur vbi masculinum est referens ἱματισμῶ mascu- linum. Et non generale pietatis nomen ponitur sed speciale: quod solum ad deum desi gnat pietatem. Paulus. ἀλλ ὅς  πρέπει γυυαιξὶν ἐπαγγελομέναις θεοσεβειαν   Δἰ ἔργων ἀγαθῶν. ¶Vulgata aeditio. Neque  dominari in virum: sed esse in silentio. ¶Potius:neque  viri siue in virum authoritatem habere. Paulus. ὀυδὲ ἀυθεντεῖν αν- δρός, ἀλλ’ εἶναι ἐν ἠσυχία. Verum authoritatem viri siue in virum habere  et do- minari viro:idem possunt. At quot sunt nunc matronae ex genere tumentes aut ex di- uitiis opulentioreque  dote  quae nunc in viros authoritatem habenti facientes dei senten- tiam irritam propter fortunae ventum ventoque  vaniorem fortunam. O caeca fortuna: quae non solum caeca sed quae vniuersum caecauit saeculum et cum qua veritas apostolica lucere   non potest. ¶Vulgata aeditio. Si permanserit in fide et dilectione  et sanctificatione cum   sobrietate. ¶ Et hic quoque  dici potest cum modestia aut castitate: vt in prima huius ca- pitis examinatione dictum est. Et in principio particulae dicendum est pluratiue: man- serint siue permanserint, respicit enim communiter vtrunque  Paulus  et virum  et mu- lierem. Paulus. ἐὰν μείρωσιν ἐν πίστει, καὶ ἀγάπη, καὶ ἀγιασμῶ, μετὰ ςω- φροσυνης.  \pend
\endnumbering\beginnumbering\section{CAP. III.}\pstart \huge\textbf{V}\normalsize T doctor gentium docuit apostolus viros et mulieres qui funt parentes carnales:nunc episcopos docet qui parentes sunt spirituales. et quales esse debeant et quae eorum officia ostendit, dicuntur eni episcopi a superintenden- di officio. superintendendi dico in omni spumalitate vitae: quod sanctum opus atque  officium est. Ac proinde dicit. ¶ Fidelis sermo. Si quis episcopen desy- derat: bonum opus desyderat. Oportet igitur episcopum: irrepraehensibilem esse  vnius vxoris virum  vigilem castum ornatum hospitalem docilem  non vinosum  non percussorem non turpis lucri sectatorem: sed mitem  im-    bellem  non auarum propriae domui bene praesidentem: filios habentem in subiectione cum   omni honestate (si quis autem propriae domui praeesse nescit: quo modo ecclesiae dei cu-   ram administrabit?) non nouicium: ne superbia elatus in iudicium incidat diaboli. Opor-   tet autem ipsum etiam testimonium bonum habere apud eos qui foris sunt:ne in vituperatio-   nem incidat et laqueum diaboli. ⁋ Episcope: visitatio  intendentia  obseruatio episcopi in suos dicitur. sicut episcopus: explorator intendens obseruator et custoai vt pastor gregis aut vt in alta specula vigilans custos ciuitatis. pulchrum sane nomem:sed rei nominis execu- tio pulchrior quod officium episcopi est. quod  cum tam pulchrum et venerandum sit: episcopum quam illi officio illique  operi praeficitur  irrepraehensibilem esse oportet, repraehensio: per omnia vicia currit superbiam iram  ĩuidiam  acediam  gulam luxurians auaritiam. si aliquo horum viciorum notatus est aut aliqua specierum quae innumerabiles sunt: non est irrepraehensibilis. Quare neque  omnino dignus: huic committi officio. Bigamus qui duas duxisset vxores:non est idoneus  vt assumatur in episcopum. Nam carnem suam diuisit: et vnitatem in symbolo non custodiuit  q̃ vni- tatem Christi et ecclesie rep̃sentat. Ideo in illa vnitate: sancti fiatuerunt eum non praeficiendum  \pend
\endnumbering
\end{pages}
\end{document}
        