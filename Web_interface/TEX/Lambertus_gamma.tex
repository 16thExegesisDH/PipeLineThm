% !TeX TS-program = lualatex
\documentclass{article}
\usepackage[T1]{fontenc}
\usepackage{microtype}
\usepackage[pdfusetitle,hidelinks]{hyperref}

\usepackage{polyglossia}
\setmainlanguage{english}
\setotherlanguages{latin,greek}
\usepackage[series={},nocritical,noend,noeledsec,nofamiliar,noledgroup]{reledmac}
\usepackage{reledpar}

\usepackage{fontspec}
\setmainfont{Liberation Serif}

\usepackage{sectsty}
\usepackage{xcolor}

\sectionfont{\normalfont\scshape\color{gray}}
\subsectionfont{\normalfont}
\subsubsectionfont{\normalfont\footnotesize\color{black}}

\lineation{section}

\begin{document}

\date{}
        \title{In D. Pauli priorem Epistolam ad Timotheum commentarius [Geneva] : [Lambertus Dannaeus], [1577]}
\maketitle
\tableofcontents
\clearpage
\begin{pages} 
\beginnumbering
\section{AD LECTOREM}       
\marginpar{[ p. ]}IN PRIOREM EPISTO- D.  PAVLI lam ad Timotheum com- IN QVO NON SOLVM IP- mentarius. sius Epistolae doctrina, et artificium singulorum- que argumentorum loci explicantur: sed etiam vera Disciplinae Ecelesiasticae forma, tum ex dei verbo, atque ex ipso Paulo, tum ex veteribus synodis repetita atque restituta est, vt legitima et Apostolica regendae Dei Ecclesiae ratio et norma hodie in vsum reuocari possit. PER LAMBERTVM DANAEVM Accessit operi duplex index. APVD EVSTATHIVM VIGNON GENEVÆ M. D. LXXVII. \pstart AD LECTOREM. Quod saepius, Lector optime, iuris Pontificii cano- nes, et Gratiani Decretum citauimus, ea solum mente fecimus, ut ex his tanquam ex quadam illius temporis historia, quae veteris Disciplinae Ecelesiasticae vestigia tunc adhuc (et si tamen corruptioribus iam Ecclesiae tem- poribus ) restarent, intelligatur:non vt ex istis scripto- ribus quicquam dubium et anceps decidatur. Nam etiam ex hoc ipso perspicient omnes, quantùm poste- riori aetate Romana Ecelesia etiam ab eo tantillos quod veteris disciplinae reliquum erat) discesserit: et in quam miseram horribilemque tyrannidem et confu- sionem iam, iusto Dei iudicio, inciderit, vt ab ea nos summo Dei Optimi Maximi beneficio liberatos esse laetemur, Deoque ipsi perpetuo gratias agamus.  \pend\pstart ILLVSTRISSIMO ET PO- TENTISSIMO PRINCIPI ac domino D. Gulielmo Principi Auracio Comiti Nassauao etc.  Christi gloriae assertori in Hol- landia et Zelandia for- tissimo, Gratiam et pacem a Do- mino.  \pend\pstart 
\section{EPISTOLA. }
\textbf{V}Etus est illud, Princeps Illu- strissime, etiam ab Aristotele vsurpatum, Nihil esse in ipsa rerum vniuersitate pulchrius ordine. Quid enim vel oculis iucundum, vel menti etiam ipsi et animo hominis gra- tum obuersari potest, quod sit confusum, et nulla partium apta separatione distin- ctum? Certe quocunque non tantum ocu- los, sed omnes animisensus conuerterimus, si quae in eos incurrunt, neque ordine di- gesta, neque apto siru inter se cohaerentia, neque conuenienti loco et modo collo- cata a nobis apparebunt: ea neque vtilita- tem, neque venustatem aliquam habere statim pronuntiabimus, tantûmque ab il- lis oculorum, animique intuitum auer-  \pend
\section*{EPISTOLA. }\pstart temus, quantum ea nos ad se rapiunt, quae commoda, |propriaque ratione|, et disposi- tione distinguuntur. Hoc in Regno, hoc in Rep. hoc in oppidis, hoc in pagis, hoc in priuatorum aedibus, hoc in hortis, et cultis sedibus, hoc in solitudine, hoc in rebus que natura gignuntur, hoc in artificiis quae hominum industria efficiuntur, verumesse ipsa rerum experientia, et publica mortalium omnium, non tantum piorum, sed etiam profanorum hominum vox testatur: deni- que Mundus ipse, pulcherrimum Dei opus, ab ordine κόσμος nominatur. Quo magis detestanda est eorum sententia, qui rerum confusione non tantum apud se delectan- tur: sed omnia publica priuataque euerte- re, et perturbationem (iusto ordine subla- to) in res omnes inducere conantur, quos homines ἀτακτους et Ἀναρχικους, tanquam mon- stra, et generis humani ἀλάςτορας, iam pri- dem veteres appellarunt. Caeterum si qua in re alia ordinis obseruatio diligens non tantum vtilis, sed etiam omnino necessaria est, maxime in Ecclesia Dei locum habere ex Pauli Apostoli praecepto debet. Est enim Ecclesia domus Dei, id est summi illius regum Regis, in qua idcirco omnia circum- specte, apte, decenter, et, vt Paulus ipse lo-  \pend
\section*{EPISTOLA. }\pstart quitur, ἀσχημόνως καί κατα τάξιν fieri oportet: nihil autein confuse, et pertubate, ne ex ista domus turpitudine ad ipsum patrem familias, id est, ad Deum Optimum Maxi- mum aliqua labes quodammodo redun- det. Itaque ipse eâdem benignitate qua in sacratissimo suo verbo omnia, quae sunt ad salutem aeternam, ipsiusque Eccle- siae suae fundamentum necessaria, praescrip- sit: it a et iam quae sunt ad eiusdem Eccle- siae ordinem, et ornamentum vtilia, de- monstrauit, nequa deesset pars ad verae Ec- clesiae constitutionem necessaria, quam non ab ipsius ore et verbo abunde disce- remus. Vt vero hae praeceptiones, quae Ecclesiae necessarium ordinem continent etiam nomine ipso ab iis distingueren- tur, quae doctrinam fidei complectuntur, generaliter disciplinae Ecclesiasticae vo- cabulo vulgo sunt appellatae. Quemad- modum enim ea verbi Dei pars, et ea ve- ritatis cognitio, quae quid de Deo Patre, de Christo, eiusque naturis et officio, de Spiritu Sancto, de causis nostrae iustifica- tionis, de vita aeterna, nobis credendum sit, ostendit ossa et neruos, ipsamque adeo verae Ecclesiae, tanquam corporis οὐσίαν, et constitutionem continet et aedificat.  \pend
\section*{EPISTOLA. }\pstart Sic illa eiusdem verbi Dei pars, quae disci- plinae Ecclesiasticae praecepta tradit, hu- ius Ecclesiae, velut corporis iam suis membris constantis, sed nudi vestimentum, cultum, et ornamentum compingit et consarcinat, quo decore vestiri possit, vt ex eo quanta sit vtriusque, id est, doctrinae et disciplinae Ecclesiae inter se consensio coniunctióque, satis iam intelligi possit. Erit enim coetus hominum, etiam eorum, qui puram Dei praedicationem amplexi sunt, velut nudum quoddam, et suis ornamen- tis carens, et destitutum corpus, nisi disci- plinam Ecclesiasticam ad veritatis lucem et professionem adiunxerit. Denique illa eadem disciplina recte custos ipsius fidei, doctrinae, mater virtutum, vinculum ho- nestatis, neruus societatis Christianae a Patribus Ecclesiasticis appellatur: cuius certe tanta vis est, vt sola illud conciliet Ecclesiae, quod decorum, venustatem, splendorem, et ornamentum appellamus, adeóque speciosam eam reddat, vt ab hae- reticis (qui istius disciplinae fuerunt sen- per hostes maximi et infestissimi ) inuidio- so nomine lenocinium vt ait Tertull. Ecclesiastica disciplina diceretur. Cum sit igitur haec disciplina illa iustitiae  \pend
\section*{EPISTOLA. }\pstart pars et regula, per quam externus ordo, et totius Ecclesiae legitima administratio praescripta est, quis eam spernendam ne- gligendamue vnquam existimauerit? Cuius institutio non tantum sub nouo testamen- to fieri coepta est: sed et iam in lege veteri a Deo ipso iam olim tradita fuit, vt in li- bello de Politia Iudaica doctissime iam pri- dem docuit Cornelius Bertramus Hebrai- cae linguae professor eruditissimus, et col- lega meus. Quid enim aliud sibi volunt illa Dei praecepta, quae Leuiticam tribum a reliquis distinguuntvt ex ea sola mini- stri templi eligantur: quae in ipsa tribu Leui- tica iura Sacerdotum et Leuitarum discernunt, quae quanta summi Sacerdotis, quanta alio- rum sit potestas describunt? quae qui ritus cui- que sacrificiorum generi, quae libamina, quae oblatio sit necessaria deffiniunt? denique tota ipsa lex ceremonialis vetus a Deo ipso dictata, quid aliud, obsecro, est, quam amplissima et luculentissima disciplinae Ecclesiasticae descriptio, non tantum autem quaedam illius delineatio? Nec vero mi- nus benigne Deus per Apostolos sub nouo testamento eandem disciplinam nobis instituit vel potius collapsam in- staurauit et restituit: vt pote omnino ad re-  \pend
\section*{EPISTOLA. }\pstart cte constituendam regendamque suam Ecclesiam necessariam, quam sub veteri descripserat. Id quod cum ex variis aliis noui Testamenti locis apparet: tum maxi- me ex libro Actorum, ex prima Pauli episto- la ad Corinthios, ex hac ad Timotheum, et vna illa, quae est ad Titum scripta:in qui- bus omnibus quicquid ad legitimam Ec- clesiae gubernationem, et disciplinam Ec- clesiasticam maxime pertinet, siue totam ipsam Ecclesiam, siue priuatos regendos spectes in qua illud totum accurate et co- piose descriptum est. Vnde mirum et ab- surdum saepe mihi videri solet, quod tot postea canones, tot synodorum decreta, tot Ecclesiae constitutiones, tum genera- les, tum particulares ad constituendam disciplinam Ecclesiasticam fuerunt con- scriptae. Quid enim his fuit opus? Sed ma- gno contemptu et ignorantia eorum pre ceptorum, quae ab Apostolis ipsis, id est, a Dei spiritu antea tradita fuerant id certe factum fuit, quod haec diuina praecepta sper- nerent homines, vt sua commenta, vt tradi- tiones, saepe etiam ineptissimas, at discipli- nae ipso Dei verbo praescriptae contrarias in Ecclesiam Deil inueherent. Quid igi- tur, dicet quis, estne verbo Dei, et in no-  \pend
\section*{EPISTOLA. }\pstart uo Testamento disciplina haec Ecclesiasti- ca tam accurate tradita, vt nihil iam addi possit: aut ea authoritate constituta, vt ab ea sine piaculo, discedi non debeat? Certe vt huic quaestioni satisfaciam, distinguen- dum dico. Sunt enim plures istius Eccle- siastice disciplinae partes, quarum alia gene- ralem tantum Ecclesiae (quacumque orbis terra rum parte esse potest) et fundandae et retinen- de, et regendae modum continet, et describit, veluti in Dei Ecclesia vocem Domini no- stri Iesu Christi audiendam esse eamque so- lam: Sacramenta piis tantum et fidelibus danda esse et caet. huiusmodi, quae generali- ter sunt conseruandae Ecclesiae praecepta. Alia magis particulares illius administran dae normas et praecepta tradit et perse- quitur. Haec vero posterior disciplinae Ec- clesiasticae pars haec 4. capita complecti vi- detur: nempe. I. Personarum, quae ad Eccle- siasticum ali quod munus vocandae sunt, electionem, 2. Muneris et officii Persona- rum iam electarum descriptionem:3. Cen surae, morumque correctionis, quae tum in consistorio tum extra consistorium fieri et applicari singulis debet, rationem. 4. Eo- rum qui censura Ecclesiastica affecti sunt, et Ecclesiae iudicio abstenti, vel excom-  \pend
\section*{EPISTOLA. }\pstart municati dicuntur, cum eadem Ecclesia- reconciliationem. Cum igitur de hoc quae- rimus, vtrum omnium horum capitum singula sic praecepta tradita sint, vt his nihil addi possit, respondeo, in quaque re quaedam esse οὐσιώδη, quaedam vero συμβεβηκότα. Sunt οὐσιώδη id est essentialia, quae rei ip- sius vel tota natura sunt, aut illius pars: ve Iuti in hoc ipso argumento, Si quis neget in pastorum aut eligendorum vocatione Iegitima eorum vitam et doctrinam exa- minandam, vel approbationem a populo expectandam esse, ille ea negat, quae sunt Electionis legitimae οὐσιώδη. Si quis item hoc examen non ab Ecclesia, non a pluri bus, sed ab vno quopiam solo fieri et me- ro vnius arbitratu stare debere doceat, is οὐσιώδη electionis tollit. Sed si quis hoc vel illo loco, hoc vel illo tempore, hoc vel illo modo examen a pluribus  fieri debere dispu- tet, is περί συμβεβηκότων tantum disputat, ve- luti, exempli gratia vetus electio et examen pastorum olim a tribus vicinioribus episco- pis seu pastoribus conuocatis cum presbyte- rio et Ecclesia (cui dandus erat Pastor) fie- bat: item in metropolitana tantum vrbe, certo tempore, nempe ieiunii publici tempore. De- nique manuum eleuatione (quae χειροτονία  \pend
\section*{EPISTOLA. }\pstart dicebatur) non solo consensu populi, aut solo vocis suffragio, sane ista omnia sunt accidentaria, non essentialia in electio- ne. Sic igitur statuimus, quae verbo Dei fieri et obseruari ad regendam Ecclesiam iubentur, siue illa de personis eligendis, siue de iam electarum munere, siue de re- Iiquis disciplinae Ecclesiasticae capitibus agant, illa, inquam, esse οὐσιώδη. Quae au- tem in quaque vel prouincia, vel Ecclesia ad haec ipsa praestanda, et ad praxin dedu- cenda commodiora iudicantur, et consti- tuuntur, ea esse συμβεβηκότα quaedam, velu- ti vt hoc potius quam illo tempore fiat ver bi Dei concio: hoc potius quam illo pres- byterorum vel diaconorum numero con- tenta sit haec Ecclesia, hoc est συμβεβηκoς minimeque οὐσιῶδες. Iam igitur ad quae- stionem propositam ex hac distinctione sic respondemus, vt quae sunt disciplinae Ecclesiasticae οὐσιώδη augeri minuiue no- uis praeceptionibus, vel constitutionibus hominum posse negemus: quae sunt autem σομβεβηκότα, et augeri et minui et tempe- rari a quaque non tantum prouincia, sed etiam coetu et Ecclesia pro sua vel com- moditate vel incommoditate posse: pro varia denique locorum, rerum, temporum,  \pend
\section*{EPISTOLA. }\pstart et personarum circunstantia. Eodemque modo respondemus ad alteram illam quo- rundam petitionem, in qua quaerunt vtrum Ecclesia a constituta semel per Apostolos disciplina recedere possit, ac vtrum sit v- na et eadem disciplinae Ecclesiasticae in omnibus euangelicis Ecclesiis forma re- tinenda? Nam si de essentialibus discipli- nae Ecclesiasticae agunt, dicimus illa ab o- mnibus Ecclesiis aequaliter eademque es- se et seruanda et retinenda, veluti vt sit re- gimen Ecclesiae non Monarchicum, sed A- ristocraticum: vt sit in quaque Ecclesia sena tus et coetus aliquis Presbyterorum ab Ecclesia electorum, sine quo nemo vel Pastor, vel alius quisquam negotia Eccle siae gerat : sit aliqua scelerum et morum vitiosorum censura, eaque non vnius ali- cuius arbitratu, sed ἡγουμένων, et vt ait Hie- ro. Presbyterorum, aestimatione et iudicio fiat: sit denique excommunicatio, non ex quorundam Iurisconsultorum et pragma- ticorum, sed ex Dei verbo iudicantium hominum sensu et iudicio, lata in non poe- nitentem, approbante et collecta Ecclesia. Itaque οὐσίαν disciplinae Ecclesiasticae in o- mnibus Ecclesiis reformatis et Euange- licis eandem esse oportere ingenue, et a-  \pend
\section*{EPISTOLA. }\pstart perte respondemus. Sin autem de exter- nis et accidentalibus huic disciplinae rebus quaeritur, in quibus commoditas incom- moditâsque praxeos, vsus, et executionis disciplinae istius versatur, dicimus illas va- riari posse, atque etiam saepe oportere pro- pter diuersam diuersorum temporum, re- rum, et personarum rationem : nedum vt loci et regiones diuersae eandem semper disciplinae formam patiantur, aut requi- rant. Itaque quae hoc loco, et in his regio- nibus commode geruntur (vti in Gallia Coenae Dominicae administratio per sin- gula anni quadrimestria) in aliis fortasse regionibus commode iisdemq́ue temporibus non fierent aut celebrarentur. Sic in ali- qua Ecclesia excommunicatio ex scripto fiet, in alia vero potius ore ipso Pastoris pronuntiabitur. Sic in hac Ecclesia eius qui post professionem Christi fuit idololatra et transfuga, posteaque ad Ecclesiam rediit confessio die tantum dominica fiet, in illa quotidie: hic ex scripto, alibi ore ipsius poenitentis. Haec igitur omnia diuersam formam habere possunt, modo tamen coercitio morum, agnitio culpae, separa- tio hoedorum ab agnis in domo Domini semper retineatur, fiatque quod fieri prae-  \pend
\section*{EPISTOLA. }\pstart cipit Deus, Denique modo sit Ecclesia et Dei domus εὔτακτος, non autem sentina im- proborum, sed recte administrata familia. Substantia igitur disciplinae vbique in Ecclesiis Dei eadem esse debet: accidens autem illius mutari potest. Ergo diligen- ter quid essentiale, quid accidentale sit, distinguendum esse respondemus. Porro nemini, ac ne quidem ipsi summo magi- stratui, vel Ecclesiae toti concedendum arbi- tramur, vt illa, quae sunt οὐσιώδη disciplinae Ecclesiasticae, tollere, vel immutare pos- sit. Etsi enim in rebus politicis Rex, sum- musque magistratus nouas leges fer- re, et veteres abrogare iure potestatis si- bi a Deo concessae potest, in disci plinae ta- men Ecclesiasticae fundamento vel vete- ri, et a Deo iam posito tollendo: vel nouo reperiendo constit uendóque idem ius non habet. Cuius rei haec ratio est, et qui- dem manifestissima, quod cum disciplinae Ecclesiasticae scopus et finis hic sit, pri- mum vt Pastores, et qui Ecclesiae Dei prae esse debent legitimam vocationem ha- beant. Postea vt hominum conscientiae gubernentur, et ad poenitentiam conuer- sionemque ad Deum adducantur, ad solum Deum disciplinae Ecclesiasticae definitio  \pend
\section*{EPISTOLA. }\pstart legislatioque sane quidem pertineat neces- se est, quia ille solus Ecclesiae, et coetus i- stius Rex est, dominus, moderator et legi- slator, idemque solus conscientiarum nostra- rum legitimus gubernator, arbiter, atque rector. Itaque solus ille potest leges, tum le- gitimae vocationis eorum, qui in sua do- mo praeesse debent, praescribere: tum etiam mouendae hominum conscientiae, ingi- gnendaeque poenitentiae modum nobis demonstrare. In quibus duobus capiti- bus laus et vis summa vniuersae Ecclesia- stice disciplinae, quemadmodum diximus, finisque positus est. De quibus rebus le- ges alias, et ab iis, quas Christus praescrip- sit, diuersas ferre neque Reges vlli, ac ne vniuersa quidem ipsa Dei Ecclesia (quae est in his terris) potest. Nam et ipsa ecclesia est duntaxat serua Dei, et non domina, et reges quoque ipsi sunt in hoc coetu membra tan- tum non caput: membra quidem honorabilia, non tamen sic membra, vt in hoc coetu regem do imperent pro suo arbitratu, sed tantum vt Dei mandata a piis et inter pios obser- uari sedulo curent. Vocationum enim ec- clesiasticarum, et conscientiarum nostra- rum procul dubio solus Deus et scrutator et gubernator et Rex et autor esse debet  \pend
\section*{EPISTOLA. }\pstart et potest. Itaque cum hominum conscien- tias immutare, inflectere, ad Deum adduce. re, huius disciplinae ecclesiasticae proposi- tum sit et finis, quomodo, quaue via id fieri recte possit, a nemine, praeterquam ab vno Deo praescribi, definiriue potest, quia so- lus ille eas fabricatus est: solus quomodo eae regendae, flectendae, ac immutandae sint, nouit, solus earum dominus est. Ipsius enim solius templa et sedes sunt nostrae conscientiae. Vnde velle ab alio, et mor- tali quidem homine, leges inflectendae, et conuertendae nostrae conscientiae propter Dei verbum habere, est hallucinari velle, est operam perdere, est summam iniuriam Deo (qui sibi vni hoc totum ius reserua- uit )inferre. Possunt igitur Reges, regum- que consiliarii, viri prudentes, leges de pu- blica inter ipsos ciues pace tuenda dicta- re et commentari: leges autem de con- scientiae hominum regendae, aut inflecten- dae ratione et modo (praeter eas, quae Dei verbo praescriptae sunt) nouas cogitare et comminisci, denique de legitima eorum qui ecclesiam (id est domum, Dei non ipsorum Regum gubernant) vocatione praeter Dei. i. ipsius Domini domus voluntatem le- ges inquam et regulas condere et sancire  \pend
\section*{EPISTOLA. }\pstart certe quidem illi nullo iure possunt. Ac, vti dixi nec Ecclesia quidem ipsa vniuersa et collecta in Oecumeniis, quae vocant, conci- liis hoc quoque potest, adeo vt mera fue- rit tyramnis, et in ius ipsum Dei inuasio im- probissima quicquid Synodi illae Oecume- nicae contra praescriptam verbo Dei disci- plinae Ecclesiasticae οὐσίαν induxerunt, et san- ciuerunt: quanquam se aliquid melius et sanctius moliri existimarent. Quod verum esse, vel ipsa Ecclesiae confusio, ruinaque, quae inde con secuta est, manifeste decla- rat. Nam si illa Apostolicis scriptis prae- scripta disciplina retenta fuisset, numquam in eam perturbationem Ecclesia incidis- set, a qua miserrime tandem pessundata est. Quid igitur, damnamûsne omnes eas Euangelicas Ecclesias in quibus vel nulla disciplina Ecclesiastica omnino obserua- tur, vel Apostolica non retinetur: sed in qui bus ea vel corruptaest: vel etiam alia, pro vera substituta ? Ad hanc odiosam qui dem quaestionem, et quae tamen saepe no- bis obiicitur, ingenue quidem, sed tamen placide et modeste respondemus. Nos propter carentiam et εἰπουσίαν? disciplinae Ecclesiasticae minime damnare Ecclesias et coetus illos hominum, in quibus vera pu-  \pend
\section*{EPISTOLA. }\pstart raque Euangelii Christi praedicatio retino tur, summota omni idololatria: ergo non diffitemur et eas etiam veras esse Dei Ec- clesias. Vnica enim est, eaque fundamen- talis verae Ecclesiae nota in his terris, pura nimirum verbi Dei predicatio a legitimo pa store facta, quae vbicumque est, illic quoque est Grex Domini, iuxta illud Christi, O- ues meae vocem meam audiunt: et si ab eo coe- tu saepe Sacramentorum administratio, sae pe etiam morum correctio, Ecclesiasti- caeque disciplinae vsus et praxis exulat, atque abest. Quod enim hae duae posterio- res notae, tanquam necessariae, ad verae Ec- clesiae designationem a nonnullis pronuntian- tur, ostendunt illi tantum, quid quantumque requiratur, vt omnibus suis numeris ab- soluta sit verae Ecclesiae facies. non autem illud volunt, omnino Dei Ecclesiam dici eum coetum non posse, in quo pure et ve- re verbum diuinum praedicatur, si nulla sit disciplinȩ Ecclesiasticae praxis. Sunt igi- tir huiusmodi Ecclesiae quidem, sed non omnibus suis partibus et numeris absolu- tae: quin potius claudicantes. Sic ille a no- bis dicitur homo, cui brachium vtrumque abscissum est, si modo idem ille ratiocina- tur. Sed hoc conari, hoc studere, hoc ef-  \pend
\section*{EPISTOLA. }\pstart ficere omnibus modis debent illae Ecclesiae, quae summo Dei beneficio idololatriam abiecerunt, verum Dei verbum et vitae aeternae fundamentum receperunt, et ad- huc tamen ista disciplina carent, vt hoc tam praeclarum ornamentum. i. discipli- nam hanc Ecclesiasticam ( qualis ab Aposto- lis praescripta est) ad corpus illud Ecclesiae suae adiungant, et addant, vt non modo neruos et ossa Ecclesiae habeant, sed etiam vt sanguinem, colorem, venustatem, vires denique Ecclesiae integras retineant. At- tendendum est enim diligenter ad illud Spiritus sancti dictum rectas faciendas esse vias nostras, ne quod claudicat tandem pror- sus abducatur a via. Nam si quae necessaria sunt per incuriam in Dei Ecclesia negli- gantur, temporis successu illa penitus a Deo tandem abalienantur. Quia vero haec discipli na maxime hac Pauli Epistola, quae prior est ad Timotheum, est explica- ta, idcirco eam potissimum delegi, in cu- ius interpretatione meam de toto hoc ar- gumento sententiam breuiter, more meo, exponerem. Scripserunt in hanc ipsam Pauli Epistolam varii, tum veteres, tum recen- tiores interpretes, eruditi illi quidem: sed qui se in hoe de disciplinae Ecclesiasticae ar-  \pend
\section*{EPISTOLA. }\pstart gumento maxime ostentare voluerit fuit Claudius Spensaeus. Is enim praeter ma- gnos in ipsum Pauli contextum commentarios etiam tres libros Digressionum, quas ap- pellat, et ipsos quoque prolixos edidit in hanc Pauli Epistolam. Tantum autem ab- est, vt, meo quidem iudicio, quisquam iis adiuuari possit, vt nihil sit et scriptore i- sto obscurius, et illius disputatione magis perturbatum atque confusum. Sunt enim isti scholastici, e quorum numero fuit Spen- saeus, iusto Dei iudicio occaecati: et tam tenebriones, vt in plerisque ne seipsos qui dem intelligant, quia sunt in tenebris et matȩologia educati. Certe cum in eos com- mentarios incidissem, nihilque in iis pro- ficerem, mature abistis digressionibus et er- rore ad ipsum Paulum et verum iter redii, fe- cique illud, quod habet prouerbium ve- tus bibi et fugi. Quantum autem hoc meo la- bore ipse sim consecutus, et rem caeteroquin propter temporis vetustatem, diuturnan- que desuetudinem, iam obsoletam ac dif- ficilem illustrarim, aliorum esto iudicium: certe magnis laboribus et lucubrationi- bus haec vetustatis rudera nobis eruenda fuerunt. Precor autem Dominum Deum vt hos meos labores Ecclesiae suae vtiles,  \pend
\section*{EPISTOLA. }\pstart piisque viris suaues esse velit. T.vero E. Princeps Illustriss. cui eos animo demisis- simo ac lubentissimo dono, dicóque, si probabuntur, certe operae meae recte col- locatae fructum hunc maximum iam ero consecutus. Quamobrem autem hoc, quicquid est operae meae, T. E. dedicarem cum variae et multae sunt, eaeque graues causae, tum hoc imprimis, quod cum summo Dei Op- timi Maximi bene ficio, tuaque illa plus heroica virtute et constantia Zelandiae et Hollandiae, totiusque illius tractus Eccle- siae Dei pace recreentur, nihil vel ad eas, iam praeclare instituendas: vel quae erunt constitutae retinendas hac disputatione ap- tius atque vtilius, adeo etiam magis ne- cessarium esse existimem. Quemadmo- dum enim Ecclesiasticae disciplinae con- temptus, atque praetermissio magnas in Ecclesiam ruinas semper induxit:ita dili- gens illius obseruatio et ad Apostolicam normam reuocatio non tantum caelestem ipsam doctrinam sartam tectam puramque inter mortales conseruauit, sed etiam mores ipsos hominum nomen Christi pro- ffitentium, si qua in parte ad vitia delabe- bantur ad meliorem frugem conuertit, et  \pend
\section*{EPISTOLA. }\pstart In Dei metu officióque continuit: est de- nique haec eadem aduersus enascentes haereses sepes et munimentum validissi- mum, aduersus iam natas et infelicissime pullulantes, tamquam aduersus magica ve- nena, remedium praesentissimum. Quan- quam autem T. E. doctissimorum virorum, imprimis autem generosissimi praestantissimi que illius viri Ph. Marnixii Domini montis Sancti Aldegondii, et fidelissimi verbi Dei Ministri Taffini, (a quibus hȩc ista discet) eruditissimis sermonibus non caret, haec tamem in commune vestrarum Ecclesia- rum commodum, tanquam symbolum, et ipse pro virili volui conferre, vt illae intelligant quantum pro T. E. et illis ipsis bene ex animo velim, ac felicissima atque opti- ma quaeque a Deo ardentibus votis exo- ptem. Tuae vero virtuti, Princeps Illustri- sime, quid non debeatur, cum ab omnibus fidelibus et Christianis viris, tum vero a nostris Gallis, quorum salus et incolu- mitas semper summae curae fuit tibi, frati- que tuo Illustriss. Comiti Ludouico ( cu- ius felix memoria nunquam caneset, ac ne seclis quidem innumerabilibus, cuius vtriusque et armis et prudentia Ecclesiae Gallicanae ab aduersariorum atrocissimis  \pend
\section*{EPISTOLA. }\pstart Iniuriis contra leges patrias, moremque maiorum illatis tandiu sunt defensae: de- nique cum tua sit hodie in orbe terrarum gloria, tantaque rerum pro Dei gloria as- serenda gestarum laus, vt parem tibi tan- que constantem principem nulla adhuc hominum aetas viderit. Tu Prouinciam ti- bi commissam ab Hispanorum ac extra- neorum tyrannide eripuisti, tu Belgium intestinis morbis profligatum ac iam de- ploratum inexpectata ratione et medici- na sanasti, ac in iura maiorum pristinum- que splendorem restituisti. Tu hostibus io triumphum iam canent ibus fugam in- iecisti, et victor ipse triumphatórque Christi vexilla praeferens rediisti, Tu pie- tatis assertor, tu Iustitiae publicaeque salu- tis vindex fortissimus ore omnium mor- talium celebraris, adeo quidem, vt qui tan- ta, tamque praeclara et pro Domini no- stri Iesu Christi regno amplificando et pro patria libertate tutanda bene geris, tu quoque bene inter omnes mortales mere- ris audire Βροτοῖς γὰρ vt ait Eurip. χρηστὰ πρα- γματα τα Χρηστῶν ἀφορμὰς ἐνδιδωσ᾽ ἀεί λογων. Ego vero Princeps Clementiss. cum tot ad- mirandas virtutes tuas commendari vbi- que gentium intelligam, et hoc quoque  \pend
\section*{EPISTOLA. }\pstart ue in T.E. obseruantia testimonium et tuarum laudum praedicationis conatum et symbolum extare volui quod vt grato benignóque animo T. E. accipiat obnixe precor. Dominus Deus solus, Æternus Sa- piens, Bonus, Iustus, Misericors, te velut alterum Danielem, Spiritu suo corrobo- ret ac confirmet, et ad Ecclesiae suae presi- dium diutissime felicissimeque nobis, et toti Belgio incolumem conseruet. Geneuae Cal. Augusti Anno temporis vltimi. CIↄ Iↄ LXXVII.  \pend
\section{CAPVT  I. } \pstart
\phantomsection\addcontentsline{toc}{subsection}{\textit{P AVLUS Apostolus Iesu Christi ex ordinatione Dei ser- uatoris nostri, et Domini Iesu Christi, spei nostrae.}}
\subsection*{\textit{P AVLUS Apostolus Iesu Christi ex ordinatione Dei ser- uatoris nostri, et Domini Iesu Christi, spei nostrae.}}IN D. PAVLI APOSTOLI priorem ad Timotheum E pistolam. L. Danaei Commentarius. CAP. PRIMVM.
\textbf{P}AVLUS Apostolus Iesu Christi ex ordinatione Dei ser- uatoris nostri, et Domini Iesu Christi, spei nostrae.  \pend\pstart Paulinarum Epistolarum genus est duplex, v- num earum, quae ad Ecclesias totas scriptae sunt. Alterum, earum quae ad Particulares quasdam per sonas quae erant, vel Ecclesiasticae, vti vtraque ad Timoth. et vna ad Titum. vel Priuatae, vt ea quae est ad Philemonem. Nam quae circunferuntur Pauli Epistolae ad Senecam et contra, sunt fictae et ementitae. Quanto vero est dignior, qui Ec- clesiae praeest priuato homini, ita vberiora et lo- cupletiora sunt earum Epistolarum argumenta, quae ad praefectos Ecclesiae scriptae sunt (quales sunt illae quae ad Timotheum, et Titum, quae pene sunt paris argumenti) quam earum, quae ad  \pend
\section*{AD I. PAVL. AD TIM. }
\marginpar{[ p.2 ]}\pstart priuatos. Quae fuerit autem huius Epistolae ad Timo- theum scribendae ratio, et occasio facile ex ipso Paulo verss.3.intelligitur, nimirum, quod iam qui- dam sanam Euangelii doctrinam, ab ipso Paulo et aliis Fidelibus Christi Ministris traditam, et explicatam varie corrumperent tum in Substan- tia ipsa et doctrina, tum in Modo et forma illius tra- dendae. Cui malo vt mature occurreret Paulus, Timotheum monet quid opus sit facto, non tam ipsius quidem gratiâ, quam aliorum, id est, isto- rum doctrinam veram et sanam deprauantium. Quando scripta fuerit infra cap. hoc ipso vers. 3. Vnde scripta sit in fine huius Epistolae dicetur, Domino fauente. Diuiditur autem totum corpus, et σύνταγμα huius Epistolae in duas tantùm partes, nempe, Sa- lutationem, et argumenti suscepti tractationem. Epilogo enim, id est, postrema illa parte caret, in qua variae salutationes esse solent, quae in aliis tamen Epistolis extant, quae hoc loco non erant neces- sariae. Salutatio autem haec tria continet vbique, ni- mirum (Quis scribat (Ad quem scribatur (Summam precationis Ergo quaeritur, quis seribat? Scribit ( Plaulus, personam designat ( Apostolus, munus suum describit (Plus est enim dixisse Apostolum, quam seruum Dei simpliciter, quod alibi facit. Chry- sosto. hoc annotat Apostolum autem se vocat, et Iesu Christi:non  \pend
\section*{CAPVT  I. }
\marginpar{[ p.3 ]}\pstart hominum: in quo est commendatio muneris. Ni- hil enim nisi apprime salutare habet legatio et munus huiusmodi et a Christo demandatum et Dei ipsius ordinatione siue decreto. In quo inest confirmatio huius muneris. In hac autem statim epigraphỹ apparet tum summum amoris Pauli erga Timotheum et luculentissimum testimo- nium: tum illius muneris et Apostolatus firmamen- tum maximum. Quod non ipsius quidem Timo- thei gratiâ fit, sed aliorum, quos admoniturus e- rat Timotheus, nec quidem tam suo, quam Pau- li nomine et authoritate. Itaque hanc authorita- tem assertam prius confirmatamque esse opor- tuit. Deinde ipsius quoque Timothei et confir- mandi et commendandi apud Ecclesiam causâ, qui a Paulo missus erat, hoc additum est, quo iustior ipsius vocatio appareat, Pauli vocatione semel stabilita, etsi de ea Ephesii incerti esse non potuerunt, inter quos vixit per biennium Paulus. Nihil autem hoc loco fit a Paulo ostentationis et iactantiae animo, sed quod muneris autho- ritatem defendi aduersus quorundam impostorum calumnias oportuit, vidit que pro temporis cir- cunstantia esse necesse. Caeterum docet hic lo- cus a quo mitti de beamus et vocari, nempe a Deo. Nemo enim vsurpat sibi honorem Hebr. 5.V.4. etsi non omnes extraordinarie (vti Paulus) vo- cantur. Finem autem, et caussam tanti Dei beneficii erga homines ostendit, dum appellat Deum ser- uatorem. Quia enim Deusnos seruare vult, etiam mitti ad nos, qui ipsius Euangelium et volun- tatem annuntient. Psal.74.Luc 13.vers.34. id est,  \pend
\section*{AD I. PAVL. AD TIM. }
\marginpar{[ p.4 ]}\pstart \phantomsection
\addcontentsline{toc}{subsection}{\textit{2. Timotheo germano filio in fide. Gra- tia sit tibi misericordia et pax a Deo Patre nostro, et Christo Iesu Domino nostro.}}
\subsection*{\textit{2. Timotheo germano filio in fide. Gra- tia sit tibi misericordia et pax a Deo Patre nostro, et Christo Iesu Domino nostro.}}qui istius viam salutis exponant nobis. Vocat autem Patrem seruatorem vti Tit I.vers.3 et 2.vers 10. et Christum spem nostram. Vti enim Pater dedit Filium, per quem seruamur. Ioan.3.V. I6.ita Filius est fundamentum spei et salutis nostrae, vt cum haec Patri et Filio communiter tribuun- tur, intelligamus esse indiuisa Trinitatis opera, et tres personas paris, non autem dissimilis maie- statis et potestatis. Hec breuius persequor, quia a multis copiose tractata sunt. 2. Timotheo germano filio in fide. Gra- tia sit tibi misericordia et pax a Deo Patre nostro, et Christo Iesu Domino nostro. Timotheus quis fuerit apparet ex Act.16. et variis Epistolarum Pauli locis: Hunc autem vocat et Filium in fide, vt distinguat a carnali filio: et Germanum. i. verum et referentem veros mores Patris Pauli: non igitur fuit hypocrita et fictitius, vt hodie multi qui magnorum Dei seruorum et filii sunt et discipuli. Quod vtrunque pertinet ad Timothei laudem et imitationem. Sic Titum appellat Filium Tit.I.v3 et Corin- thios quoque I Corinth. 4. vers.14. Tamen obstat quod ait Christus Math.23.vers.9. et Hebr.12.V.9 Vnus est Pater animorum, nempe, Deus. Resp. Verum est per se et proprie vnicum esse animo- rum nostrorum Patrem, nimirum Deum ipsum: sed communicatione quadam, et quatenus homi- nes sunt huius nostrae in fide et Ecclesia Dei ge- nerationis, aut potius regenerationis instrumenta, vox ipsa Patris benigne illis a Deo communicatur, eti et Patribus carnalibus eadem vox Patris. Plus  \pend
\section*{CAPVT  I. }
\marginpar{[ p.5 ]}\pstart autem debemus Patribus qui nos in Dei metu instruunt, quam qui nos tantum corpore gignunt. Nec quod pastores dicantur patres nostri in fide excusat patres ipsos corporum et carnales, quo minus instituant liberos suos in vero Dei metu et pietate Deut.4. vers.II.et II.V.I9. sed differentia tamen ostenditur inter vtrunque genus patrum, quorum alii sunt nobis tantum spirituales Patres, alii sunt carnales. Filii autem dicuntur, ait Augustinus contra Adimantum cap.5. (Natura aut ( Doctrina Imitatione Hic et doctrinae et imitationis ratione Timo- theus filius dicitur, non tantum aetatis, vt tamen putat Erasmus. Salutatio est in his verbis posita Gratia et haec (Deo Patre nostro Misericordia (omnia ( Pax a (Domino Iesu Christo. Misericordia addita est hic, vti in Epistola ad Ti- tum, praeter stylum et communem morem Pauli, quo magis suum erga Timotheum amorem te- staretur, et se miro erga eum affectu impelli et affici ostenderet. Recto autem ordine et natura- li vtitur in his enumerandis Paulus, et progredi- tur a causis ad effecta. Est enim gratia causa mi- sericordiae, vt misericordia ipsa causa pacis no- strae. Est igitur gratia gratuita nostri acceptatio, quae fit a Deo, quam sequitur misericordia, id est, peccatorum nostrorum remissio. Itaque vocem misericordiae hic passiue pro effecto, quod in nobis apparet: non autem actiue in Deo accipi- mus  \pend
\section*{AD I. PAVL. AD TIM. }
\marginpar{[ p.6 ]}\pstart \phantomsection
\addcontentsline{toc}{subsection}{\textit{3 Quemadmodum te sum precatus vt permaneres Ephesi, quum proficiscerer in Macedoniam, permaneto, vt denunties qui- busdam ne diuersam doctrinam doceant:}}
\subsection*{\textit{3 Quemadmodum te sum precatus vt permaneres Ephesi, quum proficiscerer in Macedoniam, permaneto, vt denunties qui- busdam ne diuersam doctrinam doceant:}}Author autem tot et tantorum donorum est Deus Pater noster, in et per Christum Dominum nostrum, in quo omnes Dei promissiones et be- neficia sunt, Amen et Etiam 3 Quemadmodum te sum precatus vt permaneres Ephesi, quum proficiscerer in Macedoniam, permaneto, vt denunties qui- busdam ne diuersam doctrinam doceant: Secunda pars huius Epistolae sequitur, in qua de re ipsa et argumento est agendum. Primum autem occasionem scribendi narrat, deinde rem ipsam aggreditur Paulus. Ex ipsa vero disputatione ap- paret magnam partem esse hanc epistolam ἐπανωρ θωτικήν et δειδακτικήν. Constat autem sex capitibus, quorum summa breuiter hac nostra tabula com- prehensa est. In summa in hac Epistola plena mi- nisterii Euangelici descriptio continetur, et disci- plinae Ecclesiasticae institutio explicatióque. Nam quae ad particulares quaestiones pertinent, suntque scitu necessaria videntur a Paulo tradita in Epi- stola ad Corinthios.  \pend
\section*{CAPVT  I. }
\marginpar{[ p.7 ]}\pstart table  \pend
\section*{AD I. PAVL. AD TIM. }
\marginpar{[ p.8 ]}\pstart Precatus) blanda et modesta ad monitio: quae eo a- cutius pungit et magis serio, quo maior est Pauli in Timotheum authoritas, id est, praeceptoris et patris in filium, et discipulum ius et affectus. Vtrunque autem potest Pastor, Denuntiare vt praefectus et pastor: et Precari vt frater2 Tessal. 3.V.I2. Nam quam hic sibi non tribuit, dat Timo- theo postea potestatem:nimirum, Denuntiandi et quasi iubendi. Hic autem nihil de Papali au- thoritate et Petri iure agitur in Timotheo E- pheso praeficiendo, cuius consensus in Timotheo vocando non interuenit, neque a Paulo est vel quaesitus, vel expectatus. Annotauit Eras. lucidiora esse Graeca, quam Latina, quod verum est, et repetendum est ali- quid ἀπό τοῦ κοινοῦ, sic Quemadmodum te rogaui, vt permaneres Ephesi, maneto, vt denunties, et c. Sic enim facile fluit sensus. Proficiscerer in Macedoniam, Haec est tertia profectio (quemad modum obseruauit D.Th. Be- za prȩceptor meus) quae contigisse videtur annoa Pauli conuersione 28. et a nato Christo 62. Huius sit mentio Act. I9. et 2O. Ergo scripta est haec E- pistola ante eas, quae sunt ad Ephesios, Philippen- ses, Colossenses. Post eam tamen vtranque quae est ad Corinthios, et ad Titum. Post vtranque eam, quae ad Thessalonicenses scripta est, vt quidam putant. Apparet etiam Paulum habuisse Ephesiorum Ecclesiam commendatissimam, cui tam sollicite prospicit. Ne aliter doceant) Cur hoc timuerit Paulus ratio satis intelligi potest ex Apocalyp.2.vers. 6.  \pend
\section*{CAPVT  I. }
\marginpar{[ p.9 ]}\pstart et quod iam Galatae, quae erant vicinae Ecclesiae, abducti fuerant a synceritate et puritate doctri- nae Euangelii per huiusmodi pseudofratres con- cionatores et nebulones, qui aliter, quam Apo- stoli, Euangelii praetextu docebant. Quid sit au- tem ἑτεροδιδασκαλεῖν hic et inf cap. 6.V.3. varia est interpretum sententia: sed ex legitimi pastoris definitione facile tanquam ex contrario perspi- cietur. Illa autem definitio extat apud Matthaeum 22.vers.I6. vbi de Christo agitur. Est igitur verus Pastor qui docet viam Domini in veritate, idque sine personarum acceptione Malach.2. Ergo peccatur vel in Re et doctrina, vel in Forma et modo docendi, quam ὑποτύπωσιν appellat Paulus  Timoth I.vers.13. vel in Applicatione et vsu do- ctrinae 2 Timot.2.vers.15. Quibus omnibus vitiis et modis aliquis ἑτεροδιδασκαλεῖ, id est peccat in docendo.  \pend\pstart \phantomsection
\addcontentsline{toc}{subsection}{\textit{4 Nec attendant fabulis et genealogiis infinitis, quae potius quaestiones praebent, quam aedificationem Dei quae est per fidem.}}
\subsection*{\textit{4 Nec attendant fabulis et genealogiis infinitis, quae potius quaestiones praebent, quam aedificationem Dei quae est per fidem.}}4 Nec attendant fabulis et genealogiis infinitis, quae potius quaestiones praebent, quam aedificationem Dei quae est per fidem. Εʹξήγησις. Illustrat exemplo quod sup. versic. praecepit. Quid sit nimirum ἑτεροδιδασκαλεῖν. Ex hoc autem loco apparet non eos tantum in do- cendo peccare, qui falsa tradunt, sed etiam eos qui vana et inutilia. Denique qui non necessaria curiose et diu inquirunt et perscrutantur. Cuius- modi olim fuit genealogiarum percensio, vel in- uestigatio. Hodie infinitae quaestiones quae sunt in scholastica. Primum de hypothesi agamus, de- inde veniamus ad thesin. Ac falsa est eorum inter-  \pend
\section*{AD I. PAVL. AD TIM. }
\marginpar{[ p.1O ]}\pstart pretatio et sententia qui volunt nomine genea- Iogiae hic intelligi, vel Mathematicas descriptio- nes natiuitatum, quas tamen damnatas esse agnos- cimus. vel longas illas series et disputationes de Æonibus, quas Valentinus ipsiusque discipuli e- tiam post Pauli mortem, nimirum circa annum a Christo passo cx. inuexerunt in Ecclesiam, tan- quam fidei doctrinam necessariam. Quod ad horoscoporum inspectionem et na- tiuitates quae ab Astrologis fiunt, illae non γενεα- λογίαι appellantur, sed γενέσεις et γενεθλιαλογίαι, et qui eas tractant et docent, non γενεαλόγοι, sed γενεθλιακοί nominantur. Hanc tamen interpreta- tionem videtur amplexus Chrisostom. quae huic loco minime conuenit. Quanquam autem inter Ephesios, qui Iudaei non erant, versabatur Timotheus, tamen inter Iudaeos maxime huiusmodi quaestiones, quas hic damnat Paulus, fiebant, vti etiam Tit. I.vers.14 eas Iudaicas fabulas vocat Paulus. Sed vel iam huius- modi fermentum in Ecclesiam Dei irrepserat, vel ex ea communicatione, quae fidelibus cum Iudaeis conuersis intercedebat hoc periculum iam Ecclesiae imminere potuit, et praeuideri a Paulo. Fuerunt autem huiusmodi de Genealogiis disputationes. Quis ex qua familia ortus esset, si- ue de se Iudaei agerent: siue de gentibus. Id quod partim ostentationis plenum est, partim inanis cognitionis. Μύθους, igitur vocat huiusmodi quaestiones, quod etsi saepe non sint prorsus res falsae et fictae pro animi libito:tamen sunt inutiles, vti fabulae et ficte narrationes Μύθος autem vnde nostri Gal-  \pend
\section*{CAPVT  I. }
\marginpar{[ p.11 ]}\pstart li duxisse videntur suum illud Mot et Latine Mu- tire, est vox aliquid significans hominis ore e- gressa et prolata, et generalis illius significatio est et acceptio. Sermonem enim quemlibet notat etiam apud Homerum vox Μῦθος. Iubet igitur so- brie potius eas delibare Paulus, quam iis inhaere- re et immorari. Ait enim Ne attendant. Nam quasdam tradit etiam Dei verbum nec inanes, nec infructuosas: nec tantum vetus testamentum, sed etiam nouum, vti Genealogiam Christi ipsius. Id quod fit ad fidei nostrae confirmationem, et vere humanitatis Christi approbationem, in quibus eatenus nobis versari licet quatenus et modum teneamus, et scopum a Spiritu sancto propositum non transiliamus. Sic fecerunt quoque Patres, vt Augustinus lib 2.3.7. contra Faustum, et lib. 2. de Consens. Euangelist. Hae disputationes et inuestigationes nimiae, vi- dentur ex eo cepisse, quod ante Christum et sub lege studiose inter Iudaeos cuiusque familiae des- criptio fiebat et tum publicis archiuis, tum pri- uatis monimentis religiose et diligenter seruaba- tur, adeo vt reiecti sint et expulsi ex sacrificato- rum numero qui suam Genealogiam ostendere non possent. Num. 1.vers.18, et Nehem. 7. V.62. Igitur iussi sunt eas habere tum in archiuis publi- cis tum priuatis scriniis Iudaei Numer. 25.vers.5. adeo vt Prophetae ipsi libros totos de iis edide- rint vt est 2 Chron. 12. vers. 15. Sed in eo ratio di- uersa fuit illis et nobis, illi expectabant Christum ex vna quadam et certa suae gentis familia, et ge- nere nasciturum, non ex quauis tribu. Idcirco di- stinctio familiarum diligenter retinenda ab iis  \pend
\section*{AD I. PAVL. AD TIM. }
\marginpar{[ p.12 ]}\pstart fuit vt melius postea Christum agnoscerent. Nos iam ex ea familia natum scimus et habemus vt hac inuestigatione nobis hodie non sit opus. Dein- de praedicatur Christus, vt non vnus tantum fa- miliae vel gentis: sed omnium populorum, omnium nationum Mediator, Deus et Redemptor, vt in sola Iudaeorum gente non sit hodie nobis Chri- stus quaerendus. Dixit, et quidem recte, August. etiam veteris Testamenti Genealogias ad Chri- stum pertinuisse lib. 19. contra Faust. cap.351. Vnde apud illos ante natum Christum fuit haec quae- stio tolerabilior, quae hodie cessat nisi si sumus infideles, et Christum vere natum secundum Pro- phetas dubitemus. Duplici nomine condemnat eas Paulus, quod et sint Infinitae, et Inutiles ad vitae pietatem in nobis serendam et gignendam. Vtrunque pugnat cum vero omnis studii et inquisitionis fine. Quae- rimus enim vt inueniamus: et vt tandem ad ali- quam cognitionem, atque etiam eam nobis vtilem perueniamus et insistamus ei acquiescentes. Quem fructum aut vsum non praestant hae quaestiones. Edificationem igitur eam dicit quae est in fide, id est, quae secundum ipsius Dei voluntatem, verbum, et praeceptum est, et solida. Nam finis Theolo- giae verae, est aedificatio conscientiarum nostra- rum, non vana et turgida scientia variarum dis- putationum et quaestionum. Ædificant enim, sed ad pugnas et clamores tantum eae quaestiones, quae sunt de huiusmodi rebus, ac propterea vt sunt ita merito appellantur inutiles 2 Timoth.2.v. 16 et 23.  \pend
\section*{CAPVT  I. }
\marginpar{[ p.13 ]}\pstart \phantomsection
\addcontentsline{toc}{subsection}{\textit{5 Porro finis mandati est charitas ex puro corde, et conscientia bona, et fide non simulata.}}
\subsection*{\textit{5 Porro finis mandati est charitas ex puro corde, et conscientia bona, et fide non simulata.}}5 5 Porro finis mandati est charitas ex puro corde, et conscientia bona, et fide non simulata. Αʹντίθεσις . Opponit ex aduerso inutilibus istis quaestionibus veram Theologiae tractationem, quae charitatem pro fine habet, Verbum Dei pro- fundamento, πρᾶξιν autem, eamque legitimam, pro laboris fructu et pro iusto finis, ad quem tendit testimonio, quae tria breuissime quidem, sed ta- men grauiter complectitur hoc loco Paulus. Finis praecepti) id est legis Dei, quam suae cu- riositati praetexebant, est charitas. Ergo non ineptae et inutiles quaestiones istae. Legis nomine Scri- pturam sacram intelligit, quae Legis nomine ap- pellatur, quod tunc temporis alia nulla sacra scri- ptio adhuc esset, nisi de Lege, etsi Legem ipsam a Prophetarum scriptis nonnunquam distinctam legimus Luc 24. Sed et Legis nomen latisstme patere certum est apud Paulum. Vnde Lex fidei et Lex Spiritus a Paulo dicitur Romanor.3, v. 8. Hic autem pro Verbi diuini tractatione. Nam docet Paulus quis sit verae Theologiae finis. Is au- tem est non theoriticus sed practicus, non inanis sed fructuosus consolatioque animarum, quia chari- tas actionem producit, nec in sola contemplatio- ne nec in verbis solis et lingua versatur a Ioan.3.v 18. Scholastici aliud de sua Theologia sentiunt, cuius finem constituunt contemplationem et Theo- riam. Aiunt enim praestantissimae scientiae est praestantissimus finis. At Theologia est ars prae- stantissima. Haec concedimus. Sed quod aiunt, Contemplatio est nobilior actione. Resp.  \pend
\section*{AD I. PAVL. AD TIM. }
\marginpar{[ p.14 ]}\pstart Actionem hic non accipi a nobis pro opere me- chanico: sed pro praestatione aliqua, cuius fructus sit Dei gloria et aedificatio conscientiarum. Con- templatio autem nuda et mera imperfectus a- ctus et humanae imbecillitatis indicium potius est quam laus potentiae, si desit scilicet istius con- templationis aliquis effectus et fructus qui cog- nitionem nostram consequi possit. Aiunt isti Dixit Christus Luc. 1o. meliorem partem elegit Maria sibi. Respond. Versabatur in actione, non in nuda disputatione, cuius exer- cendae gratiâ mundo ipsi renuntiabat Maria. Aiunt Tertiae praestantissimae hominis partis actio est praestantissima. Animus autem, cuius pro- pria est contemplatio, est corpore excellentior. Ergo et contemplatio est nobilior actione. Resp. Vera esset conclusio, si actio, de qua loquimur, solius corporis viribus fieret, non autem praeci- pue operatione, id est, intellectu et voluntate i- psius animi nostri, ad cuius obsequium postea for mantur corporis membra. Ergo quum dici- mus actionem esse finem verae Theologiae et Christianae, nomine Actionis vel Operis non in- telligimus seruile et mechanicum aliquod opus, sed actionem intellectus et voluntatis nostrae ad Deum tendentis nobilissimam. Nam ipsi illi Scho- lastici etiam constituunt int ell ectum practicum esse facultatem animi humani praestantissimam. Aiunt quarto, oculus est praestantior manu. Resp. Κατάτι. Neque etiam hic operis nomine intelligimus tantum manuales actiones, sed cul- tum Dei, reformationem vitae quae sunt actiones nobilissimae  \pend
\section*{CAPVT . I. }
\marginpar{[ p.15 ]}\pstart Charitas) Et ad Deum, et ad homines referri potest haec charitas, quia Dei verbum vtrunque docet et Deum et homines diligere. Nos tamen de charitate ea, quae proximum spectat, hoc loco intelligemus. Huic igitur charitati addit, Cor purum, Con- scientiam bonam, Fidem veram et synceram. Per quae tria non tantum vera charitas descri- pta est, sed etiam fundamentum illius et vsus de- monstratur, ne hanc charitatem, quam finem praecepti dixit, putemus esle quiddam otiosum, et contemplatiuum, vt quidam fingunt. Est.n. affe- ctus charitas qui in animis nostris est, non vt la- teat: sed vt sese proferat ipso opere. Quia igitur charitas est animi affectus huius- modi, Duo habeat oportet nempe, Veritatem et Opus, vt docet Ioann: in 1. Epist.3. vers.18. Cor purum) habet charitatis veritatem. Itaque opponitur cordi ( Impuro, quod sordida et foe- ditates amat et diligit ( Ficto seu duplici, quod non est integrum et purum a simulatione et hypo- crisi tum in Deum, tum in proximos. Conscientia bona) vsum, et opus illius habet. Nam et finem actionum nostrarum regit, et cir- cunstantias prudenter tractat et perspicit in agen- do. Tunc enim est bona conscientia in nobis cum cor nostrum nos non condemnat, sed coram Deo licet attestari nos et rectum finem habuisse, et diligenter omnia, quae circunspicienda erant, perspexisse, quantum in nobis fuit. Fides autem est moderatrix, aut potius regula et norma vtriusque, quae docet ex Dei verbo nos  \pend
\section*{AD I. PAVL. AD TIM. }
\marginpar{[ p.16 ]}\pstart \phantomsection
\addcontentsline{toc}{subsection}{\textit{6 A quibus quum quidam vt a scopo a- berrarint, deflexerunt ad vaniloquentiam}}
\subsection*{\textit{6 A quibus quum quidam vt a scopo a- berrarint, deflexerunt ad vaniloquentiam}}quo spectare debeamus, quid odisse, quid dilige- re, et quomodo pro cuiusque vocatione nos ge- rere. Itaque coniungenda est cum verbo Dei. Non vult autem Paulus hanc fidem esse hypocriticam quae speciem tantum pietatis et metus Dei prae se ferat: sit tamen ab vtroque remotissima. Haec fides igitur ad cognitionem ipsius do- ctrinae refertur, quae in nobis esse debet ἐν ἀληθείᾳ vt loquitur Coloss.I vers. 6.id est, et qualem esse ex Dei verbo oportet, non autem qualis ex nostris aut hominum commentis, et sensu haberi, for- marique potest. 6 A quibus quum quidam vt a scopo a- berrarint, deflexerunt ad vaniloquentiam Confirmatio superioris sententiae a contrario. Nam qui ab hoc scopo deflexerunt, inciderunt in miseram vaniloquentiam, et inutile studium, quantumuis magni aliorum Doctores esse vellent vel viderentur. Haec vero Pauli sententia pulcher- rimam confir mationem recipit non tantum ex haereticorum doctrina, qui vanitati illi plane de- diti fuerunt, relicto verae Theologiae scopo vt Va- lentiniani Manichei et alii: sed imprimis ex som- niis Scholasticorum, qui nunc soli Theologi vi- deri, et dici (si Deo placet) volunt. Hinc apud eos portenta quaestionum, et myriades natae inu- tilium disputationum, in quibus tota aetas tamen teritur. ac de quaestionibus quidem haec sunt pau- ca exempla. I Vtrum gratia qua nos diligit Deus, et nos vicissim Deum sit eadem. Vtrum quid creatum  \pend
\section*{CAPVT  I. }
\marginpar{[ p.17 ]}\pstart \phantomsection
\addcontentsline{toc}{subsection}{\textit{7 Quum velint esse Legis doctores, nec intelligant quae loquuntur, neque de quibus asseuerant.}}
\subsection*{\textit{7 Quum velint esse Legis doctores, nec intelligant quae loquuntur, neque de quibus asseuerant.}}sit an increatum, subsistens per se, an accidens. 2 Vtrum realitates an notionalitates sint praestantiores. 3 Vtrum personae in Trinitate vera numeri appellatione dicantur tres. 4 Vtrum Pater producat Filium ratione in- tellectus vel voluntatis. 5 Vtrum numerus personarum in diuinis pertineat ad essentiam an ad rationem 6 Vtrum Deus possit quoduis malum etiam odium sui praecipere: et quoduis bonum etiam a- morem et cultum sui prohibere. 7 Vtrum in Deo sit intellectus agens et pas- sibilis. 8 Vtrum mundus melius condi potuerit, 9 Vtrum Papa sit misericordior Chisto, quum Christus neminem a poenis purgatorii liberarit. 1o Vtrum Papa possit vniuersum purgato- rium tollere vel exhaurire. Denique vide Eras- mum hoc loco ne has feces diutius tractemus. 7 7 Quum velint esse Legis doctores, nec intelligant quae loquuntur, neque de quibus asseuerant. Descriptio est istius generis hominum, et su- perioris dicti amplificatio, vt doceat quicunque ab illo scopo aberrant, eos incidere in vanilo- quium, quantunuis magni sint doctores. His au- tem tribuit Paulus Epitheta quatuor, nimirum quod sint I Arrogantes Volunt. 2 Timoth.3 2 Ignari Coloss2. vers.18.3 Docendi munus prae se ferant, et sibi vsurpent Rom. 2.vocantur Ra  \pend
\section*{AD I. PAVL. AD TIM. }
\marginpar{[ p.18 ]}\pstart \phantomsection
\addcontentsline{toc}{subsection}{\textit{8 Scimus autem bonam esse Legem, si- quis ea legitime vtatur.}}
\subsection*{\textit{8 Scimus autem bonam esse Legem, si- quis ea legitime vtatur.}}bi Marth.23. vers.7. 4 Temerarii, assertores et habere se clauem scientiae iactant. Matth. 23. Ex quibus omnibus intelligitur, quam graphice et vere hic nobis a Paulo proposita sit Scholastica Theologia et Mag strorum nostrorum mores et ingenia descripta, ne non facile hodie agnosci possent, nisi plane vel in ipso Meridie caecutire velimus. Denique cum totam fidei doctrinam corrumpant, nulli tamen et doctiores et sanctiores videri volunt, quam furfures isti et Midae por- tantes auriculas asini. 8 Scimus autem bonam esse Legem, si- quis ea legitime vtatur. Hypophora, per quam occurrit calumniae, qua et ipse, et doctrina Euangelii grauabatur, quasi Dei Legem, quae sacrosancta est, et Prophetarum testimoniis confirmata, sperneret et damnaret Paulus tanquam vanam doctrinam, quia Genea- logias, et huiusmodi quaestiones superius re- prehenderat. Nec vero vno tantum loco Paulus cogitur se ab ea graui accusatione defendere, quasi Dei Legem damnaret: sed srpius, vti Rom. 7.v. 12. et Oalat.3.vers.24. Ex quo facile int elligitur, quam valide et astute oppugnata sit Euangelii doctrina ab istis nebulonibus et aërem ipsum sub- tilitate suarum quaestionum fatigantibus homini- bus: quasi manifesto Dei verbo esset contraria Euangelii puri praedicatio. Quod idem nobis a Doctoribus Papistarum obiicitur. Sed ex hoc i- pso loco apparet eos qui tunc temporis tam san- ctis admonitionibus repugnarent maxime fuisse  \pend
\section*{CAPVT  I }
\marginpar{[ p.19 ]}\pstart Iudaeos. Illi enim de lege gloriabantur quam acce- perant, et qua etiam abutentes circa huiusmodi quaestiones languescebant. Gentes vero ad fidem conuersae, vix, ac ne vix quidem istas de Genea- logiis quaestiones attingebant: vel Legis a Mose datae meminerant, de qua ante Euangelium non audiuerant: neque in ea, vti Iudaei, innurriti fue- rant. Praeclara est haec confessio Pauli Apostoli de vsu legis et veteris Testamenti in Ecclesia Christiana et sub Euangelio valde notanda con- tra Marcionitas, et Manichaeos. Obseruandum tamen est hic Paulum non in vniuersum de toto vsu Legis agere, sed de eo tantum vsu, qui erat maxime ad praesentem quaestionem accommo- datus, et vt vanam illam Iudaizantium doctorum ambitionem et iactantiam contunderet: qui quum nihil, nisi garrulitatem et vaniloquentiam, mon- strarent, prae se tamen ferebant magnam Legis obseruationem, praedicationem, ac reuerentiam. Ergo docet quis sit verus Legis, quam praedicant, vsus, per quem Lex ipsa nobis salutaris et vtilis esse possit. Est autem, si ex ea cor purum et sanam conscientiam habere didicerimus per fidem, quae est in Christo. Lex bona est, si quis lagitime ea vtatur) Legiti- mus Legis vsus est, quum ad eum finem ea vti- mur, ad quem nobis a Deo data est et promulga- ta. Est enim is demum rectus illius vsus, qui quis sit supra docuit vers. 5. et 9. Paulus alludit etiam apte paronomasia quadam ad vocem νόμου in verbis vόμος et νομίμως. Nec enim Lex est doctri- na vtilis non vtenti illa legitime, et vt oportet. Sed obiicitur, etsi quis non vtatur recte Lege, tamen  \pend
\section*{AD I. PAVL. AD TIM. }
\marginpar{[ p.20 ]}\pstart \phantomsection
\addcontentsline{toc}{subsection}{\textit{9 Nempe sciens Legem iusto positam non esse, sed Legis contemptoribus, et iis qui subiici nesciunt: impiis et peccatoribus, nefa- riis et profanis, patricidis et matricidis, ho- micidis. 10 Scortatoribus, maseulorum concubi- toribus, plagiariis, et mendacibus, periuris, et siquid aliud est quod sanae doctrinae sit oppositum:}}
\subsection*{\textit{9 Nempe sciens Legem iusto positam non esse, sed Legis contemptoribus, et iis qui subiici nesciunt: impiis et peccatoribus, nefa- riis et profanis, patricidis et matricidis, ho- micidis. 10 Scortatoribus, maseulorum concubi- toribus, plagiariis, et mendacibus, periuris, et siquid aliud est quod sanae doctrinae sit oppositum:}}ipsam legem per se bonam semper esse. Qui enim eam in maiorem peccandi occasionem trahunt, ea legitime non vtuntur, non desinit tamen Lex propterea esse bona. Simile quiddam de Sacra- mentis afferri potest, quae semper sunt signa sa- lutis per se, quanquam impii vtantur iis ad hy- pocrisin suam fucandam et tegendam. Respon. Hoc verum esse, sed a Paulo demonstratur hoc loco non tantùm quid res sit ipsa, id est, Lex sit in sese: sed quid in nobis, et qualis eius fructus esse et produci debeat, ne ea abutamur. Erit autem tum demum fructuosa, si ad quem finem data est, cogitemus, id est, vt fidem cum charitate ex ea discamus et exerceamus. 9 Nempe sciens Legem iusto positam non esse, sed Legis contemptoribus, et iis qui subiici nesciunt: impiis et peccatoribus, nefa- riis et profanis, patricidis et matricidis, ho- micidis. 10 Scortatoribus, maseulorum concubi- toribus, plagiariis, et mendacibus, periuris, et siquid aliud est quod sanae doctrinae sit oppositum: E’ξήγησις. Explicat enim quis sit is legitimus Legis finis, de quo dixit. Est autem vt per Legem hominum mores et vitia corrigantur, et emen- dentut: non autem vt in vanis quaestionibus oc- cupemur. Posita enim est, vt scelera hominum, veluti homicidia, adulteria, mendacia coercean- tur: non autem ad bonos, et qui iam sancti sunt,  \pend
\section*{CAPVT   I. }
\marginpar{[ p.21 ]}\pstart reformandos et reprimendos. Mirum autem hic videri potest, cur cum illae questiones, quas dan- nat Paulus, ad Legem caeromonialem et politicam pertineant, tamen de lege morali agat, non autem de caeremoniali aut politica. Respon. Quod illa parte Legis, quae est moralis, istos sibi aduersarios homines magis premi posse videt, quia se pieta- tis et sanctitatis nomine commendabant. Deinde cum praecipua Legis diuinae pars sit Moralis, reli- quae vero duae sint illius quaedam appendices tan- tum, de legis Moralis vsu apte et conuenienter agit Paulus, ex quo et Caeremonialis et Politicae legis vsus quoque pendet et intelligitur. Itaque non extra rem respondet et arguit Paulus. Simi- lis est hodie nostrorum aduersariorum et Monacho- rum fastus, et arrogantia. qui cum toti crimini- bus scateant, nihil tamen, nisi Legem Dei, man- data, et sanctitatem vitae crepant, vt videantur sancti, et nos oppugnant, quasi eadem illa sper- namus et damnemus. Quibus respondemus eos esse pares istis hypocritis a Paulo descriptis. Quod ait Paulus, Legem non esse positam iu- sto, ex eius mente facilem sensum habet, nempe quod qui sponte iam parent Legi, et virtutem se- quuntur illi nullis loris et habenis vinciri de- beant, nullis aliis praeceptis egeant, quibus ad vi- tae sanctitatem adducantur: sed illi tantum iis e- geant, qui sunt immorigeri, rebelles, et dissoluti. Haec sententia dignitatem et commendationem Legis non imminuit, quasi iustis et iis qui Doi spiritu aguntur sit vel contraria vel inutilis, sed eos tantum perstringit, cum quibus egit Paulus, quos facile quidem docet esse tales, quales hic a-  \pend
\section*{AD I. PAVL. AD TIM. }
\marginpar{[ p.22 ]}\pstart \phantomsection
\addcontentsline{toc}{subsection}{\textit{11 Quae est, secundum gloriosum Euan- gelium beati Dei, quod concreditum est mihi}}
\subsection*{\textit{11 Quae est, secundum gloriosum Euan- gelium beati Dei, quod concreditum est mihi}}lieno nomine pinguntur, id est, homicidas a dul- teros, etc. lusto etiam est vtilis Lex. primum quia viam iustitiae inchoanti perfectionem docet et o- mnis virtutis est noima. Pelagius putauit eate- nus Dei verbum et Legem vtilem esle, quatenus tantum est liber continens doctrinam morum, contra quem varie disputauit Augustinus. Exlegibus autem hominibus esse positam Le- gem potest videri mirum, sed verum est, quia vt coerceret hominum rebellionem posita Lex est, et exlegibus posita est, ne mancant exleges. Hic autem describuntur gradus nequitiae: et primum fontes ipsi duo, R. bellio, et Contumacia in Legem, ex quibus riuuli isti fluunt, duo scilicet, Impietas ad primam tabulam, et Peccatum, id est offensio proximi ad secundam tabulam pertinens. Hae res paruae primum, postea crescunt, et disso- lutiores fiunt, adeo vt tandem euadant homines, Nefarii, id est, publice et nullo pudore peccan- tes, et Profani. Inde infinita genera vitiorum na- scuntur, quae reuocat ad tria genera, Vim Libidi nem, Fraudes siue furando siue mentiendo fiant. Decimum versiculum explicuimus in Ethicis nostris. 11 Quae est, secundum gloriosum Euan- gelium beati Dei, quod concreditum est mihi Denique easdem quaestiones inutiles docet contrarias esse Euangelio ipsi. Commendat autem Euangelium nomine, Gloriosi a differen- tia scilicet quae nonest de scriptura, sed de effe- ctu intelligenda, Dei ab authore, quem vocat non  \pend
\section*{CAPVT  I. }
\marginpar{[ p.23 ]}\pstart \phantomsection
\addcontentsline{toc}{subsection}{\textit{12 Gratiam igitur ei habeo, qui me ro- bustum effecit, id est, Christo Iesu Domino nostro, quod me fidelem duxerit, vt qui me in ministerio constituerit.}}
\subsection*{\textit{12 Gratiam igitur ei habeo, qui me ro- bustum effecit, id est, Christo Iesu Domino nostro, quod me fidelem duxerit, vt qui me in ministerio constituerit.}}tantum benedictum, sed beatum. Definit deni- que illud Euangelium esse eam doctrinam, quae ab ipso et collegis praedicatur Christi mandato. Itaque videri potest argumentum ab enumera- tione partium scripturae sumptum. Habet enim Legem et Euangelium, cui vtrique hae quaestio- nes inutiles repugnant. Opponit igitur Euange- lium Dei isti Mataeologiae, quam et a Lege et ab Euangelio damnari docet. Quo ipso demonstrat Paulus consentire inter se vtranque doctrinam contra Manichaeos. Sanam quoque doctrinam hanc vocat, quia aed’ficat ad pietatem. Denique Deum illius authorem μακάριον dicit et quod in se felix sit et beatus, et quod aliis felicitatem donet et affe- rat. Benedictum alibi vocat eadem scriprura Deum id est dignum omni laude et inuocatione. Paulus autem appellat hoc idem Euangelium 2 Timot 2. vers. 8.suum Euangelium, id est, quod ipsi credi- tum erat, non cuius ipse sit author. 12 Gratiam igitur ei habeo, qui me ro- bustum effecit, id est, Christo Iesu Domino nostro, quod me fidelem duxerit, vt qui me in ministerio constituerit. E’'κβασις siue digressio ex iniecta mentione sui mi- nisterii et Apostolatus, quem tuetur et commen- dat hoc loco. Cur autem excurrat Paulus duplex ratio est. Prima, vt haec sit hypophora contra ca- lumnias, quas inuidi homines illi obiiciebant propter superiorem vitam. Quomodo enim vi- deri potest Euangelii praeco, qui tam atrociter il- lud antea fuit insectatus. Captatas autem fuisse  \pend
\section*{AD I. PAVL. AD TIM. }
\marginpar{[ p.24 ]}\pstart huiuscemodi occasiones ab aduersariis Euange- lii apparet tum ex I. ad Corinth.15.tum quod ad Epist. ad Galat. et hic suum ministerium tuetur. Secunda ratio cur suum Apostolatum Paulus commendet, est, vt Timotheo, quem exhortaturus erat ad difficile certamen suscipiendum, seipsum in exemplum proponeret, in quo summa Dei potentia, vis, misericordia, et virtus illuxit, ne fa- tisceret, aut expauesceret Timotheus. Ita vero totum suum ministerium asserit Paulus, vt illius gloriam et laudem vni Deo vindicet, non sibi, non bonis suis operibus, aut praeparationibus. Hoc significat Gartiam habeo, Gratiae enim a- guntur bene ficiorum nomine acceptorum: non autem meritorum a nobis aut operum collatorum. Christo) Deum esse ex eo intelligimus Christum, cui et vocatio, et virtus conferens dona falutaria et nobis necessaria tribuitur 2 Corinth. 3.vers.5. omnis nostra idoneitas ex Deo est, Potentem reddidit) Non dicit Paulus, qui me tantum iuuit, sustinuit, erexit: sed qui reddidit potentem, vt valerem: alibi ait ἱκάνωσε, et robustum vt persisterem. Ergo prorsus extenuat et deiicit vi- res suas Paul. quas se ab vno Deo fatetur habere Neque hic de prima sua ad Euangelium voca- tione tantum, sed de toto ministerio suo et Apo- stolatu agit. Vires enim a Deo nobis suggeri o- portet ad vocationes nostras obeundas. Vires e- tiam ad perseuerandum. Vnde ait idem Paulus collocans me in Ministerio. Hoc idem de nobis sentire debemus, et ex animo quidem, non autem simulate et fict e:nec illud Leonis x. Pontif Rom. blasphemum vsurpare, nos a Deo vocari meritis  \pend
\section*{CAPVT . I. }
\marginpar{[ p.25 ]}\pstart \phantomsection
\addcontentsline{toc}{subsection}{\textit{13. Qui prius eram blasphemus et per- sequutor et iniuriosus: sed mei misertus est. nam ignorans id faciebam fidei expers.}}
\subsection*{\textit{13. Qui prius eram blasphemus et per- sequutor et iniuriosus: sed mei misertus est. nam ignorans id faciebam fidei expers.}}licet imparibus. Fidelem duxerit) Illud ducere est efficere. Non inuenit enim nos Deus tales, sed efficit. Nec e- nim idcirco Paulum vocauit Deus quod fidelem inuenerit I. quia erat blasphemꝰ 2. quia frustra su- perius dixisset Robustum me fecit. Erat enim dicen- dum, Robustum inuenit 3. Deinde et quod illa fide litas est idoneitas, quae non est a nobis, sed a Deo. Et seipsum superioremque sententiam plane e- uerteret Paulus, si quid in eo praecessisse boni o- Peris fingamus, propter quod ipse a Deo sit vo- catus. 13. Qui prius eram blasphemus et per- sequutor et iniuriosus: sed mei misertus est. nam ignorans id faciebam fidei expers. Comparatio est, ex qua Dei gratia maxime amplificatur, et maxime quod aduersarii in ca- lumniae causam arripiebant, transfert in suum con- modum, et sui ministerii maiorem commendatio- nem. Accusabant igitur quod fuisset Paulus an- tea ( Blasphemus Insectator ( Violentus Quae omnia concedit ille, quia et vera erant, et sic Dei gloria maior, et sua vocatio ad Apostolatum illustrior apparet. Cum enim talis fucrit, mirum est vocatum tamen a Deo esse Actor.9. et 13. Non loquitur de se Paulus vt hypocritae, sua peccata minuens, sed ea exaggerat. est enim haec gradatio: Fui blasphemus. i. maledicere solitus Deo. De- inde persequutor. Denique rabiosus et contu-  \pend
\section*{AD I. PAVL. AD TIM. }
\marginpar{[ p.26 ]}\pstart melias omnes coniungens. Mirum Dei opus in Paulo vocando, vt de multis persecutoribus ver- bi Dei non sit hodie desperandum. Opponit au- tem Dei mifericordiam, per quam non tantum veniam superiorum vitiorum consequurus est, sed etiam pulcherrimam illam ad Apostolatum vocationem adeptus, omnibus suis sceleribus ne fu- turis operibus vel quae Deus post Pauli voca- tionem edenda ab eo praeuiderat, electionis huius causam tribuamus. Addit tamen rationem, quia feci, et ignorans et ἐν ἀπιςτίᾳ. id est, carens fide et cognitione. Non tantum igitur docet se ignarum fuisse prius: sed se de fide et verbo Euangelii quicquam antea audiisse vel fuisse edoctum negat. Hic igitur ἀπιςτία est species ignorantiae, quae est maxime excusa- bilis, quum quis non ignorat ea, de quibus obla- ta illi est cognitio: sed de quibus nunquam audi- uit. Ergo ἀπιςτία non est hoc loco infidelitas et incredulitas cum quadam malitia et animi obsti- natione coniuncta: sed simplex et praecedenti verbi Domini praedicatione carens. Vnde quaesi- tum est vtrum ignorantia excuset peccatores a peccato. Quidam distingunt inter ignorantiam Simplicem et Malitiosam. Alii inter ea quae i- gnorantur. Aut enim sunt vt illi aiunt, praecepta pertinentia ad Primam Legis tabulam, aut ad Se- cundam. Illa ignorare non conceditur, vt volunt. Haec autem licet, quae sunt Secundae tabulae. Res. Ignorantia excusat non a toto: sed a tanto Rom. 2.vers. 12. Nam si caecus caecum duxerit ambo in foueam cadent: et qui ducit, et ipse qui ducitur. vt est Mat th. i5, et apparet, Genes. 2o. vers.5 et7.  \pend
\section*{CAPVT  I. }
\marginpar{[ p.27 ]}\pstart \phantomsection
\addcontentsline{toc}{subsection}{\textit{14 Superabundauit autem gratia Do- mini nostri cum fide et dilectione in Christo Iesu 15 Fidus est hic sermo, et modis omni- bus dignus quem amplectamur, Christum Ie- sum venisse in mundum vt peccatores serua- ret, quorum primus sum ego. 16 Verum ideo misertus est mei, vt in me primo ostenderet Iesus Christus omnem clementiam: vt essem exemplo credituris i- psi in vitam aeternam.}}
\subsection*{\textit{14 Superabundauit autem gratia Do- mini nostri cum fide et dilectione in Christo Iesu 15 Fidus est hic sermo, et modis omni- bus dignus quem amplectamur, Christum Ie- sum venisse in mundum vt peccatores serua- ret, quorum primus sum ego. 16 Verum ideo misertus est mei, vt in me primo ostenderet Iesus Christus omnem clementiam: vt essem exemplo credituris i- psi in vitam aeternam.}}Hinc autem quidam colligunt definitionem peccati in Spiritum sanctum: sed non satis apte.  14 Superabundauit autem gratia Do- mini nostri cum fide et dilectione in Christo Iesu Αὔξησις Dei gratiae, quae vice confirmationis addita est, et cx verbis ipsis Pauli apparet, vt o- mnis superioris vitae labes deleta esie doceatur. Probat autem hoc ipsom etiam ex effectis con- trariis superiori illi vitae, nempe ex fide, quae op- ponitur blasphemiae priori: et ex Charitate in Christo, quae opponitur Persecutionibus, et Contumeliis prioribus in Dei Ecclesiam factis. 15 Fidus est hic sermo, et modis omni- bus dignus quem amplectamur, Christum Ie- sum venisse in mundum vt peccatores serua- ret, quorum primus sum ego. 16 Verum ideo misertus est mei, vt in me primo ostenderet Iesus Christus omnem clementiam: vt essem exemplo credituris i- psi in vitam aeternam. Αʹνάκλασις eaque summo illustrandae Dei glo- riae studio inducta, vt quod contra ministerium Pauli maxime videbatur antea obiici posse, hoc totum ad pios consolandos confirmandosque iam valeat, quum in Paulo, tanquam pulcherrimo quodam speculo, insigne et singulare Dei mise- ricordiae exemplum etiam maximi quoque pec-  \pend
\section*{AD I. PAVL. AD TIM. }
\marginpar{[ p.28. ]}\pstart catores habeant. Ergo primum se describit: dein- de ex aduerso Dei misericordiam confert. Demum caeteros suo exemplo consolatur et exhortatur ad idem sperandum. Magnum autem huius spei et certum fundamentum primum omnium ponit, nempe, Christum factum esse hominem, vt pec- catores nimirum redimeret, etiam sceleratissimos. Quod sundamentum, quia magni est momenti, con- mendat solita praefatione, fidus est sermo et di- gnus infr. cap.3v.1.2. Timoth.2.V.19. Quamquam e- nim vniuersa scriptura sacra est per se, et in omni- bus suis partibus amplectenda, et sermo fidus, ta- men imprimis illa, quae summa fidei nostrae salu- iis et spei capita complectitur, singulari quodam mo- do amplectenda est. qualis haec et inf. cap.3.V.16. Caeterum Christumvenisse, vt saluos faceret homi- nes, est certissimum. idcirco.n. dicitur Iesus Math.1. et est hocipsum αἴτημα θεολόγικoν quod nulli Chri stiano esse ignotum potest nec debet, ideoque diligem tius et retinendum et commemorandum. Vnde fit, vt mirum videri non debeat, si qui talis antea fuit Paulus, misericordiam iam est consequutus. Ea enim gratia factus est homo et Iesus Dei Filius Christus. At fuit maximus peccator Paulus: fate- tur id quidem ipse, non tantum modesta quadam sui deiectione vtens: sed vere et ex serio sensu a- nimi sui loquens, et quali a nobis infidelitas o- mnis maximeque obstinata debet apprehen di cum pauore ac horrore conscientiae, Et hoc suum tam graue, et grande peccatum esse illustrissimum summae Dei misericordiae erga omnes peccato- res speculum decet, ne desperent vlli, sed conuer- tentur ad Deum, et in eo spem vitae aeterng ha-  \pend
\section*{CAPVT  I. }
\marginpar{[ p.29 ]}\pstart \phantomsection
\addcontentsline{toc}{subsection}{\textit{17 Regi autem aeterno, immortali, inui- sibili, soli sapienti Deo honor, sit et gloria in secula seculorum. Amen.}}
\subsection*{\textit{17 Regi autem aeterno, immortali, inui- sibili, soli sapienti Deo honor, sit et gloria in secula seculorum. Amen.}}beant. Ex hoc loco apparet quam sit enorme sce- lus persequi Christi doctrinam et seruos, quan- quam homines zeli velo conantur excusare tam impium facinus. Persequuntur enim Christum ipsum, quo scelere nullum esse maius potest Act. 9. quanquam ex iis tamen ipsis alii aliis grauius peccant:Ii grauius, qui scientes veram esse eam doctrinam nihilominus insectantur. Leuius autem qui eam esse Dei doctrinam ignorant, inter quos fuit Paulus. In se vero primo dicit patefactam Dei clementiam illam ingentem, quod non est accipiendum, ratione temporis. Omnes e- nim sumus natura filii irae, et neminem ante Pau- lum vocauit ad se Christus, aut hodie vocat, qui non sit natura illius hostis et persequutor, si votum spectes: Sed se primum vocat aliorum comparatio ne, et sceleris sui detestatione, id est, insignem peccatorem et caeteris capitaliorem Ecclesiae Dei Definit autem fidem et a subiecto, circa quod versatur. Est autem Christus: et a fine, quem spectat vel potius sperat. Est autem vita aeterna. 17 Regi autem aeterno, immortali, inui- sibili, soli sapienti Deo honor, sit et gloria in secula seculorum. Amen. Epiphonema, id est, sententia summi ardoris et vehementiae plena, in lquam ex diuinae erga se misericordiae meditatione erumpit Paulus, quo- niam zeli sui feruorem, et suae gratiarum actionis silentium comprimere iam non possit amplius. I- taque effundit hanc praeclarissimam vocem de stupenda Dei bonitate, primum quidem erga se  \pend
\section*{AD I. PAVL. AD TIM. }
\marginpar{[ p.30 ]}\pstart patefacta: deinde quae cernitur in caeteris rebus omnibus. Quae nanque de Deo hoc loco dicun- tur, illi semper competunt, quanquam ad Pauli conuersionem optime quadrant, in qua Deus suae bonitatis, sapientiae, gloriae, et immutabilita- tis specimen eximium ostendit. Ergo si quid mi- rum et incredibile inest ac occurrit in tanta Pau- Ii conuersione, ne mirentur. Nam est Deus qui conuertit, Deus, inquam, ille rex, et aeternus, qui potest omnia, et ab aeterno fuit. Si quid alienum, et nouum apparet in eadem factum conuersione est Deus ille sapiens. Si quid praeter humanam ra- tionem extitisse cernitur, idem ille Deus est inuisibilis, vt qui alia, quam carnali ratione operatur et exe- quitur sua consilia, sua beneficia confert, et Euan- gelium suum prouehit. Itaque haec omnia Epi- theta sunt rot ἀιτιολογίαι, ne fictitium hoc esse vi- deatur, et impossibile foctu quod Paulus de se narrat, et de Dei misericordia in conuertendis sceleratissimis. Similia Epiphonemata passim apud Paulum occurrunt, vti Romanor. 9. vers.5. et infra 6 v. 16 Quatuor autem Epitheta Deo tribuit, Vocat enim Regem seculorum, immortalem, inuisibi- lem, sapientem et quidem solum. Alia quoque alibi. Deo nomina tribuuntur, sed hic selegit Pau- lus, quae magis circunstantiis sui facti et vocatio- nis conueniebant, Romanor. 1.V.2o. et Exod.34.v. 67. et passim in Psalmis et haec, et alia leguntur. Rex seculorum) dicitur quia condidit tempo- ra non tantum haec quae dicuntur Αʹιώνιά  Tit.1.V.2 et a rerum conditarum siue creatarum initio ce- peruntised illa quae fluxerunt ante mundi crea-  \pend
\section*{CAPVT  I. }
\marginpar{[ p.31 ]}\pstart tionem, si modo nobis liceat bona Peripateti- corum veniâ illam durationem tempus appella- re, vt docet Augustinus in lib.  Confe.11. Est enim Deus ante omnem temporis non tantum dura- tionem: sed etiam cogitationem. lstud tamen E- pitheton quidam de Christo proprie scribi pu- tant, quia Pater seculi nominatur Isai.9. vers. 6.sed alius est illius loci sensus. Idcirco autem Deus. i. Trinitas ipsa appellatur Rex seculorum. vt immutabilis et aeternus Deus esse intelligatur, qui nullum temporis principium habeat: Cui subsequens est, vt sit immortalis. Hoc enim signi- ficat ἄφθαρτος. Nec enim ad mores incorruptos pertinet haec vox, vt nonnunquam accipitur, sed finem temporis excludit, que madmodum aeter- nitas principium ὑπάρξεως. Inuisibilem) vocat iuxta illud Christi, Deum nemo vidit vnquam Ioan. 1.vers.18.id est, huma- no visu Deus conspici non potest, vt explicat Augustinus lib.  I.de Trinitate cap.6. Beati quidem Deum vident et videbunt, sed alia ratione et vi- sione, quam haec nostra est. Hoc de solo Patre quidam dici putant, quia Filius per carnem as- sumptam factus est nobis conspicuus Deus, quem- admodum etiam loquitur Paulus infra 3. vers.16. At nec in Christo ipsa deitas fuit conspecta, imo velo carnis fuit obducta et obtecta. Praeterea di- cit Christus ipse, qui me videt, videt et Patrem Ioa. 1a v.9 vt ad Filium quoque hic locus referri debeat, quatenus Deus est. Sapientem) vocat non eo modo, quo nos ho- mines saepe sapientes dicimur, sed incomprehen- sibili prorsus sapientia, qua solus sapiens est et  \pend
\section*{AD I. PAVL. AD TIM. }
\marginpar{[ p.32 ]}\pstart quae quemadmodum. ait id I Corinth 2. sapientia repugnat et aduersatur mundi et hominum sa- pientiae, vt si quid in ipsius operibus non compre- hendimus, obmutescamus, et Dei consilium im- mensum adoremus, non autem in temperie qua- dam animi agitati obstrepamus Deo, quod mul- ti faciunt. Soli) Quaesitum est vtrum haec vox ad Deum, an ad vocabulum sapienti sit referenda: quasi hic Deus vel solus Sapiens, vel solus Deus appelletur qui est aeternus, immortalis, inuisibilis. Variae sunt interpretationes. Ego vero potius existima- uerim illud coniungendum esse cum proxima voce, Sapienti. Hic enim miram Dei sapientiam praedicat Paulus, quae nulli homini satis nota esse potest, per quam e miserrimis et desperatissimis peccatoribus, et Euangelii persecutoribus prae- cones, pastoresque Ecclesiae suae facit mutans eos mira sua potentia et virtute, vt ex lupis saeuissimis fiant oues placidissimae Isai.11. Honorem autem et gloriam tribuit Deo. Ali- bi alia quoque tribuuntur, vti in Epist. Iudae in fine et 2. Petri 3.vers.18. quae duo videntur ita se- parari et distingui posse, vt τιμὴ sit ipsius maie- statis Dei excellentia, splendor, admiratio, et virtus: δοξα vero illius tantae virtutis praedicatio, praeconium, et celebratio, quae fit ore hominum, vel Angelorum. Prius enim Dei maiestas agno- scenda est, post praedicanda. Quaeritur autem. Vtrum haec de solo Patre, an etiam de Filio dicantur, et de Spiritu sancto, quia Arriani ad solum Patrem haec per tinere vo- lunt, quem vnum proprie quoque contendunt  \pend
\section*{CAPVT  I. }
\marginpar{[ p.33 ]}\pstart \phantomsection
\addcontentsline{toc}{subsection}{\textit{18 Hoc mandatum commendo tibi, fili Timothee, nempe vt secundum praegressas de te prophetias, milites per eas bonam illam militiam.}}
\subsection*{\textit{18 Hoc mandatum commendo tibi, fili Timothee, nempe vt secundum praegressas de te prophetias, milites per eas bonam illam militiam.}}esse Deum Chrysostomus autem ad Filium hic refert. Respond. Si verba spectentur, ad Patrem referri, cui supra quae sunt propria diuinitatis tri- buerat Paulus: si tamen res ipsa definiatur etiam ad Filium, et Spiritum sanctum referri, quia pa- ris sunt Filius et Spiritus sanctus cum Patre di- gnitatis: et confirmatur 2 Petri 3. vers. 18. nostra sententia. Quia vero superio. versiculo Christum nobis vti mediatorem, et hominem Paulus pro- posuit, ex more suo et scripturae, in Patris perso- na complectitur et restringit, quae sunt propria diuinae naturae. Est vero periculosa Erasmi anno- tatio ad hunc locum, qui id rarum esse in Apo- stolicis literis ait, vt Dei vocabulum Christo et Spiritui sancto tribuatur. Ac ne in symbolo qui- dem Apostolico, Filium et Spiritum sanctum Dei nomine concedit proponi nobis. In quo to- to coelo errat. Nam qua ratione Patrem confite- mur esse Deum, cum dicimus Credo in Patrem: eâdem etiam Filium agnoscimus Deum, quia sic enuntiamus, et in Filium: idem de Spiritu sancto et in Spiritum sanctum, praeterquam quod ita pronuntiatur prima illa pars Symboli Credo in Deum, vt illic dispunctio adhibeatur Credo in Deum, et susistat vox: deinde explicatur quis sit ille Deus, nempe Pater, Filius, et S sanctus distincte. 18 Hoc mandatum commendo tibi, fili Timothee, nempe vt secundum praegressas de te prophetias, milites per eas bonam illam militiam. Regressio est ad propositum, et huius Epistolae  \pend
\section*{AD I. PAVL. AD TIM. }
\marginpar{[ p.34 ]}\pstart proprium argumentum, quod digressione illa superiori de vocatione miraculosa Pauli fuerat interruptum. Ergo postquam a quibus vitiis Pa- stori Ecclesiae sit cauendum docuit, adiicit quae sit summa totius muneris ipsorum, nempe et vt Fidem sanam habeant, et Conscientiam bonam et integram. Fidei nomine doctrinam vniuersam intelligit, quae ex puro Dei verbo docetur in Ec- clesia:imprimis autem Euangelium significat, quae pars Scripturae fidem nobis proprie proponit, Bona conscientia ad ipsius pastoralis muneris executionem pertinet. Facit enim illa, vt syncere recte et integre, quemadmodum oportet, in to- to et vitae et vocationis nostrae cursu versemur, et nauigemus, donec ad portum peruenerimus. Est autem is portus mors. De vtraque sup. copio- se vers.5. Ex hoc autem loco facile intelligi potest, ec- quod pastoris munus esse velit Paulus. Id quod i- pse Tit.1.vers.9. amplius docet et Petrus I Pet.5. vers.2. Nempe vt pascant gregem Domini ver- bo Dei, quod proprium est animorum alimen- tum. Ad quod et doctrina pura, siue sana cogni- tione illius verbi opus est: et prudenti synceraque eiusdem dispensatione, et ὀρθοτομίᾳ vt loquitur Paulus ipse inf. 2 Timoth. 2 id est, ne sit persona- rum acceptator Minister verbi Dei: sed exactor a quoque officii, et cultus Dei. Quod si idem tunc de officio Episcopi receptum et constitutum fuis- set, quod postea et in Papatu constitutum est, cer- te neque doctrinam neque bonam conscientiam commen daret et requireret in Episcopo Paulus. Quanquam enim Episcopum explorandum et exanimandum illi  \pend
\section*{CAPVT  I. }
\marginpar{[ p.35 ]}\pstart seribunt cano. Qui Episcopus ordinandus dist.25. tamen cum munus Episcopi describunt et defi- niunt eo modo illi loquuntur, vt nihil plane cum hoc Pauli loco conueniat in cano. Perlectis dist.25 Ad Episcopum pertinet Basilicarum consecrati o Vnctio altaris, consecratio Chrismatis: ipse prae- dicta officia et ordines Ecclesiasticos distribuit: ipse sacras virgines benedicit, etc. Quae mera sunt tam sanctae dignitatis ludibria, nimirum tam procul a vero Episcopatus sine in Papatu reces- lum est. Commendat autem Paulus hanc suam, quan- quam breuem pastoralis muneris descriptionem duplici argumento, Generali, et Speciali, quo di- ligentius et a nobis omnibus, et a Timotheo ex- cipiatur et obseruetur diligentius. Generalis qui- dem ratio est quae a genere sumpta est, dum Prae- ceptum appellat, ergo non negligendum impu- ne, et Depositum. Specialis ratio commendationis spectat Timotheum, et tum a suo in eum amore, tum ab ipsius officio ducitur, qui Prophetias de se ha beat plures Praecedentes, vt certus sit de Dei au- xilio, et sua vocatione, Quod ad Prophetias, qui- dam, vt Chrisostomus, eas accipit de ordinaria via et electione, quam obseruatam esse vult in Timothei electione: alii de extraordinaria acci- piunt, qualis illa, quae clara Dei voce facta est Actor. 13. Tale de Timotheo extitisse Dei te- stimonium volunt. Actor. 16 vers. 2. Ego extraor- dinariam electionem accipio hoc loco. Vt milites per eas) Explicat Paulus qui modus sit Euangelici ministerii fideliter exequendi, ni- nimirum, durum saepe certamen subeundum esse  \pend
\section*{AD I. PAVL. AD TIM. }
\marginpar{[ p.36 ]}\pstart sed in eo debere Timotheum sustentari bona spe quae Dei promissionibus generalibus, et praece- dentibus illis Prophetiis incumbat. Ministerii ve- ro Euangelici cursum comparat militari certami- ni, nec hic modo, sed in 2 Timoth. 2.vers.3 4.v. 7. vti et Deus ipse se comparat duci. Exod.15.V.5. vt intelligatur qualis sit Ministrorum conditio, nimirum dura et certaminum aduersus haereses et vitia hominum plena. Neque vero hoc facit, vt Episcopi militaribus se negotiis immisceant. Iure enim damnantur, qui arma tractant, Episco- pi. Nam nemo militans Deo, se implicare debet illis negotiis 2 Timoth.2. vers.4 Hoc enim genus militiae est spirituale, non carnale, et arma Epi- scoporum sunt spiritualia:nempe, preces, lachry- mae, verbum Dei:non autem carnalia, qualis est gladius bombarda, etc.  \pend\pstart \phantomsection
\addcontentsline{toc}{subsection}{\textit{19. Retinens fidem et bonam conscien- tiam: qua repulsa, nonnulli fidei naufragium fecerunt.}}
\subsection*{\textit{19. Retinens fidem et bonam conscien- tiam: qua repulsa, nonnulli fidei naufragium fecerunt.}}19. Retinens fidem et bonam conscien- tiam: qua repulsa, nonnulli fidei naufragium fecerunt.  \pend\pstart Continuatio est: si militudine enim docet quan- tum sit erroris huius, qui in ministerio Euangeli- co committitur, imprimis autem in bonae conscien- tiae desertione periculum. Est enim fidei naufra- gium. In quoet magnitudo periculi, quia fit iactu- ra fidei: et terror ostenditur atque metus, quia naufragium appellatur. Est igitur bona conscientia velut arca fidei et sanae doctrinae, ac intelligentiae thesaurus: tanquam illius cynosura, a qua qui de- flectunt, se et nauem, id est, intelligentiam illam sanam perdunt, iusto Dei iudicio in eos grassante  \pend
\section*{CAPVT  I. }
\marginpar{[ p.37 ]}\pstart \phantomsection
\addcontentsline{toc}{subsection}{\textit{20 Ex quibus Hymenaeus et Alexander: quos tradidi Satanae, vt discant non blas- phemare.}}
\subsection*{\textit{20 Ex quibus Hymenaeus et Alexander: quos tradidi Satanae, vt discant non blas- phemare.}}perciti. Ex ea igitur scilicet bona conscientia torum vitae nostrae cursum regere debemus. Quod autem bonam conscientiam homines Ethnici commen- dant, quale est illud, Nil conscire sibi nulla pal- lescere culpa. Hic murus aheneus esto. et, Con- scientia mille testes. item, Bono viro nihil prae- ter culpam praestandum, et caetera huiusmodi: vim quidem testimonii conscientiae demonstrant, quanta sit: praecipuum tamen periculum, quod ex ea spreta metuendum est, non ostendunt, quod est hoc, amissio et euersio verae fidei doctri- nae, et cognitionis de nostra salute et Dei erga nos misericordia. 20 Ex quibus Hymenaeus et Alexander: quos tradidi Satanae, vt discant non blas- phemare. Hoc ipsum duobus exemplis confirmat, Hy- menaei et Alexandri, de quorum altero agit hic ipse in 2 Tim. 2.V. 17. cuius errorem qualis fuerit describit, in quem Alexandrum quoque incidisse verisi- mile est. Fuit autem hic Alexander Ephesius Act. 19.V. 13. Vtrunque autem ad terrorem caeterorum pseudopastorum punit Paulus excommunica- tione quam solenniter antea in Ecclesia factam fuisse verum est. Hic autem eam repetit, vt intel- ligant omnes, quo iudicio digni sint omnes hae- retici obstinati. Verba autem haec tradidi Satana quid significent, D. Beza in I Corinth.5. vers.5. explicuit, ad quem lectorem remitto. Nam illic melius explicatur, quam in canon. Audi denique. De forma vero Papisticae excom municationis  \pend
\section{CAPVT  II. } 
\section*{AD I. PAVL. AD TIM. }
\marginpar{[ p.38 ]}\pstart \phantomsection
\addcontentsline{toc}{subsection}{\textit{A DHORTOR igitur ante omnia vt fiant deprecationes, preces, po- stulationes, gratiarum actiones pro quibus uis hominibus.}}
\subsection*{\textit{A DHORTOR igitur ante omnia vt fiant deprecationes, preces, po- stulationes, gratiarum actiones pro quibus uis hominibus.}}vbi duodecim Sacerdotes circunstant Episcopum, et lucernas ardentes habent, quas in terram proii- ciunt. vide canon. Debent duodecim Sacerdotes 1I. quaest.3. CAP. II.
\textbf{A}DHORTOR igitur ante omnia vt fiant deprecationes, preces, po- stulationes, gratiarum actiones pro quibus uis hominibus. Transitio est ad alteram Euangelici muneris partem, quae est publicarum precum pro quoli- bet hominum genere conceptio et profusio co- ram Deo. Quae disputatio commode subiicitur tum ad susceptum argumentum, quia pastorem Ecclesiae de omnibus sui muneris partibus Pau- lus informat:tum etiam ad superiorem conclu- sionem, quia hoc ipsum est optimum pietatis Christianae exercitium, et verae fidei conseruandae modus tutissimus. Precibus enim impetramus a Deo, vt in sana ipsius doctrina et metu pergamus confirmemur, et crescamus. Hoc autem loco et capite Paulus breuiter complexus est, quaecun- que de oratione siue precibus Christianorum quaeri possunt. I Quae sit, et quot eius genera. 2 A quibus sint preces fundendae. 3 Pro quibus 4 Vbi siue quo in loco. 5 Quando. 6 Quomo- do et qua cum animi reuerentia, et totius coetus decoro: quae singula suis locis a nobis (Domino  \pend
\section*{CAPVT  II. }
\marginpar{[ p.39 ]}\pstart dante) explicabuntur. Loquitur autem imprimis Paulus de publicis Ecclesiae precibus, non de singulorum Christiano- rum priuatis precationibus, quae fiunt a quoque domi et in conclaui. Nec enim has in domibus singulorum fidelium pastores facere debent: vel sufficiunt. Sed primo loco quaesitum est. Quot sint pa- storalis curae et muneris partes, vt intelligatur quam recte et ordine omnia persequatur Paulus Respondent Canonistae esse tres. Astaria siue ba- silicas totas consecrare: virginibus benedicere: Ecclesiasticos ordines distribuere, vt diximus supra ex canon. Perlectis S Ad Episcopum Di- stinct. 25. Haec Euangelici ministerii definitio prorsus inepta est, et vana, cum nihil huic simile doceat scriptura. Quare illi ipsi se reuocant, et alibi corrigunt, et melius sapiunt tota distinct. 85. Nam docent Episcoporum munus esse vt doceant corripiant, liberales sint. Priora duo reipsa pars sunt pastoralis muneris: tertium et postremum non item, sed illi cum omnibus Christianis com- mune. in Epist. vero ad Hebraeos cap.5.vers.1, 8. vers.3. 9. vers. 24. dicitur fuisse officium Sacerdo- tis offerre, et precari pro populo. Paulus in Epist. ad Titum 1.vers.9. inf. cap.4. vers. 11. 2 Timoth.4. vers. 2. ait, ministerium Euangelicum positum esse in eo, vt pastores doceant sanam doctrinam, refutent falsam, denuntient et corripiant siue in- crepent aberrantes in Ecclesia. Hic autem addit etiam ad eosdem pastores pertinere, vt ipsi precen- tur publice pro tota Ecclesia, et praecipuis illius membris. Ratio est, I. quia, vt ait Chrysostom.  \pend
\section*{AD I. PAVL. AD TIM. }
\marginpar{[ p.40 ]}\pstart Pastor est velut communis totius coetus pater, qui omnium curam gerit.2. Quia est publicum to- tius Ecclesiae os, vt nos in doctrina, et iis quae huic connexa sunt, erudiat, qualis est oratio. Ergo et precari debet, quia eo precante tota Ecclesia orat. 3 Hoc fuit officium olim sacrificatoris, in quo nulla caeremonia continetur. Hebr. 9. vers. 24. Numeror. 6.vers. 23. 4 Denique mos veteris Ecclesiae, quae Dei spiritu regebatur, in qua pro toto coetu, et ipsius nomine Pastores fundebant preces Deo, et populo benedicebant, vt ex variis historiis constat, et annotat Augustinus Epist.9o. Ergo cum sint haec tria praecipua pastoris mu- nera et officia, Docere, Corripere, et Precari, complexus est Paulus, Docendi verbo etiam corre- ptionem. Precationem autem hic seorsim perse- quitur, quia est ministerii Euangelici pars prae- cipua, vti diximus. Magnam autem vim esse huius publicae preca- tionis, et magnam illius habendam esse curam a pastore, docent haec verba Pauli et commenda- tio, ante omnia πρῶτον πάντων. Est enim compa- ratio, et commendatio huius exercitii, tanquam omnino necessarii in Ecclesia Christi, et vtilissi- mi, denique retinendi. Neque tamen haec verba confirmant, vel iuuant Audaeorum siue Messala- niorum haereticorum errorem, qui puppim et proram salutis in precationibus, id est, certis statisque horis factis orationibus et de murmu- rationibus (quod hodie faciunt Monachi Papi- stici) ponebant: sed tantum docet Paulus, quam debeamus sedulo et diligenter precari Deum: et quam solicitum de eo debeat esse non modo pa-  \pend
\section*{CAPVT  II. }
\marginpar{[ p.4I ]}\pstart storis, sed singulorum etiam fidelium studiym. Preces enim publicas quidam impiissime, etiam hodie contemnunt. Ait autem Paulus, Adhortor, Quae vox negli- gentiam nostram perstringit, et arguit. Quan- quam enim et Dei ipsius maiestas, et promissio iussióque: item fructus ipse, quem ex precibus nostris vberrimum sentimus, satis nos ad precan- dum excitare deberet, sumus tamen natura no- stra mire ad orandum frigidi, et torpentes. Itaque exhortandi et euigilandi sumus. Christus ipse nos ad petendum et precandum adhortatur. Petite et inuenietis, Matth.7. vers. 7. Vae igitur somnio pigritiaeque nostrae etc. Luc. 17 Ut fiant deprecationes) Primum praeceptum ponit quod est, faciendas esse preces in Dei coetu cui secundum adiungit, nempe pro omnibus: ex quibus tertium iam facile colligi potest, A quo fieri debeant: nempe, a pastore. Nam agit (vti di- ximus) de publicis precibus. Sed ex hoc ipso lo- co definiri potest, Quid sit oratio, et quot illius sint genera siue species. Precandus est igitur Deus quod extra dubium est, et concedunt non modo Christiani homines: sed etiam Ethnici et profa- ni, vt et Tertullian. in libro de Testim. conscien- tiae, et Iustinus Martyr in lib.  de Monarchia Dei copiose docent: item, Iuuenalis, Persius, et Hora- tius et alii poëtae id docent. Homerus, et Virgi- lius, qui suos inducunt preces fundentes ad Deum, et eum inuocantes. Nec gens vlla est, quae si testi- monium et vocem propriae conscientiae audiat, hoc axioma neget esse verum, vel refutet. Vult autem Paulus fieri deprecationes, preces  \pend
\section*{AD I. PAVL. AD TIM. }
\marginpar{[ p.42 ]}\pstart postulationes, et gratiarum actiones. In quo pri- mum est obseruandum non temere tot voces es- se simul congestas, sed partim vt precandi assi- duitatem et studium inflammet, incitet et inii- ciat nobis his tot verbis, nosque excitet, quem- admodum dixi: partim vero vt diuersa esse pre- cationum genera ostendat, quemadmodum di- uersi sunt hominum affectus, diuersa hominum conditio, diuersa nostri status ratio, et ita pastor scite accommodet quae sunt cuiusque orationis generis propria. Denique, vt ostenderet Paulus, regum, et principum, caeterorumque hominum commemorationem in nullo precum genere o- mittendam esse, tot species commemorauit, ita- que omnes negligentiae tollit excusationes. Cap.1. Quaeritur autem primum quomodo hae voces differunt δέησις, προς ἀχὴ  , ἔντάξις, et Εὐχαρι- στία Quod vt commode fiat primum explicetur et definiatur, quid sit oratio in genere Est oratio (ait Damascenus) πρεπόντων αἴτησις, id est, Decentium postulatio petitioque a Deo. Igitur complectitur res, quae sunt a Deo postulan- dae. Illas enim solas fas est, et decet nos petere a Deo. Dicitur autem a Latinis Oratio ab ore, quia ore fit. Nec enim orationem vocabant quae intus et corde tantum conciperetur, non etiam ore proferretur. Vnde orationis vim in dicendo maximam et in gestu esse respondent. Varro lib.  5. de lingua Latina. Hebraei vocant תפלה a iudi- cando, quia deiici nos coram Deo necesse est, si vere precari velimus, id quod ex animo sensuque nostrae inopiae facile fiet. Definitur etiam oratio Dei veneratio animi nostri vota et affectus illi  \pend
\section*{CAPVT  II. }
\marginpar{[ p.45 ]}\pstart pandens, vt auxilium ab eo impetremus. Id colli- gitur ex Tertulliano, et Psal. 142.vers.2 QQuotuplex autem sit oratio quaeritur, Resp. Est duplex, Implorans eaque pro nobis, vt Bona det Deus, Mala auertat et auerruncet: vel pro aliis. Et Agens gratias, Est Εὐχαριστία Psal.22. Quae inuocat et petit αἴτησις appelletur nomi- ne generali, et oratio siue ἀχὴ  Quae vero pro no- bis ipsis postulat a Deo bona προς ἀχὴ  dicitur, vt Psal 61. Quae petit vt a nobis auerruncentur et a- bigantur mala, δένσις vti Psal. 14o. Quae pro aliis interpellat et intercedit apud Deum est ἔντάξις, vel μεσιτεία. Haec tot sunt ge- nera precationum publicarum. Vide tamen Au- gust. Epist. 59. Cap. 2. Quaeritur autem a quo fieri preces debeant et possint. Resp. Ab omnibus in vni- uersum: sed priuatae a priuatis. Publicae vero a solis iis qui Ecclesiae praesunt, quales pastores et olim Diaconi. Ac quoad priuatos scribit Chryso- stom. iam olim fuisse solitos Christianos bis in die, scilicet, mane, et vesperi domi precari, adeo vt milites ipsi in castris id consuessent, et data est illis a Constantino breuis formula et precatio, quam mane et vesperi dicerent. Eusebius lib. 4. de vita Constantini. Sed de publicis precibus hic a- gimus. Primum igitur Pastores debent precari publi- ce. Ad eos enim haec res pertinet. Sed etiam ex veteri Ecclesiae disciplina Diaconi possunt et Presbyterii etiam ii, qui doctrinam non tractant. Eorum enim muneri preces adiunctas fuisse scri- ptum est, in canon Perlectis 5 ad Diaconum dist. 25  \pend
\section*{AD I. PAVL. AD TIM. }
\marginpar{[ p.44 ]}\pstart et apud Socratem Scholast. lib, 2. Histor. cap.11. etiam praesente Episcopo, quanquam eodem prae- sente sacramenta administrare Diaconus non potuit, vt est in cano. Peruenit distinct 93.nisi ab Episcopo iussus esset. Quod autem obiici potest de Solomone, qui, adstante summo sacrificator, etamem preces publi cas pro populo fudit, vt est 1. Regum 8. vers.14. non obstat. Hoc enim semel tantum factum est, et quidem extraordinarie. Nam Azarias rex et suc- cessor Solomonis, qui in Sacerdotii munus irrue- re voluit a Deo lepra percussus est, et segrex fa- ctus 2. Reg. 26. vers.18. Cap.3. Quaeritur pro quibus fieri debeant preces Resp. pro omnibus, id est, quouis hominum ge- nere, religione, sexu, aetate, conditione. Sic enim vox πάντες hoc loco sumitur, vti et postea quem- admodum doctissimus Beza obseruauit. Ratio est, Quod homines quosuis pro Dei creaturis, et pro proximis nostris agnoscero debeamus. Ergo omnium vita, salus, et conditio nobis curae de- bet esse, nec quisquam negligendus, quantunuis pauper, humilis et abiectus. Excipiunt tamen quosdam Scholastici, impri- mis inimicos nostros, pro quibus non putant spe- cialiter orandum esse. Communia enim humani- tatis officia tantum illis tribui oportere: non au- tem haec tam egregia, quae solis amicis debemus, qualis est oratio. Durand. in lib. 3. Sentent. distin. 3o. quaest. 1.et Thomas in 2. 2ae quaest. 25. Quae sententia quam sit falsa docet Christus ipse Matt. 5,vers. 44. Precamini pro iis, ait, qui vos infestant. item docet hic Paulus qui iubet nos precari pro.  \pend
\section*{CAPVT  II. }
\marginpar{[ p.45 ]}\pstart regibus, qui tunc proculdubio erant infesti hostes Christiani nominis, et persecutores. Quaeritur vero vtrum pro excommunicatis et haereticis sit in publicis precibus precandus Deus. Nam qui orat cum excommunicato, est et ipse excommunicatus canon. Qui communicauerit 1I.quaest.3. qui canon est ex consil. Chartag. 4. cap 73. Deinde dicit loan.ne salutandos quidem eos esse, aut Aue illis dicendum 2. Ioan. vers.1o.11. quare nec pro iis videtur esse precandum. Resp. Cum aberrantes sint in viam reuocandi. vt mo- net Iacobus cap.5. vers. 19. nec excommunicario ad subuersionem, sed ad aedificationem sit insti- tuta, vt conuertatur ad Deum qui peccauit.2. Cor. 7.vers. 1o.1o. vers.8.etiam pro haereticis et excom- municatis est precandum, eóque vehementius et ardentius, quo magis sunt a Deo illi alieni et exitio suo proximi. Atque haec breuis responsio est. Quaeritur etlam Vtrum pro viuis solum sit in Ecclesia precandum, an etiam pro mortuis: haec quaestio videtur ardua et valde difficilis propter morem veteris Ecclesiae, et propter ea, quae scribit Epiph. contra Ærianos haeres.75. qui hoc fieri de- bere negarunt. Omitto enim quod Papistae ex2. Machabus  12. vers.43, et a5. afferunt. Primum enim non sic legitur, vt legunt, Bonum est orare pro mortuis, et totus ille locus pertinet ad spem de resurrectione carnis, non ad preces pro mortuis. Deinde quod est vers.4o. Res est igitur salutaris et sancta, plane glossema esse apparet. Denique nec fidem meretur ille totus liber cum sit Apocry- phus, et excusationem authoris longe a Canoni-  \pend
\section*{AD I. PAVL. AD TIM. }
\marginpar{[ p.46 ]}\pstart cis libris dissimilem contineat cap.15v.39. Affe- runtur igitur testimonia veterum, imprimis au- tem Augustini lib.  de Cura pro mortuis cap. 18. quod canonizatur (vt vocant) cano. Non aestime- mus 13 quaest.2. item aliud eiusdem Augustini di- ctum cap. 11o. in Enchirid. Sed etiam Arnobius vetus scriptor Ecclesiasticus idem confirmat lib.  4. Aduers. gentes sub finem libri nempe pro viuis et mortuis in Dei Ecclesia precatos esse Chri- stianos. Resp. Nullo verbi diuini fundamento niti August. vel Arnobium, sed sola quadam er- ga mortuos beneuolentia et humanitate, solisque coniecturis. Id quod nec Augustinus ipse dissimu- lat. Praeterea possunt adduci alii eiusdem August. loci, ex quibus quam varius in ea re fuerit intel- ligetur, quemadmodum in Enchir. pridem annotauimus. Denique opponuntur Augustino scriptores, patresque Ecclesiastici, et ii quidem pii, et Orthodoxi, quemadmodum Ambrosius in libus  qui inscribitur Abraham cap.9. vbi id tantum mortuis a viuis praestandum docet, vt eos sepe- liant: non vt pro iis orent, id quod etiam videtur confirmari posse responso Christi quod est Matt. 8. vers. 22. Obiicitur etiam Hieronymus, cuius haec est apertissima sententia. In praesenti seculo siue orationibus siue consiliis inuicem posse nos iuuari. Cum autem ante tribunal Christi veneri- mus, nec Iob, nec Noe, nec Daniel, rogare posse pro quoquam. Quam sententiam Papistae ipsi in suos canones retulerunt cano. in praesenti 13 dist.2. Origo autem huius mali ex paruis initiis cepit. Primum ex eo quod cepit piorum, qui pro nomi- ne Christi passi erant, et mortui, in Ecclesia et  \pend
\section*{CAPVT  II. }
\marginpar{[ p.47 ]}\pstart \phantomsection
\addcontentsline{toc}{subsection}{\textit{2 Pro regibus et quibusuis in eminentia constitutis: vt tranquillam ac quietam vitam degamus cum omni pietate et honestate.}}
\subsection*{\textit{2 Pro regibus et quibusuis in eminentia constitutis: vt tranquillam ac quietam vitam degamus cum omni pietate et honestate.}}imprimis in Synaxi, publica mentio publice fieri. Vnde Martyrologia nata sunt, qui sunt libri in quibus nomina Martyrum descripta erant, quae ex scripto in communicatione Coenae recitaban- tur. Inde ad eorum sepulchra ceperunt homines precari, et vigilias agere, et liba offerre, vt docet Augustinus in lib.  de Moribus Ecclesiast. cap.14. et Epist.119. quem tamen morem idem Epist.64 sublatum et abolitum esse cupit. Addunt Papistae quasdam exceptiones alias, quae sunt nullius momenti. Ac primum interdi- cunt ne omnino pro iis oretur, qui sibi manus vio- lentas attulerunt: qui si mortui sunt, vera est eo- rum sententia, sin minus, falsa. Possunt enim resi- piscere. Est tamen hoc Papistarum dictum in cano. Placuit vt hi 23. quaest.5. et est ex concil. Bracca- rens. Act. 34. Deinde iubent, ne precemur pro clericis qui in bello pro Gentilibus occubuerunt, dum eorum partes et castra sequuntur. Est ex Tiburiensi con- cilio in cano. Quicunque clericus 23. quaest.8. Sed haec exceptio vana est, quia non est pro mor- tuis orandum. Excipiuntur igitur ab hoc Pauli dicto ii soli, qui in Spiritum sanctum peccant, pro quibus mi- nime est precandum, quemadmodum docet loan. 1. Epist5.vers.16. et exemplo Saulis confirmatur 1. Samuel 16. vers. 1. 2 Pro regibus et quibusuis in eminentia constitutis: vt tranquillam ac quietam vitam degamus cum omni pietate et honestate.  \pend
\section*{AD I. PAVL. AD TIM. }
\marginpar{[ p.48 ]}\pstart Πρόθεσις quae non modo superius praeceptum illustrat: sed|etiam explicat. Addit enim illud i- psum, de quo magis aliqua et anceps quaestio e- rat pro ratione temporis, quia tum omnes pene magistratus: imprimis autem summus, qui erat Romanor. Imperator, erant et infideles, et per- secutores Ecclesiae. lubet tamen vt pro iis Magi- stratibus precemur, non tantum Summis, quales regum nomine significantur: sed etiam lnferioribus, qui describuntur ex eo, quod supra reliquum popu lum eminent Dubitari vero de iis maxime potuit, quod Euangelicae doctrinae essent hostes, qualis Nero Imperat. dissimilis locus 1. Tessalon.2.v. 16. vbi qui praedicationem Euangelii impediunt, in manifestorum reproborum numero et albo recensentur. Deinde vetus Ecclesia preces con- cepit aduersus Iulianum Apostatam imperatorem. Sed solutio ex eo est, quod ii, de quibus agit Paulus in Thessal. cap. secundo non ignorantia, non sola infidelitate animi : sed obstinata pror- sus malitia peccabant, et tanquam in Spiritum sanctum, vti peccauit Iulia. Apostata. Distinguendi vero sunt qui hoc modo peccant ab istis, dequi- bus agit Paulus hoc loco. Similis locus est, Hier. 29. vers. 7. orate pro pace Babylonis. item orate pro iis qui vos per- sequuntur. Denique vetus mos Ecclesiae idem pro- bat, in qua pro Imperatoribus et praesidibus pro- uinciarum, quanquam eam affligerent, orabatur, quemadmodum scribit in Apologet. Tertullia. et Iustin. Martyr. Donatistae tamen contra disputant, quod Im- peratoris edictoipoena in eos, si in errore persta-  \pend
\section*{CAPVT  II. }
\marginpar{[ p.49 ]}\pstart rent, indicta esset. Nullos enim' aut paucos o- mnino reges pios fuisse contendunt, vel Ecclesiae fauentes. Sed eorum argumentis ineptissimis et falsissimis respondet Augustinus in lib.  Contra secundam Epist. Gaudentii. Addit autem rationem Paulus, quo et superiorem exhortationem, confirmet, et nos ardentius inflan- met, atque stimulos addat et excitet ad officium. Est autem ducta haec ratio ab immensa quadam vtilitate, quae triplex hic enumeratur, nimirum quod Ma- gistratuum ope et ministerio Pax, Pietas, et Ho- nestas inter homines stabilitur et conseruatur. Ac pax quidem siue tranquillitas tum publica tum priuata, ad quam constituendam ordinatus est a Deo Magistratus: atque etiam Dei praece- pto gladium gerit, vt docet idem Paulus Rom.13. Vtraque vero pax est Dei donum, et summum omninoque necessarium humanae societatis reti- nendae vinculum. Pietas vero, quae Dei cultum continet etiam ad Magistratus politici, non tantum pastoris curam et officium pertinet, quia vtriusque tabulae con- stitutus est custos Magistratus. Conuenit autem haec sententia cum Psalmo 101. et cum exemplis Iosiae, Ezechiae, Theodosii, Constantini Magni et aliorum piorum regum, qui cultum Dei depraua- tum ex ipsius verbo reformarunt et restituerunt. Ex hoc autem loco concludi certissime potest e- tiam inquisitionem de haeresi, et punitionem e- orum, qui merito erroris et haeresis damnati sunt ad Magistratum pertinere, quemadmodum etiam disputauit Augustinus, quamquam Castallionistae hoc nostro seculo negant, et farraginem omnium-  \pend
\section*{AD I. PAVL. AD TIM. }
\marginpar{[ p.50 ]}\pstart que errorum licentiam concedendam scribunt et deffendunt, ne quis, aiunt propter suam opi- nionem puniatur, quasi vera religio sit opinio quaedam et hominum commentum. Honestas decorum proprie est, vt non tantum nostra cum proximo commercia regantur rectis et aequis legibus, sed etiam omnis honesta mo- destaque conuersatio inter nos locum habeat, e- tiam in rebus mediis, et quas indifferentes vocant, veluti in communi vestitu, victu, officio, et qua- tenus pro quoque hominum genere et vocatione distinctio quaedam officiorum est inter homines adhibenda. Haec ἀταξία ordo et politia honesta magnam aedificationem habet, et verae pietatis quodammodo custos est. Confutat autem hic locus pulcherrime Ana- baptistas, qui Magistratum ex Ecclesia Dei tol- lunt, vt pestiferam ἀναρχίαν inducant. Quam sit au- tem illius vsus vtilis et Ecclesiae necessarius vel vna haec Pauli sententia perspicue demonstrat. Sed quaesitum est, Num pro iis tantum Magi- stratibus sit precandum, qui suo recte defungun- tur officio, et a quibus pax, pietas, honestasque constituitur aut conseruatur. Resp. Ipsum Ma- gistratus finem, qualis a Deo praescribitur, spe- ctandum, non autem vitia personarum quae eos gerunt. Hic enim in vniuersum harum vocatio- num finis est, quem proponit Paulus, propter quem nobis commendati esse debent Magistratus, id est, qui munus publicum gerunt, quanquam male of- ficio suo fangantur. Sed pro bonis orandum est, vt eos nobis Dominus conseruet. Pro malis au- tem vt eos conuer tat et ad officium faciundum  \pend
\section*{CAPVT . II. }
\marginpar{[ p.51 ]}\pstart \phantomsection
\addcontentsline{toc}{subsection}{\textit{3 Nam hoc bonum est, et acceptum co- ram seruatore nostro Deo. 4 Qui quosuis homines vult seruari, et ad agnitionem veritatis venire}}
\subsection*{\textit{3 Nam hoc bonum est, et acceptum co- ram seruatore nostro Deo. 4 Qui quosuis homines vult seruari, et ad agnitionem veritatis venire}}excitet spiritu suo. Itaque semper pro iis orandum est. 3 Nam hoc bonum est, et acceptum co- ram seruatore nostro Deo. Αιτιολογία est, eaque duplex. Nam altera ducitur a Nostro officio vel rei ipsius natura καλόν ἐστι alte- ra a Consequenti vel connexis, sunt preces tri- buendae Magistratibus, quia et eos Dominus ad gratiae suae participationem et Ecclesiae commu- nionem vocat. Itaque hoc Ecclesiae subsidio pri- uandi non sunt. Ex hoc autem loco colligitur ec- quod sit verum nostrarum precum fundamentum nimirum Dei voluntas et promissio. Haec enim vna est optima recte legitimeque orandi regula quemadmodum etiam tradit Ioan. in 1. Epistola cap. 5.vers. 14. Denique haec eadem valet in o- mni cultus Dei parte. 4 Qui quosuis homines vult seruari, et ad agnitionem veritatis venire Altera ratio quae a connexis sumpta est. Non sunt enim priuandi et excludendi a publicis Ec- clesiae precibus ii, ex quibus Deus ipse colligit Ecclesiam, et qui ad eam spe promissioneque a Deo accepta pertinent. At Magistratus, etiam qui nunc sunt a Dei cognitione alienissimi, spe tamen ad eam pertinent. Quamobrem non sunt eo fructu, dono et ea spe defraudandi. Ac propo- sitio quidem huius syllogismi verissima est, quae non tantum hac ratione quae naturalis est confir- matur, quod quae sunt inter se connexa, non sunt  \pend
\section*{AD I. PAVL. AD TIM. }
\marginpar{[ p.52 ]}\pstart diuellenda, sed etiam authoritate Scripturae Act. 11.vers. 17. Non est nostrum( inquit Petrus) Deum prohibere, et iis gratiae testimonia aut aditum denegare et praecludere, quibus eam Dominus i- pse largitur et concedit. Voto enim Dei subser- uire debemus. Assumptio vero confirmatur a Paulo, et hic disertissime est expressa, adhibeturque argumen- tum a genere ad speciem, sic, vult Deus omnes ho- mines saluos fieri, et ad fidem et Ecclesiam vocari, Ergo et Magistratus. Quaesitum vero est, quae sit huius tam genera- lis Pauli sententiae ratio, vult Deus omnes homi- nes, etc. Respon. Explicari hac sententia anti- qua Prophetarum vaticinia, quae de vocatione Gentium loquuntur, in quibus, Dei gratia omn i- bus hominibus, sublato nationis, sexus, aetatis, et ordinis discrimine, promiscue promittitur, in pri misque illa differentia quae olim inter Gentes et Iu- daeos constituta erat, hodie cefsat. Quale vaticinium est Ps.2. Pete a me et dabo tibi Gentes in haeredi- tatem tuam Malac. 1.V.11. Nomen meum a solis ortu ad occasum magnum est etiam inter Gentes lsai.11.v.10. Et illo tempore, erit, requirent Gentes radicem Iesai, etc. Imprimis autem de vocatione magi- stratuum ad fidem et Ecclesiam iidem Prophetae Dei concionati sunt, veluti Isai.32.V.1. et2. Ecce in iustitia regnabit rex, et principes in iudicio praeerunt. Et erit ille vir velut latibulum a ven- to, receptus ab imbre, riui aquarum in terra ari- da: vmbra magnae rupis in terra laboriosa. etc. i- tem cap. 6o. vers.16. Et suges lac Gentium, ma- millam regum suges. Quae omnia Deiverba non dubi-  \pend
\section*{CAPVT  II. }
\marginpar{[ p.53 ]}\pstart tat Paulus impletum iri. Itaque vere etiam Magi- stratus ad Dei Ecclesiam pertinere pronuntiat. Deinde a natura et definitione Euangelii idem probari potest. Est autem Euangelium potentia Dei ad salutem omni credenti siue Iudaeo, siue Gentili, vt docet idem Paulus Romanor. 1. vers. 16. Nec enim, vt Lex, sic Euangelium vni tantum hominum generi et nationi destinatum aut pro- ponendum erat. Ex hoc autem loco intelligimus etiam nos pro Ethnicis et Gentilibus veluti pro Turcis, ludaeis, et iis qui adhuc in orbe terrarum idololatrae manent (quales in India et insula A- merica innumerabiles pene sunt populi) Deum orandum esse. Id quod etiam hoc loco Chriso- stom. annotauit, quanquam de eo variae fuerunt Augustini tempore quaestiones, quemadmodum ex ipsius Epistolis apparet. Modum etiam per ἐξήγησιν addit, quo ad salu- tem homines perducuntur, nimirum veritatis a- gnitionem. Hic autem veritas non est accipienda cuiusuis cognitionis doctoris et disciplinae certum effatum, certaque sententia, et cum reipsa con- sentiens, sed Euangelium tantum quod κατ' ὀνομα- σίαν et insigniter veritas appellatur, tum quod illa sit certissima doctrina et a Deo ipso immediate profecta: tum etiam quod sola sit veritas aestiman- da, a nobis et consectanda, et persequenda: reli- quae vero artes quatenus huic subseruiunt et ad vitae huius commoditatem pertinent, exercendae. Ex quo etiam colligitur non alios velle Deum saluos fieri, quam qui ad fidem Euangelii perue- ni unt, illique credunt. Non sunt enim ista duo distrahenda, quae hic Paulus coniungit, ne quis se  \pend
\section*{AD I. PAVL. AD TIM. }
\marginpar{[ p.54 ]}\pstart putet contempto spretoque Euangelio salutem aeternam consequi posse, ad quam sola fides est via. Id quod tamen hodie multi factitant, quà quemque in sua, quam vocant, religione saluum fieri sentiunt. Quae fuit Rethorianorum haeresis, nunc autem est Turcarum impiissima sententia. Varie vero de horum verborum. Vult Deus omnes saluos fieri sensu quaesitum est, quemad- modum ex Augustino apparet in Enchirid. cap. 1o3. Primum enim videntur Libertinistarum, et Origenistarum errorem confirmare, qui negant vllum hominem esse a Deo damnandum, et ae- terna morte puniendum. Deinde etiam fauere eorum errori, qui reprobationem Dei prorsus tollunt, quasi omnes homines sint a Deo electi, nulli autem roprobati. Denique hoc ipso loco perperam explicato se tuentur et Pelagiani, qui liberum ad vtranque bene et male agendi ele- ctionem arbitrium in nobis statuunt: et Semipe- lagiani quoque, qui Dei gratiam de congruo quam appellant, cum libero nostro arbitrio tanquam duo simul et aequaliter cooperantia in bene agen- do coniungunt. Hi vero errores omnes, ex eo nascuntur, quod vox ipsa omnes, non recte su- mitur hoc loco. Ac Scholastici quidem nonnun- quam ita sentiunt ideo omnes dici a Deo saluos fieri quia dedit Deus omnibus hominibus natu- ram per nos ordinabilem ad felicitatem vt loquun- tur. Contra vero nostra felicitas non ex naturae nostrae conditione, sed ex mera Dei gratia pen- det, et eatenus ordinabilis est ad eam nostra na- tura, quatenus Deus ipse nos ita fingit, destinat, et ordinat: non autem quatenus reliquis homi-  \pend
\section*{CAPVT  II. }
\marginpar{[ p.55 ]}\pstart nibus pares et similes naturâ sumus. Alii sic explicant, vt vocem omnes coniungen- dam esse doceant cum eo quod sequitur ad verita- tis agnitionem peruenire, quasi non vniuersaliter neque tam late sit accipienda quam sonat: sed ex sequenti illa sententia restringenda, vt ii tantum intelligantur comprehendi, qui credunt, vel cre- dituri sunt Euangelio. Consir matur ex Matth.14 vers.35.36. Tertia interpretatio est eorum qui volunt o- mnes saluos fieri a Deo qui salui fiunt: quasi hic non definiatur, qui salui futuri sint: sed a quo sal- ui fiant, qui saluantur. Sic nonnunquam explicat Augustinus. Verior autem sententia et iustior, meo quidem iudicio, interpretatio est haec, vt vox omnes tol- lat discrimen ordinum, nationum, sexuum et huius- modi rerum, quae inter homines percipiuntur. Ex omni enim hominum genere, sexu, aetate, Deus aliquos ad se per Euangelium vocat. Itaque non pro singulis generum accipitur, sed pro generibus singulorum vt loquuntur in Scholis, id est, non pro personis, sed pro hominum generibus. Duplex est enim huius vocis significatio. Saepe enim ita dicimus omnes, vt singulos complecta- mur: saepe vero, vt quosuis, non autem singulos. Sic dicitur Christus sanasse πάντα νόσον, id est, quem- uis morbum. Matt. 9.v.35. Sic Paulus in 2. Thess. 1.V.3. coniungit ἕκαστον cum voce πἀντων. Et hanc esse duplicem huius vocis significationem pri- mus obseruauit Aristoreles lib.  2. πολιτικ opri- mus Graecae linguae author et interpres.  \pend
\section*{AD I. PAVL. AD TIM. }
\marginpar{[ p.56 ]}\pstart \phantomsection
\addcontentsline{toc}{subsection}{\textit{5 Vnus enim Deus, vnus etiam media- tor Dei et hominum, homo Christus Iesus.}}
\subsection*{\textit{5 Vnus enim Deus, vnus etiam media- tor Dei et hominum, homo Christus Iesus.}}5 Vnus enim Deus, vnus etiam media- tor Dei et hominum, homo Christus Iesus. Αἰτιολογία superioris sententiae, ab effectu. Non esset vnus Deus, et vnus mediator omnium, nifi omnes, id est, quosuis homines saluos faceret. Si enim vnius tantum hominum generis vel ordinis salutem procurat et perficit Deus, necesse est plures deos constitui:itemque plures mediatores Quorum vtrunque blasphemum est. Similis locus est Romanor.3.vers. 29. Psal.1o5. vers.7 et 33.vers.5. Obstat autem Psal.76. vbi in sola Iudaea notus Deus esse dicitur. Responsio est, Sublatum nunc esse inter Gentes et Iudaeos discrimen, quod olim fuit, quia vtrique, diruta per Christum maceria, in vnum populum coaluerunt. Deus autem quorumuis hominum vnus neque esse, neque dici potest, nisi suae bonitatis, clemen- tiae, misericordiae et electionis testimonia et ef- fecta proferat in quosuis. Quamobrem quosuis ad salutem vocare et efficaciter quidem debet. Iam vero ex hoc Pauli responso Manichȩorum, Marcionitarum et huiusmodi aliorum haeretico- rum error fanaticus refellitur, qui duos Deos, duoque principia constituunt, et alium Iudaeorum Deum, alium autem nostrum somniant. Vnus au- tem est omnium Deus, non plures. Sed etiam mediator vnus est, non plures. Is autem Mt Christus homo. Quod variis rationibus con- firmari potest. Prima, Quod vnicum est semen illud Abrahae, in quo promittuntur benedicen- dae omnes mundi nationes Genes.15. et 17. Galat.  \pend
\section*{CAPVT  II. }
\marginpar{[ p.57 ]}\pstart 3.Vers. 16. Secunda, Quod ad vnum et eundem nos reuocat tota veteris et noui testamenti scri- ptura. Veteris enim Legis caeremoniae et sacrifi- cia nos ad eundem Christum deducunt, ad quem etiam Euangelium. Itaque Christus is, qui per Euangelium praedicatur, dicitur esse finis legis. Tertia, Quemadmodum Deus non habet duos filios natorales: ita neque duos mediatores con- stituit, sed de vno tantum illa Dei vox est, Ille est Filius meus dilectus, in quo mihi complacui Matth.3. Consentit cum hoc dogmate, et quidem verissi- mo Augustinus qui lib. 2. contra Epistolam Par- menia. cap. 8. negat plures esse mediatores ho- minum: sed vnum tantum Iesum Christum. Similis est locus in Epistola ad Hebraeos cap. 7.vers. 26. Non enim potuit quilibet Pontifex a- pud deum munere mediatoris pro nobis et offi- cio fungi: sed is solus, qui pius, innocens, segrega- tus a peccatoribus, et sublimior coelis factus est. Is autem est Christus solus. Obstare vero multa videntur quae afferri pos- sunt. Ac primum. Prima ratio, Quod alii pro aliis orare iube- mur. Iac.5. Responsio est, quod nostrae illae preces non propter nos gratae sunt Deo: sed vnius Chri- sti merito et intercessione. Eatenus enim De o placent, quatenus et fidei et mutuae inter nos charitatis sunt effecta atque fructus, quae vtraque innititur Christo. 2 Moses et Sacerdotes Leuitici dicuntur fuisse intercessores et mediatores inter Deum et po- pulum Exod.32.Resp. Puerilem esse obiectionem.  \pend
\section*{AD I. PAVL. AD TIM. }
\marginpar{[ p.58 ]}\pstart Illic enim intercedere nihil aliud significat quam internuntium et medium esse: non autem ipsam Dei gratiam nobis promereri. Tertia, Obiicitur, locus Galat.3.v,2o.Respo. Est sophistica obiectio. Negatur quidem inter- nuntius et mediator esse vnius tantum, non autem negatur esse vnus. Quarta, Angeli et Sancti vita functi, sunt no- stri coram Deo intercessores. Nam sine mediato- re non possumus accedere ad Deum, vti nec re- gem sine purpurato, qui nos intromittat, adi- mus. Resp. Praeter verbum Dei et Angeli et san- cti nostri mediatores inducuntur. Nam de Ange- lis nominatim vetat Paulus ne colantur. Coloss.2. De sanctis vita iam defunctis ratio prohibet, quod et ipsi mediatore egent: neque fuerunt sancti et impolluti. noster autem aduocatus apud Deum debeat esse δίκαιος vt docet 1. Ioan.2. vers.2. Quod enim affertur de mediatore alio inter- cessionis et salutis, conuellitur vel ex hoc ipso Pauli loco, vbi agitur de precibus et intercessio- ne. Item Hebr.7.vers.15, et 9.vers. 24. Addit Paulus Christum hominem esse. Pri- mum quod ea ratione Christus mediator noster est, quatenus homo pro nobis factus est. Deus e- nim manens pati et implere illa, quae sunt nobis ad salutem necessaria, non potuit. Deinde ne im- mensus ille diuinae maiestatis fulgor et splendor, ad quem nobis, qui sumus terrae vermes et ho- mines, accedendum est nos perterreat. Habe- mus enim qui nobis facilem ad Deum aditum prae- beat, Christum nempe, qui nostris infirmitatibus compatitur, et qui, quia nostram naturam assum-  \pend
\section*{CAPVT  II. }
\marginpar{[ p.58 ]}\pstart \phantomsection
\addcontentsline{toc}{subsection}{\textit{6 Qui semetipsum dedit redemptionis pretium pro quibusuis, Christus inquam, te- stimonium illud suis temporibus destinatum.}}
\subsection*{\textit{6 Qui semetipsum dedit redemptionis pretium pro quibusuis, Christus inquam, te- stimonium illud suis temporibus destinatum.}}psit, nobis iam formidabilis esse non potest. Heb 4.vers.15. Bernard. sermo.73. Canti. in tanta trepi- datione electis fiduciam praestat naturae similitu- do. Denique vt facilius quia naturam omnibus hominibus communem Christus assumpsit, in- telligamus quorumuis hominum eum mediatorem esse, non vnius tantum hominum generis, huma- nitatis vel hominis Christi mentionem fecit Pau- lus. Quaesitum est autem, Vtrum Christus, qua tantum homo est, noster mediator sit, quod et Stan- carus, Menno quidam, et hodie noui Arriani sentiunt, an etiam quâ Deus est. Resp. Cum me- diatoris oeconomia et officium ad totam Christi petsonam pertineat, neque aliter saluator no- ster esse potuerit Christus, nisi esset Emmanuel id est, nobiscum Deus, idcirco et qua Deus, et qua homo est Christum mediatorem nostrum esse fateamur necesse est. Hoc autem docet et confirmat August. cum in Enchirid. tum vero pulcherrime in libro de Ouibus  et lib.  Confessionum. De qua re cum co- piosissime doctiss.nostri temporis Theologi D. Caluinus in Epistola 312. et Theodo. Beza Epist. 28. disseruerint, plane superuacaneum puto latius hoc argumentum persequi. 6 Qui semetipsum dedit redemptionis pretium pro quibusuis, Christus inquam, te- stimonium illud suis temporibus destinatum. Αὔξησις est, per quam non modo superiorem rationem confirmat a connexis: sed etiam me-  \pend
\section*{AD I. PAVL. AD TIM. }
\marginpar{[ p.oo ]}\pstart diationis et intercessionis Christi pro nobis fun- damentum explicat. Est autem ipsius sacrificium pro nobis, quod hic πεεεγρασν quadam describi- tur. Sunt enim res inter se connexae, sacrificium et intercessio apud Deum: atque etiam ita inter se comparatae, vt vna alterius causa sit. Id quod his Epistolae ad Hebraeos locis confirmatur cap. 4.verf.14. 5vers.1. 7. vers. 15. 8.vers.3. 9. vers.24. at- que etiam hac ratione. Quod cum hic sit inter- cessionis finis, vt Deum nobis propitium et be- neuolentem reddat, is demum apud Deum pro nobis intercedere potest, qui eum nobis placare potest. Placatur autem Deus, non alia ratione, quam plena poenae peccatis nostris debitae persolu- tione. Id autem solum Christi sacrificium et mors potuit. Ergo illa mors Christi intercessionis pro nobis est fulcimentum et fundamentum. Vnde perperam Scholastici, qui salutis et intercessio- nis mediat ores diuersos faciunt. Quod esse non posse satis ex superiore argumento apparet. Definitur autem et ornatur hoc Christi sacri- ficium magna laude et encomio, quod ductum est ab ipsius effecto. Id autem est Redemptio nostra, cuius illud sacrificium fuit pretium integrum, plena merces, et iusta satisfactio siue persolutio. Dicitur autem Christus non tantum Seipsum de- disse: sed etiam Ipse se dedisse. Nam neque aliud pro peccatis noitris, quam seipsum dedit: neque cum dedit, inuitus aut ignorans id fecit, sed volens. In quo ipso com mendatur impense Christi erga nos bencficium et charitas. Similis locus. Tit 2.V.14. Nec caret ἐμρέσι, quod sanguis Christi illo sa- crificio effusus appellatur ἀν τιλvreςν, quia sanguis  \pend
\section*{CAPVT  II. }
\marginpar{[ p.61 ]}\pstart ille Christi agni immolati longe est maioris pre- tii, et effectus, quam vitulorum et hircorum omnium sanguis. Est enim ille agni immolati sanguis, o- mni auro lapideque pretioso aestimabilior et potior, vt est Hebr. 9.vers. 12.1.Pet.1.vers.19. et vere plena et aequalis nostro peccato et debito satisfactio et pretium 1.Corinth.1.vers.3o. Vnde non tantum λύτρον: sed ἀντίλυτρον dicitur, quod Latini dicunt contra auro venire, id est, iustum, et aequale rei ipsi pretium esse. Ex quo diluitur illa curiosorum hominum disputatio, Vtrum fuerit poena, quam pro nobis Christus pertulit, ae qualis peccatis nostris, vt peccata ex merito Christi de- leantur. Fuit enim, vt hic docet Paulus, aequiua- lens, id est, aequata ipsi peccatorum nostrorum foetori apud Deum, et plane satisfactoria poena Christi perpessio et obedientia. Secundo vero loco notandum est, Nullam a- liam satisfaciendi diuinae iustitiae pro peccatis nostris rationem esse a Deo constitutam, illique gratam et acceptam praeter sanguinem vnius Chri- sti. ltaque nec opera quae vocant bona: nec caere- moniae, nec vllius alicuius rei, quantunuis pretio- sae, donatio, aut ratio potest pars esse satisfactio- nis nostrae coram Deo, que in solidum in vna Chri- sti morte quaerenda nobis est. Sed nec ex par- te tantum Christus satisfecit, veluti vt deleat ea tantum peccata, quae dicuntur venialia: aut quae ante Baptismum commisimus, nobisque noce- bant: sed etiam expungit, et redemit: Addit Paulus τὸ μαρτύριον καιροῖς ἰδίοις. Haec a- brupta videtur esse oratio, itaque obscurum ha- bet sensum, et diuersas interpretationes. Alii e-  \pend
\section*{AD I. PAVL. AD TIM. }
\marginpar{[ p.62 ]}\pstart nim ad sequentem versiculum referunt: alii huic coniungunt: alii mutandam censent vocem μαρτύ- ριον in vocem μυςτήριον. Mihi vero videtur planus et facilis horum verborum sensus, si ab initio quin- ti versiculi ad hunc vsque locum parenthesin pro- duci intelligamus. Itaque continua orationis se- rie haec, quae iam sequuntur, coniungantur, cum illa superiore Pauli sententia. Qui quosuis homi- nes vult saluos fieri, et ad agnitionem veritatis perue- nire. (Vnus enim Deus et c. iuxta testimonium pro- priis temporibus patefactum, ad quod ipsum implen- dum et exequendum, ego constitutus sum, etc. At supplendam esse post haec verba τὸ μαρτύριον καιροῖς ἰδίοις vocem φανερωθὲν apparet ex Tit. 1. vers.2. et Coloss.1.vers.26. Vocat autem hic testimonium Paulus, antiqua Prophetarum vaticinia de voca- tione quorumuis hominum et gentium, ne id fru- stra et temere Ecclesiae polliceri ipse videatur Paulus nulloque in eo niti sacrae scripturae testi- monio et fundamento, qualia tamen multa supra cap. q.annotauimus et ipse Paulus obseruauit et affert in Epist. ad Romanos cap.15. vers 9.1o.11.12. Ergo vocem μαρτύριον non refero ad Christum: sed ad totum complexum superius Vult Deus quos- uis homines saluos fieri. Hoc testimonium et illa vaticinia antiqua ignota aut etiam obscura iis i- psis, quibus antea annuntiabantur, suis, id est, E- uangelii temporibus patefacta sunt, et illustrata atque impleta. Nouit enim solus Dominus tem- porum articulos, et rerum opportunas maturita- tes, quas ipse decernit et constituit. Itaque huius rei perficiende gratia Paulus caeterique Apostoli a Christo et vocati et missi sunt. Id quod ipse se-  \pend
\section*{CAPVT  II. }\pstart \phantomsection
\addcontentsline{toc}{subsection}{\textit{7 Cuius constitutus sum ego praeco et Apostolus (veritatem dico per Christum, non mentior) doctor, inquam, Gentium cum fide ac veritate.}}
\subsection*{\textit{7 Cuius constitutus sum ego praeco et Apostolus (veritatem dico per Christum, non mentior) doctor, inquam, Gentium cum fide ac veritate.}}quenti versiculo subiicit, et explicat. 7 Cuius constitutus sum ego praeco et Apostolus (veritatem dico per Christum, non mentior) doctor, inquam, Gentium  cum fide ac veritate. Α’πτιολoyια est et consirmatio proxime supe- rioris sententiae ab effectu vel a consequente, Deum n.id decreuisse dubitari non potest, cum executus sit, et propterea Paulum miserit, qui per Euangelii praedicationem quosuis vocaret ad agnitionem veritatis. Itaque suum apostolatum, qui a Deo e- rat, pro iusta huius testimonii et voluntatis Dei de vocandis Gentibus probatione affert Paulus. Vocat autem se Κήρυκα, id est, Praeconem et Aʹπό- στολον, quae duo ita inter se videntur differre, quod illud est generalius: hoc specialius. Plures enim sunt verbi Dei κήρυκες et praecones, quam Apo- stoli: tum deinde, quod κήρυξ dicitur Paulus ratio- ne executionis ipsius ministerii, et praedicationis verbi Dei: Α’πόςτολος autem ratione vocationis et gradus, in quem a Deo collocatus et assumptus erat. Hoc autem ipsum postea definit cum ait δι- δάσκαλος ἐθνῶν. Quae et superiorem nostram sen- tentiam et interpretationem confir mant, et finem ministerii Apostolici, imprimis autem Paulini, ostendunt. Obstat quod Romanor.1. se Iudaeorum quo- que debitorem appellat. Resp. Gentium praeser- tim causa designatus erat Apostolus Actor.13, et Galat.1.vers.16.2.vers.8. quanquam Iudaeis Euan- gelium quoque pro re nata annuntiauit.  \pend
\section*{AD I. PAVL. AD TIM. }
\marginpar{[ p.64 ]}\pstart Iureiurando etiam suam vocationem confir- mat, quod vocatio Gentium tunc res noua pror- sus et aliena a Dei consilio censeretur, qui tanto tempore solos Iudaeos pro suo populo agnouerat et res ista esset magni momenti. Quamobrem non videtur temere et de re nihili sumptum a Paulo Dei nomen. Similis autem iuramenti for- mula est etiam apud eundem Paulum Romano. 9.vers.1. Εν χρίστῳ dupliciter vel in Christo, id est, Chri- sto linguam meam et mentem dirigente, et commo- uente: vel per Christum, id est, teste Christo i- pso, quem huius rei et meae vocationis et mune- ris testem produco. In qua postrema sententia conueniunt docti interpretes. Quaeritur autem Num vero tunc per creaturas iuretur, quum per Christum iuratur. Sed responsio facilis est. Non iurari, quia Christus non tantum est homo et creatura: sed etiam Deus. Deinde hic videtur Christus potius testis, quam iudex produci. Te- stes vero possunt a nobis, vt a Prophetis, appel- lari etiam mutae creaturae, nedum Christus. Addit denique in Fide, et Veritate. Quibus verbis et eam doctrinam confirmat, quam Gen- tes docebat cum fidem appellat: et animi sui syn- ceritatem, cum huic fidei veritatem coniungit. His autem verbis breuiter comprehensa est veri pastoris definitio. Est enim is, qui in Ecclesia Dei legitime vocatus docet fidem, id est, sa- nam doctrinam in veritate.i.sana coscientia et re- cto fine. Quam definitionem confirmat Petrus 1. Pet.5.vers.2.3.4.  \pend
\section*{CAPVT  II. }
\marginpar{[ p.65 ]}\pstart \phantomsection
\addcontentsline{toc}{subsection}{\textit{8 Velim igitur viros precari in quouis loco, puras manus attollentes absque ira et disceptatione.}}
\subsection*{\textit{8 Velim igitur viros precari in quouis loco, puras manus attollentes absque ira et disceptatione.}}8 Velim igitur viros precari in quouis loco, puras manus attollentes absque ira et disceptatione. Hic versiculus varia capita complectitur, qua ad orationem pertinent, quaeque breuiter quidem sed commode Paulus hic tradit. Tria enim com- prehensa sunt. 1 Quo animi affectu seu praepa- ratione orandum sit. 2 In quo loco. 3 Quo gestu. Sunt autem haec prae cepta non Pauli, fed Dei. Itaque vox βούλομαι non priuatum quoddam hu- manae mentis commentum significat, vti nec 1. Corinth. 1o. vers.1. sed praxin horum praecepto- rum a se serio requiri docet his verbis Paulus, si- ne qua non potest probari, et Deo grata esse no- stra oratio. Ac quod ad praeparationem (nec enim illotis manibus et impraemeditati ad Deum orandum debemus accedere) duo imprimis requirit a no- bis, Sanctitatem siue puritarem vitae, et Fidem. Sanctitatem autem illam vitae designat et de- scribit ab enumeratione partium, nempe a Ma- nuum sanctitarte et puritate, et a cordis charita- te. Vult enim et manus nostras (id est, opera ex- terna) esse sanctas, vt est Isai.1.vers.13.et corda ab omni in proximum ira et irritatione vacua, vt est Matth.6. vers.15.5. vers.24. Vera enim sanctitas tum externis, tum etiam internis operibus con- stat, et definienda est: et impium est, si quis vel ani- mo vel opere ipso sceleratus et profanus ad Deum orandum accedat, sine animi poenitentia et re- Iipiscentia Psal.5.vers.6. Id quod etiam homines  \pend
\section*{AD I. PAVL. AD TIM. }
\marginpar{[ p.66 ]}\pstart nrofani senserunt vti Hesiod. lib. 1. ἔργ. καὶ ἠμερ. et Plato, et ex Platone M. Tull.libus 2. de Legibus. Videtur etiam haec sanctitas, quae vera est, hic esse opposita a Paulo omnibus illis externis riti- bus, lotionibus, et carnalibus purgationibus, quae veteri Dei lege hominibus templum ingressuris praecipiebantur, de quibus agit Apostol. ad Hebus  9. vers 1o. et Moses Leuitici cap. 6. 13.15. 16.19. quibus ingens etiam cumulus a Pharisaeis additus erat postea per δευτερώσεις, vt apparet Mar.7.V.4. Quae lotìones antiquae a Deo praȩceptae verae qui- dem animi vitaeque sanctitatis signa erant et fi- gurae: earum tamen implementum et corpus in Christo habetur et confertur. Quanquam vero de publicis precibus hic agit Paulus: in priuatis tamen eundem animum sanctum et purum re- quiri certissimum est, et ostendit Christus. Matt. 5.vers.24. Est in canon. Nihil. et canon. Non me- diocriter, quod est dictum Hieronym. De con- secrat. distinct.5 Certâ vero animi fiduciâ, quae in gratuitis Dei promissionibus acquiescat, orandum esse, docet et Christus ipse Ioan.4.v.23, et Iac.1.v.6. et Pau- lus Romanor. 1o.vers.14. quia vera precatio et Dei inuocatio est fidei effectus: quae cum animi haesitatione prorsus pugnat. Quae postea de certo quodam ieiunio, de lotio- ne manuum addita sunt a Patribus, de quibus hic Chrysost. Hom. 7 vbi quosdam sollicitos fuisse notat de huiusmodi rebus: non modo legales, sed etiam Ethnicorum caeremonias, aut potius super- stitiones reponunt nobis, et redolent, etsi ani- mus sobrius longe ardentius Deum precatur,  \pend
\section*{CAPVT  II. }
\marginpar{[ p.67 ]}\pstart quam satur et cibi plenus. Videndum iam est de loco, vbi orandum sit. Resp. vero vbique posse. Conuenit enim Paulo cum Christo Ioan.4. vers. 21. Quemadmodum e- nim Deus non est acceptor personarum, ita nec locorum. Nam Domini est terra, et plenitudo e- ius. Psal. 24. Videturque absurdum, vt Deus, cu- ius latissime per vniuersum orbem diffunditur maiestas et potentia, vbique agnosci inuocari et coli non possit, vt est Malach. 1.vers.11. Obstat tamen quod est scriptum Deuter.12.v. 5, et 2. Chronic.7.vers.12. templum Dei a Solo- mone constructum eum fuisse locum, quem Do- minus specialiter elegerat, vt ibi inuocaretur. I- taque versus illud templum conuersus etiam in media Babylone precabatur Daniel, vt ipse scri- bit cap.6.vers.1o. Respond. vero pro tempore ita constitutum a Deo fuisse, vt quanquam vbique inuocari posset nomen ipsius, sacrificari tamen solum in templo Hierosolymitano tunc tempo- ris fas esset, quia et caeremoniae tunc locum ha- bebant, et gratia Dei nondum diffusa erat in o- mnes Gentes. Denique hac paedagogia Domi- nus tunc volebat suos ad vnionem fidei et do- ctrinae exhortari, prouocare, et adducere, et futu- ram libertatem sub Christo maioremn ostendere, qualis etiam est. Ergo sublatum est illud locorumh vetus discri- men, vti nec iam cultus Dei in sacrificiis hostia- rum consistit. Quo fit vt vbique pie colatur et a- doretur Deus, et vt illa sacrificia, quae hodie exi- git, quae sunt gratiarum actiones, vbique terra- rum illi offerri possint. Atque eo praecipue per-  \pend
\section*{AD I. PAVL. AD TIM. }
\marginpar{[ p.68 ]}\pstart tinet hic locus Pauli, vt Iudaicum illud locorum discrimen sublatum esse per Euangelii praedica- tionem intelligamus, et quae quantaque sit iam nostra per Christum libertas, per quam et loco- rum et rituum seruitus nobis per Christum adem- pta est, cognoscamus. Neque tamen confusionem inducit vel inue- hit in Ecclesiam Paulus, in qua iubet ipse vt omnia τάκτως et ordine fiant, quasi iam nolit vllum esse communi totius Ecclesiae consilio et delectu con- stitutum locum, in quem Christiani ad Deum precandum certo tempore conueniant, seque a- dunent: sed vt sparsi, et, prout quemque feret ani- mi impetus, orent separati et disiecti. Hoc enim plane furiosum esset, et μανιῶδες: sed religionem propter loca vllam animis nostris inhaerere vetat quasi locus ipse sanctiorem gratioremque Deo nostram orationem efficiat. Quod superstitiosi homines, id est Papistae etiam non hodie putant. Vnde ex hoc loco votiuae illae ad terram san- ctam, et alia quaedam Martyrum loca peregri- nationes et precationes merito damnantur, quae ea de causa suscipiuntur a superstitiosis, quod nescio quid maioris sanctitatis in illis locis, et ipsi quam Dominus pedibus suis calcauit, terrae inesse iudi- cant. Quod falsissimum est. Si qua enim hodie est terra maledicta, est Iudaea: et huiusmodi co- gitatio plena est idololatriae et blasphemiae. Nec iuuantur huiusmodi superstitiosi homines exen- plo Naamani Syri, qui, vt est 2.Reg.5.terrae ipsius sanctae glebas quasdam et onera secum apporta- uit. Illo enim tempore caeremoniae locum adhuc habebant, et extra terram a se delectam sibi sa-  \pend
\section*{CAPVT  II. }
\marginpar{[ p.69 ]}\pstart crificari Dominus nolebat. Quanquam aliquid in eo Naamani infir mitati concessum esse mani- festo apparet, quod et sacrificat, et extra templum Dei. Mos autem iste visitationum monumento- rum Martyrum, et terrae sanctae, votiuarumque peregrinationum ex superstitiosa et nimia Mar- tyrum veneratione primum ortus latius postea serpsit, et ex errore errorem produxit. Quam parum honorifice de Hierusalem loquatur Chri stus ipse apparet Matth.23.vers.37. dum eam ho- micidam Prophetarum appellat: et post eum Paulus, dum eam seruae et ancillae confert. Galat. 4.et sequens aetas, quae Apostolorum tempori- bus vicina fuit, idem sentiebat, quae ne monumenta quidem martyrum saepe norat. Sepultura enim eorum contenti, vt de Stephano docemur.Act. 7, et 8. venerationem istam, quae cum idololatria coniuncta est, omittebant: imo vero detestaban- tur et damnabant. Primum Helena Constantini Magni mater mulier superstitioso et foemineo quodam zelo commota, terram, quam vocant, san- ctam inuisit, non ipsius quidem terrae gratia, sed Christi causa et fidei suae vt verisimile est, ma- gis confirmandae, quia fidem et soam et aliorum ipso aspectu rerum et monumentis passionis Chri- sti (quae a Iudaeis Ethnicis, et quibusdam haereti- cis negabatur) et quae se dulo illa conquisiuit, con- firmare voluit. Quanquam iam in eo peccatum est tamen, quod plus fidei de Domino cruci, cla- uuis, et sepulchro, quam ipsi Spiritui sancto, Apo- stoli scriptis et Euangelistis tribuisse videatur. Sed post orbem Christianum factum supplica- runt superstitiosi ad Martyrum monumenta, ad  \pend
\section*{AD I. PAVL. AD TIM. }
\marginpar{[ p.70 ]}\pstart quae concurrebant, tanquam augustiora quaedam loca, quod illic Martyrum, quorum fides Deo accepta fuerat, ossa iacerent. Sic Monica mater Augustini ex Affrica Mediolanum quotannis pergebat ad Geruasii et Protasii Martyrum sepulchra, quae tem- pore Ambrosii Episcopi eruta fuerant. Nec ta- men ossibus et cadaueri ipsorum Martyrum hic honos tribuebatur primum, sed ipsi defun- ctorum tantum fidei: peccatum est in eo tamen, quia extra verbum Dei id fiebat. Quod ipsum etiam iam quibusdam in Ecclesia bonis viris displicuit, et Ærius extitit Constantini Magni seculo, qui o- mnia illa et merito quidem damnauit, quanquam ipse, quod iam communi errore haec recepta erant, tanquam haereticus propterea habitus est. Facit e- nim communis et receptus iam inter omnes er- ror ius et confirmationem vt scribunt Iurisconsulti in L Barbarius D. de off. Praetor. Item Vigilan- tius tempore Hieronymi extitit, qui hoc totum superstitiosi cultus genus in reliquiis sanctorum colendis aperte idololatricum esse probauit. Er- ror tamen obtinuit tum negligentiâ pastorum, tum quia homines suis commentis potius, quam ex verbo coelesti Deum adorare cupiunt. Inde fu- riosa aedificatio templorum consequuta, quae in diuorum honorem facta sunt, post annum prae- sertim 6oo. a Christo passo sub lustiniano impe- ratore Constantinopol. homine iis superstitio- nibus insane addicto, vt apparet ex Procopii lib.  de aedificiis Iustinia. Et post eorum aedificationem, eamque etiam sumptuosam et magnificam reli- gio et veneratio iis addita est, post deinde ipse cultus Dei iis conclusus, aut alligatus, vt nolla  \pend
\section*{CAPVT  I. }
\marginpar{[ p.71 ]}\pstart sacra legitima et Deo grata oratio extra ea fieri posse ab hominibus etiam Christianis censeretur. Nec illa superstitio deffendi potest exemplo eorum, qui in sepulchrum Elizaei cadauer quod- dam festinantes et coacti proiecerant. vt est 2. Reg.13.vers.21. Nam illi cadauer Elizaei non co- luerunt: deinde id coacti fecerunt. Demum nulla inde superstitio nata dicitur ad sepulchrum Eli- zaei: sed tantum proptere fuit doctrina Prophe- tae, quae in animis hominum adhuc recens inhaere- bat, confirmatior facta, et Deus ipse maiori ho- nore ab illius seculi hominibus cultus. Templa igitur in Ecclesiis Dei esse vtile qui- dem est: sed tamen nec sumptuosa, nec supersti- tiosa, sed quae ad capiendum populum sint satis commode extructa, et quibus fanctitas nulla religioque ascribatur. Hoc enim est plane super- stitiosum et Iudaicum. Primum enim sine templis propriis fuit Ecclesia Christiana tempore Apo- stolorum, et post eos etiam temporibus Iustini Martyris: falsa sant enim quae in Higini et Sex- ti Roman. Episcoporum Epistolis de consecra- tione templorum in decretis et alibi extant. Sub Diocletiano primum apparet Christianos ha- buisse oratoria quaedam, et προς ευκτηρίους  ὄικους, vt vocat Eusebus  lib. 9. Histor.cap.1o. et Ruffi in Hi- stor. Ecclesiast. Templa igitur Christianorum imperatorum aetate et imperio demum aedifica- ri ceperunt, et oratoria dicebantur, quae si amplio- ra erant, Basilicae: nondum autem ναοί aut ἱερα. Hae Basilicae etiam saepe a Caesarum nominibus, a quibus fuerant extructae, vocabantur, et deno- minabantur, vti Basuica Constantini Euag.libus 2.  \pend
\section*{AD I. PAVL. AD TIM. }
\marginpar{[ p.72 ]}\pstart cap.8, et3 cap.2. postea Apostolorum et Marty- rum vocabulis nuncupari ceperant, in quorum erant honorem et memoriam constructa Sozom. libus  6.cap 8.quanquam ea Martyribus, non vt diis, fed vt hominibus, quorum memoriam colebant, consecrabant, vti ait Augustinus lib. 22. de Ciuit. Dei cap.1o. et saepe Martyrium ipsum fuit totius Basileae tantum pars quaedam cap.8. Demum ea- dem templa consecrari, dedicari, quibusdamque ritibus sancta credi et fieri ceperunt. Quod ante Constantinum Magnum factum et vsurpatum fuisse non videtur. Sed cum primus ipse in loco Caluariae templum Martyrum exaedificasset, po- stea ad maiorem, vt illi volebant, venerationem, sed potius ad adulationem et ad morem Pagano- rum, profanatum illud est potius: quam dedica- tum, et consecratum a quibusdam insulsis Epi- scopis, Sozomenus lib. 2.cap.26. et Eusebus lib. 4. de vita Constantini. Quo scelere etiam pol- lutum est templum illud augustum, quod Salua- tori nostro Hierosolymis idem Constantinus ex- truxit. Hunc tamen morem et exemplum secuti sunt postea alii Episcopi auide, non spectaro, si ex Dei verbo id fieret, neque quid ex eo mali con- sequeretur: sed placuit aliis nouitas et ille ritus et inauguratio in Constantini templo iam vsurpota et probara a quibusdam, vt ab Eusebio Caesarien- si Rethoricoteros laudata. Itaque Basilius ipse Episcopus vir alioqui doctus, ad Episcopi etad tem- pli Basilicae consecrationem solenniorem et venera- biliorem alios secum Episcopos conuocauit, vt tra- dit Sozomenus lib. 4.cap.13. Quod idem factita- tum fuisse in occidentalibus Ecclesiis, sed postea,  \pend
\section*{CAPVT  I. }
\marginpar{[ p.73 ]}\pstart apparet ex Epistolis Ambrosii. Demum certae caeremoniae sunt institutae, adhibitae, et verba, et precationes, quibus sanctitas ipsis lapidibus in- haereret. Ita repetita et reposita sunt in Dei Ec- clesia, quae profani homines in dedicandis idoliis suis obseruare consueuerant, de quibus agitur de Consecr. distinct.1. Caeterum illa iam dedicata templa appellarunt augusto nomine ναοὺς, ἐκκλησιας ἰερὰ vt apparet ex Epistolis Sidonii Apollinaris. Quorum etiam templorum inter Christianos, vt inter Iudaeos, tres partes constituerunt. Nempe Sanctum sanctosum, vbi est magnum altare Aʹ'γιον: et Sanctum siue χόρον, vbi est chorus Sacerdotum canentium: Ναὸν, vbi est plebs. Sed quaesitum est, versus quam mundi partem et plagam sit orandum. Respond. Siquidem ver- bum Dei, a quo solo pendere debemus, specte- mus perinde est, neque refert in quam mundi re- gionem conuersi precemur. Sin autem morem Papistarum, versus Orientem orant, et eôdem e- tiam templorum suorum capita conuertunt. Ve- tus quidem Ecclesia etiam sub Constantino libe- rior fuit, vt docet Socrates lib.  5. cap.22. Rationem Papistae afferunt, quod ad Orientem fuerit situs paradisus. Sed ex Ezechiel 8.vers.16.responde- mus damnatos esse a Deo, qui versus Orientem quadam religione ducti orarent, nec quia ibi pri- mum paradisum collocauerat Deus, voluit tem- plum soum eo vergere. Vrbs enim Hierusa- lem versus meridiem. Templum autem, quod in monte Sion erat, ad Septentrionem situm fuit, vti apparet ex Psal.48. et Ezechiel.4o. vers.2. De- pique videntur Papistae veteres Persas idolola-  \pend
\section*{AD I. PAVL. AD TIM. }
\marginpar{[ p.74 ]}\pstart tras imitari, qui praecise ad Orientem orandum esse docent, tantum abest, vt sit haec traditio A- postolica. Quod item idem Basil.sermo.2. de Ba- tis. cap.8. in loco non sacrato mysteria non pu- tat posse celebrari, fallitur et non videt vbi ver- bum Dei praedicatur, ibi locum esse sacrum 1. Timoth.4.vers.4. Tertio loco ex ipso versu quaeritur. Quo ge- stu sit orandum. De quo quia nihil est hoc loco diserte praescriptum, sed nec in toto Dei verbo, intelligitur habere ea res liberas obseruationes, modo ne quod offendiculum aliis nimia affecta- tione vel dissensione praebeamus. Christus ipse et stans et flexis genibus orauit, publicanus stans in templo precatus est, Luc.18.vers. 13.22 vers.41. In Actis videmus positis genibus Christianos veteres saepe orasse Actor.7.9. et 2o.Stantes orant, qui spe promissionum Dei erigantur. Flexis au- tem genibus id faciunt qui, propter peccatorum suorum sensum deiecti coram Deo humiliantur et prosternuntur. Quidam etiam prostrati ora- runt, vt Elias 1.Reg.18.Nec in eo synodorum de- creta magni esse momenti ad stabiliendam pie- tatem existimemus, sed morem potius regionis, in qua sumus, sequamur, modo ne sit in eo more et ritu aliqua idololatria apparens. De manibus etiam quaesitum est, quo earum situ, et gestu sit orandum. Resp. Alii iunctis, alii supinis, alii sub- latis orarunt, vt docet Clemens Alexandr.libus 7. Stroma. Itaque totum hoc genus obseruationum et caeremoniarum liberum est. Et quod ait hoc loco Paulus, Attollentes manus, metonymico dictum est. Nam signum pro re signata positum  \pend
\section*{CAPVT  II. }
\marginpar{[ p.75 ]}\pstart est, vti Isai.1.vers.15. nimirum pro ipso cordis af- fectu, qui ad Deum erigi debet. Hic enim gestus fauorem animi designat, quo preces ad Deum nostras concipi debere significat Apostolus: non quod homines Christianos huic caeremoniae in precandon velit esse alligatos et astrictos. Denique quaeri etiam potest de tempore, quo preces concipi et fieri a Christianis debent. Res. Quod quidem ad priuatas, non tantum quoti- die semel: sed etiam et bis et saepius in die fieri o- portere, nimirum mane et vesperi, cum surgimus aut cubitum discedimus, vt omnes actiones no- strae a Deo incipiant et finiant. Id quod exemplo Dauidis ita faciendum esse monemur Psal.55 vers. 18 et suo seculo in omnibus familiis factitatum scribit diserte hoc loco Chrysostomus. Quod autem ad precationes publicas nulla fuit ante Christianos imperatores de eo lex in Ecclesia constituta, cum non auderent Christiani homi- nes propter persecutiones libere conuenire. Post fancitam vero Ecclesiae libertatem toties ora- runt, quoties ad audiendum Dei verbum, vel ad Sacramentorum participationem colligebatur Ecclesia, quia sine precibus nunquam dimitteba- tur coetus piorum, vti docent, et lustinus Martyr et Tertullian.in Apologet. et Socrates lib. 5.cap. I6. Certis tamen horis non erant precationes pri- mum institutae, alligatae, et vinctae: sed creuit po- stea superstitio, et deuincta est certis horis oran- di consuetudo et necessitas, etiam sine publico Ecclesiae coetu, et sine predicatione verbi Dei. Id quod docet Hieronymus ad Eustochium, adeo quidem, vt etiam sioe conuocatione Ecclesiae o-  \pend
\section*{AD I. PAVL. AD TIM. }
\marginpar{[ p.76 ]}\pstart \phantomsection
\addcontentsline{toc}{subsection}{\textit{9 Itidem et mulieres amictu honesto, cum verecundia et modestia ornare sese, non, cincinnis, vel auro, vel margaritis, vel pre- tioso vestitu.}}
\subsection*{\textit{9 Itidem et mulieres amictu honesto, cum verecundia et modestia ornare sese, non, cincinnis, vel auro, vel margaritis, vel pre- tioso vestitu.}}mnibus pene diei horis preces publicae fieri de- cernerentur, diluculo, deinde tertia diei, id est, a luce solis exorta supra horizontem hora, post sexta, praeterea nona. Denique sub vesperti- num diei tempus, Vnde illae horae inter Papistas dictae sunt Canonicae nempe 1.3.6.9. de quibus etiam meminit Clemens Stromat.libus 7. Sedu- lius, Cassianus, Beda et alii. Atque haec tota res magnam secum fuperstitionem primum traxit, et verum finem orationis, atque vim extinxit: at- que in opinionem meriti, et satisfactionis conuer- tit. Denique Euchitarum errorem in Ecclesiam induxit, qui nihil, nisi precari, id est, demurmura- re certa verba solebant, et in eo proram et puppim salutis collocabant, vti diximus. Haec nos omnia breuiter hic complecti voluimus, vt tanquam lo- cus communis haberetur. Sequitur iam vt de de- coro, quod non tantum in precibus, sed in congres- sionibus publicis et Ecclesiasticis seruandum sit videamus, quod sequentibus versiculis explica- tur. 9 Itidem et mulieres amictu honesto, cum verecundia et modestia ornare sese, non, cincinnis, vel auro, vel margaritis, vel pre- tioso vestitu. Μεριςμὸς est, supra enim de viris egit, nunc autem de foeminis Christianis, quales eas in Ecclesia Dei esse et conuenire oporteat, vt ad rite orandum sint comparatae. Quo fit, vt hic de Christianarum mulierum officio et decoro duplici ratione et re- spectu Paulus agat. Nimirum quatenus sunt spe- ctandae tum In sese, tum in totius Ecclesiae coetu.  \pend
\section*{CAPVT  II. }
\marginpar{[ p.77 ]}\pstart Ac in sese debent esse Castae, et Modestae. Coetus autem Ecclesiastici ratione, Silentes, et Discentes. Primum igitur ait. Itidem, Nam ne se a con- sortio et com municatione precum Ecclesiae ex- cludi propter sexus infirmitatem, et maiora viro- rum priuilegia putent mulieres Christianae, do- cet eundem illis aditum ad Deum patere, et ean- dem orandi fiduciam dari, quam supra viris esse concessam ostendit. Deus enim suas promissio- nes vtrique sexui communes proposuit: et tam dicitur Sara parens et mater fidelium mulierum quam Abraham fidelium virorum 1.Pet.3. Communis vtrisque Baptismus, Coena Domi- ni, Praedicatio Euangelii, fides, et gloria aeterna, quia vtrisque communis est imago Dei. Ergo mu- lieres,vti et viros, orare, et Deum precari vult: sed tacitas, non autem in coetu eas preces fundere et concipere. Vult autem Paulus eas esse in coetu Domini amictas et vestitas Honeste, Verecunde et Modeste. Atque huic muliebii cultui et orna- tui opponit, Cincinnos et calamistros, Aurum et argentum, margaritas in vestibus, Pretiosum siue sumptuosum vestitum. Denique opera bona ab illis requirit et efflagitat, quae opponuntur omni impudicitiae, vanitati, ambitioni, fastui muliebri. Decent aurem maxime opera bona foeminas Chri- stianas. Ac primum similis est locus.1.Pet.vers.3.Tit.2 vers.2.Isai.3. Dissimile videri potest exemplum et factum Estherae: Esth.cap.5 et ludithae, quae se comit et ornat aliquid actura egregium, Respond. Non agi illic de precibus, nec de coetu Ecelesiae adeun-  \pend
\section*{AD I. PAVL. AD TIM. }
\marginpar{[ p.78 ]}\pstart do:sed de marito rege, eóque infideli demulcen do: item de castris infidelium ingrediendis. Nec hoc ipsum exemplum satis est tutum imitari. Ob- stare etiam videtur cap. 24.Genes.vers.47. vbi annuli et ar millae dantur Rebeccae sponsae Isaaci Resp. Non ad lasciuum ornatum, sed vt esset pi- gnus futuri coniugii, ista Rebeccae fuisse et missa ab Abrahamo et donata. Certum autem est, etsi de mulieribus Christia- nis agatur hoc loco: haec tamen praecepta etiam pertinere ad viros Christianos, quibus eo tur- pius est comi lasciue, sumptuose et dissolute or- nari quod sunt viri et mares. Nam etiam Ethnici homines sic docent. Sint procul a nobis iuuenes vt foemina compti Fine coli modico forma virilis amat. et M. T.libus 1. De off.sic. Cum autem pulchritudinis duo genera sint, quorum in altero venustas sit: in altero dignitas: venustatem muliebrem dicere debemus: dignitatem virilem. Ergo et a for ma remoueatur omnis viro non dignus ornatus: et huic simile vitium in gestu motuque caueatur. Et paulo post. Adhibenda est praeterea mun- ditia non odiosa, neque exquisita nimis: tantum quae fogiat agrestem et inhumanam negligentiam. Haec ille. Patres etiam scriptoresque Ecclesiastici veluti Tertullian. in lib.  de Habitu mulierum et Cyprian. in lib.  de Cultu virginum nobiscum fa- ciunt, qui haec praecepta ad viros perspicue refe- runt. Idem in concilio Gangrensi cap.21. et est in canon Parsimoniam distinct. 41. ne res noua aut nullius momenti esse censeatur. Idem etiam sta- tutum est aliis synodis veluti Laodicena.  \pend
\section*{CAPVT . II. }
\marginpar{[ p.79 ]}\pstart Praeterea haec ipsa praecepta non ad solas mulie- res, quae virginitatem vouent inter Papistas, et Non- nae dicuntur, pertinent: sed ad omnes et coniu- gatas et celibes, et viduas, quemadmodum prae- clare hoc loco docet Chrysosto. et Gregor. Na- zianzenus in lib.  περὶ γυυαικῶν κεκαλωπισμένων. Nam tempore Pauli nondum erant monasteria, nec i- stae virgines Vestales et Nonnae visae et natae. Hae enim cultus diuini corruptelae sub Vitaliano Pon- tifice Romano maxime receptae et probatae vi- dentur, sub quo tria haec tanquam venena Chri- stianae pietatis praesentissima nata sunt, et consti- tuta, Monachorum coenobia erecta, Reges in Monachos detonsi, Scortatio quotidiana in sta- tum sanctum canonizata. Quaesitum vero est de primariis mulieribus veluti reginis, principissis, duchissis et huiusmodi aliis primariis foeminis, vtrum ad eas quoque haec praecepta sint extendenda. Respond. Etsi di- scrimen personarum etiam in Dei Ecclesia ha- bendum est, omnia tamen etiam in illis, si modu sunt Christianae, ad modestiam, et pudicitiam composita esse debent, quanquam illis multa con- cedi et possunt et debent pro ratione dignitatis, quae aliis priuatis et priuatorum mulieribus non sunt permittenda. Nam et Solomonis vxoris cul- tus sumptuosus ille quidem et magnificus des- cribitur laudaturque Psal. 45. Denique haec eadem praecepta locum habent in mulieribus, etiam cum domi sunt, non tantum cum prodeunt in publicum, et conueniunt in coe- tu Ecclesiae. Perstringit autem Paulus breuiter o- mnem eum ornatum ascititium, qui est vel In i-  \pend
\section*{AD I. PAVL. AD TIM. }
\marginpar{[ p.80 ]}\pstart pso corpore nostro velut cincinni, vel Circa cor- pus, qualis est vestitus: cultus noster nimius quem damnat, siue ille sit nimius ratione Materie, velu- ti quia sumptuosior, vel Additae exquisitaeque lautit iae, venustatis, aut operae manus. Ad primum pertinet ἰματισμὸς πολυτελής. Palliola enim illa fe- re esle solebant sumptuosissima, magnique pre- tii, vt ex disputatione Hieronym. aduers. Iouinia. intelligitur, purpurea scilicet. Ad secundum mem- brum, et ad exquisitas lautitias vestitus pertinent ornamenta margaritarum, auri, et phrygionicae operae quae adduntur. Irem torques, armillae, annu- li, etc. huiusmodi de quibus Tertullianus lib.  de Habitu mulierum: et Cyprianus quoque. Et pul- chra est veraque illa Plauti sententia in Mostell. scen.3.Act.1. Postea nequaquam exornata est bene, si mo- rata est male. Pulchrum ornatum turpes mores peius coeno collinunt. Sed tamen, nequa hic superstitio nascatur, auri vsum prorsus non damnat Paulus, quemadmodum nec margaritarum: vti nec Petrus Palliorum. sed tantum nimiam in iis lasciuiam, arrogantiam, sumptum, fastum, immodostiaeque plenam cultus muliebris rationem et Christianis indecoram. Quanquam enim longe potius animi in muliere dissoluti aut superbi ratio habenda est, quam ex- terni ornatus: tamen nec externus iste cultus, qui impudicitiam aut proteruiam redolet, maleque modestae mulieri et probae conuenit, tolerandus est. Ac de ea ipsa re sunt leges sumptuariae a Ma- gistratibus Christianis ferendae. Quae luxu etiam  \pend
\section*{CAPVT  II. }
\marginpar{[ p.81 ]}\pstart \phantomsection
\addcontentsline{toc}{subsection}{\textit{10 Sed (quod decet mulieres pietatem spondentes) operibus bonis.}}
\subsection*{\textit{10 Sed (quod decet mulieres pietatem spondentes) operibus bonis.}}vetus Ecclesia in Synodis repressit. Oportet e- nim eo modo mulieres Christianas vestiri, quo interna earum pudicitia apparere possit, etiam infidelibus hominibus. Nam vt, et recte quidem, suo seculo est conquestus Chrysostom. quaedam mulieres ita exornatae ad templum et coetum Do- mini incedunt, et ad sacram Domini Coenam su- mendam accedunt, vt illic potius saltaturae, quam Deum precaturae videantur. Nec vero iis prodest haec exceptio, Sic culta viro et marito meo pla- ceo. Nec enim illae, vt lenoni marito placeant, debent studere: et quae sic exornata, aut fuco pi- gmentata incedit, certe naturam formamque, quam a Deo accepit, odit. Sed neque haec altera earum ratio est audienda, Sunt haec omnia ex τῶν ἀδιαφορων genere. Neque enim haec cum offendiculum cuiquam praebent, sunt adiaphora: sed damnata sunt, iisque est abstinendum. In summa cultum mu- lierum non damnat Paulus, modo sit Honestus, et Modestiae verecundiaeque Christianarum mu- lierum conueniens et consentaneus. Et recte etiam docet Plutarchus lib. 6.Sympos.quaest.7. abster- sionem sordium et spurcitiei a cultu inhonesto et sumptuoso differre. 10 Sed (quod decet mulieres pietatem spondentes) operibus bonis. A'ντίθεσις est. Fastui enim et superbiae, atque lasciuae illi mulierum pompae superiori versiculo descriptae verum earum ornatum opponit, nimi- rum bona opera, id est, pro externo ornatu, vt lo- quitur Petrus, internum requirit qui in eo posi-  \pend
\section*{AD I. PAVL. AD TIM. }
\marginpar{[ p.82 ]}\pstart \phantomsection
\addcontentsline{toc}{subsection}{\textit{11 Mulier cum silentio discito cum o- mni subiectione.}}
\subsection*{\textit{11 Mulier cum silentio discito cum o- mni subiectione.}}tus est, vt sit earum animus placidus et quietus, nullisque vitiis expugnari possit. Rationem au- tem addit ab officio ductam. Nempe, quod ho- nesta et sancta opera vere Christianas mulieres et religionem profitentes deceant: non autem il- le externus vestitus, quo pudor omnis ab illis proiectus et conculcatus videtur exulare. 11 Mulier cum silentio discito cum o- mni subiectione. Alterum et secundum de mulieribus Christia- nis praeceptum, quod ad Ecclesiae publicique coetus politiam et decorum pertinet, Ne in eo lo- quantur, sed sileant, et discant. Similis huic locus est. 1.Corinth.14.vers.34. Id quod etiam consilio Chartaginensi cap.95. constitutum est, referturque canon. mulier distinct. 23. quantumuis docta et pia sit ista mulier. Dissimilia tamen exempla videntur afferri pos- se. De Debora Iudic. cap. 4. et Olda mulieribus prophetissis 2.Reg. 22. versu 14. quae in Ecclesia docuerunt. Sed responsio parata est, illa nimirum exempla plane esse extraordinaria, itaque haec Pauli praecepta minime immutare. Affertur etiam illud quod Eusebus lib.  1.devita Constantini scri- psit, mulieres eligere e suo sexu solitas fuisse ali- quas, quae eas docerent ne in publicis congressioni- bus conuenirent cum viris, quod Licinii imperato- ris Rom. edicto, et lege in Syria simul cum viris congregari et conuenire vetabantur. Sed responde- mus, Illud praeter morem Ecclesiae, et praeter Dei praecepta iussum esse a Licinio homine profano,  \pend
\section*{CAPVT  II. }
\marginpar{[ p.83 ]}\pstart \phantomsection
\addcontentsline{toc}{subsection}{\textit{12 Mulieri enim docere non permitto,}}
\subsection*{\textit{12 Mulieri enim docere non permitto,}}vt mulieres ipsae in coetu mulierum docerent, non autem cum viris ad preces et conciones conuenirent. Denique quod de Pepuzianorum haereticorum secta, et sententia dici posse videtur habet fa- cilem responsionem. Voluerunt illi quidem, sed haeretici, mulieres esse posse? et presbyte- ridas in Ecclesia, et docere, et Sacramenta quo- que administrare (id quod hodie Papistae ex Mar- cionitarum dogmate obstetricibus in casu neces- sitatis concedunt, vt nimirum paruulos baptizarent) Verum propterea ab omnibus orthodoxis sunt Pepuziani damnati, atque etiam Synodo Char- taginensi a. vt diximus, et meritissimo. Hoc autem praeceptum Pauli locum habet, dum sunt in coetu publico mulieres: non autem cum sunt domi suae, quia familiam suam instituere pos- funt priuatim, maxime si desit paterfamiliâs, et sint viduae. Nam tunc precari Deum, et docere suos liberos Dei metum, et possunt et debent. Requirit autem Paulus et silentium, ne vices loquendi sibi dandas esse putent mulieres, quan- quam fortasse, modestiae causa primae, in Eccle- sia loqui nollent. Deinde ab iisdem subiectionem postulat, ne putent aequum, vt propter maritorum suorum dignitatem et opes:vel propter etiam fa- miliam suam ipsae saltem secundum Pastores et Ministros verbi Dei aliquod munus Ecclesiasti- cum gerant. Hoc enim illis prorsus interdictum est a Spiritu sancto. Huius autem prohibitionis variae sunt rationes, quas affert Paulus sequentibus  versiculis. 12 Mulieri enim docere non permitto,  \pend
\section*{AD I. PAVL. AD TIM. }
\marginpar{[ p.84 ]}\pstart \phantomsection
\addcontentsline{toc}{subsection}{\textit{neque authoritatem vsurpare in virum, sed esse in silentio.}}
\subsection*{\textit{neque authoritatem vsurpare in virum, sed esse in silentio.}}neque authoritatem vsurpare in virum, sed esse in silentio. Prima ratio quae explicatur hoc loco, a diffe- rentia, quae a Deo ipso constituta est inter vtrun- que sexum virilem nempe et muliebrem, ducta est. Secunda vero a genere. Deus neque mulieres praeesse vult, neque docere, neque in viros autho- ritatem vsurpare: At qui loquitur in Eccesia siue praeest ea et docet, in viros sumit quoddam imperium et authoritatem. Ergo mulieres loqui in Ecclesia neque possunt, neque debent. Nec obstat quod eas in Diaconatum alleget Paulus inf. cap. 5.vers.9. Nudum enim illae manuum mi- nisterium praebent, Diaconis autem ipsis parent. Sed ex hoc loco de eo quoque inter quosdam di- sputatum est, Vtrum honestum sit mulieres re- gnare, id est, viris imperare, et summum impe- rium et ius in viros et mares obtinere. Id quod in Hispania, Anglia, Scotia variisque aliis regioni- bus locum habet. Cuius etiam rei extant exen- pla in Semiramide Assyriorum regina, Candace Ethiopum Actor.8. vers.27. Cleopatra Ægy- ptiorum sub Augusto, et Zenobia fortissima mu- liere sub Adriano imperatore (cuius Zenobiae im- perio etiam multae Ecclesiae Christianae parue- runt. In populo autem Dei nihil quicquam tale habemus, siue Iudaicum siue Israëliticum regnum spectetur. Nam quod de Athalia potest afferri 2. Reg. 11. facile diluitur: fuit enim vsurpatio et in- uasio iniusta regni ea Athaliae dominatio: non autem regnum et potestas legitima cui populus lubens assentiretur vt postea apparuit. Quare  \pend
\section*{CAPVT  II. }
\marginpar{[ p.85. ]}\pstart uste quoque illa a summo sacrificatore loiada de regni solio deturbata et deiecta, est, atque e- tiam propter vsurpatum regnum necata. Quod de Amazonum regno commemoratur, primum vt fabulosum non fuerit (quemadmodum omni- no fabulosum non putant Arrianus, et Q Curtius quique res gestas Alexandri Magni scripserunt) tamen praeter rerum naturam constitutum est il- lud regnum, et in virilis sexus apertam contume- liam. Quid enim magis portentosum et monstro- sum et a muliebris ingenii mansuetudine alienum, quam videre armatarum mulierum exercitum con- tra viros concurrentem: ac latissimam quandam terrae plagam cernere, quae omnino viris et mari- bus careret. Itaque videntur illae contra ipsam foeminei sexus vel potius humanam naturam pu- gnasse, et auxilium, quod a Deo optimum et praestan- tissimum paratum est mulieribus, nempe mari- tos reiecisse. Vnde prorsus breuique tempore to- ta illa Amazonum gens deleta est. Sed ad propositam quaestionem reuertamur. Quanquam vero Isai.3. vers.II. pro magno Dei maledictionis signo po- situm legimus, quod pueri et mulieres imperium in aliqua gente obtineant: tamen illud ipsum non. est perpetuum. Saepe enim et pueri reges, velut Solomon et Iosias, sanctissime et felicissime re- gnarunt, fuitque eorum imperium omni bono- rum genere a Deo cumulatum. Idem de mulieri- bus quibusdam, et earum imperio dici potest, quibus Dominus mirum in modum benedixit, vt ex variis historiis apparet. Certe de illustrissima regina Angliae Elizabeta, quae nunc felicissime regnat, dici potest, nihil terrarum orbem vidisse.  \pend
\section*{AD I. PAVL. AD TIM. }
\marginpar{[ p.86 ]}\pstart vnquam illius regno felicius et optabilius. Sapien- tertamen sibi consuluisse videntur ii populi, qui legibus suis et publico iure cauerunt, ne inter se, et in sese foeminae dominarentur, summumque ius et imperium haberent, si genus ipsum mulie- bre cum virili conferamus, quia ad multa munia, quae regni administratio requirit, sunt illae minus propter sexus sui naturam et imbecillitatem a ptae et inhabiles quale est praeesse exercitui, ius dicere in publico sedentes. Quae certe res dede- cent prorsus muliebrem verecundiam. Vnde iu- re ciuili Roman. L. mulieres D. de Regul. Iuris mulieres a virilibus muneribus et officiis admi- nistrandis recte arcentur: qualia sunt etiam quae in publico exerceri debent. Et illud est August. libus 1. de Nupt. cap.9. Nec dubitari potest viros potius foeminis, quam foeminas viris principari. Vnde non potest mulier esse Roman. imperatrix et regina, et cum Athalia in ludaea et Irene mater Constantini tertii imperium Constantinopoli. gerere voluit, vtraque omnia subuertit, Idolorum cultum in Dei Ecclesiam inuexit, et haec Sarace- nis Roma. imperium lacerandum obiecit. Vnde et Carolus Magnus fuit tunc temporis in Roma- num imperium asciscendus in Occidente: et Ni- cephorus in Oriente. Vbi autem inferiores iuris- dictiones quales Ducum, Comitum, Baronum, Castellanorum, sunt patrimoniales, vt in Gallia, illae a mulieribus haberi et possideri possunt, meo quidem iudicio, quia non sunt summa imperia i- stae dignitates, et officia: sed tamen ab iisdem mu- lieribus nec possunt, nec debent hae iurisdictio- nes exerceri: sed a viris per eas delegatis. Quan-  \pend
\section*{CAPVT  II. }
\marginpar{[ p.87 ]}\pstart \phantomsection
\addcontentsline{toc}{subsection}{\textit{13 Adam enim prior formatus est, de- inde Eua.}}
\subsection*{\textit{13 Adam enim prior formatus est, de- inde Eua.}}quam hoc omnino perperam et pessime receptum est vsquam gentium, vt sit iurisdictio vlla pars patrimonii, et reditus nostri ac dominii, sed tam late patet et grassatur in omnia auaritia, vt etiam res sacratissimas, qualis est Magistratus, fecerit patrimoniales et in censu numeret : non autem virtuti et doctrinae eorum, qui sunt capaces eorum munerum, tribuat. Vide quae scribuntur can. Mu- lierem 33. quaest.5. de hoc argumento. 13 Adam enim prior formatus est, de- inde Eua. A’ιτιολoyίa est saperioris sententiae ab ordine naturali ducta, quem Dominus Deus inter vtrum- que sexum constituit rerum initio, et statim ab i- psa vtriusque creatione. Is autem fuit, vt pareret mulier. Vir autem caput et superior esset. Id quod ex ipso ordine creationis apparet. Adam enim, qui est mas et vir, prior creatus est. Eua autem, quae est foemina, posterior. Sed tamen eodem die vt docetur tum Genes. cap.I.vers.27. tum 2.vers. 18. Non tam autem ex natiuitatis ordine Paulus argumentatur et probat virum superiorem esse: quam ex fine creationis mulieris. Nam creata est foe mina, vt esset viro in adiutorium, et ad eum, tanquam suum propriumque finem respiceret. Quae autem ad finem aliquem destinantur illique subseruiunt, sunt eo fine minora, inferiora, ac illi subiecta. Id quod verum esse ratio omnis et natu- ralis et Philosophica docet. Nam alioquin a solo natiuitatis et productionis tempore, et ordine ratio ducta nihil efficit et concludit. Pisces enim  \pend
\section*{AD I. PAVL. AD TIM. }
\marginpar{[ p.88 ]}\pstart \phantomsection
\addcontentsline{toc}{subsection}{\textit{14 Et Adam non fuit seductus: sed mu- lier seducta, causa transgressionis fuit.}}
\subsection*{\textit{14 Et Adam non fuit seductus: sed mu- lier seducta, causa transgressionis fuit.}}herbae, sydera et reliquae omnes creaturae, quae homini subiiciuntur ex Dei praecepto Psal. 8.prae- ferendae tamen homini essent contra mentem et Christi, et Pauli. Omnia enim propter hominem facta sunt, etiam Sabbatum ipsum: non autem homo propter Sabbatum. 14 Et Adam non fuit seductus: sed mu- lier seducta, causa transgressionis fuit. Alterum argumentum a causa tum efficiente, tum incurrente sumptum. Efficiens igitur causa est Dei praeceptum et voluntas, quae eam mulieri poenam inflixit. Incurrens autem, transgressio mulieris, propter quam talem poenam merita est et huic seruituti et subiectioni addicta. Æquum enim fuit, vt quae seipsam regere non potuit, alte- rius consilio et imperio regenda subiiceretur a Deo. Est autem subiecta viri imperio et po- testati, vt est Genes. 3. Quanquam autem illud Mosis proprie pertinet ad virum et mulierem coniuges inter se: non autem vt quilibet viri in quaslibet mulieres imperium sibi arripiant:tamem ex eo intelligitur, in quem gradum viri supra mu- lieres euecti sint a Deo. Propter transgressionem autem mulieris facta iam est durior et seruilior mulieris conditio, quae ad virum tantum suum tanquam ad suum finem tunc spectabat. Sed nunc aerumna et iugum additum est huic seruituti, quae prius erat omnino et liberalis et voluntaria. Nunc enim dura est, etiam cum est ad virum et maritum proprium. Ex quo ipso apparet Deum et confir- masse priorem subiectionem, et eam propter  \pend
\section*{CAPVT  II. }
\marginpar{[ p.89 ]}\pstart \phantomsection
\addcontentsline{toc}{subsection}{\textit{15 Seruabitur tamen liberos gignendo, si manserit in fide, ac charitate, et sanctifica- tione cum modestia.}}
\subsection*{\textit{15 Seruabitur tamen liberos gignendo, si manserit in fide, ac charitate, et sanctifica- tione cum modestia.}}peccatum effecisse duriorem et austeriorem. Val- de autem notandum est, quod ex verbo sedacta colligit Bernard. sermo. de Duplici Baptis. his verbis, Serpens ô Eua decepit te, non impulit, aut coegit, mulier tibi, ô Adam de ligno dedit: sed offerendo, non vtique violentiam inferendo. Neque enim potestate illius, sed tua voluntate factum est, vt eius voci plus obedieris, quam di- uinae. 15 Seruabitur tamen liberos gignendo, si manserit in fide, ac charitate, et sanctifica- tione cum modestia. Hypophora est quae magnam consolationem tamen continet, ne se prorsus spe salutis orbatas esse mulieres existiment. Id quod illis potuit in animum incidere, vel nobis etiamipsis, dum eas esse nobis tanti causas mali audimus. Per eas enim factum est, vt vniuersum genus humanum siue virile, siue etiam muliebre periret. Ergo prorsus salutis exortes videbantur. Sed contra Paulus. Hanc ipsam poenam transgressionis, et hanc in coniugio subiectionem mulierum docet in ipsarum commodum et salutem cedere mira Dei sapientia nimirum, qui e tenebris lucem e- duxit, vt est 2. Chorinth. 4. vers. 6. Sic conuertit hoc suum in eas iudicium Dominus in earum i- psarum consolationem et salutis viam, dum hanc sub- iectionem erga maritos vocationem earum pro- priam esse statuit. In quo illud primum est obser- uandum vocem διὰ non denotare hoc loco cau- sam efficientem: sed mediam tantum, per quam, tanquam per iter a Deo demonstratum, est illis  \pend
\section*{AD I. PAVL. AD TIM. }
\marginpar{[ p.90 ]}\pstart pergendum et progrediendum. In quo peccant Pa- pistae et Pelagiani, omnesque Operistae, et quidem grauiter, qui salutis nostrae causas in nostris ope- ribus quaerunt et constituunt. Quid enim de vi- duis, quae nullos liberos vnquam susceperunt: de virginibus, et iis etiam quae cum sint coniuges, nullos liberos tamen pepererunt, futurum esset (vt recte hoc loco Chrysostom. animaduertit) si salus mulierum ἑκ τεκνογονίας tota pendet? Certe saluae non fierent. Sed et Thomas Aquinas hic delirat, dum laudes virginitatis et celibatus cum hoc Pauli dicto conciliare, et placare nititur, quod cum illis aperte et perpiscue pugnat. Ergo hic vocatio et munus mulierum describitur etiam in eo ipso, in quo poenam sui peccati sentiunt. Est autem, vt liberos non tantum suscipiant et gig- nant: sed vt educent, curent, et onera illa primae liberorum nutricationi, scerte quidem molestis- sima, lubentes volentesque deuorent et suscipiant. Tεκνογονίας enim hoc loco non tantum parturitio- nem ipsam mulierum significat: sed quicquid illi adiunctum est, et imminet iustae matrum curae, vt liberi editi educantur. Vnde merito matres puer- perae damnantur, quae filios, si possunt, suos non alunt, de quo Plutarchus in lib.  de Liberorum E- ducatio. et dictum Gregorii in canon. Ad eius distinct.5. Mille quidem sunt taedia, quae contra Io- uinia. enumerat Hieronym.tam minutim, vt ipse puerpera fuisse aliquando videatur, qui tamen perpetuo vixit celebs. Mille etiam dolores et mo- lestiae non tantum leuiter degustandae: sed diutissi- me deuorandae, tum in partu, tum post partum. Quae tamen omnia si libenti et grato animo de-  \pend
\section*{CAPVT  I. }
\marginpar{[ p.91 ]}\pstart \phantomsection
\addcontentsline{toc}{subsection}{\textit{F IDVS est hic sermo, Si quis Epi- scopatum apperit, praeclarum opus desiderat.}}
\subsection*{\textit{F IDVS est hic sermo, Si quis Epi- scopatum apperit, praeclarum opus desiderat.}}uorat, atque suscipit mater, Deo grata est, et in sua vocatione incedit: ad trimatum autem vsque lex Ciuilis iubet vt mater ipsa alat liberos suos, eosque praecipue curet. Sed addit Paulus alias quoque virtutes, quas esse necessarias ostendit, vt mulier Deo placeat, quarum aliae communes illis sunt etiam cum viris, nempe Fides, et Charitas: aliae illarum propriae, San- ctificatio, et Modestia. Quae enim virtus nos Deo commendat est fides. Illa enim nos et deducit ad Christum, et cum eo coniungit. Illa charitatem quoque gignit, extra quam nihil a matribus ipsis fieri Deo gratum potest, quia si inuitae, repugnan- tes, et coactae potius officium faciunt, quam ex charitate, displicent Deo. Sanctificatio, quam requirit a mulieribus, non est tantum generalis omnium vitae actionum ad Dei voluntatem conformatio 1.Thess. 4. vers. 4. sed maxime sanctimonia quaedam corporis, qua se vxores et mulieres ab omni turpitudine, im- munditia, et lasciuia puras conseruant, quae impu- dicitie opponitur: et alio nomine dicitur castitas. Modestia, luxui et sumptuoso illi cultui, de quo superius dixit, opponitur. CAP. III.
\textbf{F}IDVS est hic sermo, Si quis Epi- scopatum apperit, praeclarum opus desiderat. Transitio est. Post quam enim de pastorum of-  \pend
\section*{AD I. PAVL. AD TIM. }
\section{CAPVT . IIII. }
\marginpar{[ p.180 ]}\pstart \phantomsection
\addcontentsline{toc}{subsection}{\textit{S PIRITVS autem diserte dicit fore vt posterioribus temporibus desci- scant quidam a fide, spiritibus dece- ptoribus attendentes, ac doctrinis daemonio- rum.}}
\subsection*{\textit{S PIRITVS autem diserte dicit fore vt posterioribus temporibus desci- scant quidam a fide, spiritibus dece- ptoribus attendentes, ac doctrinis daemonio- rum.}}explicatur fere in articulo, Sedet ad dexteram Dei Patris quae cum sit ex Catechismis notissima, et ab aliis copiosissime explicata, maxime pro- pter infoelices istas de Coena Domini controuer- sias, et de vbiquitate corporis Christi, idcirco nos eam hic omittimus. Studemus enim breuitati. CAP. IIII. 
\textbf{S}PIRITVS autem diserte dicit fore vt posterioribus temporibus desci- scant quidam a fide, spiritibus dece- ptoribus attendentes, ac doctrinis daemonio- rum. Transitio est, per quam a contrario superiorem doctrinam non tantum illustrat: sed etiam con- firmat et commendat, ne verbi Dei praecones se- gnes sint in ea tuenda et conseruanda, cum ali- quando futurum esse audiant, vt misere corrum- patur et labefactetur, pessimaeque superstitiones pro ea in Ecclesiam inducantur. Similis locus et admonitio fit a Paulo Actor.2o. vers29. a Ioanne 1. Epist.2.vers.19. a Petro 2. Epist.3.vers.3. et a Iu- da Thaddaeo Apostolo, vt apparet ex ipsius Epist. vers.4. Neque eo magis terreri debemus: quod fit a pluribus ista admonitio: sed potius eo dili- gentius cauere nobis, et prospicere ne decipia- mur: minusque offendamur oportet, cum huius- modi haereses obrepunt, quas fore iampridem spiritus Dei praedixerat. Duo vero quaedam praecipue sunt hic obser-  \pend
\section*{CAPVT . IIII. }
\marginpar{[ p.181 ]}\pstart uanda, 1 Quanti momenti sit haec praedictio. 2 Quid contineat. Magni enim esse momenti ex eo apparet, quod ρντως, id est diserte et praecise a Dei spiritu praedicta fuit, et reuelata. Etsi enim omnium eorum, quae Paulus annuntiauit, Spiritus sanctus est author: hanc tamen tanquam specialem quan- dam prophetiam et admonitionem vtilem futu- ram Ecclesiae manifestauit, et obseruari voluit Spi- ritus sanctus. Quaeri vero potest, vbi hoc praedi- xerit Spiritus. Respond. Alium librum, et alium locum, quam hunc ipsum, non esse quaerendum. Satis enim est hanc fuisse prophetiam, quae Pau- lo et Ecclesiae reuelata fuerit. Nec enim fuit Pau- lus mendax: sed Dei propheta: et de his rebus, vti de aliis multis, Dei voluntatem compererat quemadmodum apparet ex 2. Tess.3. Non posset fortasse prima fronte haec doctri- na, quae hic a Paulo damnatur, vti Daemoniaca, et hypocrisis plena, tanti esse momenti videri, cum de cibis tantum agat, et matrimoniis. Re- spondeo tamen cum hac ratione verae sanctitatis definitio, et cultus Dei natura prorsus peruersa fuerit, et in meram non modo superstitionem, sed impietatem vel hypocrisin conuersa, merito detestandam et periculi plenam dici et damnari. ouu emm huiusmodi sententiae inter Christia- nos obtinent, iam rebus externis alligatur vera sanctitas, nimirum ciborum et coniugii continen- tiae. Deinde hac ratione censetur Deus melius coli. Demum hac fenestra semel patefacta certa- tim homines ex ingenii sui vanitate in ἐθελοSon- σκάας ruunt, et Deum ex sui animi commentis co- lunt: non autem amplius ex sacrosancto ipsius ver-  \pend
\section*{CAPVT . IIII. }
\marginpar{[ p.181 ]}\pstart uanda, 1 Quanti momenti sit haec praedictio. 2 Quid contineat. Magni enim esse momenti ex eo apparet, quod ρντως, id est diserte et praecise a Dei spiritu praedicta fuit, et reuelata. Etsi enim omnium eorum, quae Paulus annuntiauit, Spiritus sanctus est author: hanc tamen tanquam specialem quan- dam prophetiam et admonitionem vtilem futu- ram Ecclesiae manifestauit, et obseruari voluit Spi- ritus sanctus. Quaeri vero potest, vbi hoc praedi- xerit Spiritus. Respond. Alium librum, et alium locum, quam hunc ipsum, non esse quaerendum. Satis enim est hanc fuisse prophetiam, quae Pau- lo et Ecclesiae reuelata fuerit. Nec enim fuit Pau- lus mendax: sed Dei propheta: et de his rebus, vti de aliis multis, Dei voluntatem compererat quemadmodum apparet ex 2. Tess.3. Non posset fortasse prima fronte haec doctri- na, quae hic a Paulo damnatur, vti Daemoniaca, et hypocrisis plena, tanti esse momenti videri, cum de cibis tantum agat, et matrimoniis. Re- spondeo tamen cum hac ratione verae sanctitatis definitio, et cultus Dei natura prorsus peruersa fuerit, et in meram non modo superstitionem, sed impietatem vel hypocrisin conuersa, merito detestandam et periculi plenam dici et damnari. ouu emm huiusmodi sententiae inter Christia- nos obtinent, iam rebus externis alligatur vera sanctitas, nimirum ciborum et coniugii continen- tiae. Deinde hac ratione censetur Deus melius coli. Demum hac fenestra semel patefacta certa- tim homines ex ingenii sui vanitate in ἐθελοSon- σκάας ruunt, et Deum ex sui animi commentis co- lunt: non autem amplius ex sacrosancto ipsius ver-  \pend
\section*{AD I. PAVL. AD TIM. }
\marginpar{[ p.1o4 ]}\pstart bo: quae res omnis impietatis proculdubio fons est ac origo Coloss.2. Magnopere igitur huiusmodi doctrinae vene- num nobis cauendum esse satis docent haec Pauli verba, per quae eam vocat, Defectionem a fide, Dae- moniorum doctrinam, vti etiam Iacobus 3.V.15. do- cet eam esse plenam hypocriseos, et Conscien- tiam cauterio infectam afferre. Alterum caput huius disputationis rem ipsam explicat, et Causam huius doctrinae, et Tempus et Doctrinam ipsam siue illius dogmata. Causa vero huius doctrinae est multiplex, quaedam Supe- rior, Spiritus deceptores, et quaedam inferior De- fectio a fide, et Hypocrisis, malaque hominum, tum Docentium, tum Discentium conscientia. Vocat autem hic Paulus spiritus non homines quidem horum dogmatum et errorum praecones et ministros: sed Sathanam ipsum, et illius Angelos, quos postea Daemones nominat, quia illi sunt et flabella et authores verae doctrinae corrumpendae, ne ho- mines ad salutem perueniant. Praeterea qui haec venena spargunt, etiam spiritum iactant, sed ille est spiritus vertiginis et erroris, cui Dominus ad puniendum nominis sui contemptum dat effi- caciam. Hi igitur spiritus dicuntur πάνοι, id est, non tantum deceptores, sed etiam seductores, quia nos a recta via, cui iam insistebamus, et in qua iam incedebamus, deflectunt, id est, a verbo Dei abducunt, cui vni tanquam soli veritatis co- lumnae est acquiescendum: quod plus est, quam nos adhuc veritatis ignaros decipere, et in erro- re confirmare. Ex parte autem auditorum causa quoque no-  \pend
\section*{CAPVT  IIII. }
\marginpar{[ p.185 ]}\pstart \phantomsection
\addcontentsline{toc}{subsection}{\textit{2 Per hypocrisin falsiloquorum, quorum conscientia cauterio resecta est.}}
\subsection*{\textit{2 Per hypocrisin falsiloquorum, quorum conscientia cauterio resecta est.}}tata est, nempe defectio a fide, et a sana doctrina. Doctores igitur appellat δαδολόγες: Eorum e- nim doctrina est falsa, et contraria verae et sanae doctrinae. Discipulos autem vocat ἀποςάτας et ho- mines leues, qui veritatem iam perceptam faci- le deserunt, qua sunt ingenii vanitate. Omnis er- go mali istius et ruinae fons et origo ex eo est, quod homines a verbo Dei desciscunt, et defi- ciunt ad vana animi sui figmenta scilicet, et ad impias suggestiones Sathanae, quibus attendunt. Dixit enim attendunt. Est autem attendere dili- gentem animi curam et operam ponere, et non perfunctorie vel leuiter aliquid degustare: quod statim deseras: sed magno studio et diligen- ti nosse, perquirere, et ediscere. Sunt autem ho- mines nouarum rerum et blasphemarum auidis- simi naturâ: sed imprimis ii, qui huiusmodi do- ctrinae attendunt. Tempus quoque designauit Paulus, non quidem tempora omnium nouissima: sed tantum posteriora. Quod notandum est, vt eo magis intelligamus, vtrum istum Pauli dictum ad Papistas pertinere possit, de quo seq. versicu- lo agetur. 2 Per hypocrisin falsiloquorum, quorum conscientia cauterio resecta est. Εʹπεξεσγασία, Operose enim et diligenter Pau- lus singula notat, quae in ipsis haereseon authori- bus sunt obseruanda, vt eae sint nobis magis dete- stabiles, et earum causae magis turpes et foedae appa reant. Itaque obseruanda sunt encomia, quae istius doctrinae authoribus hic a Paulo tribumntur. Tri buit igiturhis, Hypocrisin, et Conscientiam cau  \pend
\section*{AD I. PAVL. AD TIM. }
\marginpar{[ p.184 ]}\pstart teriatam istis inquam, authoribus haereseon, quos δαδελόγος quoque appellat. Hypocrisin hic quidam referunt ad eorum do- ctrinam, qui variis superstitionibus et caeremo- niis verum Dei cultum corrumpunt. Hunc enim praetextum accipiunt haeretici et impostores ho- mines, quo melius mendacia sua tegant, vt est Coloss.2. vers.23. nimirum quod augustiorem, sanctioremque Dei cultum inducere se iactent et cogitent. Itaque in eorum ἐθελοθρησκέία et superstitioso cultu apparet et humilitas, et ab- stinentia quaedam macerarioque corporis maior quam in reliquis hominibus. Et quanquam pessi- ma conscientia praediti sunt, neque sint ipsi sibi conscii se quod agunt, propter Dei gloriam age- re: hunc tamen fucum et praetextum purioris et sanctioris Dei cultus, clamosissime obtendunt. Sic quidam vocem hanc explicant. Alii hypocrisin ad impudentem potius auda- ciam, qua vtuntur, referunt, cum eo hominum genere nihil sit magis effrons et impudens. Id quod etiam docet Paul.2. Timoth.3. Mihi vero videtur Paulus hic tria complexus esse, nempe 1 Doctrinam ipsam. 2 Mores. 3 Internum affectum animi. Sanae, doctrinae opponitur tum quod supra dixit eam doctrinam Diabolicam esse: tum quod hoc loco vocat istos εαδολόγος, vel δαδολογίας: vt alii legere malunt hoc loco. Mori- bus sanctis opponit Paulus istorum hypocrisin, vti fere in scriptura opponi solet. Interno animo affectui recto et pio, qualis in veris doctoribus inest, opponit Paulus conscien- tiam eam, quam cauterio notatam et infectam  \pend
\section*{CAPVT  MIII. }
\marginpar{[ p.185 ]}\pstart \phantomsection
\addcontentsline{toc}{subsection}{\textit{3 Prohibentium contrahere matrimo- nium, iubentium abstinere a cibis, qous Deus creauit ad participandum cum gratiarum a- ctione, fidelibus, et iis qui cognouerunt ve- ritatem,}}
\subsection*{\textit{3 Prohibentium contrahere matrimo- nium, iubentium abstinere a cibis, qous Deus creauit ad participandum cum gratiarum a- ctione, fidelibus, et iis qui cognouerunt ve- ritatem,}}vocat, id est, non sanam, et rectam. Haec autem docent haereticorum hoc loco descriptorum ne- que esse sanam doctrinam, neque sanctos mores neque rectum animi affectum in suis illis com- mentis tradendis, quicquid illi tamen contra praetendant et praetexant, vel persuadere conen- tur. Cauteriatam vero conscientiam quam appel- let Paulus varie etiam quaesitum est. Alii enim interpretantur putrem, et morbidam: alii nullam, vt pote quae iam perdita prorsus et excisa sit, vt sunt ea, que cauterio ambusta sunt: alii vexatam et perpetuo Dei iudicio tanquam cauterio quo- dam ssirbatam fluctuantem, et afflictam, atque ipse hanc postremam interprecationem sequor. Nam nulla non inest in iis conscientia, cum eam cauteriatam esse dicat Paulus, non autem deper- ditam. Putris quidem est, id est, male sana, sed quae sunt putria non sunt huiusmodi, vt nihil pla- ne sentiant et omnino emortua sint. Hic conscien- tiam explicat Paulus, quae assidue iudicio Dei vo- xetur et torqueatur propter malum testimonium et hypocrisin, quam simulat, et propter finem pessimum, propter quem istos errores propagat, cuius ipsa sibi testis ac conscia est, vt quae sit ἀυτο- κατακριτὸς vt idem Paul. loquitur Tit.3.vers.10. 3 Prohibentium contrahere matrimo- nium, iubentium abstinere a cibis, qous Deus creauit ad participandum cum gratiarum a- ctione, fidelibus, et iis qui cognouerunt ve- ritatem,  \pend
\section*{AD I. PAVL. AD TIM. }
\marginpar{[ p.400 ]}\pstart Εξήγνσις, Rem enim ipsam et doctrinam illam falsam iam explicat. Duas autem illius species e- numerat, nempe Ciborum prohibitionem, et Con- iugii, siue quod illae tuncipraecpue subnasceren- tur:siue quod primae. Certe ex iis duabus rebus superstitiose interdictis infinita postea aliarum superstitionum scaturigo et congeries vel gene- ra sunt nata, et inducta. Magno vero nostro bono explicatio harum haerescon et diabolicae doctrinae a Paulo particularis addita est, ne vlli, ac ne haeretici quidem ipsi, generalem illam falsae doctrinae descriptionem a Paulo propositam a se detorquere, et in alios torquere poslent, essetque incertum, quinam hic designarentur a Tiritu sancto. Huius autem prohibitionis meminit E- piphan. in Collyridianis, vbi etiam tertium ca- put addit, nempe Mortuorum adorationem siue cultum. Meminit etiam Socrat.libus 2. cap. 43. Eu- sebus lib. 5, Histor. Eccles. Hic autem notandum est a Paulo non certas quasdam personas, sed doctrinam potius ipsam notari, quae vel, In vniuersum, vel In parte Cibos et coniugia prohibet, atque in earu rerum absti- nentia cultum Dei et partem pietatis atque san- ctitatis vitae nostrae constituit. Id quod prorsus cum verbo Dei pugnat. Ac plures sunt sectae, quae his Pauli verbis da- mnantur. Primum, quae in vniuersum damnarunt ea quae hic commemorantur, et quae nobis con- cessa esse Deus voluit. Deinde, quae non in totum, sed ex parte tantum haec ipsa prohibuerunt, de quibus et hic doctissime Cal. et nos in lib.  Augu. de Haeres. ad Quodvultdeum diximus. Manichaei  \pend
\section*{CAPVT  IIII. }
\marginpar{[ p.189 ]}\pstart igitur comprehenduntur, Encratitae, Cataphry- ges, Montanistae, qui omnes in hac parte Pytha- goreorum delirio fascinati sunt. Sed et Priscillia- nistae quoque, vt ait Bernard.sermo. 66. in Cant. Deinde Papistae ipsi, qui eadem ista tantum ex parte, non autem in totum, vt illi priores haere- tici, damnarunt, quos hic notat Paulus: quanquam cibos omnes nemo vnquam prorsus et in totum pro- hibuit, propterea quod sine his vita haec corpo- ris sustentari non potest. Sed ex cibis, alii negant concedi Christianis hominibus Carnes, vt Mani- chaei, alii vinum, vt Encratitae. Atque hae prohibitio- nes habent speciem magnae humilitatis et corpo- ris macerationis: sed tamen cum sint ἐθελοθρη- σκείαι et Dei cultus in illis statuatur, sunt diaboli- cae et in Deum ipsum blasphemae superstitiones, quae sunt omnino ex Dei Ecclesia profligandae, vt docet Paulus Coloss2. et Tit.1. vbi de iisdem rebus agit. Ac de quadam ciborum differentia et distinctione, non ea quidem, quae postea ab haere- ticis et Papistis inuecta est, et de qua Paulus hic vaticinatur: sed quae ex Mosaica praeceptione de- fendebatur, vt apparet Leuitic. 20.V.25. et Deut. I4.iam tunc contentio et certamen erat in Dei Ecclesia, vt docet Paulus Rom.14. vers.3. 1.Cor. ISrrernej, et 27. vt de futuro super his rebus ma- iori errore, peste ac pernicie periculosiore fuerit operaepretium Ecclesiam moneri. Neque vero quod inter detestandos haereti- cos hic censentur, qui ciborum vsum prohibent, facit, vt ingluuies, crapula, luxus et intemperies probetur. Nam nec eo pertinet haec prohibitio ci- borum vt sobrie non viuatur:sed vt ne in his rebus  \pend
\section*{AD I. PAVL. AD TIM. }
\marginpar{[ p.180 ]}\pstart pars cultus Dei censeatur posita: nec quod addit Paulus, eos nobis a Deo concessos esse, aliter in- telligendum est, quam vt iis, conseruata mode- ratione et temperantia Euangelica et Christiana, vtamur. Cauendum enim est, ne crapula et cibis grauentur corda nostra, vt docet Christus ipse Luc.21.vers.34. Sic hune locum explicat quoque August. lib.  contra Adimantum Manich. cap. 14. Et aliud est prohibere cibos, aliud autem mten- perantiorem eorum vsum damnare, quod etiam nos facimus ex Dei verbo. Sed neque ieiunia quoque tolluntur et abro- gantur, quorum vsus frequens fuit in Ecclesia Dei, atque vtinam inter nos esset frequentior et religiosior eorum obseruatio, cum ab Ecclesia ex iustis causis publice indicuntur. Aliud enim est cibos ex superstitiosa opinione damnare, et in eorum delectu et vsu regnum Dei collocare, quod facere vetat Paulus et hic et Rom. 14 vers.17. a- liud autem iis ad tempus abstinere tantum ad con- primendam carnis lasciuiam, quae nimio pastu, tanquam ferox equus, superbit, et furit. Itaque ie- iunia non damnantur hoc loco a Paulo. Sic Au- gust aduersus Manichaeos lib. 6. contra Faustum cap.8, et 3o, cap.5. Sic Bernar. sermo. 66. in Cant. qui plane cum hoc loco conuenit, et huius loci Pauli interpres est optimus. Hanc autem esse haeresin, et diabolicam doctri- nam variis argumentis et rationibus probat Pau- lus: sed imprimis a fine, propter quem cibi a Deo creati sunt. Duplex quidem erroris genus hic a Paulo propositum est, vnum de cibis: alterum de coniugio:tamen hoc primum solum persequitur,  \pend
\section*{CAPVT IIII }
\marginpar{[ p.187 ]}\pstart et refutat : alterum praetermittit, quod hoc malum nondum vicinum esse praeuideret Paulus. Sed a- pertissime in Epist. ad Hebr. cap.13.vers.4 ref ata- tur, et iugulatur iste error de prohibitione con- iugii. Etsi vero praecipuus confutationis locus a fine ducitur, alia tamen argumenta admiscentur, quae singulis pene Pauli verbis comprehensa sunt. Haec autem optima erroris cuiusuis tollendi via, et ratio est, si ad Dei ordinationem et verbum nostras sententias et definitiones reuocemus. Quod ipsum in disputatione de Coena facit alibi Paulus 1.Corinth.11. Sed singula verba examine- mus. Nam in iis varia latent argumenta, quem- admodum diximus. Deus creauit) Locus refutationis a temeritate istorum hominum et a sacrilegii crimine. Nemo inrem alienam, et praesertim Dei, ius habet. At cibi sunt Dei creaturae, non istorum. Ergo Deï est, non horum, leges de cibis praescribere. Et qui aliter faciunt, ingerunt se in aliena quae ad eos non pertinent, de quibus nullum ius habent definien- di et statuendi, vt ait Paulus Coloss.2. vers.18. Est enim quisque rei tantum suae, non autem alienae legitimus moderator. Hinc sane refutantur varii haeretici, quorum alii vinum, alii carnes, alii aliud esse Diaboli, non autem Dei opus et creaturam pessime atque falsissime contendebant. Ad participandum) Ait Paulus ἐις μετδ λ̓ηαιν. vox ἐις finem declarat. Μετάληδις vsum, more di- cendi in quauis lingua vsitato, vt etiam in Odys- seam Homeri annotat Didymus interpres anti- quissimus, vt ex antecedenti consequens, et contra concludatur et intelligatur. Nec enim tantum a  \pend
\section*{AD I. PAVL. AD TIM. }
\marginpar{[ p.192 ]}\pstart nobis cibos assumi posse vult, et concedit Deus, vt tractentur, spectentur, conseruentur: sed vt e- dantur, consumantur, et iis famem nostram suble- uemus. Sunt enim cibi ex istarum rerum genere, quae vsu ipso consumuntur. Itaque in genere suo functionem, vt aiunt Iurisconsulti, recipiunt. Est autem hic locus ex praecipuis refutationis superio- ris erroris argumentis. Ducitur a fine, propter quem a Deo ipso illae res sunt conditae et creatae Ad eum enim sunt omnia referenda, nisi et Dei ordinationi resistere, et eius placitum euertere impie, at que impudenter velimus. Ergo cum Do- minus hominem creasset, tamen noluit eum sibi suo arbitrio ex creaturis ipsius rapere ad victum quas vellet: sed ipse tanquam optimus paterfa- milias quas et quantum vidit necessarias esse, illi concessit benignissime. Concessit autem eas, quas cibi nomine propter vsum appellamus. Confirmatur autem hoc Pauli dictum Genes.1. vers.29.9. vers.3. Matth.15.vers.11. et Actor.1o.V.13 Explicatur etiam a Fulgentio in lib.  de Fide ad Petrum cap.42: Dissimiles tamen loci quidam, qui solebant a Manichaeis huins erroris summis patronis afferri et a Papistis etiam regeruntur, sunt sane a nobis di- luendi. Ac primum Papistae excipiunt, cibos istos non ipsos per se suaeque substantiae ratione pol- luere edentes, sed propter Ecclesiae prohibi- tionem et authoritatem, quae violatur a sumenti- bus. Respondeo vero ipsam Ecclesiam id prohi- bere non posse. Neque enim in res alienas ius habet, neque conscientiis nostris legem et iugum imponere potest, quod Dominus ipse non impo-  \pend
\section*{CAPVT  IIII. }\pstart soit. Itaque petunt illi principium, vt loquuntur, cum volunt illam Ecclesiae prohibitionem de ciborum vsu et differentia firmam et legitimam esse. Quam quaestionem breuius tracto, quia ab infinitis nostro seculo, imprimis autem a D. D. Lu- thero, et Caluino explicata est. Manichaei vero et Tatiani alios aliquot scriptu- rae locos obiiciunt aduersus Paulum. Primum Genes.1.vers.29. vbi nulla fit carnis mentio inter eos cibos, qui homini conceduntur a Deo: sed tantum herbarum, olerum, et fructuum arboris. Ergo si ad primam Dei institutionem reuoce- mur, carne nos abstinere oportere aiunt. Resp. Paulum spectare non illam modo concessionem, quae homini primum a Deo facta est: sed eam, quae postea, tanquam illius beneficii ampliatio quae- dam, accessit, et extat Genes.9.vers.3 vbi carnium esus diserte conceditur, et exprimitur. Secundo locum Pauli afferunt, ex 1. Corinthe 9. vers.7. vbi lactis tantum, et eius esus meminit Paulus, non etiam carnis. Resp. Non excludere illam Pauli sententiam esum carnis, cuius ipse meminit, vti rei licitae etiam in eadem Epistola 8. vers.13. Nam illic alia Pauli mens est et scopus, quam vt explicet vtrum homini Christiano car- nem edere liceat, nec ne. Tertio dicunt inter res ad victum humanum necessarias non recenseri carnes in Ecclesiastic, cap.39, vers.31, et 32. cum tamen multa ciborum genera enumerentur. Respond. Quod illic prae- termissum est, alibi additum esse. Neque ex eo libro regulam fidei vitaeque instituendae petendam esse, cum sit liber Apocryphus. Denique illic oepen-  \pend
\section*{AD I. PAVL. AD TIM. }
\marginpar{[ p.194 ]}\pstart δοχικῶς ex specie vna reliqua ciborum genera in- telligi oportere. Quarto, docent Deum ipsum llanc ciborum differentiam instituisse maioris cuiusdam sancti- tatis gratia: et olim carnium esu interdixisse populo suo. Deut. 22. Quae res non est visa leuis momenti, cum laudentur qui etiam mortem obi- re non recusarunt, ne vetitas a Deo carnes esita- rent, vt apparet ex historia Macchabaeorum. Re- spond. Cum adhuc sub paedagogia versaretur Dei populus, nec eius libertatis, quam per Chri- stum adepti sumus, plenum vsum haberet (quo inter Christi nondum manifestati et manifesta- ti tempora perspicuum discrimem appareret) Do- minum voluisse quibusdam vmbris vitae spiritualis sanctitatem, quam exigebat, illis repraesentare, quae illis quidem partim variis lotionibus: partim ista ciborum distinctione proposita fuit, tanquam ru- dioribꝰ adhuc puerulis. Quae causa in nobis cessat, qui per Christi aduentum et gratiam ad legitimam Ecclesiae aetatem peruenimus, et pueri esse desii- mus. Itaque haec causa fuit instituti istius delectus ciborum a Deo, nempe vt illis adhuc rudioribus esset vitae sanctioris delineatio et significatio:non quod cibi prohibiti ipsi per se impuri essent, aut quod pet sese ipir ciorconceli puriores, cum tota ea res carnalis fuerit (vt ait Scriptura) quae proprie animos non attingit, vt docetur in Epist. ad He- braeos cap.9. vers. 1o. 7. vers. 16. Atque haec ipsa hodie firmissima est Papistarum ratio, quam a Ma- nichaeis sumpserunt. Cum gratiarum actione) Tertium argumentum ab haereticorum blasphemia ductum. Est ea do-  \pend
\section*{CAPVT  IIII. }
\marginpar{[ p.195 ]}\pstart \phantomsection
\addcontentsline{toc}{subsection}{\textit{4 Nam quicquid creauit Deus, bonum est, nec quicquam reiiciendum est, si cum gratiarum actione sumatur.}}
\subsection*{\textit{4 Nam quicquid creauit Deus, bonum est, nec quicquam reiiciendum est, si cum gratiarum actione sumatur.}}ctrina blasphema, quae Deum spoliat suo hono- re, et ea gratiarum actione, quae propter summa ipsius in genus humanum beneficia illi debetur. At haec Deum isto honore priuat. Ergo, etc. Addit autem quibus maxime hic creaturarum vsus sit con- cessus a Deo, nimirum piis, qui eum noue- runt, et in eum credunt. Hic enim, nosse, non sim- plicem et historicam agnitionem significat: sed cordi inhaerentem, et cum animi fiducia coniun- ctam, per quam in Deum homines acquiescunt vera animi fiducia et πλνρςφορία. Similis locus Romanor.4.vers.13. Dissimilis locus Matth.5. vers.45. et Psal.17.y. 14. vbi dicuntur etiam infideles a Deo impleri i- psius thesauris. Respond. Dantur quidem a Deo terrena bona etiam infidelibus hominibus: sed in eorum exitium, et perniciem, quo magis reddan- tur inexcusabiles. Itaque non tam iis vtuntur, quam abutuntur: neque tam vsus ab iis fit, quam depraedatio quaedam, quoniam Deum dantem spernunt, in cuius contumeliam et contemptum ista dona sumunt et consumunt, quem non agno- scunt eorum authorem. 4 Nam quicquid creauit Deus, bonum est, nec quicquam reiiciendum est, si cum gratiarum actione sumatur. Est hic locus canon, et quidem generalis, ac maxime necessarius. Iis omnibus nimirum vti nobis licere, quae in nostrum commodum crea- uit Dominus, modo cum gratiarum actione illa sumamus. Ex quo colligitur, 1 ad eum finem  \pend
\section*{AD I. PAVL. AD TIM. }
\marginpar{[ p.196 ]}\pstart rebus creatis vtendum ad quem a Deo conditae sunt. 2 Non modo Deo gratias agendas, post- quam iis rebus vsi sumus: sed etiam Deum inuo- candum, vt vti liceat, cum ex eius manu sint eae res sumendae: cuius proprie sunt, vt apparet ex oratio- ne Dominica. καλὸν vero non refertur hoc loco ad mores honestos: sed ad vsum, qui a nobis Dei nomen inuocantibus recte fieri et salua conscien- tia potest. Aliâs enim perinde, atque fures dan- naremur etiam ex minutulae cuiusuis rei vsu: et merito quidem, quasi rem alienam contrectan- tes: nisi Dominus ipse nobis eam in vsum con- cederet, eamque commoditatem ex rebus a se creatis permitteret. Bonitas item ciborum hic non pertinet ad valetudinem et sanitatem cor- poris, nec ad salubritatem ipsorum ciborum. Ne- que enim hic medicum agit Paulus, sed Theo- logum et Apostolum: ne quis forte obiiciat si- bi aliquos cibos non esse salubres, dum iis vtitur. Atque propterea hunc Pauli locum calumnietur, quia generaliter dicit Paulus οὐδεν ἀπόβλητον, id est, nihil esse reiiciendum ex iis, quae Dominus in v- sum et cibum hominum creauit. Quid enim li- ceat conscientiae nostrae ratione spectat Paulus, non autem quid conferat stomacho nostro, vel valetudini corporis. Hic tamen et inferiori versiculo duplicem ci- borum bonitatem proponere viderur, Vnam, quae ex Ipsorum substantia: et Alteram, quae ex vsu nostro consideranda et aestimanda nobis est. Quod ad primam:illa nobis testata est Genes.cap.1. Quae- cunque enim condidit Deus, illa substantiae, in qua sunt creata, ratione bona sunt, vt Dominus ipse,  \pend
\section*{CAPVT  IIII. }
\marginpar{[ p.1. ]}\pstart pronuntiauit. Nec obstat quod quaedam ex iis sunt nobis venena. Respondeo enim non idcirco mala esse. Malum enim, quod bonitati rerum crea- tarum opponitur, non est vsus noxius (nam ar- bore scientiae boni et mali impune vti non licuit Adamo) sed est malum hoc loco qualitas interna quae peccatum in nobis producit: et Dei ordina- tioni contraria est, nobisque propterea noxia, et detestanda. Deinde quae creaturae sunt nobis vene- na, illae non sunt nobis in cibum a Deo destinatae. Denique hoc ipsum vitium, quod rebus creatis nunc inest et superuenit, non ex ipsa Dei creatione, sed ex peccato hominis accessit. Distinguendum au- tem est vitium et accidens a rei natura et substan- tia, vt docet Augustinus in articulis contra Felicem Manichaeum cap.8.9. et in lib.  de Fide ad Petr. cap.42. Si cum gratiarum) Altera bonitas ciborum ex vsu ipso, et nostri ratione bonitas appellanda. V- sus autem hic certis legibus a Paulo circunscri- ptus est, vt a profanorum mensis piorum homi- num conuiuia et victus separetur. Cur autem has leges addat Paulus, ratio est, quod pro concesso sumit, Nihil nobis, tanquam a nobis ipsis, in Dei creaturis iuris esse vel licere, sed id tantum, quod Deus, qui est earum Dominus, quantumque con- cessit, et largitur. Ergo si, vel alio Modo, vel fine, illis vtamur, sane peccamus. Quoad modum au- tem, Dominus non vult nos immoderate suis re- bus et donis vti, quia hic plane abusus est, non vsus: atque ipsius dona non sunt spernenda, aut con- temptim habenda, quod ii faciunt, qui per crapu- la m'et gulam sese ingurgitant, sed magni sunt ae-  \pend
\section*{AD I. PAVL. AD TIM. }
\marginpar{[ p.450 ]}\pstart \phantomsection
\addcontentsline{toc}{subsection}{\textit{5. Sanctificatur enim per verbum Dei et preces}}
\subsection*{\textit{5. Sanctificatur enim per verbum Dei et preces}}stimanda. Quoad finem vero, ideo sumi vult, tan- quam a sua ipsius manu hos cibos sumamus, vt ipse eorum author agnoscatur et laudetur. Similis locus Psal. 16. et Psal. 5o. Huius autem totius rei rationem reddit sequenti versiculo Pau- fus. 5. Sanctificatur enim per verbum Dei et preces Α’ιτηλογία est ducta a causa instrumentali (quae refertur ad genus causarum efficientium) per quam ciborum vsus fit nobis purus, sanctus, et legitimus. conscientiae nostrae ratione. Ergo san- ctificatio hoc loco opponitur non tantum profa- nationi ciborum:sed etiam malae conscientiae, et ei vsui, qui nobis in exitium et condemnationem apud Deum cedit. Sanctificatur cibus cum eum salua conscientia sumimus, et eo vtimur, quia no- bis Dominus ipse eum largitur ad finem huius- modi. Neque vero sic sanctificari cibus dicitur, quemadmodum homo. Nam qui a spiritu Dei sanctificatur homo, is immutatur, et nouam tan- quam naturam induit. At qui cibus sanctificatur in nostrum vsum, totam integramque naturam suam retinet, fit tamen Deo gratus ex eo, quod Deus nostra inuocatione illius author esse a no- bis agnoscitur et praedicatur. Ex hoc autem loco perperam explicato natae sunt magicae illae ciborum, velut ouorum, vini, pernarum, aquae execrationes et incantationes, quas Pontificii faciunt, quosque D. Martinus Chemnitius prolixe in Examinis Trident. concilii postrema parte persequitur et refutat doctissime. Duplex autem huius cibi san-  \pend
\section*{CAPVT  IIII }
\marginpar{[ p.19 ]}\pstart \phantomsection
\addcontentsline{toc}{subsection}{\textit{6 Haec si subieceris fratribus bonus eris minister Iesu Christi, innutritus in sermoni-}}
\subsection*{\textit{6 Haec si subieceris fratribus bonus eris minister Iesu Christi, innutritus in sermoni-}}ctificandi ratio proponitur, verbum Dei, et Precatio. Verbum Dei, quia tum demum sana conscientia sumimus hos cibos, cum per Dei verbum intelligimus eos nobis in esum conces- sos et datos esse. Itaque hoc Dei verbum fide a nobis apprehendi prius debet, quam hic cibus sanctificari possit. Cur autem ad Dei sermonem sit nobis recurrendum, ratio est, quod per Adamum omne ius illud, quod prius in Dei creaturas, et mundum hunc habuimus, perdidimus. Quod nobis restitutum esse nonnisi ex Dei verbo fide apprehen so possumus agnoscere. Hanc fidem sequitur pre- catio nostra, quae a Deo petit, qu'od est ipse mera sua benignitate pollicitus, et agnoscit eum tan- torum bonorum authorem esse. Atque tunc de- mum nostrum esse et a Deo nobis donatum, quod apponitur, intelligimus, vt sine animi serupulo vtamur. Haec autem sanctificatio communis mensae differt ab ea, quae fit in mystica Coena Domini- vti praeclare docet hoc loco Caluinus. Quaerit Chrysostom. de deliciis ciborumo v- trum iis vti liceat homini Christiano. Respond, vero cum nobis profusa liberalitate sua dona largitus sit Dominus, nec vnius tantum generis cibos in vsum nostrum creauerit, posse nos cum gratiarum actione, etiam cibis lautioribus vti. neque tantum vulgaribus, modo omnis luxus, ingluuies, intemperantia, et crapula absit. 6 Haec si subieceris fratribus bonus eris minister Iesu Christi, innutritus in sermoni-  \pend
\section*{AD I. PAVL. AD TIM. }
\marginpar{[ p.200 ]}\pstart \phantomsection
\addcontentsline{toc}{subsection}{\textit{bus fidei, bonaeque doctrinae, quam assectatus es.}}
\subsection*{\textit{bus fidei, bonaeque doctrinae, quam assectatus es.}}bus fidei, bonaeque doctrinae, quam assectatus es. Παράκλησις et exhortatio, per quam quae de pastoris officio et vera doctrina contra haereses tuenda in vniuersum dixit, applicat ad Timoth. Sic enim nobis agendum est, vt quae funt nostri muneris diligenter perpendamus, et ex iis intelli- gamus, vtrum fungamur officio. Duo vero prae- cipit Paulus I.vt superiorem doctrinam Timot. suggerat, 2 vt fratribus ipsis. Suggerere (inquit Chrysostomus) est frequenter proponere. Vti enim cibis quotidie ad pastum corporis egemus: sic Dei verbo et sana doctrina ad pabulum ani- mi, vt semel sanam doctrinam proposuisse non sit satis:neque odiosa censeri debeat huius doctri- nae tam necessariae et salutaris repetitio, vt est 2.Pet. I.vers.12. et 3.vers.1. cum Sathanae impetus in op- pugnanda Dei doctrina nunquam fatiscat et con- quiescat. Dixit autem Paulus ὐποτιθέμενος i.subiiciens, vel suggerens non autem imperans, vt obseruat hoc loco Chrys. quod distincta sit muneris et mini- sterii Ecclesiastici ratio a ciuili imperio. Magi- stratus imperat: denuntiat autem tantum. Minister verbi Dei. Quae tamen denuntiatio habet con- iuncta Dei iudicia et minas, si non illi pareatur, longe omni poena corporali formidabiliores et tetriores. Paulus tamen 2. Thess.3. vers.12. vtrunque coniunxit et imperium et obsecrationem, quia Domini ipsius nomine agunt Ministri Euangeli- ci:non autem suo et priuato. Ronus erus minister) Duplex exhortationis locus  \pend
\section*{CAPVT  IIII. }
\marginpar{[ p.201 ]}\pstart nimirum tum ab officii ratione, tum a prima Ti- mothei vitae institutione. Ergo vtrunque nobis sedulo cogitandum est, et quid munus, in quo versemur, requirat: et quid susceptae vitae ratio et institutio. Vtrunque enim est nostrae vocationis a Deo factae certissimum testimonium, Quatenus igitur Minister munus suum consi- derat tria haec spectare debet, vt sit Minister vt Iesu Christi, vt Bonus siue fidelis. Quae his tri- bus opponuntur, nempe Ministris Otiosis, Hu- manae doctrinae, et Infidelibus. Hi vero sunt, qui vel Seipsos, id est, suam gloriam, vel quaestum suum quaerunt. Ergo sana doctrina a fidis Mini- stris verbi Dei proponenda est: non autem inanes, non futiles, et subtiles quidem, sed futiles quae- dam quaestiones, quae neque nostras conscientias aedificant:neque Dei gloriam, salutisque doctri- nam vllo modo explicant vel illustrant. Quod autem addit, Jnnutritus sermonibus fidei. etc, magnam vim et pondus exhortationi addit. Est enim prima educatio tanquam tacitum futu- rae vocationis indicium a Deo demonstratum. Denique turpe est illa talenta, quae Deus in nos contulit, vel omnino perdere, vel infructuo- sa habere, vt docet Christus Matth. Vtitur autem mis verbis Paulus, ex quibus intelligitur, seriam prius operam Timotheum nauasse in sacris lite- ris, et doctrina fidei Euangelica. Id quod appa- ret etiam ex 2.Timoth.1.vers. 5. sup. cap.1.vers.18. Ex hoc autem loco quaeri duo possunt 1. Vtrum eos solos in ministerium Euangelicum eligi o- porteat, qui a primis et teneris annis sacris sçri- turie sunt innutriti et eas consectati:an etiam alii  \pend
\section*{AD I. PAVL. AD TIM. }
\marginpar{[ p.202 ]}\pstart possint. Respond. Deum quidem in quem vult do- na sua conferre. Itaque eum nobis eligendum esse, cui maiores dotes et χαρίσματα Deꝰ ipse dederit. Tamen longe tutius esse, vt, nisi magna aliqua causa subsit, ii potissimum eligantur, qui diutius in sacrae Scripturae lectione versati sunt, et a tene- ris annis sanam Domini doctrinam tum didice- runt, tum etiam professi sunt. Secundo quaeritur, An a suscepta semel profes- sione Euangelici ministerii abduci et reuocari homines pastores possint. Vult enim Paulus Ti- motheum in ea cognitione et arte perseuerare, quam semel assectatus est. Deinde damnantur et Demas et Marcus, qui Paulum deseruerant, et susceptam semel ministerii vicariam delegationem reliquerant, mutato vitae genere ac instituto. Res. Deum nullum huiusmodi tam arctum et strictum vinculum inter pastorem et gregem instituisse, vt nunquam solui possit. Saepe enim ex multis, et iis quidem iustis, causis liberantur pastores, et habent a suo susceptoque munere legitimam missionem. Id tamen, nisi ex iusta causa fiat, proculdubio in temeritatis et eius quidem turpissimae non tan- tum suspicionem, sed etiam criminationem meri- to incidit, qui consilium suscepti semel ministerii mutauit et deserit. Et quod Demas, et Marcus damnantur, ex eo est, quod seculum Deo praetu- lerant: non tamen propterea excommunicantur, aut eiiciuntur ex Ecclesia a Paulo, etsi nulla fuit deserendi ministerii iusta, quae ab iis praetenderetur causa. Et quod ait Bernard.epistola 87.Aut ergo oportuit te gregem dominicum minime seruan- dum suscipereraut susceptum nequaquam relin-  \pend
\section*{CAPVT . IIII. }
\marginpar{[ p.293 ]}\pstart quere iuxta illud Alligatus es vxori, noli quaere- re solutionem. Primum falsa similitudine nititur coniugii, et vinculi pastoralis: deinde nec eum i- psum Ogerium, qui gregem suum deseruerat, dan- nat, tanquam qui sit ab Ecclesia expellendus, sed qui sui facti et temeritatis poenitere debeat, non de eo gloriari. Et sane in eo falluntur homines, qui non putant magis licere pastori Ecclesiastico ministerii professionem semel susceptam mutare, quam maritato coniugem non adulteram dimit- tere. Etsi sine ratione ista vitae mutatio fieri non debet.  \pend\pstart \phantomsection
\addcontentsline{toc}{subsection}{\textit{7 Caeterum profanas et aniles fabulas reiice: sed exerce teipsum ad pietatem.}}
\subsection*{\textit{7 Caeterum profanas et aniles fabulas reiice: sed exerce teipsum ad pietatem.}}7 Caeterum profanas et aniles fabulas reiice: sed exerce teipsum ad pietatem. Αʹντίθεσις. Sanae enim et superiori doctrinae opponit Paulus fabulas, quas Prophanas, et Ani- les vocat. Fabularum nomine intelligit tum supe- rius damnatam de prohibitione ciborum doctri- nam: tum etiam reliquas, quae in rebus externis cultum Dei constituunt Prophanas voca, quod sint blasphemae. Aniles autem, quod sint nihili. Est autem hoc postremum dictum ex forma qua- dam vulgaris prouerbii sumptum, quia solent a- nus mulieres fabulas et fictitia quaedam comme- morare lubentius, quod propter aetatem multa sibi tribuunt: atque sunt natura verbosae. Has ta- men fabulas Chrysost. refert non ad superiorem doctrinam, sed ad Iudaicas traditiones, et caeremo- nias: sed perperam, vt quilibet facile animaduerte- re potest. Sed exerce teipsum) Transitio est, per quam ad alteram ministerii et muneris Euangelici parte acce  \pend
\section*{AD I. PAVL. AD TIM. }
\marginpar{[ p.204 ]}\pstart dit Paul. Est autem haec altera pars Pia, et religio- sa vita. Neque enim pastores tantum sanam do- ctrinam tradere debent, sed etiam vitae suae ho- nesto exemplo eam confirmare, et caeteris praelu- cere. Quemadmodum et Christus ipse Matt.5.v.13. et Petrus 1.Pet. 3.V.2.3. docent. Ratio igitur huius additionis duplex est. Prima, quod Ministri verbi Dei caeterorum comparatione esse debent lux mun- di, et lucidum exemplar sanctae vitae non tantum fidelibus, sed etiam ipsis infidelibus, vt docet inf.v. Paul.12. et Tit.2.v.7. Secunda vero, quod doctri- nae suae fidem illi derogant, qui sunt turpis, et inhone- stae vitae pastores. Si quis enim docet non esse fu- randum et ipse furetur, eleuat suae praeceptioni vim et authoritatem, que madmodum ait Paul. Rom. 2. Contra vero ipsa suae doctrinae praxi et prae- statione constanti caeteros in ea doctrina confir- mat. Cauendum igitur nobis est, ne praedican- tes aliis ipsi reprobemur, nobisque vere obiicia- tur illud, quod Christus Pharisaeis, Dicunt, sed non faciunt 1.Corinth.9.V.27. Matt.23.vers.3. Pulchra igitur est harmonia recte docentis et sancte vi- uentis Ministri verbi Dei. Similis huic loeus est 2. Timoth. 2.vers.15.Tit. 2.vers.7. Dissimiles vero loci afferri videntur posse. Pri- mum dictum Christi Math. 23. vers.1. Etsi enim erant Pharisaei moribus impuris, eorum tamen sanam doctrinam sequendam et retinendam esse docet Christus. Respond. Gregis quidem ratio- ne illa Christi sententia vera est, eorum nempe pastorum doctrinam esse amplectendam: qui a veritate non discedunt. Sed tamen illi ipsi si male  \pend
\section*{CAPVT  IIII. }
\marginpar{[ p.205 ]}\pstart vixerint, eo grauius sunt damnandi, quod eos, quos verbis aedificare student, vitae turpissimae exen- plo perdunt. Postea vero aduersus hunc locum obiicitur illa Christi sententia, Martha sollicita est circa plurima Luc. 1o. vers. 41. Ex qua vitam contemplatiuam Actiuae, quam vocant, et quam hac voce, γέμναZε, significasse videtur Paulus, prae- ferri colligunt. Respond. Nihil illum locum cum hoc Pauli habere commune. Alia enim plane Christi mens est illic, quam vt piae et actiuae, quam vocant, vitae studium restinguere in nobis cone- tur Christus: quod actiuae vitae genus hic procul dubio etiam Paulus commendat: at que vanae et Mo- nachicae hominum speculationi. i. vitae conten- platiuae postponit, quicquid de loco illo Lucae Monachi imperiti sentiant. Ait γύμναζε Exerce) Qua voce duo demonstrat Paulus, 1 Continuum vitae nostrae studium et exercitium in eo esse debere, vt sancte, et pie, et religiose viuamus. Is enim est finis, propter quem salutaris illa Dei gratia per Christum ap- paruit. 2 Vt intelligamus quam natura inepti et indociles ad huiusmodi opus simus, ad quod nos ideirco formari longa et perpetua exercita- tione, cura, meditatione, et vigilantia necesse est, Ataque ad ilau non nascimur apti. Nam ad male agendum non est nobis vlla exercitatione opus, cum sponte in illam partem propendeamus et procliues simus. Denique ex hoc loco verum esse intelligitur, quod ait Bernardus epist. 201. Pasto- ris officium esse, vt pascat gregem suum verbo: pascat exemplo, pascat et sanctarum fructu ora- tionum. Nam vocis virtus est opus, et operi et  \pend
\section*{AD I. PAVL. AD TIM. }
\marginpar{[ p.206 ]}\pstart \phantomsection
\addcontentsline{toc}{subsection}{\textit{8 Nam corporalis exercitatio paululum habet vtilitatis: at pietas ad omnia vtilis est, vt quae promissionem habeat vitae praesentis ac futurae.}}
\subsection*{\textit{8 Nam corporalis exercitatio paululum habet vtilitatis: at pietas ad omnia vtilis est, vt quae promissionem habeat vitae praesentis ac futurae.}}voei gratiam efficaciamque promeretur oratio. Pietatem) Non tantum doctrinam ipsam de side complectitur hoc loco vox ἀσεβάας: sed i- psius doctrinae praestationem et praxin, quae in vita, conuersatione, charitate, cultu Dei, caete- risque similibus aliis rebus posita est, vt Paulus ipse inf.vers.12. interpretatur. Id quod ex sequen- tis versiculi ἀντεθέσαι manifestissime apparet. Ex hoc loco facile colligi potest, Christianae Theologiae cognitionisque veri Dei verum finem non in mera contemplatione et θανρεία constitui a Paulo: sed in vita et actione ipsa, per quam ta- les efficimur, quales nos esse oportere doctrina ipsa iubet, atque ipso Christianorum nomine pro- fitemur. Vnde fit, vt omnino Monachismus ille Papisticus subruatur, qui Christianae religionis professionem in nuda rerum contemplatione, et doctrinae meditatione collocat et deffinit, si mo- do veram Christi doctrinam haberent tamen. 8 Nam corporalis exercitatio paululum habet vtilitatis: at pietas ad omnia vtilis est, vt quae promissionem habeat vitae praesentis ac futurae. Τeποφορὰ est. Occurrit enim tueltae obiectioni. quae ex corrupto iam hominum iudicio fieri po- tuit: vel quam in posterum fieri debere sanctus Apostolus et spiritu prophetico plenus iam tum praeuidit. Cum ergo mirum in modum semper homines ad externas illas actiones quae speciem pietatis prae se ferre videntur, stupeant, mirum e- tat Paulum vnius tantum pietatis meminisse, in  \pend
\section*{CAPVT  IIII. }
\marginpar{[ p.207 ]}\pstart qua nos exerceri, et operam ponere vellet: non etiam externarum vllarum caeremoniarum, quas tantopere homines tamen laudant et admiran- tur. Respond. Paulus, illam vnam pietatem vtilem esse tum in sese, tum aliarum exercitationum et operum, quae homines consectantur, ratione et comparatione. Apparet vero tum ex hoc loco, tum etiam ex epistola ad Coloss.2.iam aetate Pauli plus satis ex- ternae illi vitae rationi et austeritati homines etiam, qui Christum profitebantur, tribuisse, illique ad- dictos fuisse. Quod ex eo profectum videri po- test, primum quod fermentum illud Pharisaeo- rum, quod tamen toties Christus reprehenderat, iam inualuerat apud plerosque, quemadmodum ex quorundam animis falsa illa de rituum Papi- sticorum, et habitus monachalis sanctitate per- suasio euelli hodie non potest Matth.23. Deinde haec haereseon et superstitionum semina, quae po- stea creuerunt, et Papismum et Monachismum penitus, vti foetus infelicissimos, pepererunt, a Sathana iaciebantur, et serebantur in animis Chri- stianorum hominum, id est, mysterium iniquita- tis iam agebatur vt docet Paulus 2. Thess.2. Deni- que quod humanum ingenium vti suapte natura vanum est ita vanarum rerum amans est et auidum, quales sunt omnes illae externae cultus Dei vel vi- tae sanctitatis species. Itaque mature tantae huic verae fidei et religionis pesti occurrere voluit Dei spiritus: sed propterea tamen non est excisa tanti, tamque noxii mali radix: sed statim in Eccle- sia pullulauit, primum per Nicolaitas, et Ebio- naeos: deinde per Gnosticos, denique per Mona-  \pend
\section*{AD I. PAVL. AD TIM. }
\marginpar{[ p.208 ]}\pstart chos, per quos tandem vis et authoritas huiut tam sacrosancti Pauli praecepti, non tantum elusa, sed abrogata et prorsus antiquata est. Quaesitum autem est, Quid corporalis exerci- tationis nomine complectatur Paulus. Putat Chrysostom. corporis agitarionem, quae sanitati conferat, significari, qualis est venatio, ludus, fos- sio, aut quae exercitamenta iubentur a Medicis ad superfluos corporis humores consumendos fieri. Rationem addit ille, quod corporis exerci- tationem cum ea, quae animi est, et spiritualis, confert, vt hanc vtilem: illam nullius momenti homini Christiano esse doceat. falsa est haec in terpretatio. Nam non agit medicum Paulus, sed Theologum: Et vt supra cum de cibis agebat, de iis eatenus disputauit, quatenus ad conscientiae tranquillitatem pertinere censebantur, non au- tem ad corporis valetudinem:sic hoc loco de ex- ercitatione corporis disserens eam intelligit, in qua cultus Dei et nostrae sanctitatis pars quaedam constituitur, quoniam in rebus externis eam homines collocabant. Est igitur exercitatio ista, de qua nunc agit Paulus, alia, quam censet Chrysostomus, et est ea, per quam carnis mortificatio, vitae sanctitas, et externa quaedam pietatis cultusque Dei professio demonstrari videtur inter homines, qualis quae in ieiuniis et afflictione magna corporis versatur. Haec tamen dicitur per contemptum σωματικὴ, ne per eam animus vere reformari censeatur. aut quisquam reipsa propterea religiosior sapien- riorque fieri iudicetur: cum tota vis et effectus istarum rerum circa corporis tantum maceratio-  \pend
\section*{CAPVT  IIII. }
\marginpar{[ p.209 ]}\pstart nem afflictationem et vexationem cernatur. Sed et σωματικὴ quoque nominatur a subiecto. Est enim corpus, quod his modis atteritur, affli- gitur, et has exrernas actiones agit: quanquam ratione finis, quem spectant qui haec taedia per- ferunt, dici quoque potest haec exercitatio spiri- tualis, quod spiritualis vitae gratia fit. Haec igitur omnia externa sibi imponunt ho- mines, et quae vocantur nunc regulae ordinis suscepti et professi, eas appellarunt Patres Α’σκησν. Vnde veteres Monachi et Μονάζοντες et Α’ταηταὲ dicti sunt: Paulus autem nominat γνιναςίαν, quod Α'σκνσις magis quiddam spirituale significare vi- detur : γυuνοCα vero quandam tantum corpo- ris vexationem ac agitationem, quae non tam commendat monasticum statum, quam illi con- mendatum esse inter homines voluefunt. Sed praestabat vocem Spiritus sancti retinere. In his autem externis rebus praescriptis obser- uandis versatur tota Monachismi essentia, quam vocant. Nam in tribus hisce potissimum posita dicitur, in voto Obedientiae, Paupertatis, et Con- tinentiae, tum matrimonii, tum certorum quo- rundam ciborum. Sed et illud quoque notandum est, Paulum hic non loqui de superstitiosa earum rerum ob- seruatione, qualis fit in Papatu, et praesertim a Monachis (eam enim prorsus damnasset Paulus, est enim superstitiosa, cum pars salutis nostrae aliqua in iis rebus statuitur vti fit a Monachis) sed de rerum indifferentium discrimine constituto, et in iis obseruatione quadam carnis conterendae gratia, ad tempus indicta, qualis a piis viris pro  \pend
\section*{AD I. PAVL. AD TIM. }
\marginpar{[ p.210 ]}\pstart varia rerum, temporum, locorum, circunstantia fieri et suscipi potest. Hanc tamen totam vitae rationem et ob- seruationem, etiam ad tempus factam, dicit Pau- lus esse minimi momenti. At qui Monachi in iis ipsis rebus vitam Angelicam sitam esse volunt, et proram et puppim salutis et sanctitatis collocant. Neque tamen prorsus tales exercitationes damnat Paulus. Inter eas enim est ieiunium, et huiusmodi seruitus et maceratio corporis, qua- lem se etiam obseruasse ad carnem subigendam scribit ipse 1, Corinth.9. vers.27. et Psal.35.vers.13. Similis locus est Matth. 9.vers.14, et 15.vers.2. Romano. 14. vers.17.1. Corinth.7.vers.5. Dissimi- lis autem, primum, quia ait Paulus Ephes.5v.29. fouendam esse carnem: at qui istas rigidas exer- citationes obseruat, vexat sese, suumque corpus. Ergo videtur inhumanus et saeuus, vt hae non tan- tum parum vtiles dici debeant, sed sint prorsus damnandae. Respond. Cum moderatione istas esse a nobis suscipiendas, et ad tempus tantum: non autem cum superstitione, vel in perpetuum, vt docet Paulus 1. Corinth.7.vers.5. Nam omnes illae austerae, et seuerae Monachorum regulae, ab iis etiam cum periculo vitae et valetudinis de- trimento Obseruatae, fuerunt superstitiones merae et Diaboli praestigiae, qualia quaterduana ieiunia, perpetuae humi cubationes, stationes continuae, vt sibi sedere non permitterent: et reliquae huius- modi saeuae praeceptiones vel potius crudelitates in ipsam humanam naturam. Obstat etiam quod ait Paulus Rom.14. vers.17 Regnum Dei non esse in esca et in potu. Ergo fru-  \pend
\section*{CAPVT  IIII. }
\marginpar{[ p.211 ]}\pstart stra suscipiuntur ista vel ad breuissimum tempus. Respond. Cum ista sine causa et temere fiunt, frustrâ reuera suscipi a nobis. Cum autem ad car- nem lasciuientem domandam, vel vt paratiores promptioresque ad Deum precandum simus, non frustra: quanquam in iis rebus tamen regnum Dei non est positum, sed in vera fide, regeneratione, spe, et obsequio eo, quod mandatis Dei praesta- tur. Pietas autem ad omnia) Α'ντίθεCις est, ex qua su- perior sententia illustratur. Pietatem autem vo- cat verum Dei cultum, qui ei ex ipsius verbo ex- hibetur. Primum autem hic fidem continet, dein- de charitatem: deinde cordis sanctitatem et pu- ritatem, quae externis actionibus se patefacit. Haec vna res vtilissima est, et inf. capx̃. vers.6. magnus quaestus appellatur. Caeterum exemplo Christi, qui nihil praeter caeterorum hominum morem habuit: et Ioannis Baptistae, qui istas exercitatio- nes retinuit, nec propterea tamen Christo me- lior fuit, aut sanctior:imo vero nec cum eo vlla in re conferendus, hoc Pauli dictum perspicue con- firmatur. Fuit enim sanctior illa Christi vita con- munis, quam austera et rigida vita, quam tenuit Ioannes. Huius autem suae definitionis et laudis de pie- tate argumentum Paulus profert ab effectu, vel signo, quod pietas vera sine his exercitationibus habet promiffiones, nimirum vitae, tum Praesentis, tum futurae. Magna certe laus verae religionis. Sed cur vitae praesentis promissiones habeat pietas, ra- tio est, quod Deus est suorum in vniuersum et per omne tempus Pater et ουτὴς, non tantum  \pend
\section*{AD I. PAVL. AD TIM. }
\marginpar{[ p.212 ]}\pstart post hanc vitam: sed etiam in hac ipsa vita. Bern sermo.4. in Psal. Qui habitat. Credo in his scapu- lis geminam Dei promissionem intelligendam, vitae scilicet eius, quae nunc est: pariter et futurae. Si enim solum promitteret regnum, et in itine- re deesset viaticum, omnino conquererentur ho- mines et responderent: Magnum quidem est, quod promittitur, sed illuc perueniendi facultas nulla datur. Vnde omnem excusationem Deus ade- mit. Deus igitur iam se eorum Deum ex hac vi- ta esse ostendit, et iis benefacit Psal.36. Incipit e- nim Dei promissio impleri in iis, etiam in hac vita. Obstat quod fides res spirituales, et eas etiam absentes spectat, non autem praesentes. Neque proprie corpus: sed potius beat et spectat animum. Bona enim spiritualia pollicetur Deus potius, quam temporalia Hebr.II. vers.1. Respond. Bo- norum quidem spiritualium promissionem po- tissimum et praecipue habent et spectant fides et pietas: sed tamen eriam de suis curam habet in hoc mundo Deus, et illis bona terfena, quantum satis esse perspicit, largitur. Nan ea a se peti et vult et praecipit in oratione Domimca. Sunt autem quaecunque a Deo pii obtinemus ex mera ipsius bonitate, et ea gratuita, non ex me- Fito. Ilue bona terrena, siue coelestia spectentur, vt docet perspicue hic locus, qui ea omnia Dei promissiones appellat. Admonet hic locus, vt de Monachis aliquid di- camus, quia hi Α'σκηταὶ primum sunt appellati: ve- luti a Philone Iudaeo, si modo de Monachis Chri- stianis potius, quam de Essaenis et Iudaeorum Mo- nachis agit Philo:deinde a Basilio, et aliis, qui e-  \pend
\section*{CAPVT  IIII. }
\marginpar{[ p.213 ]}\pstart tiam Monasticas regulas A'Cηητηρίων nomine in- scripserunt. Μονάζον τες et Μοναχοι appellati sunt, quod vitae solitariae genus elegissent, qui hoc modo viuere instituissent. Μοναχὸν autem putant Canonistae ex can. Placuit S Agnoscat 16. quaest.1. dictum com- posita voce, ἀπὸ το͂ μόνον et ἄχος. Quae sane vera est hoc tempore Monachorum etymologia. Sunt e- nim illi re vera vnica pestis mundi, vt et ex veteri quoque Epigrammate iampridem intelligeba- mus. Sed falluntur et Papae et Papistae et Mona- chi in nominis sui etymo. Est enim syllaba xos ἐπενθέ βς tantum, ad vocem μόνος, a qua fit deriua- tio vocis huius, Μοναχὸς. Vnde et Moναχᾶ et Moναχο et Μοναχῶς vocabula Graecis scriptoribus etiam ante monachorum exortum defunctis vsitatissima. Basil de Trinitate agens quanque personam Μοναχῶς ἐκφωνεῖθς, id est se orsim pronuntiari dixit. Latine ipsi se Religiosos vocarunt, quasi reliquus popu- lus Christianus sit irreligiosus. Item et Canoni- cos quoque et Regulares, et aliis nominibus ad sui commendationem. Sed et Mandritae etiam sunt appellati a cauernis, quas isti Troglodytae subibant, et vbi latebant. Monachus quid sit nondum perspicuede fini- ri potuit, etsi variae a variis descriptiones tentatae sunt. Neque enim ex vestitu et professionis ve- stimento definiri potest, quia, vt vulgo dicitur, ve- stitus non facit Monachum: et tale est monstrum Monachus, vt vna forma non constet. Tandem definitum est, esse eum Monachum, qui haec tria vota solenniter suscepisset seruanda, Paupertatem Continentiam coniugii, et Obedientiam vni cui  \pend
\section*{AD I. PAVL. AD TIM. }
\marginpar{[ p.214 ]}\pstart dam certo praeposito, et in eadem vitae regula degenti. Viuendi quidem instituta et regulas posse esse in quoque ordine Monachorum diuersas, haec ta- men tria vota esse ipsius vitae Monasticae essen- tialia, et ab ea semper indiuulsa. Quod Thomas Aquinas in 2 2ae et in opere Contra impugnant. Religion. prolixe docet. Nam qui definiunt Mo- nachum esse militem Christi: vel agricolam ter- rae: vel Angelicum hominem, non modo rem non ex- plicant: sed quod experientia huius aetatis demon- strat, mentiuntur, cum nullus hodie Monachus sit huiusmodi. Vtcunque maxime superstitiosum vitae genus est, et a Turcis etiam hodie in blas- phema ipsorum doctrina semper retentum, quia fuco faciendo hoc genus hominum est aptissimum. A Iudaeorum Essonis, vt multi existimant, primum introductum fuit, et institutum: et inde postea ad Christianos translatum a superstitiosis homi- nibus primum Ægyptiis, deinde Syriis: postea Asiaticis, demum Graecis et Europaeis, ne vlla pars orbis tanti mali expers esset, et immunis. Certe post Christum natum Anno 300. maxi- me in Ægypto commendari coepit. Fuit enim ea regio, vti Arrianismi, et Meletiasmi, maximarum narcicomita etia Monachismi satrix et ferax aut siue propter solitudines, quas hoc hominum genus ferum prorsus et agreste captat: siue propter inna- tam illi genti superstitionem praeter caeteras mum- di nationes. In occidente hoc hominum genus fuit posterius et notum et receptum, et demum sub Vitaliano Pontifice Roman. coeperunt Mona- chorum coenobia erigi. Sed tamen, vt externis  \pend
\section*{CAPVT  IIII. }
\marginpar{[ p.215 ]}\pstart rebus homines fere faciliusque capiuntur, quam vera sanctitate: sic hoc nouo hominum genere exorto et terris monstrato, omnes miro stupore affici, nouum hoc vitae genus in coelum tollere, et cum ad scripturae regulam non examinarent, (quod tamen Basil. Magnus vitae Monasticae prae- co fieri vult) solum beatum praedicare coeperunt. Hinc laudes horum hominum immensae extiterunt, et hyperbolice decantatae, etiam a priscis scri- ptoribus quibusdam, qui Basilium sunt sequuti- Nam Iustini Martyris, Tertulliani, Cypriani, Ire- naei, Ignatii seculo hoc monstrum hominum non- dum natum in orbe Christiano fuerat, quicquid Eusebius ex Philone narret: cum Philo non de Christianis, sed de Essaenis Iudaeis loquatur, quae res non fuit, vt aliae multae, ab Eusebio obseruata, quemadmodum praeclare Chemnitius in Exami- ne Tridentini concilii probat, quicquid Diony- sius ille consecrationis Monachorum in Hierar- chia meminerit. Quidam hoc hominum genus ex persequutio- nibus, iis praesertim, quae saeuissimae contigerunt sub Maximino, et postea Diocletiano, exortum censent, quod cogerentur pii homines in vastas solitudines confugere, vt persequutorum imma- nitatem euitarent. Et certe haec causa erat probabilis nisi ἔξ ἀιρέβως potius animi, quam ex vlla neces- sitate appareret homines hoc sibi iugum impo- suisse, et eo tempore, quo nullae erant persequu- tiones: sed summa pace fruebatur Ecclesia Dein- de nisi ipsa historia Ecclesiastica eo tempore, quo Monachismus adhuc pene recens erat, scri- p ta, aliud diceret, quae refert hoc institutum vitae  \pend
\section*{AD I. PAVL. AD TIM. }
\marginpar{[ p.216 ]}\pstart ad quendam Ammonium virum Ægyptium et con- iugem. Eremitarum enim author fuit Salas, vt ait Nicephorus, quanquam Isidorus in lib.  de Offi- ciis non tantum a Ioanne Baptista: sed etiam ab vsque Elia Propheta huius vitae primordia repe- tit, tam est bonus Monachorum patronus. Cer- tatim igitur omnes ingenii sui aciem in hoc vitae genus extollendum contulerunt. Alii enim, vt Ambrosius epist.82. Monasteria scholas et offici- nas virtutum appellarunt, Chrysostom.dixit, Vt in coelo sunt astra:sic in terra esse ordines diuer- sos Monachorum. Alius dixit, vt e paradiso terre- stri quatuor flumina fluebant, sic ex Ecclesia quatuor Monachorum ordines extitisse. Alius, ve- lut Hieronym. et Bernard. susceptionem Mona- chismi vocat seçundum Baptismum, per quem remissio omnium peccatorum obtineatur. Alius Monachismum esse dixit Angelicam vitam, vt Thomistae omnes, quo sibi maiores in vitae suae errore delicias faciant. Nos et ex superstitione primum inuentum esse hoc vitae institutum de- fendimus: et mere esse Diabolicum, cum pars quaedam cultus Dei, et vitae sanctitatis in hoc ex- terno rerumgenere posita sit, et praeter Dei verbum salutaris, et beata vita definita. Etsi enim admittimus minus fuisse olim cor- ruptos Monachorum mores, quam nunc sunt, tamen haec ipsa vitae institutio fuit a Christi vita prorsus aliena, et superstitionis plena, et nulla a- lia de causa, nisi ex ἐθελοθρηβεεία et idololatrica rerum externarum admiratione inuenta, et fu- scepta. Admitto igitur quod de Monachis suae ae- tatis scribit Chrysost. Homill. 14. in Timoth.5.  \pend
\section*{CAPVT  IIII. }
\marginpar{[ p.217 ]}\pstart his verbis, quae recito vt appareat, quantum no- stri Monachi ab illis veteribus differant. Verae, ait ille, domus luctus sunt Monasteria, vbi cinis atque cilicium: vbi solitudo, vbi risus nullus, nul- lus negotiorum secularium strepitus, vbi ieiunia, vbi terrenorum duritia lectulorum, vbi munda sunt omnia. Nullus ibi nidoribus locus, nulli san- guinis riui, etc. quae quantum ab Monastica vita nostri seculi differunt vident omnes. Et Bernar- dus in Apolo. ad Guill. Abbatem verissime mul- ta de luxu Monachorum et ἀσωτία dixit, quae ho- die, vti alia vitia omnia, in iis maior est et intole- rabilior, quam vnquam: imo vero prorsus insa- nabilis et inueterata. Sed et hoc vitae genus viris quibusdam bonis statim displicuit, et postea Sy- nodis vti Gangrens. et Matisconensi, improbatum est et velut superstitiosum damnatum. Modum vero non habuit, cum quotidie noui ritus, inauditaeque superstitiones inducantur, vt nouus Monachorum ordo excudatur, veluti Stilidae olim: nunc Iesuite, adeo vt non si mihi centum sint linguae, sint ora que centum, ferrea vox, possim eorum varias com- prehendere formas. Sed et illud immane, quod inter eos ipsos huiusmodi genera reperiuntur, quae humanam exuant naturam prorsus et in bel- luinam mutent qualia sunt haec Armenta Bασκοὶ et ἀπαθεῖς vt est apud Nicephorum lib.  14. cap.50. Tamen nunquam istae laudes tantum obtinue- runt, vt propterea reliqua Christianorum gene- ra, quae nihil cum ista ἀυθαδεία et vitae genere con- mune habent, minus ad Christum pertinere cen- serentur. Dixit Paphnutius Episcopus in Ni- caena Synodo, in omni ordine, et non in solo Mo-  \pend
\section*{AD I. PAVL. AD TIM. }
\marginpar{[ p.218 ]}\pstart nachatu, posse homines placere Deo. Bernard.in in sermo. in Psal. Audiam:tam coniugatos, quam Monachos esse saluandos affirmat, et idem in ser- mone 66.in Canticum cantic. Eorum genera fuerunt primum duo tantum, Eremitarum, et Coenobitarum, vt ait Sozomen. libus 3.cap.16. Quae multiplicata postea, vt docet Euagrius lib. 1 cap.21. in quatuor excreuerunt, vti etiam ex Cassiano apparet Collat. 18.cap.4. Alii enim fuerunt Coenobitae, alii Anachoretae, alii Sarabaitae, alii Rhomoboitae, vt etiam docet Hie- rony. Coenobitae qui sub vno capite in coetu vi- uunt: Anachoretae qui in coenobiis primum insti- tuti, seorsim postea vel singuli, vel bini in solitu- dines secedunt, ibique habitant, non in coetu. Ho- rum author fuisse dicitur ab aliis Paulus quidam, ab aliis Antonius. Quod vitae genus plane ferum est et agreste et ab ipsa hominis natura et defini- tione dissentaneum, qua dicitur homo esse Cῶον πολιτκὸν, et praeclare M. Tullius docet ne si qui- dem alicui omnia virgulta diuina suppeterent, o- ptaturum tamen eum vt viueret in solitudine et procul a reliquis mortalibus more ferarum, quod isti Anachoretae faciunt. Sarabaitae in cellulis ha- bitant, et Ana choretis sunt peiores. homines vagi de auari quiin Ægypto praesertim fuerunt. Rho- moboitae qui terni habitant in cellulis suo arbi- tratu viuentes, effrencs, et nulli capiti parentes. Hi pessimi omnium vt in epist. ad Eustoch. testa- tur Hierony. qui eos vidit. Stilidae siue Columna- rii a Simeone instituti, vt ait Euagrius, qui in co- Iumnis habitant, vt nec in terra, nec in coelo vi- uere videantur. Socrat. lib. 4.cap.23 4.haec gene-  \pend
\section*{CAPVT  IIII. }
\marginpar{[ p.219 ]}\pstart ra enumerat Gnosticos, Coenobitas, Practicos, Eremitas, vt varium fuisse hoc totum genus facile appareat. Hi etiam ordines et distinctiones poste- rioribus temporibus sunt mutatae, et aliis nomini- bus appellatae. His enim successerunt quatuor alii Monachorum ordines, nempe Benedictinorum, Augustinianorum, Carmelitarum, Praemonstrensium, qui postea se in infinitas propagines diffuderunt, Mendicantium (quos verius alii fratres Manducan- tes vocant) et non Mendicantium, et variae distin- ctiones inter eos inductae a colore vestis, a gene- re cibi, a forma calceorum, ab amplitudine mani- carum, a forma pallii, et huiusmodi aliis nugis, quasi eae res sint porta coeli et salutis nostrae cau- sae. Antea tamen tres tantum ordines Monacho- rum enumerabantur scilicet, Sancti Basilii, San- cti Augustini, et Sancti Benedicti, vt est in cano. Pernitiosam 18. quaest 2. Anno vero Domini 500. ista non apum, sed fucorum potius examina varie et sparsa sunt et multiplicata per totum terrarum orbem, impri- mis autem per Palaestinam, Cappadociam, Asiam, Graeciam, et Ægyptum, tanquam Apocalypseos locustae, sese diffuderunt, et praecipue in Sceti, siue Staitide Ægypti Nomo siue praefectura, vt Sozo- menus lib. 6.cap,7. tradit. Monasteria vero in vrbibus etiam furiose eri- gi coeperunt Anno Domini 700. et postea Velut autem scholae quaedam aedificari primum coepe- runt, et velut officinae literarum in summa totius mundi barbarie: quae per Gothos, Vandalos, Hunnos, et Erulos inducta fuerat. Vnde Iuo scri- bit in Iustiniano Florum quendam Gallorum  \pend
\section*{AD I. PAVL. AD TIM. }
\marginpar{[ p.220 ]}\pstart Francorum procerem prope Aurelianensem ciui- tatem coenobium sancti Benedicti condidisse, v- bi Maurus ipsius Benedicti discipulus, vt ait Sa- bell. Ennead. 8 lib 2. docuerit, et Floriacenum Mo- nasterium adhuc hodie appellatur. Certe ab hoc vitae instituto prorsus nunc nostri Monachi de- generarunt, quorum coenobia, non sunt literarum officinae, sed harae porcorum, et pascua hominum cantillando viuentium, more cicadarum. Ta- men postea vsurpatum est, vt poenae loco in Mo- nasteria detruderentur homines inertes, scelera- ti et immorigeri, et tonderentur, vt ex historiis praesertim Francogallorum apparet, et Grego- rio Turonensi. Quis autem vsus fuerit istorum hominum varie quaesitum est, quae item regulae. Ac regulae vt Nicephorus ait lib. 9. cap. 14. hae primae erant, vt omnia essent libera, votum au- tem perpetuum ista professio non haberet: adeo vt nec ciborum, nec ieiuniorum certa tempora et indicta essent, sed quantum vellent ederent, et biberent, et operarentur: item quando vellent, ieiunarent. Atque ita Pachoimus primas regulas instituerat. Sequuti sunt alii qui restrinxerunt illam libertatem. Ac primum in vestitu: post autem in cibis. Qui primus autem instituta ista scrupulosius et mi- nutius vel explicasse vel tradidisse censetur, est Ba- silius, siuecis est, qui Magnus dicitur: siue alius, eo- que recentior. Ergo quoad habitum, cucullis sunt vsi sumpto more ab Ægyptiis Monachis, vt Cas- sia. scribit. Melotis item.i. ouilibus vestimentis, et vilibus pannis, Colobiis lineis ad imam vsque cubiti partem pertingentibus, vt expeditas ad  \pend
\section*{CAPVT  IIII. }
\marginpar{[ p.221 ]}\pstart operandum manus haberent, Cilicio, et Succin- ctoriis siue subligaculis, et pelliceis, quae lineo co- lobio operiebantur, vnde superpellicea dicta, Gal. des superplis. Colores in habitu varios habuere, et gestauere: calceamentorum nullus inter cos, vt- pote in calidis regionibus primum habitantes, fuit vsus, vt Cassianus docet. Quoad ipsam autem corporis eorum formam, comas alebant et bar- bam resecabant, contra quam tamen Epiph.hae- res. 8o. decorum viris Christianis esse censet. Bar- bam enim μορφην ἀνδεος vocat, et in concilio Car- thaginensi clerici barbas radere veriti sunt: sed non sunt Monachi de numero clericorum. Ac- cincti semper lumbis incedebant, et pecunias cu- mulare vetabantur. Nuptiis item et festiuitati- bus interesse prohibebantur. Operari autem iu- bebantur quotidie, id est, in agro aliquid fode- re, aut arare, aut ferre, vt victus vel ipsis vel toti sodalitio in quo viuebant, superesse posset. Praecepta quoque in ciborum delectu et ge- nere contra Pauli regulam illis instituta sunt. Alii enim herbas tantum, alii oua, alii pisces: alii o- mnia comedere permittuntur. Ieiunia tamen quaedam illis ad certam mensuram et dies certos praescripta fuerunt: et mirum silentium, quod non est hodie in nostris Monachis, cum nusquam inuidia et murmur maius frequentiusque locum habeat, Denique nulli Monacho esse sine cella licuit, ne velut piscis siue aqua viuere existimare- tur. Hae sunt nugae, sed potius hae sunt superstitio- nes et idololatricae caeremoniae, in quibus hoc vi- tae institutum, et Christiana (si Deo placet) sancti- tas fuit tandem, abolitis verae sanctimoniae prae- ceptis, collocata.  \pend
\section*{AD I. PAVL. AD TIM. }
\marginpar{[ p.222 ]}\pstart Quasi illa in his externis etάδιαφόρσις rebus insit: quae in mentis renouatione, veteris hominis de- positione, et noui susceptione posita est, in fide Christi, in charitate proximi, in omni internae et externae immunditiae expurgatione. Sed ita ho- mines sibi praestigias istas fecerunt, vt aliud quid esse sanctum, quam istam viuendi rationem non arbitrarentur. Denique isti homines et seditionum in Ecclesia magnarum authores fuerunt, vt Eua- grius lib. 3.cap 32. totaque historia Ecclesiastica docet: et pessimorum errorum tum ipsi inuento- res, tum propagatores, veluti Messalianorum, Antropomorphitarum, Pelagianorum, Iudaeo- rum, Abelonitarum, Samaritanorum. Quod non tantum testatur Theodorit. lib.  1.cap.1 Histor. sed ipsa rerum veritas et experientia: cum nulli alii sint hodie Antichristi, totiusque Papisticae perfi- diae maiores patroni, et fulcimenta, quam tota i- sta Monachorum factio et coitio. Sed quaesitum est, cuius in Ecclesia vsus fuerint isti homines. Respond. Certe nullius, cum nec Presbyteri, nec Diaconi, nec Episcopi essent. Itaque de se vere illi istud Horatianum dicere possunt. Nos numeri sumus. et fruges cousumere nati. Proci Penelopes. nebulones, Alcinoique. monachi officia populo celebrare non possunt. Monachus officium plangentis habet, non docen- tis. Ergo nullum munus Ecclesiasticum gerunt, sed nec gerere, quatenus Monachi sunt, debent aut possunt, non concionari apud populum: non Sacramenta administrare. Claret enim et certum est, ait Bernardus, quod publice praedicare Mo- nacho non conuenit, nec Sacramenta admini-  \pend
\section*{CAPVT  IIII. }
\marginpar{[ p.223 ]}\pstart strare, quia ne clericorum quidem loco et nume- ro censentur. Nullam enim in Ecclesia vocationem legitimam habent Monachi, vt est in cano. Alia est causa et cano. Nemo potest 16. quaest.1. et a- pud eundem Bernardum epist.42. labor et late- brae, ait, et voluntaria paupertas, haec sunt Mona- chorum insignia, haec vitam solent nobilitare Mo- nasticam. At, Audite nostri seculi Monachi, vestri oculi omne sublime vident, vestri pedes omne forum circumeunt: vestrae linguae in omnibus au- diuntur conciliis: vestrae manus omne alienum diripiunt patrimonium. Quod si Monachorum sceleratissimam vitam pingere aut narrare vellem: nulla mihi Ilias satis esset. Sed quaero, cuius pri- mum rei et vtilitatis gratia hoc vitae et hominum genus institutum, inuentum, et productum fuit? Respondeo ego quidem, Nullius. Nam in canon. Generaliter S Hoc idem 16. quaest. 1. Abbatem quemlibet et cuiuslibet Monachorum coetus prae- fectum simplici ostiario Ecclesiae postponendum esse et subiiciondum aperte responsum est, dura- uitque vsque ad Syricii Papae spurcissimi tempo- ra, vt Monachi Monachi simpliciter manerent, clerici autem nullo modo fierent, non praedicarent, non celebrarent officia in Ecclesia. Postea vero paulatim gens ista, instar locustarum, e suis deser- tis, et e suo coeno velut ranae emergere coeperunt, et tandem effecerunt, vt speciali priuilegio (sed toti ordini a Bonifacio quarto, vt alii volunt: vt autem alii, ab Innocentio secundo concesso) qui- libet Monachus Sacerdos et sacrificulus fieri possit et Doctor, et praedicator, et Sacramentorum administrator de consensu Abbatis, et Episco-  \pend
\section*{AD I. PAVL. AD TIM. }
\marginpar{[ p.224 ]}\pstart \phantomsection
\addcontentsline{toc}{subsection}{\textit{9 Fidus est hic sermo, et dignus qui o- mnibus modis recipiatur.}}
\subsection*{\textit{9 Fidus est hic sermo, et dignus qui o- mnibus modis recipiatur.}}pi, et curionis, in cuius paroecia ista agit. Ex quo postea factum est, vt in maximas Ecclesiae Papisticae dignitates, veluti Episcopatus, Archie- piscopatus, Papatum denique ipsum, isti cra- brones et locustae inuolarent, et iam ipsius Papi- smi solae columnae restarent. Quo hominum ge- nere nihil habet vniuersus orbis pestilentiosius, superstitiosius, impudentius, audacius, crudelius, ambitiosius, aut denique magis auarum et rapax. Hoc autem omne vitae institutum quam procul absit a Pauli praecepto, et ἀσκήσει, cuius hic mem- tionem facit, vident omnes: imo vero vt nihil quicquam cum Dei verbo consentaneum habeat sed potius mortis Christi beneficium, et nostrae coram Deo iustificationis causam (quae in vno Christi sanguine est) ad cibos et colores vestium merasque superstitiones et blasphemias transfe- rant, in hac Euangelii luce intelligunt omnes, vt mihi non duxerim necesse diutius ista refellere, quae vel indicasse satis est ad refutasse. Redi- mus igitur ad Paulum nostrum. 9 Fidus est hic sermo, et dignus qui o- mnibus modis recipiatur. Commendatio superioris doctrinae ab vtilita- te et illius praestantia. Tum enim fidelis est haec do- ctrina: tum vero omnibus vtilis, qui illam audiunt, illique credunt. Praecautio quoque et praemunitio quaedam haec esse videtur aduersus varias animi du- bitationes, quae ex superiori doctrina solent oriri. Tantum. n. abest, vt fideles felices iudicentur, vt et miseri et probrosi passim et vulgo esse censean- tur. Quod iudicium falsum esse pronuntiat hoc loco  \pend
\section*{CAPVT  IIII. }
\marginpar{[ p.228 ]}\pstart \phantomsection
\addcontentsline{toc}{subsection}{\textit{10 Nam idcirco etiam fatigamur, et probris afficimur, quod speremus in Deo vi- uo, qui est conseruator omnium hominum, maxime vero fidelium.}}
\subsection*{\textit{10 Nam idcirco etiam fatigamur, et probris afficimur, quod speremus in Deo vi- uo, qui est conseruator omnium hominum, maxime vero fidelium.}}Spiritus sanctus et seq. versiculo. Quae sit autem sententia huius orationis supra cap.1.V.15. fusius ex posuimus. 10 Nam idcirco etiam fatigamur, et probris afficimur, quod speremus in Deo vi- uo, qui est conseruator omnium hominum, maxime vero fidelium. Occupatio est, quae per digressionem iniecta est, ne superior illa pietatis commendatio falsa esse censeatur, propter varios rerum euentus et duram piorum sortem, quae in hoc mundo pror- sus misera esse apparet et afflicta, tum in corpore ipso, tum in honore. Respondet autem et negat eam sortem Paulus propterea esse miseram et infeli- cem existimandam, idque respond propter Cau- sam propter quam patiuntur, pii: et Exitum, qui cernitur, et inde speratur. Acfinis siue causa, propter quam pii affliguntur, praeclarissima est. Est enim pietas et ipsa doctrina coelestis, et spes in Deo viuo, non in idolis, qua nihil melius. Exitus etiam laetus et felix est. Id quod pro- bat argumento sumpto a minori ad maius. Deus enim, qui omnium hominum est conseruator, mul- to maxime suorum et fidelium custos est et libera- tor. Haec est summa totius uoctrinae I'auli. Dispu- tat Chrysost. atque ambigit, vtrum vox συτὴρ hoc loco referri ad vitam, et salutem aeternam debeat. Sed hic non potest. Neque enim omnes homi- nes vitam aeternam consequuntur, sed soli fide- les, quanquam tamen omnium est custos, conser- uator, et benefactor Deus, vt est Ps.36. quatenus nempe omnium hominum est creator et opifex.  \pend
\section*{AD I. PAVL. AD TIM. }
\marginpar{[ p.226 ]}\pstart \phantomsection
\addcontentsline{toc}{subsection}{\textit{11 Haec denuntia et doce.}}
\subsection*{\textit{11 Haec denuntia et doce.}}Spemautem coniungit cum doctrina Paulus, cum ait (quod speremus in Deo) quia haec cogni- tio, quae Per fidem et Euangelium a nobis habe- tur, non est euanida quaedam phantasia de Deo, sed est huiusmodi agnitio, quae spem, eamque cer- tam in Deo nobis afferat, et inserat. In quo differt a reliquo cognitionis genere. Nam aliae sunt tan- tum in animo et intellectu versantes cogitatio- nes, neque cor hominis mouent, afficiunt et mutant, haec vero etiam in corde ipso est certi- ssima fiducia, quod afficit et inclinat ad Deum amandum, eiusque promissiones certissime spe- randas et expectandas, vt est in cap. 5. Epistolae Pauli ad Romanos. Sed et vox haec speramus pro credimus posita videri etiam hoc loco potest. 11 Haec denuntia et doce. Commendatio superioris doctrinae ab vtilita- te, quae ex illius praedicatione colligitur. Est e- nim haec vtilis doctrina, cuius praedicatio est sic a Paulo praecepta, vt et saepius inculcetur, et Pro imperio et potestate ministerii Euangelici pro- ponatur, vti Dei ipsius vox. Ergo comparatione quadam superiorem doctrinam magni esse in Ecclesia momenti docet probatque. Nam etsi alia capita doctrinae doceri a pastore debeant: hoc tamen doctrinae caput, per quod falsus Dei cultus oppugnatur, inanes caeremoniae et super- stitiones euelluntur, saepissime et frequentissime repeti, doceri, ac inculcari debet. Atque hoc fa- ciendum esse affirmat Dei spiritus, id est, doctor ille, qui optime nouit, de quibus sedulo et saepius  \pend
\section*{CAPVT  IIII. }
\marginpar{[ p.227 ]}\pstart \phantomsection
\addcontentsline{toc}{subsection}{\textit{12 Nemo tuam iuuentutem despiciat, sed esto exemplar fidelium in sermone, in con- uersatione, in charitate, in spiritu, in fide, in puritate.}}
\subsection*{\textit{12 Nemo tuam iuuentutem despiciat, sed esto exemplar fidelium in sermone, in con- uersatione, in charitate, in spiritu, in fide, in puritate.}}edocendi et commonefaciendi sumus. Quo mi- nus nobis molestum aut fastidiosum esse opor- tet, si saepe saepius in Ecclesia hanc eandem do- ctrinam a pastoribus audiamus, quia maxime no- bis est salutaris. Paulus Philip.3.vers.1. dicit tutum et vrile esse, vt eadem nobis saepius repetantur. Idem docet et affirmat Petrus in 2. Epist. cap.1. vers.13. Vnde vae delicatis hominum auribus, qui abusus doctrinae et superstitiones Papisticas nun- quam a concionatoribus Euangelii taxari vellent. Denique ex hoc ipso apparet, eam maxime a pastoribus doctrinam proponendam, quae neces- saria est, et vtilis: non autem quae inutiles et vanas quaestiones habeat. Colligit vero ex hoc Pauli dicto Chryso. du- plex esse pastoris munus in proponendo Dei ver- bo pro rerum, quas tractat, et personarum, apud quas agit, ratione, nimirum vt Doceat, et Ea quae sunt obscura, et Eos, qui sunt obsequentiores. Item Denuntiet, propotestate et imperio sibi a Deo dato, Ea, quae omnes concedunt esse vera, Eis, qui sunt contumaciores et magis rebelles. Quod cer- te verum est. Debet enim pastor nonnunquam increpare et virga.i. seuetitate vti, vt ait Paulus 1. Corinth 4.vers.21. et 2.Timoth.4. 12 Nemo tuam iuuentutem despiciat, sed esto exemplar fidelium in sermone, in con- uersatione, in charitate, in spiritu, in fide, in puritate. Νεθέτηοις. Admonet enim qua ratione et gra- uitate superior doctrina sit a Timotheo propo-  \pend
\section*{AD I. PAVL. AD TIM. }
\marginpar{[ p.228 ]}\pstart nenda ac illi ipsi sit in omni ministerii Euangeli- ci quod obibat ratione conuersandum et agen- dum. Adeo vt etiam haec praecepta ad reliquos pastores pertineant, qui nobis in Timotheo, tan- quam in perfecti ministri idea, a Paulo depicti proponuntur. Est vero hic locus etiam tacita re- gressio ad superius institutam disputationem de officio veri et Euangelici pastoris, quam et hoc, et sequenti capite persequitur. Haec est igitur ra- tio ordinis, ne putemus hoc scriptum Pauli esse scopas dissolutas. Quod Athei obriciunt. Summa vero huius doctrinae eo pertinet, vt doceantur pastores superiorem doctrinam reuerenter et grauiter tractandam esse: et in reliquis ministerii et muneris suipartibus ita illis esse conuersandum, vt nemo eos, quantunuis aetate iuuenes, despi- ciat vel contemnat: sed sint totius gregis exen- plar in Verbis, et factis, et in Doctrina, et in vita, Videri autem potest non ad omnes in vniuersum verbi Dei ministros pertinere haec Pauli exhorta- tio: sed ad eos tantum, qui adhuc iuuenes sunt: neque dum sibi per aetatem eam authoritatem compararunt, per quam sint populo commenda- biles et venerandi, quales sunt adolescentes ad- huc pastores, et nondum prouectae aetatis. Sed cum etia fiilino lonorentur, qui, quanuis aetate grandae- ui, moribus tamen sunt iuuenes, certe ad senes etiam haec praecepta pertinent, ne hi ipsi in ea aetate pro- pter morum dissolutionem contemnantur, si idem agant, quod homines moribus iuuenes et proter- ui, et lasciui et leues. Quae certe vitia adolescentiam sequi et comitari solent. Dicit Paulus reorντος, id est, Iuuentutis. Differt  \pend
\section*{CAPVT  IIII. }
\marginpar{[ p.229 ]}\pstart haec aetas, a Pueritia, Adolescentia, Virili ma- turaque aetate, et Senectute. Est enim iuuentus post adolescentiam, et ante virilem aetatem velu- ti a decimo septimo anno aetatis ad vicesimum quintum, intra quos annos existimo tunc fuisse Timotheum: quemadmodum adolescentia est post pueritiam, et ante iuuentutem a septimo anno aetatis ad decimum septimum. Sic Pythagorici distin- xerunt vt apparet ex Laertio lib. 8.in vita Pytha. gorae etsi plures annos cuique aetati tribuit. Solon Atheniensis in epigrammate distinguit hominis totam vitam in septem aetates. Sed Ioannes in pri- ma Epistola cap.2.vers.13. et 14. tres tantum con- stituit. Nos nostram distributionem sequimur, ex qua cuius tunc aetatis fuerit Timotheus intelligi potest, et quantis in ea aetate Dei donis, et iis qui- dem summis et egregiis, ornatus fuerit. Id quod etiam apparet ex secunda Epistola ad eundem et ad Philip.2.vers.18. Remedium autem per subiectionem affert Paulus, per quod efficiatur, ne vlla pastoris aetas contemni vel possit, vel debeat. Est autem si illi se exemplar caeterorum fidelium praebeant in vi- ta et doctrina. Igitur ex hoc loco colligitur iuue- nes eligi posse in ministros, si iis Dei donis excel- lant, quae ad tantum munus requiruntur. Supra cap.3 vers.6.diximus, quae aetas olim cuique mu- neri Ecclesiastico exercendo praescripta fuerit a Synodis vt hic repetere nihil sit necesse. Sed apo- logo lepidissimo docet Æsopus non semper ca- nitiei et barbae inesse sapientiam. Describit autem fusius Paulus illa duo, quae diximus, id est, quibus in rebus pastores gregis ex  \pend
\section*{AD I. PAVL. AD TIM. }
\marginpar{[ p.250 ]}\pstart emplar, vt ipse cum Petro 1.epist. cap.5. loquitur esse debeant. Ac primum summatim quidem in- quit, ἐν λόγω id est, Verbis siue dictis, ἐν ἀνας ροφν id est, Conuersatione et factis, toti gregi praeluce- re pastorem oportere. Postea vero quasdam vir- tutes addit, ex quibus distinctius haec duo paran- tur et docentur: et tum vitae, tum ipsius linguae vera reformatio oritur, tanquam ex ipsis fonti- bus. Obiter dixerim hoc genus dicendi esse ἀσύνθε- τον, quod maiorem vim habet, et facit, vt singula diligentius perpendantur a nobis. Ergo inter eas virtutes praecipue commemorat Paulus, Cha- ritatem, cui subiicit Spiritum siue zelum. Cha- ritas autem hic tum ea, quae in Deum, tum in pro- ximum est, significatur, cui subseruire debet spi- ritus ipse, id est, feruor animi:atque illa etiam do- na, quae Dominus cuique fidelium donat et con- cedit, charitati ancillari oportet. Deinde addit, charitati suum fontem, nempe Fidem, cui subiicit Puritatem siue Castitatem. Ergo fides haec non tantum est doctrina, vel cognitio: sed etiam ipsa in Dei promissiones per Christum acquiescens assensio et voluntas, quae parit ex se tanquam v- berri mum fructum ἀγνείαν, id est, castitatem. Haec autem castitas non tantum abstinet alieno thoro, E libiume fasciua: sed omni vitae impuritate, quae Christianae sanctitati contraria est. Ex his igitur quatuor fontibus pene omnis vitae fanctitas re- formatio et exemplum nascitur. Ergo vanae illae Episcoporum commendationes ex insulis, lituo siue pedo chirothecis, annulis, cothurnis, fanda- liis, et caeteris huiusmodi nugis facessant, quae in Papatu, tamen tanquam certissima Episcopatus  \pend
\section*{CAPVT  IIII.. }
\marginpar{[ p.231 ]}\pstart \phantomsection
\addcontentsline{toc}{subsection}{\textit{13 Interim dum venio, attende lectio- ni, exhortationi doctrinae.}}
\subsection*{\textit{13 Interim dum venio, attende lectio- ni, exhortationi doctrinae.}}insignia, ornamenta, et testimonia dantur, et so- lenniter gestantur. 13 Interim dum venio, attende lectio- ni, exhortationi doctrinae. Α’κολύθησις est. Nam sequuntur variae exhor- tationes, quas muneris Ecclesiastici dignitati con- sentaneas esse vidit Paulus, et necessario adden- das. Ex his autem quantum sit hoc onus, quan- que sedulo capessendum et diligenter exercen- dum intelligere certe debemus, ne segnes so- cordesue ipso Episcopatus titulo, vel fastigio tur- gidi et insolentes, vel etiam reditu et bonis contenti torpeamus. Sunt enim hae cohortationes, tot Spi- ritus sancti aduersus pastores negligentes et mu- neris sui grauitatem non apprehendentes fulmi- na vel tonitrua. Neque enim tam Timothei gra- tia dicuntur ista, quam omnium pastorum, quo- rum verissimam et optimam ideam nobis in Ti- motheo format et pingit. Denique si tam ardens in Dei metu et tantus vir Timotheus iis stimulis eguit, quid nos? et quot quantisque exclamatio- nibus et puncturis sumus excitandi? Est autem hic exhortationis et admonitionis locus non vnus, sed quadruplex. Omnes quidem eo tendunt, vt sedulo et diligenter pastores suo muneri inten- dant et incumbant. quilibet tamen locus exhor- tationis habet quoddam peculiare et necessario considerandum. Prima igitur haec est exhortatio et admonitio, attendendum et assidue versandum Ministro Euan- gelico in Sacrae scripturae lectione. Vox enim, le- ctionis, generaliter hoc loco enuntiata ad Sacrae  \pend
\section*{AD I. PAVL. AD TIM. }
\marginpar{[ p.432 ]}\pstart scripturae tamen lectionem specialiter restringi debet, quia non simpliciter hoc requirit a pasto- ribus Paulus, vt aliquid legant quodcunque tandem sit, sed vt facrum Dei verbum euoluant. Hanc esse Pauli mentem apparet ex fine, propter quem attendendum lectioni docet, nimirum vt pastor Doceat gregem suum, et Exhortetur, quae sunt duae muneris Ecclesiastici partes praecipuae. Do- ctrina, tum in Verorum dogmatum traditione, et Falsorum refutatione posita est. Quorum neutrum nemo facere, nisi apprime et abunde verbo Dei instructus et diligenter in eo versatus potest, cum ex eo vno fonte et veritas Dei fit haurienda, et mendacium illis armis oppugnandum, illoque gladio iugulandum Actor. 18.vers.28. Exhortatio est in Bene iam currentibus con- firmandis, Improborum acri reprehensione: de- nique in frigidis excitandis. Illi autem stimuli:non nisi ex Dei promissionibus et minis, addi pos- sunt. Haec igitur omnia Dei verbo continentur, quod idcirco et versandum et meditandum est diligenter. Ergo non satis est Episcopos et pasto- res quaedam legere ( nam nec fabulas Poëtarum, nec vanas et inutiles Philosophorum scriptita- tiones terere debent, aut iis immorari) sed opor- tet eos Dei verbum scriptum assidue contrecta- re. Vt autem ad Pharisaeorum scripta et scholas non remittit suum Timotheum Paulus, neque nos etiam ad Scholasticorum et Sophistarum Sor- bonistarum Academias reuocandi sumus, vt veri Theologi euadamus: sed ad sacrum Dei ver- bum et ad eos scholas, vbi illud pure docetur. Sic Paulus 2. Timoth.4. vers.13. libros suos et  \pend
\section*{CAPVT  HIII }
\marginpar{[ p.233 ]}\pstart membranas, in quibus aliquid descripserat priua- tae memoriae causa, sibi Troade afferri vult Similis locus 2. Pet.1.V.19. Dissimilis Mat.1o.v19. Respo. Christum quidem nolle pastores esse in legendo Dei verbo negligentes: terreri tamen nos non vult vel conspectu vel doctrina vel fama nostrorum aduersariorum, cum ipse nobis sit suo spiritu po- tentissime affuturus, et quae ad conuincendos ve- ritatis Dei hostes necessaria sunt suggesturus. Addit Paulus, donec venio. Quae limitatio non excludit tamen reliquum tempus, quasi praesen- te Paulo Dei verbum contemnere Timotheus debuerit. Nec facit pastorum, et praeceptorum nostrorum magna et excellens doctrina, vt in Dei scriptura euoluenda fiamus desidiosiores: sed contra nos magis illorum egregiae dotes debent excitare, vt quae illi proponunt doctissime, ad ex- emplum Thessalonicensium, cum scripto Dei verbo conferamus, diligentiusque et lubentius a tantis viris ediscamus. Denique quod nullum tempus praescribit Pau- lus, ac ne longissimum quidem, post quod cessa- re debeat a studiis Timotheus, ostendit nos per totum vitae tempus proficere posse: tamen non facit, vt videatur inepta certorum annorum cer- tique spatii definitio, intra quod tum Theologiae: tum aliis artibus et facultatibus sit studendum, o- peraque ponenda, quanquam potius ex ipso cu- iusque profectu id diiudicandum est, qui apti sint ad ministerium vel gradum adipiscendum, quique plurimum profecerint, quam ex ipso temporis curriculo et spatio, quo versantur in Academiis. Quod autem quidam putant, et ex hoc loco col-  \pend
\section*{AD I. PAVL. AD TIM. }
\marginpar{[ p.234 ]}\pstart \phantomsection
\addcontentsline{toc}{subsection}{\textit{14 Ne negligito donum quod in te est: quod datum est tibi ad prophetandum, cum impositio ne manuum presbyterii.}}
\subsection*{\textit{14 Ne negligito donum quod in te est: quod datum est tibi ad prophetandum, cum impositio ne manuum presbyterii.}}ligunt fuisse scriptam hanc epistolam, cum Pau- lus in Syriam postremum venturus esset, non vi- deo habere satis firmum argumentum et funda- mentum. 14 Ne negligito donum quod in te est: quod datum est tibi ad prophetandum, cum impositio ne manuum presbyterii. Secunda exhortatio ab officio vel fine donorum Dei, quam amplificat etiam a circunstantia, eaque duplici nempe a Doni Dei excellentia, quo prae- ditus erat Timotheus, et Prophetia quae de Ti- motheo praecesserat, quod vtrunque delebit et deturpabit Timotheus, nisi suo muneri diligen- ter operam nauet et impendat. Eodem vero mo- do debet quisque animaduertere, quae a Deo do- na peculiaria acceperit, ne apud ipsum otiosa et infructuofa lateant, aut inutilia maneant, et de- fossa: sed vt illa in Ecclesiae vsum communem, propter quem illa talenta data sunt a Domino, conferantur. Similis locus Matth.25.vers.14.1. Pet.4.V.1o. Quo autem maiora sunt Dei dona erga nos, eo diligentiores in caeteris lucrandis esse debe- mus, ne illa apud nos otiosa iaceant, vel rubigi- nem contrahant. Nam cum foenore a Deo repe- tentur. De Prophetia autem, quae de Timotheo prae- cessit, quod dicit hoc loco Paulus, etsi quidam sic explicant, quasi finis illius Dei doni ostendatur: non autem vlla insignis praedictio' Dei et testimo- nium de futuris! in Timotheo donis antegressa sit:ta men malo accipere:vti supra 1.vers. 18. et 2.  \pend
\section*{CAPVT  IIII. }
\marginpar{[ p.235 ]}\pstart \phantomsection
\addcontentsline{toc}{subsection}{\textit{15 Haec exerce, in his esto: vt tuus profe- ctus manifestus sit inter omnes.}}
\subsection*{\textit{15 Haec exerce, in his esto: vt tuus profe- ctus manifestus sit inter omnes.}}Timoth.1. vers.6. Explicatum autem est a nobis supra eo ipso cap.1.vers.18, ΚάριQμα autem dixit Paulus, non χe Quα, ne quis hebes Papista putet chrisma, quo ipsi superstitio- se vtuntur ad suos Sacerdotes et Episcopos vn- gendos et inaugurandos, hic esse a Paulo com- memoratum et quaesitum. De impositione manuum infra, vti et de voce hac Presbyterii, quae coetum significat Presbyterorum, contra quam sentit Erasmus: putat enim ille E- piscopatum significari hoc loco, et finem designari non instrumentum et modum, quo vocatus est Timotheus: infra dicetur cap.5.vers.22. Hic locus commendat authoritatem Presby- terii, et vocationes ordinarias, quae hominum mi- nisterio fiunt, quam qui maxime, ne quis praetex- tu ingentium donorum et egregiae suae doctrinae in Ecclesiam intrudat ipse sese, cum Timotheus tantus vir etiam de quo praecesserant spiritus Dei apertae prophetiae, tamen ordinario modo desi- gnatus sit minister et praeco verbi Dei. 15 Haec exerce, in his esto: vt tuus profe- ctus manifestus sit inter omnes. Tertia admonitio de ipsa operis difficultate, vt ea perspecta diligentius pastores suo muneri in- cumbant. Habet autem summos stimulos ipsa styli, quo Paulus vtitur, ratio, meditare (ait) in his esto. Vrget enim pastores ne alibi diutius aut lu- bentius occupentur, quam in hoc munere suo, tanquam in sua Sparta exornanda. Ergo damnan- tur ex hoc Pauli dicto pastores omnes, qui in iis,  \pend
\section*{AD I. PAVL. AD TIM. }
\marginpar{[ p.430 ]}\pstart quae propria sunt sui muneris, perfunctorie et no- τιργως versantur: in aliis autem rebus, quae pro- prie ad eos non pertinent, serio, assidue et diligen- tius insunt eaque satagunt. Faciunt enimη̃ργον πά- ρεργον: et πάρεργον ἔργον, vt ait Prouerbium vetus, contra exemplum Apostolorum, qui cum se vi- derent a vero sui muneris officio propter curam mensarum distrahi et auocari, quanquam ea res et sancta erat et pia, tamen voluerunt se ea cura leuari, vt liberius meliûsque veris et potioribus exercitiis suae vocationis inseruirent ac incumbe- rent. Nec valet haec quorundam pastorum, qui sunt πολυπράγμονςς excusatio, quod Reipubl. com- moditas, et ratio postulet, vt in istis ciuilibꝰ et ex- traneis rebus ipsi versentur, et occupentur. Facit enim haec ratio, vt hodie quidam Episcopi potius sint militares Duces, vel Praefecti aerarii: vel fo- renses iudices, vel Cancellarii regni, vel Consili arii, regni, quam pastores, ministri et praecones verbi Dei coelestisque doctrinae, et pabuli gregi suo di- spensatores. Denique quid illi Episcopatus est a- liud, quam reditus quidam pinguis et opulentus vel ad fruendas voluptates, vel ad familiae onera sustinenda, quae ampla est, interea dum aliis ne- gotiis vacant, quae tamen ab Episcoporum mu- nere, sunt alienissima? Μελέτα Meditatio non est leuis quaedam co- gitatio, sed assidua et frequens eiusdem cogitatio. nis repetitio, cura, et mentis volutatio, vt aptio- res ad id, quod meditamur, fiamus. Sed non so- lum horum praeceptorum cogitationem requirit Paulus, imo vero etiam executionem et praxim, ne contemplatiuae vitae homines hinc sibi patro-  \pend
\section*{CAPVT  IIII. }
\marginpar{[ p.237 ]}\pstart cinium quaerant. Ex hoc etiam ipso ostenditur, quam sit difficile et arduum hoc ministerii opus, quocirca diligenter est in illud incumbendum. Profectus tuus) Est finis quem in munere suo faciendo spectare debet pastor Euangelicus, nen- pe propagatio Dei gloriae, et ipsiusmet Ecclesiae maior in dies profectus et incrementum. Itaque vocem, Profectus, non refero ad personam Timo- thei:sed ad Ecclesiam potius, Manifestus sit) Quid si nullus profectus, nul- Ium incrementum Ecclesiae appareat, estne pro- pterea pastori desperandum. Estne etiam ipse semper propterea, vt deses et negligens damnan- dus? Respond. Hoc quidem lugendum esse, sine vllo profectu praedicari Dei verbum:si tamen pa- stor quantum in se est, syncere pascat gregem suum, vigilet, in eo sit totus, et officium faciat, neque illi desperandum esse:neque eum, vt negli- gentem, damnandum 1. Pet.5.vers.2.  \pend\pstart \phantomsection
\addcontentsline{toc}{subsection}{\textit{16 Attende tibi ipsi et doctrinae: persi- ste in istis, id enim si feceris, et teipsum ser- uabis et eos qui te audierint.}}
\subsection*{\textit{16 Attende tibi ipsi et doctrinae: persi- ste in istis, id enim si feceris, et teipsum ser- uabis et eos qui te audierint.}}16 Attende tibi ipsi et doctrinae: persi- ste in istis, id enim si feceris, et teipsum ser- uabis et eos qui te audierint. Quarta exhortatio a fructu huiusmodi labo- ris, et eo quidem vberrimo. Est autem is fructus, salus animarum, tum in Pastore ipso, tum in Gre- ge et Ecclesia. Quod vt fiat vult Paulus vt quili- bet pastor Euangelicus, Attendat sibi, et Persistat in officio. Attendat autem, id est, diligenter vi- deat, et non perfunctorie consideret, et Qualis i- pse sit in sese et moribus, et Qualis sit ea doctri- na, quam docet gregem Domini: vt si quae vitia  \pend
\section*{AD I. PAVL. AD TIM. }
\marginpar{[ p.238 ]}\pstart vel in sese:vel in sua doctrina inesse animaduertat corrigat, et emendet. Denique ea vitia potius a seipso, quam ab aliis discat, ne videatur negligen- ter aut perfunctorie officium suum facere. Persistat vero, id est, continuet, neque fatige- tur aut fathiscat, neue susceptum semel opus pro- pter infinitas pene molestias deserat. Id enim est non modo leuis animi: sed etiam infidelitatis et diffidentiae de Dei misericordia signum verissi- mum. Ergo perseueret, quantunuis difficile opus hoc videatur. Finis tandem et fructus suauissimus, atque adeo salutaris sequitur, salus nimirum ipsius pa- storis dum in munere suo obeundo fideliter Deo paret. Habent enim in eo ipso optimum miseri- cordiae Dei erga se testimonium pastores, quia opera bona sunt nostrae adoptionis et aeternae sa- lutis certa testimonia et indicia, praesertim quae in cuiusque vocationis opere implendo et obeum- do apparent 1. Timoth.2. vers.15. Ipsius quoque gregis salus consequitur, quia ea ratione et per ministerium Euangelii Domi- nus adducit suos ad salutem. Vnde Ministri Euan- gelici sunt salutis hominum ministri et organa. Deus autem ipse est illius aurnor et causa. Eadem phrasi alibi quoque vtitur scriptura, vt Iacob5. vers.21. Prouerbus 23. vers.14. Haec autem magna laus est ministerii Euangelici, per quod seruari dicimur, vti quod operarii cum Deo iidem mini- stri vocantur 1. Corinth.3.  \pend
\section{CAPVT  V. }
\marginpar{[ p.239 ]}\pstart \phantomsection
\addcontentsline{toc}{subsection}{\textit{S ENIOREM ne increpato, sed hortare vt patrem, iuniores vt fra- tres.}}
\subsection*{\textit{S ENIOREM ne increpato, sed hortare vt patrem, iuniores vt fra- tres.}}CAP. V. 
\textbf{S}ENIOREM ne increpato, sed hortare vt patrem, iuniores vt fra- tres. Μετάβαζις. Transitenim, et ordine quidem Paulus, a doctrina ad mores, cuius vtriusque rei cura ad veros pastores et Ecclesiae praepositos pertinet. Sed prius de vitiis, quae doctrinae ratione obre- punt, et haereses dicuntur, fuit dicendum. Postea autem de morum vitiis, et eorum emendatione, quam hoc loco persequitur, vti et in Epistola posteriori ad hunc ipsum Timotheum, et ad Ti- tum cap 2. Dicitur autem alio nomine haec cor- reptio et emendatio, censura Ecclesiastica, quae potestatis clauium pars est. Sic autem et hoc loco et alibi loqui videtur Paulus, vt vni episcopo eam tribuat, non autem toti ipsi Ecclesiae, non item coetui et senatui Presbyterorum. Id quod perpe- ram intellectum duplicem errorem peperit. Pri- mum eorum, qui totam Ecclesiae ipsius iuris- dictionem vnius tantum personae, id est, Episcopi arbitrio vindicarunt et asseruerunt, ex quo ipso inducta est postea superba Episcoporum tyran- nis. Alterum eorum, qui vt hunc errorem cor- rigerent, ipsi secundum induxerunt, et ad vniuer- sam, quam vocant, Ecclesiam, id est, ad singulos de Ecclesia hanc potestatem retraxerunt, et ita per malum malo mederi studuerunt, malo nodo vt est in Prouerbio, malum cuneum quaerentes. Denique inciderunt in Scyllam, dum cupiunt vi-  \pend
\section*{AD I. PAVL. AD TIM. }
\marginpar{[ p.240 ]}\pstart tare Charybdin. De quo toto argumento et v- tilissimo, et futurae disputationis se quentiumque toto hoc capite praeceptorum fundamento nobis aliquid est dicendum. Est autem hic status causae et controuersiae (quae etiam non his nostris tempo- ribus acerbe a quibusdam agitatur) Ad quos nimirum Ecclesiasticae censurae ius et potestas pertineat. Hoc enim pro concesso sumimus, quod tota docet scriptura, cum vitae sancte et honeste instituendae gratia, et vt Dei gloriae inseruiamus, scriptura et doctrina coelestis nobis praedicetur, omnino necesse esse, vt aliqua sit Ecclesiae disci- plina et turpium morum emendatio, per quam homines tum in officio contineantur, tum, si se- mel abducti fuerint, reuocentur. Nam qui Eccle- siasticam disciplinam prorsus tollunt, quid aliud Eccle- siae quam laruam nobis inducunt, et omnem pec- candi effrenem licentiam probant? contra quos Psal.5o. et Hebr.12.vers.8. et dictum Tertulliani de Habitu virginum, item Cypriani lib.  a. epist.7. valere debet. Disciplina enim Ecclesiae, inquiunt illi, et castigatio Ecclesiastica morum est fidei custos, quam qui repudiant, fidem ipsam nomi- ne tenus habent. Secundo autem loco postulamus et sumimus hoc nobis concedi, quod est verissimum. Hanc mo- rum correctionem et vitiorum emendationem, quae Ecclesiastica iurisdictio et ius clauium dici- tur, latissime differre et toto genere dissimilem esse a ciuili et politica, quam vocant, iurisdictio- ne, quae Magistratibus competit. Tertio denique et hoc quoque tanquam ἀιτη- uae praesupponendum est, quod alibi verum esse  \pend
\section*{CAPVT  IIII. }
\marginpar{[ p.241 ]}\pstart probabimus, Deo dante, et iam ex parte docuimus supra cap.3.ex veteri et bene constituta Christia- nae Ecclesiae institutione, non vnicum Episcopum id est, non vnam quandam duntaxat personam ad regendam totam Ecclesiam electam et ordinatam fuisse (id quod nunc fit in Papatu) sed plures si- mul delectos, quibus Ecclesiae cura demandare- tur, tum in iis quae ad doctrinam pertinent et hi erant Pastores, et Doctores: tum etiam quae ad mores, et Pauperum aerarium, quales erant Pastores iidem, Presbyteri, Diaconi. Ex quo fit vt a Tertulliano in Apolog. cap, 39. Rectores Ecclesiae dicantur non duntaxat soli Episcopi, sed cum illis etiam Diaconi, et Presbyteri, tan- quam ipsorum Episcoporum collegae. Quod idem in lib.  de fuga in persequutione confirmat, et in- finitis pene locis Cyprianus. Atque etiam Augu- stinus, veluti in Enchirid. cap.65. Ergo his ita praemunitis quaesitum est. Vtrum ad vnum tantum Episcopum, vel delegatum ab eo aliquem, ius censurae et emendationis morum, (quae Ecclesiastica dicitur) eiusque cognitio tota pertineat: an potius ad plures? Quod si pertinet ad plures, vtrum ad totum coetum Ecclesiae, an ad singulos fideles qui sunt de eo coetu, an ad eos tan- tum qui Ecclesiae praefecti, Authores, ηγομένος, et Praepositi dicuntur, quales sunt Presbyteri et pa- stores simul, qui alio nomine dicuntur senatus Ec- clesiasticus. Ac primum ad vnum duntaxat totam eam po- testatem et cognitionem pertinere non posse pro- bat Christi et Pauli dictum. Christus vult per- uicaciam rebellium denuntiari et dici Ecclesiae,  \pend
\section*{AD I. PAVL. AD TIM. }
\marginpar{[ p.242 ]}\pstart Ergo non vni tantum, Matth. 18. vers.17. Paulus quae geruntur in Ecclesia a Presbyterio fieri do- cet, id est, a quodam coetu hominum. Ergo non a solo illius Ecclesiae Episcopo, supra cap. 4.V.14, Philip. 1.vers.1. Actor.11.vers.30. Deinde Eccle- sia ipsa Dei transformaretur in regnum terrenum, essetque iam Monarchia, non Aristocratia, si vnus in ea dominaretur, et omnia solus ageret. Quod fieri et esse vetant hi scripturae loci Matth.2o.V.25 1.Pet.5.vers.3. Hieronym. ad Rustic. penes coetum Presbyterorum, quem vocat senatum morum, cen- suram esse aperte scribit. Denique mos ipse vetustissimus primae Eccle- siae huic vnius dominationi in Ecclesia repugnat. Nam omnia ex communi consensu totius coetus praepositorum Ecclesiae geri solita fuisse, non v- nius autem hominis duntaxat arbitrio, probant hi scripturae loci, vbi congregata fuisse Ecclesia dicitur ad res et negotia Ecclesiastica gerenda, et decernenda et dispicienda Actor. 15. vers.4.22. vers.3o. Ioan.9. vers.24. Actor.21.vers.18. 1. Cor. 5.vers.4. 16.vers.3. Nec in contrarium efficit quicquam eorum ratio, qui omnia, quae tum in coelo, tum in terra fiunt, ad vnum principium, non ad plura reuocari va- riis exemplis docent, quae fuit ineptissimi cuiusdam Itali et Veneti Cardinalatum ambientis recens pro Papistica et Romana tyrannide in Ecclesia tuenda defensio. Hoc enim verum esse concedi- mus, et ipsam Ecclesiam vnicum caput habere ipsi contendimus: sed Christum, non autem ho- minem vllum mortalem, a quo capite, et eo qui- dem vnico et summo, sunt caeteri, tamquam a sum-  \pend
\section*{CAPVT  V. }
\marginpar{[ p.243 ]}\pstart mo iudice, delegati, qui pastores in Ecclesia prae ficiuntur. Et vt inter ipsos delegatos nullus sine sacrilegio et laesae maiestatis crimine summi ma- gistratus ius sibi vindicare potest: ita nec in Ec- clesia vel vniuersa, vel particulari quisquam ius summi pastoris et ipsius capitis sine aperta in Christum blasphemia nomem sibi tribuere debet. Quod item obiiciunt, olim sub lege fuisse inter Leuitas vnum quendam summum sacrificatorem et Sacerdotem, habet facile responsum. Quia cer- tis figuris sub lege, Christi dignitas, vti Sacrificium adumbrabatur, donec ipse apparuisset, fuisse vnum quendam inter multos fratres summum Sacer- dotem, qui Christi figura erat. Itaque fuit ea res legalium caeremoniarum pars, quae omnes Christi aduentu abolitae sunt, et quae sine piaculo a Chri- stianis in vsum reuocari non possunt. Nec efficit etiam quicquam haec quorundam obiectio, qui idcirco purant ab vno quodam (cui hae partes sunt totius Ecclesiae consensu deman- datae) veluti Episcopo, aut ipsius officiali, censuram Ecclesiasticam fieri, quia quod agit, vt delegatus ab vniuersa Ecclesia agit, non vt priuatus. Delegatio enim ista nunquam vel ab Ecclesia fieri, vel a quoquam suscipi sana conscientia potest, cum alium ordinem ad seipsam regendam instituere Ecclesia ipsa vniuersa non possit, quam illum, quem Christus et ipsius Apostoli praescripserunt, et obseruari praeceperunt. Illi autem regiam istam vnius in Ecclesia regenda authotatem, vti docui- mus, siue per vsurpationem, siue per delegationem institutam damnarunt. Neque enim vti homines in rebus politicis, quae ad hanc corporis vitam  \pend
\section*{AD I. PAVL. AD TIM. }
\marginpar{[ p.244 ]}\pstart pertinent, multa mutant, instituunt, corrigunt, figunt et refigunt, idque iure et prudenter:ita in ipsius disciplinae et censurae Eccleliasticae essentia immutare et nouare quicquam possunt, quia huius disciplinae finis, et scopus animas spectat, et ad earum gubernationem refertur, vtque ad poeni- tentiam homines adducantur. Cuius rei cognitio, efficacia, modus, ratio, et iurisdictio solius Dei propria est, ei nota et relicta:nullius autem regis aut coetus mortalium imperio aut prudentiae con- cessa. Ergo immutare ipsius censurae et discipli- nae Ecclesiasticae fundamentum (quale est, vt ex Aristocratica fiat monarchica administratio Ec- clesiae) nemo homo, quamuis Rex et Imperator potest, ac ne vniuersus quidem ipsius Ecclesiae coetus. Accidentalia quidem moderari possunt, quae pertinent ad faciliorem istius disciplinae et cen- surae vsum vel praxin. Atqueeo sensu concedimus formam disciplinae vel censurae Ecclesiasticae ean- dem in omnibus gentibus constitui non posse. Nempe cum de accidentalibus, non autem essen- tialibus vel disciplinae, vel censurae Ecclesiasticae disputatur. Nam essentialia eadem vbique gen- tium in Ecclesiis Dei esse debent: accidentalia mutari et variari ex variis causis possunt. Ergo ad prurts mad lus, ea cognitio, et potestas gerendi et decernendi et dispiciendi pertinet. Sed vtrum sic ad vniuersam Ecclesiam, nimirum, vt suffragium singuli de Ecclesiade qualibet re Ecclesiastica fe- rant, vbi aliquis admonendus vel reprehendendus erit. Res. Per solos Ecclesiae ηγρμένες et prepositos eam potestatem oportere exerceri, etsi totius Ecclesiae communis potestas est, et in ipsius coe-  \pend
\section*{CAPVT  V. }
\marginpar{[ p.245 ]}\pstart tus aedificationem a Christo concessa et relicta: non solum autem praepositorum ipsorum gratiâ aut ratione est. Atque nostra haec sententia, cui multi obsistunt, plane est confirmanda, scili- cet ad solos Ecclesiae praepositos eius iuris et po- testatis exercitium pertinere, non autem ad sin- gulos de coetu et populo Ecclesiae. Probatur au- tem his fere rationibus. Primum authoritate verbi Dei Actor.11.v.15. 21.et 22.Petrus et Paulus coram Apostolis et prae- positis rationem reddunt sui facti, non coram v- no quodam Episcopo: quanquam, si quibusdam credimus, iam erat Iocobus frater Domini Hiero- solymorum Episcopus. Non etiam singuli de Ec- clesia in suffragia mittuntur. Sed coram praeposi- tis Ecclesiae, qui Ecclesia vocantur, ea tota dispu- tatio disceptatur. Petrus ipse priore Episto- la cap.5.V.5. negat se vllum in coetum Domini habere principatum. Iudaeorum mores tempore Christi multa quidem in doctrina corrupta ha- bebant: tamen veteris disciplinae et a Deo insti- tutae vestigia retinebantur. Itaque cum Christus ipse suae vitae et doctrinae rationem redditurus si- stitur, conuocantur Scribae, Pharisaei, principes Sacerdotum, id est, praepositi Ecclesiae, totumque synedrion: non autem solus summus Pontifex si- bi de Christo sumit iudicium. Non item iudicant singuli de populo, sed praefecti Ecclesiae dunta- xat. Denique Christus nos iubet ad plures: non ad vnum tantum aliquem referre, si qui nos of- fenderunt nobis sponte non reconciliantur Matt, 18.vers.16.1. Timoth.5.vers.19. Secundum: ratione. Est morum coercitio et  \pend
\section*{AD I. PAVL. AD TIM. }
\marginpar{[ p.446 ]}\pstart censura Ecclesiastica cum verbi Dei praedicatio- ne indiuulse coniuncta, et illius tanquam appen- dix quaedam 2. Timoth.3. vers.16. Malach.2.vers.4. Itaque Paulus ipse praecipit, vt qui Ecclesiam pa- scere volunt, iidem eam increpent 2. Timoth.4. At Presbyteri sunt Pastores et ποιμένες Ecclesiae, non autem quilibet de populo. Ergo et ii ipsi, non autem singuli de populo, Ecclesiae correctores, censores, iudices esse et constitui debent. Deinde iidem’illi curam animarum nostrarum habere|di- cuntur, vt est Hebr. 13. vets.7. Iidem reprehemn- duntur, nisi nos admonuerint sceleris nostri, Eze- chiel. 34. Iidem verbo Dei iubentur Ecclesiae curam habere. Ergo ad eos, non ad populum co- gnitio et exercitium huius censurae et potestatis per- tinet. Denique vetustissimus Ecclesiae ipsius Chri- stianae mos et vsus idem probat. Nam semper Presbyteri cum eo, qui Episcopus dictus est, iudi. carunt de scandalis et censuris quae cuique fieri debent. Ambros.epist.28. Cyprian.epist.39. et 40. non autem vnus ille Episcopus, quantunuis ma- gnis animi dotibus excellens, non item quilibet de populo. Ergo haec iurisdictio est totius quidem Ecclesiae, ratione potestatis, Praepositorum autem, ratione exercitii et administrationis. Qui vero contra sentiunt, ad totum Ecclesiae coetum et multitudinem hoc exercitium potesta- tis pertinere, afferunt. 1 Dictum Christi Matth.18. vers.17. Dic Ec- clesiae. Resp. Ecclesiae nomen illic accipi pro iis, qui totius Ecclesiae authoritate fulti et vocati legitime Ecclesiae praesunt ac inuigilant. Sic quod Senatus Roman. decreuit, dicitur Ro-  \pend
\section*{CAPVT  V. }
\marginpar{[ p.247 ]}\pstart mana ciuitas decreuisse. Ita explicatur etiam can. Nullus distinct.63. 2 Obiicitur Paulus 1. Corin.5. qui conuocasse videtur totam Ecclesiam, cum ait vers.4. vobis et meo spiritu conuocatis in nomine Domini, cum de incestuoso iudicando ageretur. Ergo ad illius suffragia haec censura pertinet.i. ad singulos de populo. Resp. Paulum decernere quidem quid sit faciendum, sed a Presbyteris et a Senatu Eccle- siae:id enim suffragiis populi non permittit. Ita- que non pertinet ad populum ea deliberatio: sed ad praepositos Ecclesiae, populo assentiente. Aiunt Quod ad omnes pertinet, ab omnibus fie- ri debet. Respond. ex Ambrosio in lib.  de Digni- tate Sacerdotal. cap.3. Agi quidem debere a quo- que pro loco et munere, quod in Dei Ecclesia ge- rit. Aliud enim est, quod ab Episcopo requirit Deus: aliud quod a Laico. Sed cum sint pastores et Presbyteri Ecclesiae ηγόμενοι, ad eos censurae iu- dicium pertinet, ad quos eosdem pertinet immo- rigeros Deo diuini iudicii metu iniecto com- pescere. Ad totam vero Ecclesiam spectat disce- re alieno exemplo, quenque Deo rebellem poe- na esse dignum. ltaque publicae censurae in Eccle- sia fieri debent ad ipsius aedificationem, et vt in- telligant omnes scelera in coetu Christianorum non esse tolleranda in can. Certum est s sed ad- huc 24.quaest.4. Aliae obiectiones afferuntur, sed omnino leues. Executioni igitur publicae censurae interuenire debet notitia et consensus Ecclesiae audientis et admonitae de poena scelerati homi- nis et obstinati:ipsius tamen iudicium et execu- tio ad solos praepositos Ecclesiae pertinet.  \pend
\section*{AD I. PAVL. AD TIM. }
\marginpar{[ p.248 ]}\pstart Quod autem quidam existimant, vbi est vere Christianus Magistratus, ibi nullam esse censuram Ecclesiasticam debere, illi certe Ecclesiasticam et Politicam iurisdictionem inter se pessime con- fundunt. Deinde non vident se refutari Exod.1o. vbi et Ecclesiastica censura, et poena politica locum in vno et eodem homine et negotio habet, di- uerso tamen respectu. Augustinus certe vixit sub imperatorib' Chri- stianis, et tamen idem lib. 11.de Genes.ad lite. cap. 4o. docet solitos esse Episcopos vti aduersus scan- dalosos homines excommunicatione et censura Ecclesiastica. Item confirmatur, omnibus Syno- dis, quae sub Christianis imperatoribus habitae sunt, vbi censurae Ecclesiasticae et confirmatae et retentae sunt. Idem denique, si quis hoc etiam genus probationis admittat tota 24. quaest. 4 Sed et Deus ipse poenam exigit a nobis scelerum nostrorum tum in corpore, tum in anima. Nec ea sane est duplex poena: sed vna tantum. Tanta e- nim peccato debetur, et ea, per quam et animus et corpus peccatoris puniatur, quia peccatum hominis et totius suppositi actio est. Quibus expositis iam accedamus ad ipsa Pau- li verba. Distinguit autem hoc loco homines, Paulus ab Ætate, et Sexu. Ab aetate, alii sunt Se- niores, alii Iuniores. A sexu, alii sunt Mares, alii Foeminae. Hic enim πρωβυτέρων vocem ad aetatem refero, non autem ad dignitatem Ecclesiasticam, quemadmodum tamen saepe accipitur, velut inf. vers.17.19. et 1. Pet.5 vers.1. Id quod ex ἀντιθέσει et voce νεωτέρςυς apparet. Cur autem haec praece. pta danda existimauerit Paulus, ratio est, quod  \pend
\section*{CAPVT  V. }
\marginpar{[ p.249 ]}\pstart oporteat ministrum verbi Dei et pastorem ὀρθoτο- με͂, id est, recte et pro cuiusque captu et vtilita- te dispensare sanam doctrinam, et quae ex ea pro- ficiscuntur admonitiones 2. Timoth. 2.vers.15.Id quod fecisse Ioannem animaduertimus 2. Epist. cap,2, vers. 12.13.14. Sed obstat quod hac ratione, quam inducit hoc loco Paulus, videtur haberi et praescribi acceptio personarum, quam in vero et fideli dispensatore verbi sui esse vetat Deus Ma- lach.2.vers.9. Praeterea cum in Christo ne que sit mas, neque foemina, neque seruus, neque liber, neque senex, neque adolescens, cur ista vel sexus vel aetatis ratio et distinctio fit? Respond. Perso- narum acceptionem non induci, cum omnes re- prehendantur qui peccarunt. Sed aliud est pru- dens et fidelis verbi doctrinaeque coelestis dispen- satio, quam Paulus ὀρθοτομίαν appellauit: aliud au- tem personarum gratiosa et damnata acceptatio, quae dicitur πρπωποληδία. Quod autem affertur in Christo neque marem esse, neque foeminam Galat.3, vers.28. alio prorsus pertinet, quam ad hoc argumentum. Quaeritur etiam vtrum haec praecepta ad pri- uatas tantum, an vero ad publicas etiam repre- hensiones, quae fiunt a Ministris, pertineant. Pu- blicae vero sunt admonitiones, quae et in consistorio fiunt, et nomine totius Ecclesiae a senatu Ecclesia- stico, vel ab aliquo Presbytero, quanquam priua- tim et domi fiant. Respond. Quidam ad priuatas tantum admonitiones et reprehensiones haec praecepta referenda esse putant. Quidam autem ad publicas duntaxat. Ego vero ad vtrasque, quia semper ratio aetatis et sexus a nobis habenda est.  \pend
\section*{250 AD I. PAVL. AD TIM. }\pstart In quaque enim aetate, vt oportet tractanda ma- gna quaedam non tantum prudentiae, sed etiam charitatis pars posita est, quae vtraque et a pu- blicis et a priuatis admonitionibus nunquam a- besse debet, vt suum vsum et fructum habeant. Illud autem notandum est, cum ita Paulus nos agere iubet, nolle tamen nos vel personarum ac- ceptatores vel assentatores esse, vt alienis vitiis blandiamur et fau eamus: sed ne nimium acerba et acri agendi ratione et oratione vtentes pecca- tores ipsos a Dei doctrina auertamus. Quae certe acerbitas prorsus est inutilis, quanquam tamen grauitas est a pastore retinenda Tit.1.vers.7. Haec igitur sunt de his rebus Pauli praecepta et rationes I Pastor Euangelicus, tanquam pe- ritus animorum medicus, pro suae aetatis et sexus ratione tractato quenque, neminem imprudenti et acerba agendi ratione offendito. 2 Seniores tum mares, tum foeminas, peccantes tanquam pa- tres et matres pastor Euangelicus leniter admo- neto. 3 Iuniores, tum mares, tum foeminas, peccantes velut fratres et sorores idem commo- nefacito. 4 Foeminae imprimis autem iuniores ita a pastore Euangelico moneantur, vt summus castitatis pudor in ipsa agendi et loquendi cum illis ratione eluceat et appareat. Επιπλήξης. Quidam ad pugnos et verbera refe- runt, quae quidem infligi viris senioribus a Ti- motheo vetet Paulus: sed perperam. Pertinet e- nim hic locus ad verba tantum acerbiora in pa- store damnanda: non ad verbera. Nam supra Pau- lus docuit non debere pastorem esse percussorem et πλήκτην. Ratio vero horum omnium praece-  \pend
\section*{CAPVT  IIII. }
\marginpar{[ p.251 ]}\pstart \phantomsection
\addcontentsline{toc}{subsection}{\textit{2 Mulieres natu grandiores, vt matres iuniores et sorores, cum omni puritate.}}
\subsection*{\textit{2 Mulieres natu grandiores, vt matres iuniores et sorores, cum omni puritate.}}ptorum est in vniuersum duplex. Prima, quod omnis reprehensio est ipsa per se iam acerba. Ita- que quadam verborum lenitate est mitiganda, et tanquam condienda, vt sit vtilis peccatori, luben- tiusque ab eo audiatur. Secunda, quod ista ratione pastores agentes, non impetu quodam animi ferri, sed summa cha- ritate eos impelli apparebit, cum vitia reprehen- dunt. Maiorem autem vim et pondus habet no- stra oratio, vbi sibi persuadent homines, se a no- bis non animo iniurandi: sed summo beneuolen- tiae affectu commoneri, et reprehendi. Quod ad speciales cuiusque praecepti rationes, sunt duae. Prima, quod aetas illa iuuenilis, qualis fuit in Timotheo, omnino rationem hanc agendi necessariam esse suadet. Multo enim odiosiores sunt reprehensiones, quae nobis ab adolescenti- bus fiunt, quam quae a senibus et maturae iam ae- tatis viris. Senum enim prudentiam admiramur: atque illis propter aetatem facile cedimus. Secunda ratio, quod haec senum et iuniorum appellatio, quam praescribit hic Paulus, cum lege Dei maxime conuenit et consentit. Illa enim iu- bet, vt senes, et ipsam senectutem veneremur. Vn- de Seniores pro patribus agnoscere suadet. Qui vero sunt nobis aetate aequales, fratrum nominum et sororum fere vulgo, amicitiae ergo, a nobis ap- pellantur. 2 Mulieres natu grandiores, vt matres iuniores et sorores, cum omni puritate. Cum omni castitate) Etiam in ipso agendi cum illis iunioribus foeminis modo, ne praetextu mu-  \pend
\section*{AD I. PAVL. AD TIM. }
\marginpar{[ p.252 ]}\pstart \phantomsection
\addcontentsline{toc}{subsection}{\textit{3 Viduas honora quae vere viduae sunt.}}
\subsection*{\textit{3 Viduas honora quae vere viduae sunt.}}neris et officii nostri, imo vero verbi Dei, videa- mur impurae libidinis semina quaedam aut irrita- menta iacere. Illud enim certe est accipere no- men Dei in vanum. 3 Viduas honora quae vere viduae sunt. Πρόθεαις. Nam inter eas personas, quarum ma- xima et peculiaris quaedam a nobis ratio haben- da est in Dei Ecclesia, merito censentur viduae, de quibus idcirco hoc loco quaedam praecepta Paulus tradit. Viduarum autem, quae in Ecclesia versabantur, aetate Pauli, duplex fuit genus, Vnum earum quae Simpliciter viduae erant: Alterum ea- rum, quae Allectae et cooptatae erant in ministe- rium Diaconorum. De primo genere agit primo loco Paulus, de secundo postea hoc ipso capite a versiculo 9. et seq. Viduae nobis a Deo commendantur passim in scriptura veluti Exod. 22.vers.22. Zach.7.vers. 1o. Psal.146. vers.9. et Luc.7. vers.12. Haec vidua e- rat, et aspexit super eam Dominus. Itaque in iure ciuili Vidua, Pupillus, et Peregrinus dicuntur Mi- serabiles personae lib.  2. codic. id est, tales, quae sint ipsae per se miseratione dignae:non autem Deo exosae, sed cui sunt plurimum curae, vt mirum vi- deri non possit, si earum curam habuit Ecclesia Dei, praesertim fidelium et piarum. Itaque hos tres canonas de viduis subleuandis tradit Paulus. 1 Viduarum fidelium curam, quam fieri po- test, maximam pastor Euangelicus habeto. 2 Viduas egenas liberi earum qui possunt, sine onere Ecclesiae alunto. Quae viduae piae sunt et fideles et omnino  \pend
\section*{CAPVT  V. }
\marginpar{[ p.83 ]}\pstart destitutae bonie, et aliis auxiliis ab ipsa Ecclesia aluntor, et exhibentor. Hos tres canones ex tota hac disputatione colligimus. Ac quidem duos ex hoc versiculos nempe 1. et 3. Secundum autem canonem ex se- quenti versiculo, qui est 4. Tίμα Hebraeorum mo- re dictum est, vti in praecepto quart o Honora pa- trem et matrem, qua voce non tantum reueren- ter et sollicite curandas esse viduas a pastore in- telligimus, sed etiam alendas ab Ecclesia, et ex- hibendas, si aliunde non possint se sustentare. Hoc idem praeceptum transferri ad vniuersum paupe- rum fidelium genus et potest et debet, qui si a suis possunt, debent ali sine onere Ecclesiae: sin mi- nus, ab Ecclesia. Debet enim inter nos ea vigere charitas et tanta, vt nemo inter nos egeat, quem- admodum Dominus ipse praecepit, neque sit mem- dicus Deut.15. vers. 4. non quidem vt pauperes qui sunt eiiciantur: sed vt alantur a nobis et sub- leuentur. Vnde sicolim etiam nondum plane corruptis Ecclesiae Romanae moribus a Gelasio primo con- stitutum fuit, vt ex omnibus Ecclesiae reditibus fierent quatuor partes (in quibus. ipsis etiam fi- delium oblationes continebantur) ex quibus v- na tantum pars erat Episcopi:secunda clericorum: tertia insolidum pauperum, quarta vero fabricae templi applicanda, quemadmodum relatum est in cano. Quatuor autem et can.seq.12. quaest.2. Adeo vt Bernard. epist. 348. velit etiam sacros Ecclesiae calices vendi et distrahi, vt pauperibus suecurratur. Quod antea fecerat quoque Exupe- rius Tholosanus Episcopus, vt Hierony. scribit.  \pend
\section*{AD I. PAVL. AD TIM. }
\marginpar{[ p.254 ]}\pstart Vere vidua) Quaenam autem illa sit veravidua postea docet paulus ins.vers.5. est autem quae Pia est, et Destituta omni alia ope. Vnde illa Hieron. definitio, quae est in illius quadam epistola ad Fa- biolam non satis huic loco conuenit, vbi de offi- cio pastoris euangelici agitur. Vidua est, ait, cuius maritus mortuus est. Primum enim de piis tan- tum viduis hic locus intelligitur, non de omni- bus: deinde proprie quod ad subsidium dandum Ipectat, de ns tamtmmtemgitut, quae destituun- tur omni alio auxilio et ope: non de omnibus e- genis, etiam viduis.  \pend\pstart \phantomsection
\addcontentsline{toc}{subsection}{\textit{4 Quod siqua vidua liberos aut nepo- tes habet, discant in primis in propriam do- mum pietatem exercere, et vicem rependere parentibus. hoc enim est honestum et acce- ptum coram Deo.}}
\subsection*{\textit{4 Quod siqua vidua liberos aut nepo- tes habet, discant in primis in propriam do- mum pietatem exercere, et vicem rependere parentibus. hoc enim est honestum et acce- ptum coram Deo.}}t 4 Quod siqua vidua liberos aut nepo- tes habet, discant in primis in propriam do- mum pietatem exercere, et vicem rependere parentibus. hoc enim est honestum et acce- ptum coram Deo.  \pend\pstart Πιαρακολήθησις, siue Τάξις. Prospicit enim viduis et docet a quibus illae sint alendae. Nempe a pro- priis liberis, id est, filiis, nepotibus, abnepotibus in infinitum, si quos tamen illa habet, qui id facere possint. Videtur autem in genere ipso dicen- di Paulus alludere ad cap.21. Genes. vers.23. vbi vidua sic describi videtur. Vox igitur Μανθανέτωσαν ad liberos viduarum pertinet. Æqua sane ratio, vt etiam hoc loco ne- potes, et in infinitum alii, liberorum nomine con- prehendantur, quia ab auia tanquam a propria stirpe suam originem habent, et vt sint. Itaque eos, vti authores, post Deum gencris vitae que suae  \pend
\section*{CAPVT  v. }
\marginpar{[ p.255 ]}\pstart debent agnoscere pii homines et filii. Ratio huius praecepti hic multiplex affertur a Paulo. I A consequenti vel consentaneis. Sibi i- psis hoc quodammodo liberi impendunt, quod in parentes suos conferunt: habent enim a pa- rentibus, vt sint, et sunt cum illis vna caro et vnus sanguis 2 Ab officio, oportet liberos rependere parentibus quodammodo quod ab iis acceperunt. Id enim lex ipsa naturae iubet et dictat, quae nos obligat ad ἀντιδῶρα, vt Iurisconsulti doceut, et ad ἀντιπελαργίαν, quam hoc loco vocat ἀμεελὴν Paulus. Hoc exemplo Ciconiarum facere doce- mur. 3 Ab honesto. Hoc ipsum Deo gratum est et acceptum. Quod ex praecepto Honora patrem et matrem, satis intelligi potest. Illa enim voce, Honora: non tantum reuerentia quaedam exter- na significatur, sed etiam subsidium vitae, quod est a liberis parentibus praestandum. Quod offi- cium etiam pietas hoc loco nominatur, et a Pla- tone iustitia. At haec omnia iura semel violant, qui religionis et Monasticae vitae, quam profiten- tur, praetextu, se ab ea de parentibus alendis cura liberatos esse contendunt, quod Monachi facti sint. Denique omne liberorum in parentes obse- quium abiiciunt, et officium suum, propter votum Monachatus a se semel susceptum, cessare et nul- lum iam vsum aut locum habere debere sentiunt. Quae sane est horrenda impietas et omnis huma- ni sensus violatio. Quod autem de parentibus alendis a propriis liberis et nepotibus Paulus hoc loco praecipit et tradit, omnino latius patet. Omnes enim eos, qui parentum loco sunt, etiam complectitur:non tan-  \pend
\section*{AD I. PAVL. AD TIM. }
\marginpar{[ p.256 ]}\pstart \phantomsection
\addcontentsline{toc}{subsection}{\textit{5 Porro quae vere vidua est ac sola, spe- rat in Deo, et permanet in supplicationibus et precibus nocte ac die.}}
\subsection*{\textit{5 Porro quae vere vidua est ac sola, spe- rat in Deo, et permanet in supplicationibus et precibus nocte ac die.}}tum eos, qui nos e suo sanguine genuerunt, qua- lis est patruus, amita, matertera, frater, et soror, denique quales sunt omnes, qui nobis aliqua san- guinis necessitudine sunt coniuncti: quos egentes et inopes deserere, cum ipsi abundemus opibus, est certe criminosum, nefas et impium. Est enim Charitatis regula amplianda, et ex pari ratione, quid Dominus nobis hic praescribat, satis intelligi potest. Itaque Leuitic.18, illae personae, quas com- memoraui, nobis parentes ipsos referre dicuntur et pietas Abrahami nota est, qui patrem Thare assumpsit. Dauidis, qui omnes e sua familia secum in deserto habuit, et aluit, itemque patrem et matrem tuto loco deposuit apud regem Moabi- tarum 1.Samuel.22. Idem in l.alimenta C. de ne- gotiis Gest. ab hominibus profanis responsum est. 5 Porro quae vere vidua est ac sola, spe- rat in Deo, et permanet in supplicationibus et precibus nocte ac die. Ε’ξηγησις est siue ὀαισμός. Exponit enim quaenam sint illae vere viduae, quas supra honorare praece- pit. Ex hac autem ipsa descriptione, quales viduae piae esse debeant, intelligitur. Haec autem verba poti? Indicandi, quam imperandi modo sunt accipienda, quemadmodum fecit Ambrosius, idque omnino recte, et ex propria Graecarum vocum significatione. Ait igitur Paulus vere vidua est ea, quae manet in precibus: non autem sic ait quasi praecipiens, ve- re vidua maneto, vel maneat in precibus. Primum igitur harum, quarum cura ab Ecclesia maxime suscipi debet, conditionem proponit, et quam sit  \pend
\section*{CAPVT  V. }
\marginpar{[ p.257 ]}\pstart pastorum Euangelicorum sollicitudine digna, docet Paulus: deinde ab effectu siue proprio ea- rum munere easdem depingit, nempe a preci- bus et pietate eximia. Est in cano. Vidua distinct. 34 aliqua de viduis disputatio ex Hieronym. sed quae nihil, nisi superstitiosum et mancum, habeat. Eae igitur viduae Ecclesiae eleemosynis subleuari debent, quae describuntur hoc loco. Nam exhis Pauli praeceptis etiam aliae regulae et canones de pauperibus ab Ecclesia alendis sunt decerpendi, vt et Pastores, et Diaconi Ecclesiae intelligant, quos in albo pauperum habere, quibusque ex ae- rario Ecclesiastico succurrere debeant, ne, si aliud hominum genus alant, fucos potius quam paupe- res nutriant et pascant: et temere tenues Ecclesiae opes profundant. Magna enim cura est adhiben- da, vt prudenter, et quibus oportet, fidelium e- leemosynae distribuantur, vt etiam 2.Thessalon.3. docet Paulus. Videtur autem hoc loco proponi duplex ve- rae et pauperis viduae descriptio, vna ab Etymi ratione et explicatione sumpta, altera ab officio et muuere earum. Nam cum ait Paulus χῆραἔστιν ὸ μεμονομένη vocis ipsius χήρας significationem sy- nonymo explicauit. χῆρα enim ἀπὸ το͂ χηρο͂ν, quod est desolare, destituere, orbare proculdubio de- ducitur. Hoc autem ipsum vox sequens μεμονομe- vn plane designat, quae solam esse, desertam, et praesidio destitutam significat. Id quod ex Ge- nes. 15. vers.3.videtur sumptum. Sed a Latinis ipsis viduertas dicta est calamitas, vt scribit Sext. Pom- peius. Haec igitur μονώτης et solitudo non tantum marito priuatam esse oportere hanc viduam do-  \pend
\section*{AD I. PAVL. AD TIM. }
\marginpar{[ p.258 ]}\pstart cet (neque enim aliter vidua dici posset) sed et li- beris, et cognatis, et omni alio humano praesidio destitutam. Quae iam conditio plane est afflictissima et commiseratione, atque Ecclesiae et pastorum cura dignissima. Quare ipso viduae nomine, id est, desertae et destitutae mulieris appellatione, quae certe statum miserabilem denotat, illae nobis con- mendatissimae esse debent. Vbi enim melior elee- mosina, aut vbi iustior misericordia esse potest, quam vbi est maior miseria atque afflictio? Neque tamen, si quae viduae eos liberos aut propinquos habeant, a quibus praeter officium deserantur, sunt ab Ecclesiae eleemosyna arcendae, et deserendae, quia perinde est tales cognatos habere, atque o- mnino nullos. Haec igitur est prima viduae defini- tio. Sequitur altera, ex qua maior etiam commen- dationis earum ratio apparet. Eas enim vere vi- duas appellat, quae Pietatem habent, et Eam ex- ercent suo specialique modo, qui eas deceat. Quod enim de viduis ait hoc loco Paulus, proprium quodammodo earum esse videtur, non etiam con- mune caeterarum mulierum. Est vero vtraque definitio coniungenda, vt quae sit vera vidua, pos- sit agnosci. Neque enim omnis quae sola est et de- stituta alieno auxilio, est vera vidua, si piam se non praestet. Neque omnis mulier, quae pia est, vidua quoque est, aut ea vidua, de qua hic Paulus agit, quae ab Ecclesia debeat ali, veluti si non est μεμo- νωμένη. Neque enim omnes, etiam, quae solae sunt et desertae, piae sunt: sed quae vtrunque misera ex- peritur, ea demum vere est vidua. Spes autem in Deum hoc loco continet πειγραφην  \pend
\section*{CAPVT  V. }
\marginpar{[ p.29 ]}\pstart fidei, cuius spes est effectus. Itaque perinde est, atque si diceret Paulus eam esse viduam, quae veram fidem habet. Apposite tamen spei potius ( quae aduersus res duras et afflictas pugnat) quam fidei nunc meminit, quia de afflictis personis agit, quas spes corroborat, vt intelligamus qualem se prae- stare Christianam viduam oporteat in rebus ac- cisis, et cûm se omnino destitutam videt. Nimi- rum igitur in Deum sperare debet, non autem animi diffidentia desperare, aut propter huius- modi res aduersas veram religionem abiurare, aut instar infidelium mulierum in Deum ipsum impie murmurare, et blasphemare Permaneat in supplicationibus,) Hoc iam proprium est Christianarum viduarum munus, et earum quae sunt maxime ab Ecclesia subleuandae, vt ipsis externis actionibus et operibus se totas Deo esse deditas atque consecratas ostendant. Itaque σu- νεκδοχικῶς Paulus a parte et specie totum genus piarum externarumque actionum expressit. Si- mile quiddam de Anna Phanuelis filia et pro- phetissa narrat Lucas 2.vers.37. et 1. Samuel.2.v. z2 facta est eius consuetudinis mentio. Ergo assi- duitas ista precandi demonstrat huiusmodi vi- duam esse debere, quae, omnibus aliis rebus post- positis soli Deo, ipsiusque negbtliis de Ecclesiae inseruiat et vacet. Denique ea curet, quae sunt Domini, non autem mundi huius, vt loquitur i- dem Paulus 1.Corinth. 7. vers.32. Nam quae nu- pta est mulier, varia sollicitudine, variisque curis et cogitationibus saepe propter familiam distra- hitur, ne tam libere tamque assidue, et frequen- ter precibus publicis interesse, et de Deo ipso  \pend
\section*{AD I. PAVL. AD TIM. }
\marginpar{[ p.260 ]}\pstart cogitare possit. Haec igitur sunt quodammodo propria viduarum opera, quia haec assiduitas pre- cationum non potest pari modo a reliquis exigi quae non sunt, vti viduae, ab omnibus impedimen- tis solutae, sed occupantur in iusta rerum dome- sticarum cura, qualis erat Martha Luc 1o.vers.4o. Obstant vero duo huic Pauli praecepto: e qui- bus primum est, videri Paulum magnam homi- num animis superstitionem iniicere, cum Chri- stianas mulieres vult diem et noctem in preci- bus permanere. Atque istae videntur fuisse super- stitiosae Monacharum et Monialium instituenda- rum causae, nimirum vt illae nihil nisi precarentur: et tum pro se, tum pro aliis, preces Deo assidue funderent. Cuius etiam vitae exemplum Messalianorum haereticorum erroribus, et moribus inductum est, sed est prorsus a Dei praeceptis alienum. Itaque Psallianorum haereticorum surculus vel potius propago miserrima fuit istud Monachorum in- stitutum, vt ex Cassiano, qui prolixe omnes istas nugas descripsit, facile colligi potest. Respond. Non laudari hic a Paulo otium illud inutile et superstitiosam precationum institutionem, qua- lem hodie habent obseruantque Monachi, vel e- Ciuiii alii Papistae superstitiosi, qui suas etiam pre- ces venales habent, quemadmodum olim inter Ethnicos praeficae illae in funeribus adhibitae mer- cenariae: sed tantum ostendit Paulus ecquod sit verum et legitimum earum, quae nullis negotiis domesticis distrahuntur, exercitium, nempe, vt in iis quae sunt Dei, prorsus se occupent, qualis est ni- mirum precatio, concionum assidua frequenta-  \pend
\section*{CAPVT  Ve }
\marginpar{[ p.261 ]}\pstart tio, etreliqua huiusmodi pietatis et charitatis ve- rissima officia et exercitia. Neque tamen cum haec officia a viduis requi- rit Paulus, ab iisdem reliquos homines et mulie- res, veluti coniuges, excusat: sed ex comparatio- ne, quae etiam versiculo proximo sequitur, docet, quod sit potissimum viduarum mulierum, quae solae sunt atque sine familia, iustum studium ac exercitium. Neque etiam hoc loco vlli verborum demur murationi, quantunuis assiduae, Paulus pie- tatem alligat, sed ex opere externo, quae sit inter- na fides et pietas, et vt deprehendi possit, solum demonstrat. Nam potest in ea ipsa assiduitate precandi inesse damnabilis superstitio, quam non probaret certe quidem Paulus, quales quaedam mulierculae cerauntur maxime in Papatu, quae omnes aras atque omnia templa quotidie cir- cumeunt, tanquam in iis Deus habitet, aut cola- tur sanctius: aut etiam ipso precum et vocum re- petito numero delectetur, contra Domini nostri Iesu Christi sententiam Matth. 6. vers.7. Itaque res suas domesticas, curamque familiae et mariti praeter officium deserunt. Quid de iis dicam, qui coniuges cum sint, etiam in Iesuitarum, id est, pessimorum idololatrarum ordinem, inuitis ma- ritis, cooptari se volunt? Obstat item quod in Actis Apostolorum 9.v. 39. affertur, vbi alia viduae Christianae exercitia describuntur, nempe opera manuum, labores corporis, qualia sunt nere, suere, sarcire et cae- tera huiusmodi, quae magnam humanae societati commoditatem afferunt. Respond. vtrunque re- cte inter se conuenire et consentire. Nempe Deï  \pend
\section*{262 AD I. PAVL. AD TIM. }\pstart \phantomsection
\addcontentsline{toc}{subsection}{\textit{6 At quae in luxu viuit, ea viuens mor- tua est.}}
\subsection*{\textit{6 At quae in luxu viuit, ea viuens mor- tua est.}}cultui interesse assidue, quod hic describitur: et pauperum curam habere, ac ea, quae prodesse illis possunt, parare, quod faciebat Dorcas. Vnum c- nim alterius comes est. 6 At quae in luxu viuit, ea viuens mor- tua est. Αντίθεσις est, eaque duplex, Vna quidem, quae superiorem sententiam illustrat ex contraria vi- duae descriptione. Altera, quae in lpsius versiculi sententia, quanquam breuissima, latet, habetque magnam emphasin, pulchramque insolentis vi- duae picturam, quam inf. vers. 13. copiosius postea persequitur Paulus. Primum igitur lasciuientem et luxu diffluen- tem viduam opponit verae viduae Paulus. Neque enim huiusmodi ab Ecclesia ali vult, imo potius serio admoneri, vt exitium suum intelligat et ap- prehendat. Vocat autem eam παταλῶσαν, qua voce idem significare videntur Graeci quod vo- cabulo ἐντρυφαν, vt apparet ex Iac.5. vers.5. Eae vero sunt huiusmodi omnes, quae cum viris sunt orbatae, delitiis, corporis voluptatibus, lasciuiae indulgent, item sumptuosis vestimentis, magnis virotum coetibus inuisendis delectantur, seque ita et corpore et vultu et cultu componunt, co- munt, et comparant, vt viris placeant, arrideant, ac allubescant, quales pene sunt hoc nostro secu- lo infinitae. Itaque illae etiam ipso Dei flagello miserrimae gaudent, et laetantur, quod viros suos, id est, optimum vitae suae subsidium amiserint. Nam hac ratione se ad omnem lasciuiam perpe- trandam liberiores esse iam putant. Certum est  \pend
\section*{CAPVT V. }
\marginpar{[ p.263 ]}\pstart autem semper apud probas mulieres, et bene moratas etiam ciuitates alium fuisse viduarum, quantunuis iuuenes adhuc essent relictae, et alium coniugum mulierum habitum, et cultum vti ex libro ludith. et historia Thamaris Genes. 38.v. 14. apparet et ex Tertull. non quidem, vt aliqua superstitio in cultu induceretur: sed vt modestia et honestas personarum et ordinum in Ecclesia conseruaretur. Quae certe distinctio honestati, atque etiam ipsius naturae sensui consentanea propter dissolutissimos nostri huius seculi mores iam sublata et antiquata est, adeo vt magis com- ptae et dissolutae viduae nostris his temporibus nunc incedant et cernantur, etiam quae se Chri- stianas, si Deo placet, appellant, quam et coniuges et maritatae. Alterum autem est, quod etiam hoc loco ἀντί- θεσν continet, nempe quod ait Paulus, ζῶσα τόθνηκε id est, viuens mortua est. Quae enim potest esse haec vita, quae tota offendiculi plena est, et certis- simum futurae mortis aeternae indicium ac testi- monium? Haec autem sententia quanquam con- tradictionem implicare videatur, vt aiunt, et prorsus ἀδύνατος esse, facile tamen explicari po- test, quemadmodum docet hoc loco Chrysosto- mus. Est enim duplex, ( Vita, 5 Corporis, et - et Animi  Corporis (Mors? Animi Saepe autem vita corporis cum animi morte con- iuncta est, vt hoc loco. Saepe autem disiuncta ab ea, vt cum pii corpore moriuntur, vt animo vi-  \pend
\section*{264 AD I. PAVL. AD TIM. }\pstart \phantomsection
\addcontentsline{toc}{subsection}{\textit{7 Haec igitur denuntiato, vt fint irre- prehensae. 8 Quod si qua suis et maxime domesti- cis non prouidet, fid em abnegauit, et est infi- deli deterior.}}
\subsection*{\textit{7 Haec igitur denuntiato, vt fint irre- prehensae. 8 Quod si qua suis et maxime domesti- cis non prouidet, fid em abnegauit, et est infi- deli deterior.}}uant: mundo moriuntur, vt Deo viuant. Ergo vi- duae lasciuae viuunt quidem corpore, animo ta- men mortuae sunt. Sic infideles dicuntur et a Christo et a Paulo mortui 1. Corinth.15. vers.29. Matth.8.vers.22. Sunt enim ad omnia quae ad vi- tam aeternam pertinent, immobiles, et inepti, et inutiles, vt ait Chrysostomus: vnde perinde, atque mortuae merito inter pios censentur hae viduae. 7 Haec igitur denuntiato, vt fint irre- prehensae. Haec repetitio, siue potius admonitio a Paulo facta superiorem doctrinam valde vtilem esse docet, neque semel tantum pronuntiandam, sed saepius in Dei Ecclesia inculcandam, vt animis ho- minum inhaereat altius. Eiusdem autem maximus fructus hic ostenditur, cum, si obseruetur, omne offendiculum sublatum iri dicat Paulus, foreque Ecclesiam ipsam irreprehensibilem, et boni apud omnes odoris. Fidelibus autem et infidelibus bene olere debemus omnes. 8 Quod si qua suis et maxime domesti- cis non prouidet, fid em abnegauit, et est infi- deli deterior. Α’ιτιολογ́α est superioris canonis de parenti- bus alendis a liberis, et contra, quae tum a turpi ducta dici potest, tum etiam a repugnantibus: est autem haec sententia generalis, quae tam ad ma- res, quam ad foeminas pertinet. Itaque relatiuo communis generis vsus est Paulus. Hoc autem pro certo et confesso sumit, quod pii fatentur, debe-  \pend
\section*{CAPVT  V. }
\marginpar{[ p.265 ]}\pstart re Christianum quemlibet, siue marem, siue foe minam longe sanctiori viuendi ratione omnibus innotes cere, quam infideles homines soieant: adeo vt quaecunque vel honestas ipsa, vel iustitiae et pieta- tis leges postulant, ea lubentissime omnia prae- stare studeat. Sic Christus Math.5.vers.46. Nam si dilexeritis, eos qui diligunt vos, quam mercedem habebitis, nonne publicani haec faciunt. Est etiam huius hypotheseos et sententiae ratio apertissima ex definitione Christiani viri, et Christianae do- ctrinae siue religionis. Est enim vir Christianus, qui caeteris hominibus vitae sanctioris facem in omnibus praefert. Math.5. Est etiam Christianis- mus vt Basil. definit Homil.1o. in Examero. no- stra cum Deo similitudo, quantum humana natura illius capax esse in hoc mundo potest. Est igitur vir Christianus non tantum, vt caeteri, ad Dei glo riam natus: sed ad Dei optimam similitudinem et perpetuum obsequium deo praestandum et for matus et intentus. Denique vt ait Paulus. 2. Ti- mo. 3.vers. 17. ὁσιν ἀνθρωπος πρὸς πᾶν ἔργον ἀγαθὸν ἔξηρτισμένος. Id quod de infidelibus hominibus di ci minime potest. Itaque Paul Rom.13. et Petrus I.epist. cap.3.latum discrimen inter fideles et insi deles constituunt: veluti quod appellanturfideles filii lucis, homines Dei, et serui Dei. At infideles appellantur filii tenebrarum, filii diaboli, et serui peccati. Ergo hic canon est, Christianus quilibet vt caeteris exemplum sit pietatis, suorum omnium, imprimis autem domesticorum curam habeat. Qui secus faxit, fidem abnegasse, et infideli deterior esse censetor. Voce προνοεῖν et Animi et corporis no strorum curam complectitur demandatque Pau  \pend
\section*{AD I. PAVL. AD TIM. }
\marginpar{[ p.266 ]}\pstart lus quamquam de alimentis maxime agit hoc loco. Neque enim qui corporis et vitae huius caducae curam habet, is animum, qui longe potior homi- nis pars est, debebit negligere: sic interpretatur Chrysosto: et confirmatur Deuter.11.vers.19. E- phes.6.vers.4 in verbis ἐν παιδεία καὶ νυθεσία κυείκ. ι Suorum et Domesticorum) omnes intelligit Paulus, qui ad nos siue iure sanguinis, siue iure fa miliae et subiectionis pertinent. Est enim omnium illorum a nobis cura habenda. Quamquam enim om nes non sunt pari gradu coniuncti nobis, etiam qui sanguinis necessitudine sunt deuincti, nulli tamen sunt a nobis spernendi et deserendi. Quanquam etiam saepe in alienas familias transiisse videntur nostri liberi, vel per adoptionem ciuilem: vel per coniugia: propterea tamen ius illud et vinculum sanguinis non est abolitum, neque abruptum, vt eos negligere debeamus. Iura enim sanguinis nul- Io modo tolli possunt. Vnde perpetuo ad nostrum erga consanguineos et cognatos officium prae- standum, ac curam de iis habendam iure naturae, di uinoque praecepto manemus obligati. Hos autem gradus statuit hoc loco Paulus, vt Domesticos primum nobis commendet, et praeferat iis, quos ἰδυς appellauit. Ratio est, quod hi dome- stici, quos intelligit Paulus et sanguis sunt noster: atque etiam nostrae familiae pars, quae nobis praeci pue commendatur: et cui a Deo preficimur spe- cialiori atque arctiori quodam vinculo. Non vult autem seruos a nobis propinquis et cognatis prae ferri, etsi serui sunt domesticorum numero fidem ab- negat qui secus facitait Paulus: duplicem notam huius modi hominibus inurit et Infidelitatis ferinae vel  \pend
\section*{CAPVT  V. }
\marginpar{[ p.267 ]}\pstart beluinae crudelitatis: vel etiam saeuioris quam ferina est. Quae vtraque ab homine debet certe quidem esse alienissima. Itaque hoc argumentum est a re- pugnantibus humanae et Christianae charitati mo ribus ductum. Fidem abnegare est non verbis ipsis disertis abiurare fidem in Christum, illius doctrinam, quam antea amplexus eras, iam exspuere: sed vim et fructum illius ac praecepta, quae fides Christiana tradit, spernere et conculcare. Qua ratione dicit idem Paulus Tit. 1. vers. 16. multos ore quidem profiteri Christum, qui factis eum abnegent, quoniam illi sunt Dei verbo immorigeri et re- belles. Docet autem fides Christiana non tan- tum omnem paternitatem et cognationem ho- miuum inter ipsos essea Deo. Itaque sperni sine Dei ipsius contemptu non posse, vt docet Pau- lus Ephes.3 vers,15. sed etiam eam, ex illa paterni tate, qua Pater Filium generat, quodammodo profluere. Docet etiam nos esse hac lege Deifi- lios, vt illi tanquam patri obsequamur, et quam cuique nostrum dedit pater ille clementissimus curam, prouinciam, locum, et vocationem, eam lu bentissime tueamur: neque eam per incuriam aut negligentiam deseramus. Praeceptum enim Dei Honora patrem et matrem etiam parentes ad cu- ram liberorum et Domesticorum suorum ob- stringit. Videtur autem dici posse in istos crudeles ali- quid amplius, nimirum eos qui suos negligunt, esse non tantum infidelibus peiores: sed etiam bru tis ipsis animantibus deteriores et stupidiores, quae exse nata suscipiunt, fouent, amplectuntur, eteducant,  \pend
\section*{AD I. PAVL. AD TIM. }
\marginpar{[ p.268 ]}\pstart ipso naturae instinctu tantum impulsa, id quod cer te verum est. Obstare autem videtur huic compa- rationi quod Infidelitatis et contempti muneris notam fere facit ἀσορίαν Paulus, vti in epistola ad Roma 1.vers.31.2. Timoth.3. vers.3. At hic etiam σοργάς φυτικὰς Infidelibus hominibus relinquit. Resp. Docere Paulum, cum ita Roma.1.vers.31. lo quitur: quo tandem delabantur infideles, qui Deum contemnunt: non propterea tamen eos omnes, qui caelestis veritatis luce destituuntur, ἄςτόργυς esse omnino pronuntiare. In quibusdam. n. infidelibus manent istae storgae. At obiici potest, Multi infideles fuerunt ἀςτοργον, et suorum negli- gentes, quos etiam feris ac bestiis saepe obiece- runt ac exposuerunt. Respon. Paulum hic loqui de iis, qui naturam humanam non prorsus exue- runt quales inter eos sunt multi: qui naturales il- los et insitos a Deo omnibus hominibus affe- ctus retinent et sentiunt. Disputat hic insulse Tho mas, vtrum ex hoc Pauli dicto colligi debeat pa- tres fideles suorum liberorum et domesticorum negligentes, esse ἀπλῶς et omnino peiores et de- teriores infidelibus et ethuicis, vt nullaiam in re iis sint praeferendi: sed sint vtrique inter se aequa les et per omnia conferendi, adeo vt quemadmo dum infidelium paruuli baptizari non possunt, ne- que etiam istorum sidelium paruuli baptizari de- beant. Respon Hoc Pauli dictum esse κατάτι acci- piendum. Qyoad ipsos sideles hypocritas, plane in exitium illis cedunt Dei dona vt mitior sit fu- tura conditio et poena Sodomae, quam istorum con- temptorum Dei. Quod autem ad ipsa Dei dona, qualis est baptismus suum habent effectum, et firma  \pend
\section*{CAPVT  V. }
\marginpar{[ p.269 ]}\pstart \phantomsection
\addcontentsline{toc}{subsection}{\textit{9 Vidua allegatur non minor annis se- xaginta, quae fuerit vnius viri vxor.}}
\subsection*{\textit{9 Vidua allegatur non minor annis se- xaginta, quae fuerit vnius viri vxor.}}sunt, atque adeo valent ad eum finem, ad quem Deus dedit illa. Itaque vti ingratitudo Iudaeorum non aboleuit signum foederis nempe circuncisio nem: sic neque improbitas parentum efficit quo minus liberi ex foedere in populo Dei censean- tur. Particularisquidem effectus donorum Dei in ipsis improbis nullus est, generalis autem effe- ctus eorundem manet propter baptismum et foe dus Dei, vt propterea non sint omnino cum in- fidelibus exaequandi, nec debeant eorum liberi a Baptismo arceri. 9 Vidua allegatur non minor annis se- xaginta, quae fuerit vnius viri vxor. Τάξις. Ingressus enim Paulus disputarionem de viduis, quae ab ecclesia aluntur, canonem ad- dit de iisdem valde necessarium, maxime pro ra- tione et illius temporis, et huius quod postea sub secutum est. Nam nostro hoc seculo in omnibus reformatis ecclesiis cessat huius canonis et prae- cepti vsus. Ex quo apparet, quae ad particularem ecclesiae politiam pertinent et pro ratione tantum temporum aut locorum aut personarum consti- tuta et recepta sunt, ea non esse perpetua:sed pro diuer sa temporum, locorum, personarum etc.cir constantia mutari posse: quemadmodum etiam ex iis apparet, quae a prima synodo Hierosolymita- na, quae ab Apostolis habita est, decreta sunt A- ctor.15. Neque enim illa semper fuerunt obserua- ta, quod in eo cernitur, quod de suffocato consti- tutum est. Illud enim prorsus hodie et cessat et sublatum est. Hic igitur canon de Viduis a Paulo sancitur,  \pend
\section*{AD I. PAVL. AD TIM. }
\marginpar{[ p.270 ]}\pstart vt inter diaconissas ecclesiae, ea demum vidua al legetur, quae sexagenaria est. In quo vox haec, καταλσγόθω non tantum additionem ad numerum iam constitutum et ascriptionem significat: sed etiam nouam eius ordinis in aliqua ecclesia, si quae iis egeret, nullas dum tamen haberet, electio- nem. Id quod fieri necesse erat in iis ecclesiis, vti dixi, in quibus nullae adhuc diaconissae erant, et quae tamen iis opus haberent. Itaque et deligi et allegi significat hoc loco de nouo creare, si o- porteat, et in album siue numerum iam electarum adscribere. Loquitur autem Paulus de Vidua fideli, et quae pietatem veram iam professa sit: non autem de infideli, id quod exsuperiori disputatione satis apparet. Duo autem spectari in iis viduis iubet Paulus e quibus alterum Abesse debet, alterum Adesse. Abesse haec duo iubet, vt sit minor sexagenaria et Duos maritos habuerit. Adesse autem vult vt I Liberos suos educauerit. 2 Hospitalis fuerit erga extraneos. 3 Misericors erga afflictos. 4 De omni bono opere testimonium idoneum habeat. Itaque ait Chrysosto. eadem pene hic a Paulo in vidua requiri, quae supra in episcopis ip- sis. Et certe cum hae viduae munus laliquod in ec- clesia habuerint, idque publicum, tales deligi o- portuit, quae essent aedificationi omnibus siue fi- delibus, siue insidelibus et extraneis futurae: ne Dei gloria, et ecclesiae progressio impediretur. Hoc autem vt intelligatur, totum hoc argumem tum altius arcessendum est, nempe ex eo loco, qui Actor.6.versr. et 9. vers.39. Rom.16. vers.1.2. Nam  \pend
\section*{CAPVT V. }
\marginpar{[ p.271 ]}\pstart legimus primam et Apostolicam ecclesiam ab ip- so sui exordio habuisse mulieres Viduas, impri- mis ad certa quaedam et quotidiana tum erga v- niuersam ecclesiam ipsam, tum erga fideles pau- peres et egenos fratres ministeria destinatas, quae postea Diaconissae ad imitationem marium, qui Diaconi dicebantur, appellatae sunt. Sic enim Phoe ben appellat Paulus Rom.16.vers.1. Cur autem id factum sit et inductum in ecclesiam quaeri po- test. Nam non modo Paulus 1. Corint.14. et supr. 2.vetat mulieres loqui in ecclesia, et publicum aliquod munus gerere:sed etiam quaecunque le- ges politicae recte vtriusque sexus discrimen ag- nouisse videntur, quales sunt Romanae, idem fa- ciunt. Illis enim mulier prohibetur publicum ali- quod munus exercere l. Mulier.5.de Regul. Iuris. Cur igitur et sibi ipsi iam non consentire vide- tur Paulus: et recte constitutum illud inter mares et foeminas discrimen tollere, atque adeo natu- rae ipsi tanquam aduersari? Resp.neque sibi, ne- que naturae repugnare hoc Pauli praeceptum: vel antiquissimum illum ecclesiae Christianae morem, quae huiusmodifoeminas ad pauperum ministerium accersiuit, deditque illis in ecclesia munus atque aliquam dignitatem: publicam quidem ratione fructus et vtilitatis, quae ad omnes ex ea proue- niebat. priuatam autem ratione iniunctae ipsis pro- uinciae, et descripti deffinitique certis augustisque finibus carum muneris. Differunt enim hae Dia- conissae, atque earum munus a Diaconis: quod Diaconi toti aerario ecclesiastico et curae vniuer- sorum ecclesiae pauperum praeficiuntur: hae vero certo cuidam ministerio tantum, atque adeo sub  \pend
\section*{AD I. PAVL. AD TIM. }
\marginpar{[ p.272 ]}\pstart Diaconis sunt, quibus subiiciuntur et parent, red- duntque rationem suae vocationis: Diaconi vero ipsi, toti ecclesiae. Vnde non tam fuit munus hoc publicum, quam subsidium aliquod a Diaconis quaesitum: partim, quod vniuersis ecclesiae pauperi bus subuenire propter numerum tunc non pos- sent: partim, quod vt et Epiphanius et Chrysosto. ait, pleraque circa pauperes vel egroros, vel men sas ipsas ecclesiae parandas commodius a foeminis administrantur, quam a maribus: nec modo commo dius: sed etiam honestius fiunt et geruntur: vt si aegrotet Christiana mulier, vel aemorroide labo- ret, aut aliis huiusmodi morbis occultis, et quae ho neste a maribus tractari non possunt. Ad eam e- nim honestius alia mulier admittitur, quam mas aliquis Diaconus et medicus: ergo honestas hoc munus suasit admittendum. Deinde necessitas. Neque enim omnia munia et ministeria primum obire potuerunt soli Dia- coni, cum tota ecclesia, more Laconum, simul epu- laretur, et Α’γάπας et conuiuia communia celebraret in testimonium mutuae charitatis. Sic enim fiebat et vt alii aliis. i. ditiores pauperibus subuenirent, ac ea quae erant victui necessaria tanquam in con- muni suppeditarentur Actor.4. et 6. Itaque ali- quae subsidiariae oper ae et ministeria, eaque com- moda Diaconis quaerenda ad mensarum admini- strationem fuerunt, ad quae ascitae sunt hae foeminae. Quas vetustiss. et primaeua ecclesia habuit. Dein- de vero caeterae ecclesiae, imprimis autem eae, in quibus diligens pauperum cura et ratio habita est, eum morem, easque Diaconissas retinuerunt etiam post Apostolorum tempora.  \pend
\section*{CAPVT  V. }
\marginpar{[ p.273 ]}\pstart Tertio videntur hae mulieres diaconisse ad Mar thae imitationem toleratae vel inductae in ecclesia, atque ad exemplum earum mulierum piarum, quae Christo ministrabant, de quibus Lucas 1o. vers. 41. 8.vers.3. Vtcunque vtilis visa est eccle- siae pro tempore illa aecessio et subsidiaria mulie- rum opera, antequam essent in Christiana eccle- sia Ptochodocheia, Nosodocheia et Hospitalia quae dicuntur, quod commodissimum esset et ma xime idoneum mulierum ministerium quibusdam in rebus, atque adeo aptius, quam Diaconorum. Vnde quae aetas secuta est tempora Apostolica, Diaconissas habuit, vt ex conciliis Neocaesarien- si, Laodiceno, et Chartaginensi apparet, et ex ve- terum scriptis, imprimis autem Epiphanio, qui vixit sub Theodosio maiore, et historia ecclesia- stica: adeo vt magnae dignitatis postea huiusmodi Diaconissae in ecclesia fuerint. Nam Ischyrias Dia- conissa interfuit synodo Chalcedonensi quemad- modum Euagrius scribit lib. 2. cap.16. et Pulche- ria etiam ipsius Theodosii Iunioris Constantino pol. Imperatoris soror in Diaconissam electa est vt ex Hermio Sozom. facile colligi potest. Praeter Diaconissas autem neque Presbyteri- das, neque episcopissas mulieres habebat Chri- stiana ecclesia, quanquam Pepuziarii haeretici nullam Maribus dignitatem in ecclesia tribuerent, quam eandem etiam foeminis non concederent. At ex Pauli praecepto aliud semper orthodoxa eccle sia obseruauit. Nullum enim huiusmodi munus vn- quam foeminis delegauit, quod in docendo versare- tur. Itaque Epiphani. Haeres. Collyridianorum quae est 79. sic scribit. Hρεςθυλερίδες ἠιερίσταιούκ ἐιοιν  \pend
\section*{AD I. PAVL. AD TIM. }
\marginpar{[ p.*24 ]}\pstart ἐνἰεκλνσία ιερκεγῆν φἐν ἐακλνσιά μὸ δυναται. Iam vero ad Paulum veniamus, et ea quae requi- rit in istis Diaconissis. Ne minor sexaginta annis. Ratio est duplex, quod Iuniores oportet dare operam liberis et q lasciuiunt quae nondum eam aetatem attigerunt, quia nondum plane in iis extinctus est vigor gene- randi. Quaeritur vero cur sexagesimum annum statuat Paulus, eum quinquagesimus annus foeminis cen- seatur prorsus sterilis et infoecundus, vt vulgo an notant interpretes in Luc.1 ver.36. et Iurisconsul- ti in l.viuiC. de Caduc. Tosl. Resp. Anno qudem 50. fieri vt plurimum steriles foeminas: non ta- mem semper. Quaedam enim post 5o. annum peperis- se dicuntur. Ac post oo nulla est, quae concepisse referatur, nisi miraculo id factum sit, et praeter na- turae ordinem, id quod Sarae vxori Abrahami ac- cidit Genes.18. vers.ii. . In sexagenariis autem foeminis videtur vigor il Ie naturae et prolificus prorsus periisse et defe- cisse. Sed cur tantum tribuit aetati senili Paulus, qui supra Timotheum adolescentem nobis com mendabat.c.4. vers. 12. Resp. Chrysosto, proprie non tribuere Paulum id honoris aetati, quasi se- Ragenariae lasciuae fieri non possint: sed quod mi- nus in hac aetate offendicula illa nascantur, quae euitanda postea docet. Neque enim ipsa per se aetas facit sapientes vel prudentes homines: sed Deus potius et rerum praeteritarum experientia. Deinde cum leges ferantur de iis, quae ὥς ὀτι τὸ πο- λὸ accidunt, non debet propter paucas aliquot sexagenarias insolentes toti etati suus honor  \pend
\section*{CAPVT  V. }
\marginpar{[ p.27 ]}\pstart detrahi. Iam vero, quia haec capacitas non ex aeta- tis senio et beneficio proprie pendet, imminutum est annorum tempus et spatium a Paulo hoc loco praescriptum, volueruntque homines viderispiri- tu ipso Dei sapientiores. Itaque concilio Chalce- donensi constitutum est, vt statim post 4o annum çtatis eligi mulieres ad istum Diaconatum pos- sent est in cano. Diaconissas 27. quaest. 1.id quod est confirmatum concilio Carthaginensi 4. vt est in cano. sanctimoniales 2o. quest. I.in quo postea va rie peccatum est, vt docebimus. Addit Paulus. Vnius vir vxor. Hoc vero pro- bari potest exemplo Annae, de qua Luc.1.ver. 36. Respondet autem illi, quod de episcopis et Diaconis agens idem Paulus dixerat. Vnius vxoris vir. Itaque ex illo loco facile iam intelligi potest que sit vis huius loci sententia atque mens. Pri- mum enim non damnat nuptias Paulus siue pri- mas, siue secundas, siue tertias siue vlteriores. Hanc enim esse diabolicam doctrinam supra asseru- it cap.4. vers.1.2.3. Deinde non iniicit hunc la- queum viduarum conscientiis, quasi praecipiat, vt in viduitate permaneant, etiam eae, quae denuo vel iterum nubere vellent: neque hoc loco etiam vi- duitatem aut coelibatum tanquam per se sanctio rem praefert honesto coniugio Pauius; vti neque 1.Corinth.7. vers.8.9.34. et 35. Denique cum ne- que ante Pauli aetatem, neque eius seculo vsquam terrarum obtineret: vt vna mulier eodem tempo re duobus nupta esset viris (etsi inter Iudaeos vir vnus solebat ex vitiosa patrum imitatione duas vel plures vxores ducere) satis intelligitur eam dici plurium maritorum vxorem esse, quae quan-  \pend
\section*{AD I. PAVL. AD TIM. }
\marginpar{[ p.276 ]}\pstart quam est ex iniusta causa repudiata a viro, tamen eo viuo alteri non nubet, sed propter vinculum coniugii, et Dei praeceptum, quandiu prior ille maritus, a quo iniuste repudiata fuit, viuit, estque superstes, alterius coniugio abstinet. Quae enim ex iniusta, vel potius nulla causa repudiatur, ma- net vxor: et vel reconciliari debet vel innupta ma nere 1. Corinth.7. vers.1o. Neque tamen impedi- tur, quae ita immerito repudiata est a viro, viro ipso ad alteras nuptias conuolante, quin et ipsa alii nubere possit. Sed hoc praesertim in ea, quae in Diaconatum allegenda est, requirit Paulus, quoniam maioris continentiae signum et exen- plum in ista muliere cerni debeat. Erit autem hoc certissimum, si immerito repudiata, tamen alteri non nupsit, sed continuit. Sed quid si meri- to illa repudiata est, veluti propter adulterium, vel quod ipsasit desertrix coniugii? Nemo dubi- tare tunc potest, quin ea ad hoc munus ecclesia- sticum eligenda omnino non sit, sed ab hochono re sit arcenda. Vult autem Paulus, vt quae est allegenda, fuerit Vxor. Ergo si fuit vel concubina, vel pellex, vel meretrix eligi non poterit. neque enim sine ma- gno ecclesiae offendiculo tales cooptari in vllam ceclesiasticam dignitatem possunt, quamlibet poe- niteant. Ait Paulus. Quae fuerit. Quid si adhuc illa sit v- xor, etiam in illa aetate sexagenaria, num allegi et ascisci poterit, cumillud, quod de episcopo supra dicitur, Sit episcopus vnius vxoris vir, locum ha- beat: siue adhuc sit ille maritus et vir vnius vxo- ris: siue mortua vxore maritus fuerit et iamesse  \pend
\section*{CAPVT  V. }
\marginpar{[ p.2 ]}\pstart \phantomsection
\addcontentsline{toc}{subsection}{\textit{10 In operibus bonis idones testimanio ornata, si liberos educauit, si fuit hospitalis, si sanctorum pedes lauit, si afflictis subuenit si omne bonum opus est assidüe sectata.}}
\subsection*{\textit{10 In operibus bonis idones testimanio ornata, si liberos educauit, si fuit hospitalis, si sanctorum pedes lauit, si afflictis subuenit si omne bonum opus est assidüe sectata.}}desierit. Resp. non esse ex mente Pauli eligendam sexagenariam, quae adhuc vxor manet, quia de vi- duis loquitur Paul': et variis illis muneribus et cu ris nimium illa distraheretur, si iam alligata sit viro suo: et tamen cura etiam pauperum illi sit ge- renda. Vtrique enim oneri ac curae vna satis esse non potest. Nam quae nupta est, inquit Paulus, curat ea, quae sunt mundi, et quomodo placitura sit viro 1.Corinth.7.vers.34. Ergo ea, quae funt Dei et pauperum, libere, quemadmodum Diaco. nissam decet iam curare aut administrare non po- test. Alia vero et diuersa est ratio viri et episcopis qui quanquam cum est maritus, curat, quomodo sit vxori placiturus. 1.Corint.7.ver.31. tamen non perinde in ea cura subiicitur vxori, quemadmodum vxor viro. Itaque manens maritus, munus eccle- siasticum gerere ac obire potest. 10 In operibus bonis idones testimanio ornata, si liberos educauit, si fuit hospitalis, si sanctorum pedes lauit, si afflictis subuenit si omne bonum opus est assidüe sectata. Αύξησις. Amplificat enim ornamenta futurae et eligendae Diaconissae, quod eam publica voce to- tius ecclesiae et aliorum etiam extraneorum ho- minum, qui a fide Christi sunt alieni, consensu vult habere idoneum testimonium et probitatis, et charitatis, et beneficentiae, et curae et diligen- tiae. Hoc idem supra in Episcopis requirebat cap. 3vers.7. ex quo apparet quanti sit momenti pu- blica illa fama et opinio de nobis concepta, modo non temere, sed idoneis argumentis et testibuz fulta spargatur. Est enim velut quidam odor?  \pend
\section*{AD I. PAVL. AD TIM. }
\marginpar{[ p.278 ]}\pstart quem nobiscum circumferimus: et siquidem illa bona est, bonus est et suauis gratusque ille odor, quem emittimus: sin vero illa est mala, est etiam odor a nobis effluens malus, et male olens qui a- lios, cum quibus versamur, offendit. Itaque iubet Paulus, vt quaecunque sunt ἕυρημα.ι. bonum nomem bonis conciliant, et cogitemus et procuremus. Philip. 4. vers. 8. Id quod etiam Hesiod. lib1. ἔργ.καί ημίς. praeclarissime versibus monet. Α’ύξησιν superiorem sequitur ἐξήγησις. Aliquot enim exem plis, et generibus actionum explicat atque illu- strat quod generaliter antea dixerat Paul. Affert autem quatuor eorum bonorum operum species, quae maxime futuro muneri et allegendae diaconis sae quadrant atque conueniunt. Cum enim paupe- rum cura illi committenda sit, quae res et Miseri- cordem animum, et diligentem atque etiam de missum requirit, ea bona opera commemorat Pau lus, ex quibus huiusmodi animus facile argui et deprehendi in nobis potest. Diligentiae signum erit, si liberos proprios stu- diose, et diligenter ipsa educauit et curauit.  Misericors et humanns illius deprehendetur animus, si hospitalis fuit, et afflictis subuenit. Demissus autem illius animus, si sanctorum pe- des lauare non est dedignata. Denique pietas et probitas eius intelligetur, si quodlibet opus bonum assidue et studiose con- sectata est. Haec est breuitaer mens Pauli. Quod si haec tam diligenter in iis, qui ad mi- nora ecclesiae munera vocantur obseruari vult Paulus, id est, Dei spiritus, et ab illis requirit has virtures, quid sentiendum statuendumque nobie  \pend
\section*{CAPVT  V. }
\marginpar{[ p.279 ]}\pstart est de iis, qui ad maiora munera ecclesiastica ac- cersuntur et eliguntur? Haene virtutes ab iis abesse debebunt? Minime sane. Itaque in iis et diligen- tia et misericors animus, et demissus et pius spe- ctetur. Quod ait. Si liberos educauit mouet dubitatione quasi Paulus steriles sexagenarias, quae vxores fuerunt nullosque liberos pepererunt, ab hoc mu- nere arceat, illisque tanquam hanc notam inurat propter sterilitatem, vt eligi ad hoc munus non possint. At ee saepe sunt aptissimae, An non igitur quae fuerunt steriles allegi et cooptari poterunt? Sane poterunt. Neque enim vniuersum iudicium, quod de mulierum piarum diligentia haberi po- test, Paulus restringit ad vnicam curam, et educa tionem susceptorum liberorum: sed hoc tanquam verissimum et optimum inter caetera earum dili- gentiae testimonium protulit. Ergo que mulieres nunquam liberos sustulerunt in coniugio, aut e- tiam quae nunquam fuerunt coniuges, posfunt, si erunt idoneae, in diaconissas eligi.  Quod ait. Si fuit hospitalis, si afflictis subuenit, mouet etiam quaestionem, quasi solas diuites eli- gi et cooptari ad hoc munus velit Paulus. Illae e- nim sunt, quae et extraneos excipere apud se et suis opibus afflictis subuenire ἐπόρκεσιν enim dixit Paulus id est, quantum satis est, sumministrauit egenis) possunt. Quae autem sunt egenae et ipsae 424 pauperes idem praestare nequeunt. Ergo ab hoc munere summouebuntur, quas tamen saepe ma- 4. gis esse ad hoc ministerium idoneas, quam dites mulieres certum est, et experientia comprobatum Resp. Non excludit pauperes Paulus imô maxi-  \pend
\section*{AD I. PAVL. AD TIM. }
\marginpar{[ p.A0O ]}\pstart me de iis allegendis agit hoc loco. Sed neque e- tiam diuites ab eodem munere suscipiendo arcet: caeterum quae pauper est, potest tamen et miseri- cors et benefica, et hospitalis esse, etafflictis sub- uenire, si quantum in se est, illa egenis praestat, si corporis ministerium exhibet, si operam suam il- lis impendit et praebet, quando pecuniam confer re non potest. Denique et vidua paupercula, quae minuta duo in gazophylacium misit, maiorem, quam alii Iudaei opulentissimi eleemosynam contulit: et Petrus ac Ioannes cum pecuniam non haberent, summo tamen beneficio claudum affecerunt, vt est Luc.21.vers.2.Actor.2. Quod idem Paulus ait. Quae lauit pedes.) sum- ptum videri potest partim ex Gene.18.vers.4.19. vers.2.43.vers.24 vbi haec eadem ratio dicendi v- surpatur: partim ex recepto tunc more inter O- rientales homines. Nam cum qui fere pedibus iter faciunt, fole, nimioque labore incalescant, sunt eorum pedes abluendi vt refrigerentur, recreentur et refocillentur. Qui autem equo incedunt et iter conficiunt, non egent fere hoc auxilio et ministerio. Ergo ex consuetudine loci hoc annotauit Paulus. Hoc autem ipsum magnam animi demissionem significat, cum vilissima et sordidissima hominis pars fere sint pedes: quot quutunque ita contre- ctare, et curare non dedignatur, certe nec vilia quaelibet alia munera et ministeria, quae pauperi- bus aegrotantibus impendi necesse est, obire ea- dem vnquam recusabit. Christus ipse in signum humilitatis lauit pedes discipulorum Ioan 13.Idem fecit et illa mulier, quae lacrymis suis lauit pedes Christi. Luc.7 vers38.  \pend
\section*{CAPVT  V. }
\marginpar{[ p.281 ]}\pstart \phantomsection
\addcontentsline{toc}{subsection}{\textit{11 Porro iuniores viduas recusas post- quam enim lasciuire coeperint aduersus Chri- stum, nubere volunt.}}
\subsection*{\textit{11 Porro iuniores viduas recusas post- quam enim lasciuire coeperint aduersus Chri- stum, nubere volunt.}}11 Porro iuniores viduas recusas post- quam enim lasciuire coeperint aduersus Chri- stum, nubere volunt. Αʹντίθεσις est, ex qua superior sententia non tantum illustratur: sed etiam probatur. Ex eo e- nim, quod subiicit Paulus, apparet, cur demum sexagenarias mulieres allegi voluerit et praescri- pserit. Varia enim offendicula, quae a iunioribus oriuntur, id suaserunt. Itaque non ea est causa cur sexagenariae demum eligantur, ne diutius a- lantur diaconissae ab Ecclesia: sed ne scandalum praebeant vitae mutatione et lasciuia. Vult autem vt recusentur iuniores, etiamsi se offerant, et paratae sint hoc munus acceptare: vel etiam si aliâs idoneae viderentur. Quod vero, ait Paulus παραιτο͂, non est eo modo accipiendum, quasi in solius Timothei arbitrium totam hanc electionem conferat, quam tum ex Presbyterii, tum ex totius. Ecclesiae consensu fieri solitam esse certum est: sed ita loquitur Paulus, quod pastor omnium Ecclesiae actionum et electionum mo- derator est. Notandum est autem, quod viduas iunio- res tantum recusari et electas a munere diaconis- sae deponi ac reiici vult, etiamsi postea nubant: non autem praecipit eas excommunicari et eiici de Ecclesia: vel vsu et participatione Sacra- mentorum arceri et prohiberi, quemadmodum de Nonnis et Virginibus postea Synodis ineptis- simis sancitum est. Atque vtinam haec Pauli moderatio semper in Ecclesia locum habuisset: sed postea longius progressi sunt Episcopi, maxime postquam nata est superstitiosa illa de coelibatus merito et  \pend
\section*{AD I. PAVL. AD TIM. }
\marginpar{[ p.262 ]}\pstart dignitate opinio. Primum enim sanciuerunt Ec- clesiastici canones, vt vidua iunior, atque etiam vt Monialis, quae post votum nubit, inter diga- mos tantum censeretur, et reponeretur, can. Quot- quot 27. quaest.I Post autem inualesc ente super- stitione sancitum est, in eas aliquid acerbius. Vo- luerunt enim, vt si qua Monialis post votum nu- psisset, vt non tantum Sacramentorum vsu pri- uaretur: sed etiam a toto Ecclesiae coetu excom- municaretur, quia votum de coelibatu et conti- nentia violarat, quod Deo gratum esse falso et insulse crediderunt. Est haec poena constituta in can. Virgines 27. quaest.1.qui ex consilio Eliberti- no decerptus est. Sed iam quae sint illa offendicula, quae praebent viduae iuniores ad Diaconatum electae videamus. Sunt autem imprimis haec duo nimirum, quod I Institutum vitae genus in magnam contumeliam Christi commutant magnam animi leuitatem et carnis lasciuiam demonstrantes. 2 In hoc ipso vitae genere suscepto si perstent, magna tamen Eccle- siae praebent offendicula, dum sunt Otiosae, Di- scursatrices, Nugaces, Garrulae. Caeterum cum improbat Paulus quod nubant hae viduae, quae iam electae sunt, non ideo facit, quasi secundas nu- ptias damnet, vel etiam tertias, et quartas, id quod Montanistarum haereticorum, aut Papistarum Montanizantium proprium est, sed quod tale quid fit, oritur ex animi magna, tum Leuitate, quae gra- ues matronas et eas quae sunt hoc munus publicum adeptae, non decet: tum Lasciuia siue infreni ac indomita carnis petulantia, quam tamen abiecis- se debuerant, et exuisse. Id quod etiam cum Chri-  \pend
\section*{CAPVT V. }
\marginpar{[ p.283 ]}\pstart \phantomsection
\addcontentsline{toc}{subsection}{\textit{12 Ex eo damnandae quod primam fi- dem reiecerint.}}
\subsection*{\textit{12 Ex eo damnandae quod primam fi- dem reiecerint.}}stiani nominis et doctrinae contumelia coniun- ctum est. Irridetur enim Dei Ecclesia, quod tam leues homines habeat, vt qui munus a se susce- ptum statim abdicent, relinquant, et huic mundum et carnis delicias praeferant. Quasi Christus a suis parui fiat et contemnatur aperte etiam iis, ipsis qui tamen maiores in fide progressus fecisse videbantur. Vox καταςρηνιάζην cum sequenti, id est, χειςοͅ construitur. Est autem ςρηνιαζεην ἀπὸ τωὺ ςερεῖν vel στρηνὲς deductum, quod durum significat et perti- nax: qualia sunt animantia nimium saginata. Ita- que sunt petulantiae plena et indomita, vnde ci- bus est illis detrahendus, vt in ordinem cogantur et pareant, quemadmodum et Xenophon et i- psa rerum experientia faciendum docet. Hac igi- tur similitudine vtitur Paulus, vt carnis nostrae lasciuiam eo turpiorem ac foediorem esse demon- stret, quod brutis et petulantibus animalibus con- paratur. Chrysostomus autem interpretatur hanc vocem fornicari, sed Apocalyp.18.vers.7. et 9. explicatur eo significato, quo nunc accipimus. 12 Ex eo damnandae quod primam fi- dem reiecerint. Αʹὐξισις est. Amplificat enim harum iuniorum viduarum lasciuiam a consequenti, quod hac ra- tione, et vitae genere tandem sibi extremum exi- tium acceasant. Nam eo tandum prolabuntur, vt Munus a se susceptum deserant, et Fidem ipsam primam Christo datam abnegent. Voces autem has πρώτηνπιςιν varie interpretes accipiunt. Alii  \pend
\section*{264 AD I. PAVL. AD TIM. }\pstart \phantomsection
\addcontentsline{toc}{subsection}{\textit{13 Simul autem etiam otiosae discunt circumire domos: imo non solum otiosae, sed etiam nugaces et curiosae, garrientes quae non oportet.}}
\subsection*{\textit{13 Simul autem etiam otiosae discunt circumire domos: imo non solum otiosae, sed etiam nugaces et curiosae, garrientes quae non oportet.}}de voto non nubendi sumunt, quasi ab iis inter- poni huiusmodi votum solitum fuerit, cum eligeban- tur: quod nullo certe scripturae loco probari po- test. Alii de fide Christo in Baptismo data:alii de munere Diaconatus acceptato, quod diligenter se praestaturas esse, in eoque perseueraturas polli- cebantur. Quod postea cum nubendi pruritu et libidine deserunt, fidem a se datam violant, ac frangunt. Sed ea sententia magis placet arg. vers. 15.infra, quae fidem primam refert ad eam pro- missionem, quae primum fit a Christo nobis in Baptismo, quam reiicere dicuntur, quae a fide Christi et professione Euangelii postea deficiunt, et ad Gentilismum, vel priorem idololatriam reuertuntur. Sic Athanas.explicat lib. 6. de Tri- nit. Hierony. in praefat. ad Tit. Vincentius Lyri- nensis, et est similis locus Apocalyps.2.v.3. Veritae enim probrum inter Christianos propter eam a- nimi sui leuitatem malunt prorsus a Christo de- sciscere, quam huiusmodi contumelias audire et pati. Nicolao, de quo est in Actis Diacono id sane accidisse narratur, qui ne opinionem leuitatis et temeritatis inter Christianos subiret, maluit con- dere nouam haeresin, nouum sibi coetum collige- re, et ab Ecclesia discedere. Quae autem hic narrat Paulus, rerum ipsarum experientia edoctus didicerat, quemadmodum ipse testatur. 13 Simul autem etiam otiosae discunt circumire domos: imo non solum otiosae, sed etiam nugaces et curiosae, garrientes quae non oportet.  \pend
\section*{CAPVT  V. }
\marginpar{[ p.285 ]}\pstart Altera superioris amplificationis pars, in qua iuniorum viduarum in munere Diaconatus per- manentium offendicula et mala describuntur. Tria vero enumerat Paulus, non quod sola haec obser- uari possint: sed quod maxime soleant. Deinde quod ex his paucis, quae sint reliqua incommoda, quae ex eodem fonte nascuntur, id est, ex huius- modi iuniorum mulierum electione, facile iam per se potest quilibet intelligere. Ergo eas, etiam dum hoc munus administrant, docet esse, Otio- sas (cui vitio discursationem adiungit) et Nuga- ces, et Garrulas. Garrulitatis autem vitium bre- uiter definit Paulus dum ait, asdem loqui quae non decet. Porro haec vitia ideo pariunt magna inter Christianos offendicula, quod sint contra diuinae legis praeceptum 9. quod est, Non falsum testimo- nium aduersus proximum tuum dices, quodque ex his infinita mala nascuntur. Sunt otiosae) At obstat, quod eas ipsas postea vocat πεήργος, id est, plusquam oportet, occupatas et curiosas. Respond. Haec duo vitia inter se re- cte conuenire. Nam nihil agentes illae rerum, quae ad se pertinent, sunt in alienis perscrutandis cu- riosae. Itaque ἀργαι appellantur, quod nihil suum agunt. Περιίργοι vero quod aliena negotia plus quam sua curant, atque etiam plus quam opor- tet, quemadmodum ait ille. Excussis propriis a- liena negotia curo. Imo vero etsi saepe habent domi, quae agant: magno et turpi animi veterno illa negligunt, vt nimirum de rebus alienis co gitent, et disputent. Circumeunt domos) Annexum pene semper o- tio et segnitiae vitium, nimirum discursatio. Hoc  \pend
\section*{AD I. PAVL. AD TIM. }
\marginpar{[ p.286 ]}\pstart autem praetextu maxime id facere videbantur i- stae mulieres, quod tanquam publicae quaedam personae, et pauperum curam habentes in alienas aedes possent, solerentque liberrime ingredi 2. Timoth.3.vers.6. Videtur autem etiam his ver- bis detractrices easdem appellare Paulus, more dicendi Hebraeis vsitatissimo, qui et detrectare et discurrere eadem voce 2לτ Ragal dixerunt, quod detractores iugiter eant, et referant verba hinc inde ex aliis audita. Peccant igitur hi in nonum legis diuinae praeceptum. Imo vero) Amplificatio est superioris mali per adiectionem et προδεσι. Non folum enim sunt otiosae: sed etiam Nugaces, et Curiosae vel plus quam oportet negotiosae φλυάρος appellat Pau- lus, quos Latini Nugaces, quae vox videtur deri- uari a voce. φλύαξ, ακος, quae tumulentum ho- minem significat, quod haec sint duo maxime inter se connexa vitia, Temulentia nempe, et Nugacitas. Itaque vt Plutarch. lib. 8. Symposia- con Problema. 1. nihil aliud esse videtur φλυz- εία quam λήρησις πάροινος. Nugari quoque, ait idem, proprie nihil aliud est, quam vano et futili sermo- ne vti, et eo, qui vel mendax est: vel qui quanquam verus est, ad propositum tamen nihil facit et at- tinet. Vnde postea rixae, iurgia, et contumeliae o- riuntur, atque varia offendicula, Vocem autem i- psam nugae Latinam ab Hebraea man hagaḿh quod more auium garrire significat deduci putat Iose- phus Scaliger in lib.  Varronis de lingua Latina. Qanquam autem Plutarchus in lib.  πεὶ ἀδο- λέσχίας videtur facere φλυκρίας siue ληρήσεως, duo genera, Temulentiam quam ὀινωσιν vocat, et ἀδο-  \pend
\section*{CAPVT  v. }
\marginpar{[ p.287 ]}
\marginpar{[ p.0 ]}\pstart λοχίων quam garrulitatem vertunt, quas ita di- stinguit, vt temulentia sit nimia inter pocula vel inter potandum orta ex vino sumpto loquacitas. Garrulitas vero sit stultiloquium nimiaque lo- quacitas vbique, et ex qualibet, vel etiam nulla o- blata occasione. Itaque garrulus in foro, in thea- tro, in deambulatione, domi, interdiu, et noctu nugatur, et loquax est. Temulentus autem tantum in mensa et inter pocula: ista tamen di- stinctio etiam hunc Pauli locum illustrat, et ex- tendit. Sed videtur vtranque quidem, sed maxime secundam speciem hoc loco damnasse Paulus vti etiam Psal.14o. vers.1o. aperte inter vitia pessima recensetur ista nugacitas, Curiosas vocat tamen eas, quae plura quam de- ceat, scire affectant, tum etiam, quae diligentius et pe- nitius ea, quae sciri possunt, curant et inuestigant, quam sit necesse. Itaque sunt illae nimium nego- tiosae, sed vbi non oportet. In quibus autem oportet otiosae. Alio nomine πολυπράγμονες eaedem appellan- tur. Quanquam Plutarch.in libro de Curiositate eos tantum videtur definire πολυπράγμονας, qui in alienis erratis siue peccatis, vnde alteri nascitur aliqua infamia, curiosius indigandis occupant se- se. At siue in peccatis, siue in aliis etiam rebus, quae ad nos non pertinent, inuestigandis nimium laboremus, hic πρέργων nomine a Paulo signifi- camur. Quanquam vero haec vitia faere sunt omni. bus foeminis communia, maxime tamen eas dede- cent, quae caeteris praelucere vitae exemplo debent. Inquirere autem in alienum factum, quatenus il- lud ad nos spectat: vel Dei gloria proximiquev- tilitas postulat, vel inde promouetur: denique  \pend
\section*{AD I. PAVL. AD TIM. }
\marginpar{[ p.286 ]}\pstart quantum ex muneris nostri ratione vel Christianae charitatis officio tenemur, non est curiositas: sed est officium, quod est sane laudandum. Eo deni- que pertinere videntur postrema huius versiculi verba loquentes quae non oportet. Nec enim de alio loqui prohibemur semper, sed videndum est. Quid, Quomodo, et Quatenus loqui de eo liceat. Priusquam autem totum hunc de Diaconissis locum finiamus, duo scitu necessaria sunt enodan- da. Primum quidem, vtrum eas hodie in refor- matis Christianisque Ecclesiis retinere sit operae- pretium, cum nos quam maxime fieri potest, proxi- me ad veterem et Apostolicam Ecclesiam acce- dere tum in doctrina: tum etiam ln disciplina et politia Ecclesiastica oporteat. Quid igitur? Re- spondeo aliquod discrimen esse nobis statuendum inter doctrinam veteris Ecclesiae, et inter gene- ralem qua illa vsa est, disciplinam. Nam cum doctri- na Apostolica sit Christi ipsius vox et sapientia, cui tanquam vero et vnico fidei et salutis funda- mento innititur Ecclesia, certe ab ea doctrina v- niuersa ne tantillum quidem a nobis recedi vllo tempore debet. Itaque nulla occasio affer- ri vel fingi vnquam potest, propter quam vel in vno apice, quanquam minimo, doctrina A- porterorun.immutetur Matth. 5. vers.18. At illus disciplinae vniuersae, quam illa vsurpauit, non eadem omnino ratio est. Cum enim sit disciplina tan- quam decorum quoddam vestimentum ad res, tempora, personas accommodatum, fit, vt quem- admodum haec variari possunt:ita possit quoque aliquibus in rebus mutari etiam disciplina vetus. Tantum distinctionem hanc adhibeamus, vt in  \pend
\section*{CAPVT  V. }
\marginpar{[ p.289 ]}\pstart disciplina ipsa Ecclesiastica, distinguamus ea, quae sunt Fundamentalia politiae Ecclesiasticae, ab Accessoriis et leuioribus. Fundamentalia ve- ro sunt haec in politia Ecclesiastica. Primum vt sit aliqua politia Ecclesiastica, et quidam ordo in Del domo, qui tollat confusionem. Deinde, vt legitimae personarum vocationes, per quas Dei verbum legitime administratur, charitasque et sanctitas vitae exercetur in Ecclesia, retineantur, veluti Pastores, Presbyteri, Diaconi. Tertio, vt ii electi legitime et ex praescripto Dei verbo suo munere fungantur, vel deponan- tur, et remoueantur. Accessoria dico, quae, vt haec fiant et obseruen- tur, in Ecclesia quaque pro tempore, personis et locis statui possunt. Accessoria incolumi manen- t disciplina in Ecclesia ipsa variari possunt. Ma- neat enim semper disciplinae fundamentum et pars praecipua, sed tantum ea mutentur, quae commo- dioris illius obseruationis gratia tunc visa sunt necessaria: veluti vt futurus et eligendus Episco- pus a tribus Episcopis, non a duobus tantum eli- geretur. Hoc mutari potest incolumi Ecclesia et disciplina. Vt in vrbe Metropolit, electio fiat et examen futuri Episcopi, non alibi. Hoc etiam mutari potest. Itaque vetus Ecclesia in iis rebus saepe diuersam formam habuit et secuta est, inco- lumi nihilominus Dei verbo, ipsoque fundamen- to Ecclesiasticae disciplinae retento. Ex quo fit, vt cum hae Diaconissae fuerint tantum subsidiariae Diaconorum manus et operae (propter egenorum et aegrotorum ingentem numerum et multitudi- nem pene infinitam) quo tempore Nosodochiiss  \pend
\section*{AD I. PAVL. AD TIM. }
\marginpar{[ p.290 ]}\pstart Xenodochiis, et Ptochodochiis careret Ecclesia postea pauperum numero decresente et Ptocho- dochiis constiturtis, non fuerunt eaedem Ecclesiis necessariae. Itaque neque sunt hodie a nobis re- tinendae, nisi eadem et par aliqua causa, qualis illo seculo fuit, occurreret, et eas retinere suade- ret. Quae cum nulla sit, cessante causa, cesset effe- ctus necesse est. Eodem modo videmus veteres habuisse plures officiorum Ecclesiasticorum gradus (vti apparetex aliquot Synodis etiam quan- do adhuc erat tollerabilis Ecclesiae forma) quam nunc habeamus. Itaque quemadmodum illae dignitates tanquam nobis inutiles hodie propter diuersam locorum rerum, et personarum inter nos et illos rationem explosae sunt, et cessant, qualis Subdiaconatus, Acolythatus, Archidiaconatus, Archipresbyteratus: sic et hic Diaconissarum mu- Iierum or do desiit, nullo prorsus detrimento vel in doctrina, vel in disciplina ipsa ab Ecclesia, ac- cepto. Secunda vero quaestio, quae ad hunc quoque locum maxime pertinet, est huiusmodi. Vtrum recte a Papistis sanctimoniales sint in istarum Diaconissarum, quas nullius iam esse in Ecclesia vsus videbant, locum substitutae: at que vtrum eo- rum consilium et institutum sit probandum, et sequendum necne. Respond. Minime id quidem, quia in eo statu hominum et genere instituendo nihil nisi plane Satanicum, superstitiosum, et in Deum ipsum blasphemum a Papistis inductum est. Primum, quia coelibatum illis suis sanct imo- nialibus tanquam legem ad vitam aeternam asse- quendam necessariam, et verissimae sanctitatis  \pend
\section*{CAPVT . V. }
\marginpar{[ p.291 ]}\pstart partem indicunt. Deinde, quod in eo ipso cultum Dei situm esse contendunt. Tertio quod in iis ipsis a Patrum consuetudine recesserunt, et a me- liori more in Ecclesia iam recepto et inueterato. Quod vt melius intelligatur, rem totam fusius sic explico. Cum κακοζηλία quadam homines o- mnia quae Apostolorum temporibus obseruata fuerant, sibi putarent imitanda, neque locorum, neque temporum, neque rerum dissimilium ra- tionem haberent, etiam et ipsi suas Diaconissas retinere praecise voluerunt. Sed cum posteriori- bus temporibus et mutatis circunstantiis earum nullum in Ecclesia vsum esse cernerent, magno Ecclesiae malo, magnoque conscientiarum detri- mento pessima tamen eius ordinis instituendi ratio reperta est. Instigabant virgines Christianas quidam E- piscopi, vt se Deo dicarent, atque, vt illi loque- bantur, consecrarent. Deinde vt in signum istius consecrationis se velo, tanquam castitatis et ve- recundiae testimonio, semper tegerent, et obnu- bilarent. Quod ex Pauli consilio facere se et iu- bere iactitabant I.Corinth.11. Hinc variae Hiero- nymi, Augustii, Ambrosii, Fulgentii epistolae ad varias virgines scriptae, quia etiam ante aetatem horum omnium huiusmodi votuiiimi ihagno a- pud Ecclesiam pretio iam haberetur. Tamen, vt ex iisdem scriptoribus apparet, et imprimis ex Cypriano et Tertulliano in lib.  de cultu et habi- tu virginum, hoc totum voti genus etiam conce- ptum et nuncupatum adhuc liberum erat. In pri- uatis enim parentum vel cognatorum aedibus huiusmodi virgines, quae sanctae vel consecrata  \pend
\section*{AD I. PAVL. AD TIM. }
\marginpar{[ p.292 ]}\pstart Deo dicebantur, manebant non in vllo monasterio et simul, etiamsi in eadem ciuitate plures huiusmo- di essent. Atque etiam saepe ipsae domos seorsim, quaeque suam, habebant, ex quibus vel solae, vel co- mitatae cum volebant, et cum libebat egredie- bantur, nihilque a caeterarum virginum conditio- ne differebant, nisi quod eas caeteris castitatis ex- emplum esse oporteret. Id apparet ex can. Nu- ptiarum 27. quaest. 1. Quin etiam postea eaedem, si volebant, poterant nubere. Et quanquam in eo o- pinione quadam leuitatis apud pios grauabantur propterea tamen neque sacra Coena iis interdi- cebatur, neque etiam ipsae Ecclesia excommu- nicabantur, vti postea factum est. Hoc igitur ma- li initium et exordium fuit, sed latius postea Sa- tanae astu prouectum atque propagatum est. Pri- mum quidem his nominibus vocari coeperunt, vt Moniales, Sanctimoniales, Virgines et Nonnae dicerentur. Postea vero Monachae et ad propha- narum imitationem, virgines Vestales: ad quarum instituta reuocatae vetitae sunt, ne priuatas aedes haberent, id est habitacula vel cellulas feorsim incolerent: sed iussae sunt vt in communi habita- rent et viuerent. Hoc primum fuit, deinde fancirum est, ne aliae Sanctimoniales censerentur, quam quae diicur i'ionasterio certo, certaeque regulae sese in perpetuumaddixissent can. Perniciosam 18 quaest. 2concil.tamem Agath. vetabantur ista Monacha- rum Monasteria et cellae (quibus istae conclude- bantur) iuxta Monachorum et virorum caulas et lustra propter periculum et offendiculum aedisi- cari, vti est in cano. Monasteria puellarum quaest. 2Coepta sunt autem aedificari earum coenobia ma-  \pend
\section*{CAPVT . V. }
\marginpar{[ p.293 ]}\pstart xime post annum Domini 700. Antea enim etsi fuerant quaedam, omnino tamen pauca et rara erant foeminarum et virginum Monasteria extru- cta. Atque tres tantum earum ordines concessi sunt primum, et probati eodem illo seculo, nen- pe post annum Domini 700. quod tunc demum ex Synodo Hispalensi cap.11. ad instituta Mona- chorum coeperunt reuocari.i. ex magna super- stitione regi et institui Moniales: id quod antea non fiebat. Tres autem illi Monacharum ordi- nes primum hi fuerunt, ordo Basilii, Benedicti, Augustini, extra quos nulli foeminae esse licuit Moniali, quodcunque votum vouisset et suscepisset can. Perniciosam 18. quaest.2. Hodie vero infinita prorsus sunt earum genera, et adeo diuersa, vt nihil inter se simile pene habeant. Sunt enim hoc nostro seculo Monachae, Benedictinae, Bernardi- nae, Franciscanae, Dominicanae, Carmelitanae, Cle- ranae, Augustinianae, et aliae plures species. Ita vero primum fuerat in suscipienda profes- sione constitutum, vt nulla ante 4o.aetatis annum votum vlliꝰ ordinis susciperet, aut si suscepisset ne obliga retur, can. Diaconissam 27.q. I.can. Sanctimonia- les 20.q.1. Deinde hoc tempus, tanquam longius esset, restrictum est, atque constitutum vt statim post vicesimum quintum aetatis annum posset virgo quaelibet votum Monasticum suscipere, et 2o. quaest. 1. Sed et hoc ipsum quoque annorum spatium postea est imminutum, atque sancitum, vt statim post pubertatis annos possent mulieres virginitatem perpetuam vouere: quanquam con- secratio earum differebatur vsque ad annum aeta-  \pend
\section*{AD I. PAVL. AD TIM. }
\marginpar{[ p.294 ]}\pstart tis vicesimum quintum: sed postea receptum in- stitutumque est, vt statim post annos pubertatis, id est, post duodecimum aetatis annum posset, quae votum ordinis et vitae Monasticae suscepisset etiam a Praelato consecrari cano. Puella 2o. quest. 2. quem ex concilio Triburiensi sumptum esse volunt. Ante autem duodecimum aetatis annum, ne ex parentum quidem iussu et confirmatione, votum ab iis factum ratum esset constitutum est. Denique ne qua inuita in Monasterium detrude- retur, sed si qua ergastulum illud iubentibus pa- rentibus esset ingressa, decimo quinto anno aeta- tis interrogetur a Praelato de voto, cui si ipsa non assentitur, ne votum ante illam aetatem factum o- mnino valeat. Sic can. Firma can.Sicut et seq .20. quaestione I.sancitum est, a quibus tamen hodie plane discessum est. Coguntur enim etiam an- tequam pene vsum rationis habeant miserae puel- lulae, vt se in Monasterium aliquot detrudi sinant, quod vel non satis formosae sint, vel caeteris liberis pars hereditatis maior accrescat his in pistrinum istud et Monasterium relegatis. Quae virgo vel vidua votum ipsa sibi imposue- rat, et habitum Monialis assumpserat domi suae vel in aedibus patris: non autem ab Episcopo vel praelato illud acceperat, primum poterat illud, si vellet postea deponere, neque cogebatur in eo vitae statu inuita permanere, nisi fuisset postea ab Episcopo consecrata. Post autem aliud decretum est. Nam neque deponere illud velum potuit, ac ne ea quae illud priuatim domi suae assumpserat: sed cogetur etiam ante consecrationem factam in eo statu et habitu permanere perpetuo vitae tempore, et Monasterium ingredi can. Si quis sa-  \pend
\section*{CAPVT V. }
\marginpar{[ p.295 ]}\pstart cra 27.quaest.1.et can. Mulieres cano. De viduis et Puellis eod. Ita nimirum paulatim auctae sunt superstitiones, et durior conscientiis laqueus in- iectus. Cum sanctimonialis a praelato consecratur, sto- la alba induitur, et chrismate vngitur, vt ait Al- cuinus. Primus autem eam consecrari instituit Basilius magnus, vt plaerique volunt. Post autem eadem foemina vel virgo tondetur. Quanquam ne tonderetur Christiana mulier, et concilio Gangreno interdictum fuerat, et expresso Pauli praecepto, quod ipsius naturae voci et sensui con- sentaneum est 1.Corinth.11. Socrat.libus 2. cap.43. Sed boni scilicet illi patres omnia iura diuina et humana, vt suas superstitiones stabilirent, per- uerterunt: et susque deque habuerunt. Nam et ex cano. Quaecunque distinct. 5o.ita in Ecclesia fuis- se olim obseruatum quemadmodum dicimus, con- stat. Consecratae Moniales habent Monachos qui illis praesint, nimirum oues lupis committuntur, (quanquam tamen vt ab earum accessu isti arcean- tur ex concilio Hispal. et can. in decima 18. quaest- 2.sancitum est, sed frustra) Post istam consecra- tionem nubere istae mulieres ex iure Pontificio iam non possunt. Qanquam olim infamia tantum quadam in Ecclesia notabantur, et tanquam Diga- mae tantum censebantur, si quae nupsissetst, at que adeo ex concilio Ancyrano et can. Quotquot 27. quaest. ita erat constitutum. Postea vero ipsa su- perstitione' nullum iam modum seruante pror- sus ab Ecclesia sunt excommunicatae, quae post i- stam consecrationem nupsissent can. virgines 27.  \pend
\section*{AD I. PAVL. AD TIM. }
\marginpar{[ p.296 ]}\pstart \phantomsection
\addcontentsline{toc}{subsection}{\textit{14 Velim igitur iuniores nubere, libe- ros gignere, domum administrare, nullam occasionem dare aduersario ad maledicendum.}}
\subsection*{\textit{14 Velim igitur iuniores nubere, libe- ros gignere, domum administrare, nullam occasionem dare aduersario ad maledicendum.}}quaest.1.Atque ita miserrimus conscientiis homi- num laqueus semper durior, et maior inuectus est: et προς τὸ ἀδύνατον tandem contra rerum natu- ram miserae illae mulieres sunt obligatae, nempe vt perpetuo coelibes manerent, quantunuis intus aestuarent et magno igni vrerentur. In Monasteriis autem illis et coenobiis, tanquam carceribus, conclusae horas canonicas, quas Papi- stae appellant, et aliquot Psalmos tantum cantillant. Nam neque Missam cancre ipsae possunt, neque vasa sacra aut vestimenta, quibus in Missa sacrificuli vtuntur, contrectare, ac ne quidem sine piaculo attingere audent cano. Sanctimonialis et can.seq- distinct.23.Atqui etiam seculo Hieronym. Sacris literis quae virginitatem vouebant diligenter o- peram dabant, illudque exercitium, non autem cantillandi in templis habebant, quod magis erat tolerabile. Atque has superstitiosas pro iis Dia- conissis, de quibꝰ Paul. hic agit, Papistae sustituunt: quae quales sint. et vt a verbo Dei alienae, atque etiam a veteribus canonibus et more Ecclesiae dissentaneae, omnes iam vident qui modo aliquid videre possunt. Sed pergamus ad Paulum. 14 Velim igitur iuniores nubere, libe- ros gignere, domum administrare, nullam occasionem dare aduersario ad maledicendum. ἈντίθεCς est vel potius Ε’πανόρθοβις. Rationem enim subiicit, per quam superiori malo Eeclesia mederi potest. In quo ipso canon vtilissimus, et quidem generalis, qui ad disciplinam Ecclesiasti- cam pertinet, continetur. est autem hic canon Iu-  \pend
\section*{CAPVT  V.. }
\marginpar{[ p.29 ]}\pstart niores viduae Christianae nubunto, et liberos sus cipiunto, familiasque suas habento, quas curent, et administrent, vt omnia superiora mala euitent, et omnem nominis Dei blasphemandi occasio- nem cuilibet calumniatori et aduersario adimant. Singula vero huius canonis verba momentum et pondus aliquod habent. Volo, inquit Paulus. Pri- mum videtur alienum ab Apostolica dignitate et doctrina praeceptum, mulieribus iam plus nimio nu pturientibus et nuptiales laetitias concupiscentibus praecipere vt nubant, Illud enim est oleum adde- re camino, non autem frigidam soffundere, vel lasci- uas illas carnis libidines extinguere Resp. Du- plici nomine hoc praeceptum subiiei a Paulo, 1.vt ecquod sit optimum vitae genus, quod viduae iu- niores eligere debeant, vt sint aedificationi, do- ceat. Nec em ad nuptiales laetitias eas inuitat Pau lus, sed docet quae sit ratio omnium superiorum of fendiculorum tollendorum. 2. vt non minus Deo gratas esse, quae nubunt, quam quae viduae manent contra superiores errores, de quibus. c.4. sup. dixit demonstret. Nam ex hoc ipso apparet secundas nu ptias in mulieribus a Paulo minime damnari, imo vero et iuberi et probari, nedum vt viris sint interdicte. Q as tamen Papistae ex Encratitarum deliriis et haeresi damnant. Quaesitum est autem ad quas viduas hoc prae- ceptum pertineat, vtrum ad eas tantum, quae sunt pauperes, an ad omnes: Vtrum etiam ad eas tantum, de quibus ad Diaconatum eligendis a- ctum est in ecclesia:an ad quaslibet. Resp. Ad om nes. Id quod ex ratione, quae subiecta est, apparet manifeste. Non minus enim diuites, quam pau-  \pend
\section*{AD I. PAVL. AD TIM. }
\marginpar{[ p.190 ]}\pstart peres esse possunt otiosae, garrulae, et nugaces, i- temque propter periculum carnis, atque nuptia- rum offendiculum infidelibus hominibus praebere. Volo) Sic et sup.cap.2.vers.8. et 1. Corinth. 7. vers.7. sed diuerso sensu. Haec enim vox modo vim praecepti habet, vti in cap.2.vers.8. Modo au- tem consilii tantum cuiusdam et exhortationis, vti hoc loco. Itaque pro subiecta materia huius vo- cis, volo, significatio accipienda est. Nec tamen illam Papistarum distinctionem probamus inter consilia et praecepta, quasi consilia quiddam et pius et sanctius contineant: cum hoc ipsum consilium Pauli etiam ex eorum sententia minime conti- neat eam vitae rationem, quae ab illis sanctior cen setur, sed eam potius quae minus ex eorum sen- tentia, sit et seuera et sancta. Suadet enim Nu- ptias, non coelibatum. Praeclare autem hoc loco Chrysosto. negat hic tradi praeceptum a Paulo absolute cui quisquam obstringatur, nisi vt supe- riora mala euitet. Alias enim nubere aut non nu- bere viduis liberum est, vti apparet ex 1.Corinth. neque cuiusquam conscientia irretitur et laqueo constringitur hoc loco, si nullum est offendiculum. Atque etiam ex hoc ipso loco apparet secundas nu ptias, dequibus proculdubio Paulus agit cum de Vi duarum nuptiis hic disputet, minime esse vel dan- natas, vel Deo exosas, quicquid vel Montanistae, vel post eos Papistae sentiant: e quibus illi secundas nuptias omnino improbant: hi vero eas tanquam infames impedimentum esse censent ad ecclesia- stica munera subeunda. Nihil certe moror dictum vel Tertullia. vel Bernard.serm.59 cant. quorum ille fuit Montanista, iste monachali superstitione  \pend
\section*{CAPVT  V. }
\marginpar{[ p.299 ]}\pstart iam fascinatus. Quod si secundae nuptiae non dam nantur, nec tertiae Aug. lib.  de fide ad Petr. cap.3. in fine. Bernard.ser. 66. cantic. eam quoque sen- tentiam refutat quae vult honestiora esse inter vir gines connubia, quam quae iam nupserunt. Uiduas. Ex hoc loco apparet, quam sit lasci- uiae dedita natura, et ingenium omnium hominum: sed imprimis mulierum, de quibus hic agit Paulus: quanquam tamen ab eo malo non sunt mares li- beri et immunes. Nam et quae.1. Corint.7. tradit Paulus et hic, aeque ad vtrunque sexum pertinent, quod si offendiculum oritur, et nascitur. Iuniores sexaginta annis supra definiuit Pau- lus: non quod tamen omnes, quae ea aetate mino- res sunt, nubere velit: sed eas tantum, quae liberos suscipere, et familias administrare possunt. Nam si quae 5o annos nata est, vel ea sit corporis consti tutione, vt haec mala fugiat, illi nullam legem dat Paul.hoc loco. Nubere Tria praescribit, in quibus omne mu- liebre officium contineri videtur. Nimirum Nu bere, liberos procreare familiam administrare. Nu bere. Hoc pertinet, vti diximus, ad secundas nu- ptias: et si iam secundo nupserat, et vidua restet, ad tertias et quartas et quintas nuptias, qualis illa, de qua Hierony. ait, septies eam nupsisse. Erat Vercellis apud Insubres. Liberos procreare τεκνογονεῆν. Quid significet vide supra, cap:2.ver.15. Oἰκοδεσ ωοτεῖν Est in verbis Graecis ομοιτελευτὸν, et orationis ornatus. Hoc autem ἐκοδεσποτειν est familiam administret, non vt marito suo praesit, et sit superior vxor, sed pro eoiure, quo debet tanquam mater familiâs caeteris quidem prae-  \pend
\section*{AD I. PAVL. AD TIM. }
\marginpar{[ p.300 ]}\pstart \phantomsection
\addcontentsline{toc}{subsection}{\textit{15 Iam enim nonnullae deflexerunt, se- qundae Satanam.}}
\subsection*{\textit{15 Iam enim nonnullae deflexerunt, se- qundae Satanam.}}esse eam vult Paulus interim tamen agnofcere virum caput suum, illique subiici et parere. Nul- lan occasionem. Subiectio est, quae totius vitae Christianae scopum demonstrat. Nos enim aedifi- cationi omnibus esse oportet Fidelibus et Infide Iibus. Infirmis et Validis. Itaque hac ratione vsus est saepissime Paulus vti sup.3.ver.6.Tit.2.vers.8. sed et Petrus quoque 1.epist. cap.2. vers.12.15. et Christus ipse Math.5. in si.idem argumentum af- ferunt. Ἀν τικειμένον autem generaliter accipio pro calumniatore et aduersario vti.2.Thessa.2.ver.4. Nec enim Satam ipse maledicit per se de doctrina, sed homines profani, Infideles et calumniatores detrahunt et de nobis et de doctrina Christi, quam profitemur. 15 Iam enim nonnullae deflexerunt, se- qundae Satanam. Confirmatio est superioris sententiae ab exen- plo. Debent enim aliena pericula reddere nos cautiores ac prudentiores. Nam felix, quem fa- ciunt, aliena pericula cautum. Hoc ipsum pro exen- plo et documento Papistis esse debuit, vt tyranni cas suas de coelibatu leges tollerent, et abroga- rent, cum inde tot mala nasci et emergere quoti- dit cernerent: sed mordicus doctrinam suam non modo retinent, verum etiam ipsi Dei legi im- piissime praeferunt. Neque tamen sic semper con cludendum est. Illud omne nimirum esse mu- tandum, vnde aliquod periculum, et aliqua di- scessio ab ecclesia sit exorta. Sic enim omnis et correptio et disciplina ecclesiastica tolleretur, propter quam multi a fide defecerunt, et ab ec-  \pend
\section*{CAPVT  V. }
\marginpar{[ p.301 ]}\pstart \phantomsection
\addcontentsline{toc}{subsection}{\textit{16 Quod si quis fidelis aut si qua fidelis habet viduas, eis subuenito, et non onerator Ecclesia, vt iis qua vere viduae sant subue- niat.}}
\subsection*{\textit{16 Quod si quis fidelis aut si qua fidelis habet viduas, eis subuenito, et non onerator Ecclesia, vt iis qua vere viduae sant subue- niat.}}clesia Christi recesserunt, vti Nouatiani et Dona tistae. Resp.igitur caussam ipsam secessionis spe- ctandam esse. Nam si quis idcirco deflectit a fi- de, quod iugum Christi ferre nolit, discedat. Sin autem quod tyrannidis praeter Christi legem iugum sibi impositum ferre non potest et recu- sat, certe illud iugum tollendum est, et demen- dum. Neque enim quisquam iugum Christi du- rum sentiet, nisi qui omnino sceleratus et im- pius esse volet. Ergot quuuaiir imno isto rigore coelibatus iam offensae discesserant atque deflexe- rant a fide, et ad Satanam reuersae erant, abiurata nimirum Euangelii doctrina. 16 Quod si quis fidelis aut si qua fidelis habet viduas, eis subuenito, et non onerator Ecclesia, vt iis qua vere viduae sant subue- niat. Conclusio est totius superioris dispucationis de viduis pauperibus, quae quidem sunt fideles. Continet autem hic locus etiam canonem illi primo si millimum, qui sup. vers.4. explicatus est. Est autem hic canon sextus numero, Quisque fi- delis suas viduas alito: Illis, quae a cognatis ali et exhiberi possunt, ecclesia ne grauator, vt aliis omni auxilio destitutis subuenire et suppeditare possit quod satis est. Primum quoad verba, Erasmus censet haec a πςὴ a studioso aliquo qui mentem Pauli ple- nius explicare glossemate voluit, fuisse addita Vult igitur hunc locum sic restituendum esse ὐτιςπςὸς ἔχει χόρας. Sed perinde est. Nam etiam  \pend
\section*{AD I. PAVL. AD TIM. }
\marginpar{[ p.302 ]}\pstart voxista πιςις non addatur, tamen ex mente Pau- li intelligitur. Nihil enim hic de viris praeceptum est, qui propinquas et cognatas viduas alere pos- sunt, quod idem ad foeminas non pertineat. Nam etiam illae si sunt diuites, pauperes consanguineos suis opibus sustentare et subleuare certe debent, exemplo illius Tabithae de qua est facta mentio Actor.9. et Electae, de qua Ioannes 2.epistola, quae vtraque suis opibus alienos foult, nedum cog- natos fideles egentes, si quos habuit, Item et Ioannae mulieris, et aliarum de quibus est Luc. 8. vers. 3. Quaesitum est autem. Quid si illae viduae quae se a piis propinquis ali postulant, ipsae non sunt piae. Resp. Etsi omnibus est benefaciendum, vt docet idem Paulus Galat. 6. imprimis tamen de domesticis fidei hoc Pauli praeceptum intelligen- dum esse, et de Viduis piis, quae nomen dederunt Christo, quarum maxima et potissima ratio sem per haberi debet a cognatis piis et Christianis. Dicit autem Paulus Επαρκεῖτο id est, abunde et copiose sumministret. Αρκεῖν enim est copiose, et quantum satis est atque necessarium, suppedi- tare. In quo tamen praeclare Chrysosto. annota- uit, non praecipi a Paulo, vt deliciae illis Viduis vel a cognatis vel ab ecclesia, a qua aluntur, prae- Seantur: sed quod ad victum modestum, sobrium ac earum conditioni aptum sit satis et conue- niat. Neque tamen sic, vt Viduae illae alantur otiosae et inertes: sed vt illae ipsae victum sibi, quantum arte vel manibus suis vel labore possunt, comparent etiam, veluri lanificio, serificio, suendo, nendo, aliisque rationibus honestis. Quod si istae viduae et pauperes laborent, iam cognati vel ec-  \pend
\section*{CAPVT  V. }
\marginpar{[ p.303 ]}\pstart clesia addat, si quid illis ad necessar ium victum deest. Idque pro ipsorum praebentium facultati- bus. Neque enim vel cognati, vel ecclesia cogi- tur omnia quae viduis et aliis pauperibus desunt, suppeditare, nisi id fieri ipsorum facultates et o- pes patiantur. Vera est enim etiam inter Christianos illa Ennii sententia a M. Tul. Cicerone recte explicata, ita aliis largiendum esse, vt nostrae fami- liae etiam subuenire possimus. Pro eo enim quod quisque habet dare debet, et in eo Deo acceptus est: non autem pro eo, quod non habet. Exhauri- re enim seipsum prorsus nemo tenetur, ne sit hu- iusmodi subuentio ipsi subuenienti oppressio, et ne postea ipse mendicet, vt docet Paulus.2. Co- rinth.8.vers.2.13. Etsi parce tamen grauate et te- nuiter non est subueniendum proximo sed luben ter et vti dixi, quantum possumus. Vtiis. Æquissima ratio quae ducta est a fine ae- rarii ecclesiastici. Ecclesiam enim ad suos paupe- res alendos pecunias a piis collatas habuis- se apparet ex Sozom. lib.  7. cap.26. et Euang. libus  5. capite. 5. ne praeter aequum se onerari existiment propter hoc praeceptum istarum vi- duarum cognati. Etsi enim ecclesiae aerarium est commune omnium pauperum aerarium ta- men videndum est et prospicienduti ine statim exhauriatur: omnibus emm sufficere diu non pote rit Reseruetur igitur illud maiori inopiae, et eo- rum praesertim egestati, qui non habent, vnde se aliunde exhibeant, siue sint Viduae: siue Mares pau peres. Qui habent vnde aliunde adiuuentur, eccle siam ne grauent. Haec sane de pauperibus viduis cura perstitit diu in ecclesia Christiana puriori-  \pend
\section*{AD I. PAVL. AD TIM. }
\marginpar{[ p.304 ]}\pstart \phantomsection
\addcontentsline{toc}{subsection}{\textit{17 Qui bene praesunt Presbyteri, duplici honore digni habentor: maxime qui la borant in sermone et doctrina.}}
\subsection*{\textit{17 Qui bene praesunt Presbyteri, duplici honore digni habentor: maxime qui la borant in sermone et doctrina.}}17 Qui bene praesunt Presbyteri, duplici honore digni habentor: maxime qui la borant in sermone et doctrina. Πόεσβασς Transitio est. Tractat enim iam aliud caput disciplinae et politiae ecclesiasticae Paulus, quod ad Ministerii ecclesiastici munus eque per- tinet, atque superior cura et disputatio. Est au- cen noc caput de Presbyterorum, et eorum, qui praesunt ecclesiae munere id est, tum aliorum er- ga ipsos: tum eorum ipsorum erga seipsos: tum eorundem etiam erga alios officio et cura in vni- uersum. Ac quidem de aliorum erga ipsos, et ipsorum cura erga seipsos, quid nimirum praestare inter se debeant, ostendit Paulus vsque ad vers.24. varios  \pend\pstart bus seculis maximeque episcopi iubebantur illi prouidere et prospicere vt apparet ex variis Sy- nodotum decretis et imprimis Synodo Chartag.4. postea pontificii παρέπισηοποι, qui nihil omnino, ni si suum lucrum ex ecclesia captant, totam hanc pauperum curam neglexerunt, minimeque ad se pertinere existimarunt: sed ipsi dona Viduis, pu- pillis, et aliis ecclesiae Dei pauperibus destinata sacrilege inuaserunt. Quorum certe bonorum sunt Deo rationem reddituri: atque ii omnes etiam, qui in reformatis ecclesiis, bona, quae ad paupe- res pertinent, hodie alio transferunt, et sibi pro- priisque commodis applicant. Quod faciunt. nimium multi, imprimis aut Nobiles et Princi- pes Christia. Quid autem de bonis ecclesiasticis sit inter nos constituendum, vide Cal. epist. 372. 394.384.  \pend
\section*{CAPVT  V. }\pstart que canones constituit omnino et semper ec- clesiae necessarios, si modo eam recte administra- re ipsi velint. 1011d0 Cur autem haec praescribat Paulus facilis ratio est, Nam si vel minimi cuiusque ratio in Dei ec- clesia haberi debet, multo magis eorum, qui prae- sunt ecclesiae. Si singuli, qui sunt in ecclesia ordi- nes, regendi sunt certis legibus, vt omnia recte fiant in Dei ecclesia, multo magisii, a quibus maius periculum imminet. Ergo tria quaedam hic pro- ponit, I Quid praestandum Presbyteris. 2Quibus Presbyteris. 3 Cur praestari illis aliquid debeat. Presbyterorum autem nomen hoc loco, et in toto hoc argumento et capite, dignitatis nomen est, non aetatis, quicquid Chrysosto. et alii contra sentiant vti saepe alibi et I.Petr.5.vers.1.accipitur. Id autem esse ita accipiendum apparet ex his ver bis, Qui laborant in sermone. Ergo illis, debetur honor, isque duplex, ait Paulus Honoris voce subsidium omne piumque offi- cium, more Hebraeorum, significatur, vti supra eodem significato vers.3. Viduas honora et 1. Pe- tr.3.vers.7. Vir vxori honorem tribuat. Sumpta est haec dicendi ratioex lege Dei et praecepto 5.Ho- nora patrem. Sunt enim nomine parentum, sed spiritualium comprehensi, qui nos verbo Dei pas cunt: et praesont nobis, nosque in Christo vel ge- nuerunt: vel iam genitos coelesti cibo alunt, quod faciunt hi Presbyteri. I taque hic canon Pauli ex quinto legis diuinae praecepto ductus est. Con- sentit autem hic locus cum 1.Corinth.9. etHebr.13.  \pend
\section*{AD I. PAVL. AD TIM. }
\marginpar{[ p.306 ]}\pstart vers.7.1.Thess.5.vers12.13. Cum autem hoc praecipit tam diserte Paulus, oblique taxat illius aetatis ingratitudinem in suos Pastores, nec tantum cauet in futuram posterita- tem. Hoc tamen malum regnat adhuc hodie, et hac ipsa ratione verbo coelesti priuare ecclesias tentat Satan: multos certe a suscipiendo coelesti ministerio deterret, quod eos efficiat famelicos, qui Deo ipsi in tam honesta necessariaque causa seruiant: quorum operae et labori nullus alius, vt ait Chrysostomus, est conferendus. Et recte. Ergo quod aliis collatum diceretur eleemosyna, Presby- teris praestitum dicitur honorarium, stipendium, Debitum, a Christo ipsoμιθὸς. Ait Paulus Duplicis, vt caeterorum compara- tione, quibus aliquid etiam specialiter debemus, eos nobis magis commendet. Sic tamen quidam interpretantur, Duplici honore perinde et tantum esse, atque si dixisset, Magno. Alii vero comparate hoc dici putant, alii absolute. Absolute quod et in hac vita digni sint verbi Dei ministri stipen- dio et subsidio victus: et in futura vita, felicitate aeterna. Quicerte errant, si nos Ministrosve ipsos a- liquidapud Deum mereri posse putant. Chrysosto. rectius, qui Duplicem honorem explicat Venera- tionem ipsorum personae id est reuerentiam: et Subsidium vitae ac alimoniae. Certe vtrumque Pa storibus, vti Parentibus ipsis, deberi verissimum est. Qui comparate vero hunc locum accipiunt, referunt ad superiora praecepta de viduis alen- dis et honorandis. Istis enim viduis longe sunt magis honorandi Presbyteri et pastores. Quae sententia videtur huic loco consentanea et com-  \pend
\section*{CAPVT  V. }
\marginpar{[ p.307 ]}\pstart moda. Non vult autem magnas opes et honores congeri in Ministros Paulus, etsi duplici honore dicit dignos. Recte Chrysosto. Ego ecclesiae prae sules audenter dixerim, ait, nihil praeter victum et vestitum habere oportere, vti inf. cap.6.ver.8. expli cabitur. Praeclare quoque Erasmus. Sacerdotes et praedicatores quidam nihil sunt, nisi decimatores. Obstat autem quod propriis manibus victum sibi quaesiuit ipse inseruiens ecclesiis: ergo ipsi Βιωτικὰ sibi comparare debent. Actor.2o. ver. 34 Respond. ex Chrysosto ne pastores inopia coa- cti a studiis et melioribus rebus distrahantur, iis victum esse ab ecclesia suppeditandum. Nam et victum sibi manibus artificioque quaere re, et suo muneri superesse simul non possunt Mi nistri, nisi dona illa prophetiae et cognitionis scri pturarum egregia haberent, extraordinaria qua lia certe habuit Paulus. Plusquam belluina ingrati tudo est denegare carnalia ei, qui tibi coelestia prȩbet vt loquitur Paul. Ro.15.v.27. Neque ne- gat Paulus se ab vllis ecclesiis vnquam accepisse stipendium.2. Corinth.11.vers. 8.sed eo tm absti- nuisse, cum ecclesiae aedificatio id fieri postulabat. Eodem modo est respondendum, si quis obiiciat Ne he.5.ver.15 Res. Nehemiam id fecisse propter popu li paupertatem: non autem qu victum oniptre ab eo non licuisset. Itaque non est lex generalis illud Pauli vel Nehemiae factum, sed singulare quoddam pro temporis ratione assumptum. Et certe nimis sunt impudentes Catabaptistae, qui illius facti et exen- pli pretextu omnes verbi Dei Ministros ad ma- nualia opera cogendos esse contendunt, vt victum comparent: aduer sus quos preclare Bulling lib. 2.  \pend
\section*{308 AD I. PAVL. AD I'IM. }\pstart Tract.7. contra Catabaptistas. Nec fures nec la- trones sunt verbi Dei Ministri, qui iusta a grege suo stipendia, quantum quidem ad victum et ve- stitum satis est, exigunt, vt isti calumniantur. Nec sunt fuci cum strenue verbo Dei nauantes operam, ita viuunt siue manuali labore. Secundum caput, est quibus Presbyteris de- beatur? Resp. Iis presertim qui in verbo Dei et Doctrina laborant. Ergo est duplex Presbytero- rum genus, Vnum eorum, qui Doctrinae coele- stis sant Ministri: qui nihil ab episcopis et Pasto ribus differunt: Alterum eorum, qui moribus fi- delium tantum inuigilant, et cos inspiciunt, ne quid offendiculi in ecclesia Dei ex quoquam o- riatuc. Hi sunt, quos in Gallia vulgo Supervigilan- tes dicimus qui vna cum Ministris verbiDei con sistorium siue Senatum ecclesiasticum constituunt. Sed quaeritur, vtrum hoc postremum genus Pres- byterorum sit negligendum ab ecclesia. Respon. Minime vero, si quidem isti quoque egeant: sed primum genus magis commendatum nobis esse de bet Ex quo fit, vt Paulus addiderit. Ma xime. simi lis locus Galat.6.vers.6. Ait enim communicet qui instituitur in sermone ei, qui se irstituit, omnia bona. Obstat quod ait Christus Matth.1o. vers.8.Gra nj ilccepistis, gratis date. Resp. Eum gratis praees- se et pascere gregem, qui nec lucrandi noc ditescen di animo id facit: quanquam victum sibi necessa- rium a grege, idest ab ecclesia accipit, quemad- modum docet et Christus eodem loco, et Paulus I.Corinth.9.2. Corinth.11.vers.8. Vult autem istos Presbyteros laborare, er- go non otiosos esse vult, non etiam perfunctorie  \pend
\section*{CAPVT  V. }
\marginpar{[ p.309 ]}\pstart \phantomsection
\addcontentsline{toc}{subsection}{\textit{18 Dicit enim Scriptura, Boui trituranti non obligabis os. Et, Dignus est operarius mer- cede sua.}}
\subsection*{\textit{18 Dicit enim Scriptura, Boui trituranti non obligabis os. Et, Dignus est operarius mer- cede sua.}}tantûm fungi suo munere: non denique pro sua tantum commoditate inseruire ecclesiae: sed ma- gno animi zelo, totisque et animi et corporis viri bus incumbere, vt optimam muneris sui rationem Deo reddere possint. adeo vt saepe de se testetur Paulus se nontm laborasse, sed etiam lassitudinem, quasi exhaustis viribus, passum esse, tantae aerumnae corporis in eo faciendo nimirum saepe sunt (vt 2. Thes.3.ver.8. docet ipse) perferendae, ne sint muti canes, vt ait Isaias. Sed et laborare in sermone et Doctrina, non qualibet, sed coelesti et Euangelica iidem iubentur, per quam homines adveramn ag- nitionem Dei deducant. Quidam sermonem pro exhortationibus accipiunt. Doctrinam pro ipsa traditione methodica coelestis mysterii Quod v- trunque fit concionando. Optime Chrysosto. qui docet exigi a pastore his verbis, vt Prendicet, doceat, et concionetur: non tantum vt studeat et vr operam det priuatim Dei verbo, illudque n ditetur et speculetur ipse, vel vt vite sanct? „ plo gregem suum instituat. Docere igitur det pastor, et eatenus officium facit, quatenus viua vo ce illud exponit. Ite, ait Christus, docete. Ergo qui absunt a suo grege: qui in grege ipso muti sunt: qui etsi per alios docent, ipsi nunquam id faciunt, omnino sunt indigni nomine Pastorum et Pres- byterorum: et duplici illo honore, de quo agit hoc loco Paulus, priuandi 18 Dicit enim Scriptura, Boui trituranti non obligabis os. Et, Dignus est operarius mer- cede sua. 1 2 Αιτιολογιά est, in qua tertium caput eontinetu?  \pend
\section*{AD I. PAVL. AD TIM. }
\marginpar{[ p.310 ]}\pstart nempe. Cur hic honor Presbyteris debeatur. Af- fertur autem a Paulo ratio duplex, quarum Pri- ma ducitur ab exemplo, quod etiam diuina lege comprobatur. Secunda ab officio et communi hominum sententia. Ergo videtur hic vti duplici autoritate Paulus, nimirum Diuina et Humana: Diuina, cum ait. Dicit enim scriptura, Boui trituran ti etc. qui locus sumptus est ex Deut.25.ver.4. etI. Corint.9. ver. 9. explicatur fusius ab eodem Paulo. Humana vero autoritas est, cum ait, Dignus o- perarius mercede sua: quae in prouerbium abiit, ip- sa naturae voce inter omnes mortales confirma- ta: adeo quidem vt cum Christus ipse hoc idem dixit Math.1o.vers.1o. Luc.1o.vers.7. ex communi prouerbio sumpsisse videatur. Nihil enim hoc di cto magis tritum est, et vulgare. Nisi quis fortasse ductum malit ex eo, quod saepe in scriptura legi- tur, esse dignum mercenarium mercede sua Deut. 24.vers.15. Genes.31.vers.12. Ad quod ipsum quo- que respexit Iacobus cap.5.vers.4. Neque tamen, etsi dignus est Pastor suo stipen- dio et suo honorario, tanquam merce de, propter- ea pascere gregem debet eo animo, vt inde sti- pendia accipiat, et mereatut. Nam huiusmodi ani- mus est Auaritiae (quae est omnibus fugienda, im- primis autem Pastoribus Luc.12.vers.15.1. Petr.1. vers.2.)sorculus, quanuis non expetat inde dites- cere, vel opes cumulare. Praecipuus enim Pasto- rum scopus esse debec Dei ipsius gloria. Vi- ctus autem ipsorum est accessio quaedam tantum, quem idcirco petunt et spectant, quia ad faciendum et melius, et alacrius, et liberius officium, ille ne ceslarius est.  \pend
\section*{CAPVT V. }
\marginpar{[ p.311 ]}\pstart \phantomsection
\addcontentsline{toc}{subsection}{\textit{19 Aduersus presbyterum accusationem ne admittito, nisi sub duobus aut tribus testibus:}}
\subsection*{\textit{19 Aduersus presbyterum accusationem ne admittito, nisi sub duobus aut tribus testibus:}}Mirum autem est praetermissam hic esse confir mationem huius sententiae, quae fieri potuit exen- plo Leuitarum, qui a reliquo populo Israelitico ale bantur Dei iussu ex decimis, et aliis oblationibus: sed ab eodem Paulo in 1.Corinth.9. vers.13. haec ratio non est neglecta. Vnde errant, qui stipendia Ministris tribui pu- tant tanquam meras et gratuitas tantum eleemo synas putantque liberum esse ea persoluere vel non: debentur enim vti iustissima merces, etsi nulla est tanto eorum labori aequalis, quemadmodum ait Chrysosto. Scholastici et Canonistae in cano. Qui pro pecunia 1.q.1.cano. Altare.1. q.3. distinguunt inter illud quod accipitur tanquam Pretium ope rae et Sustentatio vitae, Illud quidem accipi a Pastore negant posse, haec autem concedunt Nec est sane inutilis vel incommoda.ista distinctio. Vide Du- rand.in lib. 4. Sent. dist.25.quaest.3. et 4. 19 Aduersus presbyterum accusationem ne admittito, nisi sub duobus aut tribus testibus: Transito est: iam enim agit de|secunda huius ar gumenti parte nimirum ecquod sit ipsorum Pres byterorum aduersus scipsos et collegas suos offr cium, ne quid ex ea parte offem diculi nascatur. Impri- mis enim cauendum est, ne illi se mutuo excusent et sibi parcant, dum tamen alios castigant seuere: atque ne in illis locum habeat prouerbiumvetus, Mu tuo muli fcabunt. Potius illis in mentem veniat illa Christi sententia, vos estis sal terrae si sal euanuerit in quo salietur? vos estis lux mundi. Math.5.verf.13.14. Item illa, sint lumbi vestri praecincti et lucernae ardentes in manibus vestris. Luc.12.ver.35. Sed et illa quoqe  \pend
\section*{AD I. PAVL. AD TIM. }
\marginpar{[ p.312 ]}\pstart Petri debere pastores toti gregi esse exemplar vi tae sanitate doctrinaeque puritate, caeterisque praelu cere.1.Petr.5. vers.3. Ergo quemadmodum vna ex parte diligenter illis cauendum, est ne propriis collegarumve peccatis conniueant: sic ex altera parte prospiciendum, ne ad facilem criminatio- nem et accusationem aditum aliquem proteruis hominibus et iugum disciplinae ecelesiasticae re- nuentibus et excutientibus praebeant. Falsa est qui dem illa Papistarum ratio, qui postquam vniuer- sum Domini gregem distribuerunt in Laicos et Clericos, fingunt laicos esse infestos acperpetuos Clericorum hostes, eosque innato quodam ex cle- ricalis dignitatis fulgore et priuilegiis odio pro- sequi. Itaque cano. Nullus laicus, et cano. Laicos. 2.quaest.7.omnes laicos, quamuis probatae vitae vi ros pontificii ab accusatione et testimonio in cle- ricos dicendo summouent. Sed peruerse. Primum enim contra Christia. charitatis praecepta illud o- dium in omnibus, qui subiiciuntur, inesse praesu- munt, cum non sit charitas suspiciosa. Deinde res ipsa docet, qui Christiani aliis subsunt, eos pro- pter Dei verbum ex animo pastores, praepositos- que suos diligere, illisque subiici lubenter velle, quia non sunt pastores euangelici animorum ty- ranni. Quid igitur? Possunt fateor in ecclesia Dei esse efftenes homines: possunt esse simplices: pos- sunt esse hypocritae et astuti, qui cum doctrinae ipsius firmitatem concutere non possunt: neque aliter mysteria arrodere: in personas ministrorum debacchantur impudenter, vt eorum autoritatem eleuent et imminuant quantum possunt. Atque ca semper arte Satan tacite tanquam  \pend
\section*{CAPVT  V. }
\marginpar{[ p.33 ]}\pstart per cuniculos Ecclesiam Dei conatus est euerte- re. Itaquehuic et Satanae technae, et improbo rum audaciae fuit occurrendum, praecludendusque huiusmodi criminationibus aditus. Imprimis quod multorum accusationi et calumniis obno- xius esse solet, qui in omnium oculis versatur, qui in vitas omnium inquirit, qui omnium vitia re- prehendit, tam sunt nimirum homines correctio- nis impatientes, vt recriminatione etiam falsissi- ma sese vleisci studeant aduersus eos, a quibus tamem merito arguuntur. Est autem hoc pastorum of- ficium, vt est Ezechiel. 18. tanquam hominis in specula collocati qui despicere omnia, ac turpia quaeque redarguere debet. Ex quo fit, vt sint mul- tis odiosi, et malignorum obtrectationibus pa- teant, sintque obnoxii. Id quod in causa. Bonifa- cii. docet Augustinus. epist. 137. Vnde huic quo- que periculo cauendum et prospiciendum fuit Vtrunque autem hoc malum, de quo diximus, ex- cludit et summouet canon, quem hoc loco sancit Paulus, aut potius Dei ipsius spiritus. Est autem huiusmodi, Presbyteri ne aduersus Presbyterum facile vel accusationem ipsam admittunto, nisi duorum vel trium testimonio crimen cum de- fertur, sit statim confirmatum. Qui canon vt commode declaretur, sunt haec tria ordine nobis perquirenda. 1 De quibus Presbyteris agat Paulus. 2 Quid de illis praecipiat. 8 Cur hoc iubeat et velit. Primum caput facile est, et ex superioribus versiculis satis apparet. Agit enim de iis, qui di- gnitatis, non autem aetatis ratione Presbyteri sunt.  \pend
\section*{AD I. PAVL. AD TIM. }
\marginpar{[ p.51 ]}\pstart quorum officium et munus in fine huius capitis persequitur et exponit sigillatim. Quanquam a- liud sentit Chrysostom. et ad aetatem refert, cui, tanquam minus ad peccandum procliui, vult parci, minusque facile aduersus eam vel solam ipsam accusationem admitti. Sed haec ratio Chrysost. saepe quidem apparet falsissima cum senes quos- dam videamus ad omne scelus propensissimos. Secundum igitur caput est. Quid de Presbyte- ris hoc loco Paulus praecipiat. Resp. Ne contra eos vel sola duntaxat accusatio, citra duorum vel trium testimonium, excipiatur vel a toto colle- gio Presbyterorum, vel ab vno aliquo Presbyte- ro, nedum vt condemnatio de iis temere a quo- quam fiat. Quaerunt autem, quid hoc nouo prae- cepto fuerit opus, et in Presbyteris praesertim, de quibus hic aliquid speciali' sancire Paülus vi- detur, cum in omni accusatione hoc ipsum locum habere videatur. Nam Deut. 17.vers.6. sic ait lex Ex ore duorum vel trium stabit et ratum erit verbum et accusatio, quae fiet de aliquo. Itaque hoc ipsum tam leuis momenti quibusdam hoc loco visum est, vt omnino delendum censerent, tan- quam inutile et superuacaneum, cum tamen sin- gulare priuilegium, et a Deo ipso pastoribus et Presbyteris concessum contineat. Respon. igitur aliud esse Accusationem ipsam, aliud Iudicium, quod de accusato fit, continetque vel absolutionem vel condemnationem rei a iudice factam. Iudi- catio igitur siue sententia sine testibus et legitimis probationibus in vniuersum a nemine ficri de- bet. Ac eo pertinet locus Deuter. 17. Accusatio autem ipsa certe admitti sine testibus a Iudice  \pend
\section*{CAPVT  V. }
\marginpar{[ p.815 ]}\pstart de quouis alio potest, praeterquam in hoc casu verbo Dei expresso: si modo, qui accusant non sunt omnino infames personae: vt est praeclare re- sponsum ab Eusebio Vercellensi Episcopo, et extat in canone De accusatoribus3 quaest.5. Pre- sbyteri igitur admitti a collegis sine testibus, iis- que duobus vel tribus, ne accusatio quidem ipsa potest. In quo est Presbyterorum praerogatiua praeter caeteros homines hic a Paulo expressa, quia variis hominum calumniis, multorumque inuidiae sunt obnoxii. Testes autem eos hic requirit Paulus, qui in caeteros reos audiri et admitti possunt, non alios. Ergo huius canonis et priuilegii ratio (in quo ter- tia huiꝰ versiculi pars posita erat) apparet esse, non quod presbyterorum peccatis, et vitiis parcere et conniuere suadeat Paulus (quos grauius pro- pter dignitatem peccare certum est) non quod priuilegia fori et immunitatis personis Ecclesia- sticis tribui velit, qui Romanor. 13. omnes atque etiam pastores ipsos magistratibus subditos esse iubet, vti et Petrus I.Pet.2.vers.17. Non quod semina odii et suspicionum inter gregem et pa- storem serere, aut alere cupiat: sed quod rerum ipsarum experientia doceat quosdam odisse prin- cipes et eos omnes qui sibi praesunt: atque etiam tantam esse humanae mentis malignitatem, vt eos qui vitia nostra reprehendunt, maxime in odium ac inuidiam aliorum adducere conemur. Cui ma- lo et effraeni licentiae occursum esse sapienti isto remedio voluit Paulus. Illud autem priuilegium, vt Episcopi nonnisi s septuaginta testibus, isque non Laicis homini.  \pend
\section*{AD I. PAVL. AD TIM. }
\marginpar{[ p.310 ]}\pstart \phantomsection
\addcontentsline{toc}{subsection}{\textit{20 Eos qui peccant, coram omnibus ar- gue, vt et reliqui timeant.}}
\subsection*{\textit{20 Eos qui peccant, coram omnibus ar- gue, vt et reliqui timeant.}}bus conuicti damnari possint prorsus impunita- tem peccandi dat Episcopis. Quod tamen canon: Placuit 2.quaest.4. continetur. Item illa clericorum praeuilegia, per quae vetantur in vniuersum Laici clericos accusare, et inter clericos ipsos ii, qui in- ferioris sunt gradus, prohibentur superiores deferre, aut in eos testimonium dicere, sunt o- mnino tollenda et abroganda, tanquam nefandis- sima omnium scelerum in Episcopis vela. Atque illa omnia late 2.quaest.7. per totum explicantur. Fenestram enim ad nequitiam patefaciunt, et nullum Episcoporum corrigendorum locum re- linquunt, planeque pugnant cum mente Pauli. 20 Eos qui peccant, coram omnibus ar- gue, vt et reliqui timeant. Alter canon de poena Presbyterorum sic con- uictorum. Est autem hic, vt Presbyteri, qui pu- blice et notorie peccarunt, publice, id est, in coe- tu totius Ecclesiae arguantur. Ratio autem est, qu huiusm odi crimina et facta, quae pluribus testibus probantur, etiam antequam delata sunt, pluribus sunt nota, praesertim vero si a Presbyteris commissa fuerint. Vere enim ille Nam lux altissima fati, Occultum nihil esse sinit. Cum igitur crimina illa pluribus iam innotuerunt, et in ore hominum et cognitione versantur, sunt iam publica. Ex quo fit, vt etiam sint publica cen- fura notanda, et publice arguenda, quemadmo- dum hoc loco praecipit Paulus. Deinde cum Presbyteri caeteris exemplo esse debeant vitae sanctitate, ita et poenitentia, si deli- querint:et censura atque castigationc, si non resi-  \pend
\section*{CAPVT  V. }
\marginpar{[ p.317 ]}\pstart puerint, vt sint de populo reliqui ad iniuriam, et peccandum tardiores. Sed etiam olim reliquis Ecclesiis significabantur ii, qui ob scelus erant depositi, vtex concilio Carthaginensi I. et 2. apparet, et ex Cypria. Atque has literas formatas appellabant. Certe Presbyter Flauianus Episco- pum suum publice reprehendit, quod in Chri- stiana religione veram fidem simularet, vt ait Theod.libus 5.cap,3. et idem lib.  5.cap.28. Laudat Chrysostom. quod suos Presbyteros libere, tum moneret, tum publicis censuris castigaret. Vt autem male agentes Presbyteri sunt semper repre- hendendi: ita bene agentes sunt et cohortandi et laudandi, vti Episcopi Ægyptii Athanasium exulem propter fidem aliis Ecclesiis commenda- runt, vt apparet ex Tripar, histor. lib.  4.cap.19. Plato annotat tria esse supplicii, per quod cri- men punitur, genera κόλαση, τμην, οθάδεγμα. κό- λασις est in ipsius delinquentis corpore et sensu inflicta poena, vti ictus fustium, abscissio auricu- lae. Tiun poena ea, quae ob publicam tranquillitatem violatam imposita est, qualis est infamia, exilium, multa pecuniaria. Neutra castigatio pertinet ad Ecclesiam: sed ad magistratum quanquam ab Eu- genio primo constitutum est, vt liceret Episco- pis suos Presbyteros criminis conuictos carce- re multare, qui certe fuit manifestus potestatis a- busus. Παράδ'ειγμα est poena, quae caeterorum eru- diendorum vel terrendorum causa publice con- stituta est, atque haec ab Ecclesia imponi potest. Verbum inde est ὀραδειγεώζεθς, quo vtitur Mat.1 Duo vero posse obiici videntur. 1 Quod est Matth.18. vers.15. Mitius repre-  \pend
\section*{AD I. PAVL. AD TIM. }
\marginpar{[ p.Jle ]}\pstart hendendos esse qui peccant, quam hic Paulus iu- beat, nimirum inter nos et illos solum. Res.Illud Chri sti dictum de priuatis offensis intelligendum esse, nimirum cum nos priuatim laesi sumus, necdum offensio publice innotuit. Hoc autem delictum, quod publice docet arguendum Paulus, est pu- blicum. Nec enim nisi propter publica aut publi- ce nota crimina vel peccata reprehendi in coetu Ecclesiae quisquam debet, siue is sit pastor et Pre- sbyter, siue priuatus. Atque ita hunc locum ex- plicat Augustinus, nempe de peccatis Presbyte- rorum, quae non latent, sed publica sunt. De Cor- rept. et Grat. cap. 16. cuius sententiae plane as- sentior. 2 Obiicitur, peccata quae occulta esse opor- tuit, nota fieri hac ratione. Id quod et ab officio hominis Christiani alienum est, et pugnat cum aedificatione Ecclesiae, ac illa veneratione, quae de- ferenda est praefectis nostris. Respond. Ea tantum argui publice ex hoc Pauli praecepto posse, quae publice nota sunt: non autem quae occulta sunt, quaeque celari, et ad caeterorum aedificationem tegi oportet, vt in exemplo Petri a se publice in Ecclesia reprehensi apparet Galat. 2. vers.14. In summa omnino disciplina aliqua Ecclesiastica o- pus est in Ecclesia Dei, si modo hominum effrae- nem licentiam coerceri, et morum sanctorum neruum aliquem, atque vinculum inter nos reti- neri volumus. Id quod, etsi verissimum est, Do- ctoris tamen eximii, et primarii D. Ro. Gualteri verbis potius, quam meis intelligi malo, qui in praefatione doctissimorum illorum suorum in Ga- latas commentariorum ita scribit. Quare, ait,  \pend
\section*{CAPVT  Ve }
\marginpar{[ p.319 ]}\pstart Ecclesiis disciplina opus, est sine qua non magis consi- stere possunt, quam res domestica, quae patrefamil. et moderatore idoneo destituitur: non magis item, quam Respubus quae nullis legibus gubernatur, aut nul- los habet legum vindices. Etenim non minus vere quam prudenter olim quidam dixerunt, malum esse sub prin- cipe viuere, sub quo nihil liceat: sed multo peiorem esse corum fortem, qui principem habent, sub quo licent o- mnia. At qui huic loco commodum adiicere libet quod idem Doctor praestantissimus et pientissi- mus in iisdem commentariis ad cap.2. Homil.14. his verbis disertissime, sanctissimeque scribit. Admonemur hoc exemplo (de Petro a Paulo pu- blice reprehenso agens) ait nullam vel in Ecclesia, vel in Republ. dignitatem esse tantam, quae censurae et re- prehensioni eos eximat, qui aliquid prater officium faciunt. Etenim cum omnium Dominus sit Deus, eius quoque verbo omnes homines cuiuscunque sint loci, aut ordinis, subiici, et obedire conuenit, et ad omnes ver- bi Dei praecones pertinet, quod olim Hieremiae dice- bat Deus, Ecce ego te facio hodie ciuitatem munitam. columnam ferream, et murum aeneum contra omnem terram, contra reges Iuda, contra principes eius, contra sacerdotes, contra vniuersum populum terrae. Haec Gualterus mihi obseruandissimus. Ex hoc autem loco colligitur, iam tempore Pauli in Dei Ecclesia duplex censurae Ecclesia- sticae genus fuisse vsitatum, vnum Priuatum, alte- rum Publicum. Priuatum est, cum qui offendit, a- gnoscit culpam eamque fatetur ei, quem offendit, nullo confessionis teste adhibito, sed inter se et eum, quem offendit, tantum et priuatim id facit. Hanc descripsit Christus Matth. 18. descripsit  \pend
\section*{AD I. PAVL. AD TIM. }
\marginpar{[ p.320 ]}\pstart quoque Iacobus cap.5.vers.16. quae saepe a Pasto- re fieri praescribitur ad mutuam reconciliationem: et non praecepta, tamen ex officio a quoque no- strum fieri debet, vt bonum conscientiae testi- monium habeamus. Publicum cenfurae genus est, cum agnitio offen- sionis non fit fratri soli offenso: sed adhibitis aliis velut illius confessionis testibus. Haec etiam a Christo commemorata est Matth.18. vers.16. Hie autem Bernard. sermo. 44. consilium locum habet, qui oleo suauium admonitionum morda- cia medicamenta, et vinum compunctionis saepe docet esse adhibendum, duris videlicet et prae- fractis. Denique si duritia tanta est, etiam censu- rae Ecclesiasticae baculo percellendum conten- ptorem docet, qui priuatim reconciliari cum fra- tre nolit. Huius autem postremae censurae multae sunt species et differentiae, quae ex loci et multi- tudinis, in qua fit, ratione sumuntur. Eo enim ma- gis publica est haec confessio et censura, quo ma- ior est coetus, et frequentior celebriorque locus, in quo fit, atque quo plures illius testes adhiben- tur. Ergo sic a nobis videtur diuidi posse vt Alia dicatur Consistorialis: alia Ecclesiastica censura. Consistorialis est, quae coram solo senatu Eccle- hartieo ne au to, qui peccauit. Coetum enim Pre- sbyterorum senatum Ecclesiae vocat Hieronym. Ecclesiastica, quae fit praesente et aduocata v- niuersa illius loci et vrbis Ecclesia, et non tan- tum senatu Ecclesiastico praesente, id est, eo tem- pore et loco, quo coetus Ecclesiae conuenit, et fre- quens est. Publica vero nunquam, nisi pro publico deli-  \pend
\section*{CAPVT  V. }
\marginpar{[ p.52 ]}\pstart cto, indici debet. Publicum autem delictum duobus modis dicitur, vel quod Pluribus, vel quod Omni- bus notum est: idque, vel ex ipsius facti perpe- trandi ratione vt quod in publico loco, vel coram magno coetu hominum committitur: ex et sparso postea illius facti rumore, et pluribus communi- cato, vti multa peccata priuatim et clam primum admissa postea ex fama innotescunt, et diuulgam, tur. Ergo quae delicta omnibus innotuerunt, ea coram omnibus agnosci operaepretium est, et de iis publicam in totius Ecclesiae coetu exomolo- gesin fieri oportet, si verum remedium morbo adhibere velimus. Quae pluribus tantum et non toti Ecclesiae nec maiori parti innotuerunt peccata, illa in senatu tantum Ecclesiastico:non autem coram vniuersa Ecclesia confiteri oportet, ne latius illud pate- fiat, quod occultum esse debuit. Illa vero delicti nostri, quo offendebatur Ec- clesia, confessio publice praesenteque tota Eccle- sia facta, exomologesis speciali et antiquo nomi- ne dicta est, quemadmodum ex Latino Cypria- no, et Graecis scriptoribus apparet. Haec in vsu fuit etiam primis Ecclesiae vetustissi- misque temporibus, vti ex hoc loco satis probari quoque potest: et ex Ecclesiastica historia pene to- ta Philippus imperator qui post Decium Romae regnauit, sic poenituit, atque confessus est, vt testa- tur Eusebus  lib. 6.Histor. cap. 34. Ecebolius quidam, etiam ex consuetudine iam recepta, sic peccata sua agnouit Socrat. lib. 3.cap.13. Theodosius im- perator, qui Maior dictus est, sic suum delictum, quod erat publicum in Ecclesia Mediolanensi a-  \pend
\section*{322 AD I. PAVL. AD TIM. }\pstart gnouit, Theodorit. lib. 5.Histor. cap.18. et Sozom- libus 7, cap.24. Atque hic locus est vtilissimus. Aliquot igitur quaestiones in hoc argumento sunt nobis explicandae, quae magnum in exercen- da disciplina Ecclesiastica vsum habere solent. Ac primum quaeritur. In quo peccati genere locum haec publica exo- mologesis habere debeat. Resp. Cum differat iurisdictio ciuilis ab Ecclesiastica, et haec tantum voluntaria sit, illa vero contentiosa: haec solius poenitentiae animique salutis testificandae gratia comparata sit: illa publicae tranquillitatis tuendae causa sit constituta: in eo tantum locum habere- posse siue publicam siue priuatam delicti confes- sionem, qui agnoscit se deliquisse, idque sponte fateri paratus est. Nam in eo qui negat a se pecca- tum esse (quia vel factum ipsum a se commis- sum pernegat, vel hoc suum factum negat esse peccatum, et damnandum) nulla certe vel publi- ca vel priuata exomologesis proprie locum ha- bere potest, etsi hic testibus ipsis fuerit in senatu Ecclesiae conuictus fecisse Id, quod tamen perne- gat. Itaque huiusmodi pertinax est ad magistra- tum remittendus, vt de facto ipso amplius inqui- rat. Sed tamen obstinatum hominem et peruica- ceni Lecieria nonnunquam excommunicatione aut suspensione Coenae perstringere potest, id est si factum a se commissum esse agnoscens, tantum qualitatem facti, id est, ἀνομίαν et turpitudinem il- lius non vult talem videre, et idcirco non poe- nitere. Negantem autem id a se, de quo accusatur factum, nunquam Ecclesia damnabit, neque ad confessionem delicti, in cuius perpetrati dene-  \pend
\section*{CAPVT a v }
\marginpar{[ p.323 ]}\pstart gatione persistit, coget, aut adiget: sed confiten- tem tantum et vere poenitentem, inquam, Ec- clesia ad confessionem adducet. Probo igitur id quod in can. Presbyter xv. quaest. I. in eandem sen- tentiam scriptum est. Et haec quidem in vniuer- sum de iis, qui censura Ecclesiastica ad delicti confessionem compelli debent et possunt. Quod si de iis, qui publice confiteri debent, quaeritur, Respondeo eos tantum ad id esse compel- lendos, iisque folis indicendum esse, quorum deli- ctum et factum iam publicum et notorium est Ecclesiae. Quaesitum est, Quomodo fieri haec publica con- fessio debeat. Atque hoc partim ad formam ex- ternam spectat: partim etiam ad ipsius peccato- ris publice confitentis officium. Ac quidem quod ad ipsum confitentem attinet, nemo per nuntium, neque per scriptum facere illam confessionem olim potuit, sed corporali praesentia, et ipso ore suo, vt est praeclare responsum ab Augustino in libus  de Poenitent. et est relatum in cano. Quem poenitet distinct. I. de Poenitentia, et hoc ipsum retinendum censeo. Quod autem ad formam et ritus externos. Pri- mum haec publica confessio simplicissime in Ec- clesia fieri solebat. Deinde vero huic simplicita- ostentationem et pompam. Ergo concilio Aga- thensi constitutum est, vt publice confitens ini- tio Quadragesimae staret nudis pedibus pro fori- bus templi, sacco indutus, cilicio caput tegens, et supra caput cinerem spargens, vultu in terram demisso. Deinde exomologesi facta, Ecclesiae con-  \pend
\section*{AD I. PAVL. AD TIM. }
\marginpar{[ p.344 ]}\pstart ciliabatur per manuum impositionem a pastore siue Sacerdote. Quae res fuerunt pompae potius externae, quam vera castigatio delicti: sed quaesi- ta ratio est nugarum in Ecclesiam pro vera poe- nitentia substituendarum, et affectata inepta quaedam seueritas. Ista referuntur in can. Poeniten- tes et cano. In capite quadragesimae distinct.50. quae prorsus a nostris Ecclesiis exulate debent. Quid enim signa haec poenitentiae praescribi ne- cesse est, cum qui vere poenitet, satis illa exhi- beat:imo vero potius attollendus sit, et erigen- dus, quam deprimendus, ne incidat in despera- tionem? Quaeritur qui publicam exomologesin facere possint et permittantur. Respo. Omnes debere, et non tantum posse, qui eo peccati genere delique- runt, cuius poena debet esse consessio publica. I- taque nullius dignitatis, sexus et conditionis ra- tio habenda est, ne sit in Ecclesia Dei damnabilis illa personarum acceptio. Quod fieri vetat Do- minus Malach.2.vers.9. Apparet etiam id ipsum ex hoc loco, vbi ne Presbyteris quidem ipsis par- cit Paulus:item exemplo Theodosii imperatoris qui ab Ambrosio Mediolanensi Episcopo ad eam compulsus est. Quare perniciosissima est Cano- nici Iuris et Romanici decisio, atque sententia, quae negat clericos vllos quantumuis grauiter errantes ad hanc confessionem et exomologesin publicam damnandos ac compellendos, vel etiam si velint, admittendos, ne toti ordini clericorum fiat iniuria. Itaque elericos detrudunt in Monasteria vbi poenitentiam agant solitarii, et nulli noti. Ita prae ceptum est in canone Alienum, et can. Con-  \pend
\section*{CAPVT  V. }\pstart firmandi distinct.5o. qui plane et apertissime cum hoc Pauli praecepto pugnant. Quaesitum est quoque, quoties in vita fieri pu- blica haec exomologe sis possit. Respond. Augu- stinus in epistola ad Macedoniam semel tantum in vita, ne vilescat Ecclesiae authoritas: aut ne fiant audaciores homines ad peccandum, si saepius in vita cuiquam liceat publice poenitere canou. Quanuis caute distinct. 50. Sed cum nulli pecca- tori poenitenti sit spes veniae, et remissionis pec- catorum deneganda: toties in vita fieri haec pu- blica confessio potest, quoties quis eo modo deli- quit, quo iuste publica confessio indici atque im- poni debet. Et iste fuit certe nimius Patrum ri- gor, et pene Nouatianorum error, nimiaque au- steritas, quae homines ad desperationem tandem adducebat, quasi denegata peccatorum remissio- nis impetrandae spe. Quoties igitur vocat ad se peccatorem Dominus, toties ille est ab Ecclesia admittendus!, etiam in die septuagesies septies, vti docet Christus ipse Matth. 18.nisi (vti diximus) velimus esse Nouatiani. Praeclare vero Bernard. quatuor esse docet, quae nos a publica hac exomologesi deterrent, quae sunt omnino nobis et cauenda et debellan- da, si peccauerimus Sunt autem illa quatuor, Pu- dor, Timor, Spes, Desperatio. Itaque recte idem Ad milit. Templi cap.12. Et quidem verbum in cor- de pectoris operatur salutiferam contritionem. Ver- bum vero in ore noxiam tollit confusionem, ne impe- diat necessariam confessionem. Ait enim scriptura. Est pudor adducens peccatum, et est pudor adducens gloriam. Bonus pudor est, quo peceasse, aut certe pec-  \pend
\section*{AD I. PAVL. AD TIM. }
\marginpar{[ p.326 ]}\pstart care, confunderis. Huiusmodi proculdubio pu- dor fugat opprobrium, parat gloriam, dum aut peccatum omnino non admittit: aut certe admissum et poenitendo punit, et confitendo expellit: si tamen gloria etiam nostra haec est, testimonium bona con- scientiae. Hic igitur pudor est secundum Deum, qui nos ad eum reuocat, facitque vt peccati nos pudeat: non autem resipiscentiae, et melioris fru- gis. Praeclare enim Augustinusin Ioan. tract. 12. Qui confitetur et accusat peccata sua, iam cum Deo facit. Cum autem coeperit tibi displicere quod fecisti, inde incipiunt bona opera tua, quia accusas mala o- pera tua. Denique haec censura est pars potestatis clauium, quae a Domino Deo data est Ecclesiae. Quam qui impugnat, optimum Ecclesiae doctri- naeque Christianae retinendae neruum et vincu- lum tollit, nimirum. authoritatem illi a Deo i- pso concessam, et ius tuendae conseruandaeque Dei gloriae ac reformandae hominum vitae in me- lius. Obiiciunt tamen quidam atque afferunt ali- quot rationes, per quas istam disciplinae Ecclesia- sticae partem abrogent, vel saltem in odium addu- cant si possunt. Primum autem eorum argumentum est. Quid prodest haec confessio, si iam peccator voce do- minica et interna Spiritus sancti vi resipuit et re- suscitatus est. Cui argumento respondet August. in sermone 8.de Verbus  Domini secundum Mat- thae. Quid prodest Ecclesia et authoritas ipsius confitenti, cui Dominus dixit, Quae soluetis in terra, etc. Ipsum Lazarum attende, cum vinculis prodit, iamviuebat: sed nondum liber ambulabat  \pend
\section*{CAPVT . V. }
\marginpar{[ p.27 ]}\pstart \phantomsection
\addcontentsline{toc}{subsection}{\textit{21 Obtestor in conspectu Deï, et Domi- ni Iesu Christi, et electorum Angelorum, ut haec serues absque eo, vt vnum alteri prae-}}
\subsection*{\textit{21 Obtestor in conspectu Deï, et Domi- ni Iesu Christi, et electorum Angelorum, ut haec serues absque eo, vt vnum alteri prae-}}vinculis irretitus. Quid ergo fecit Ecclesia et con fessio, nisi quod ait Dominus continuo ad disci- pulos, Soluite illum, et sinite abire. Ergo non est satis ei qui publice peccauit, interni animi com- punctione affici, nisi datum a se offendiculum tollat, et resarciat aperta culpae suae confessione, vt si qui prius illius exemplo peccare vellent vel didicissent, iam contrario eiusdem exemplo resi- piscant, et auertantur. Fit enim haec confessio non ad peccatorum remissionem impetrandam a Deo: sed ad Ecclesiam aedificandam, quae scelere illius destrui videbatur. Obiiciunt irem, Haec confefsio publica affert infamiam, cum iam peccatum negare non possit is qui publice agnouit. Respond. Minime vero, I- mo et Ecclesiae, et ipsis adeo Angelis illa gaudium affert, cum sit certissimum resipiscentiae animi testimonium, et aberrantis ouis receptio, atque inuentio. Malach. 18.vers.1o. Luc.15.vers.1o. Neque certe laudem dare Deo (quod fit hac publica con- fessione apertissime, quemadmodum docet Au- gustinus in Psal.141.) cuiquam Christiano vnquam irrogauit infamiam. Et longe distat haec Eccle- siastica censura a ciuili confessione. quae Amenda honoraria vocatur: quod haec hominum ratione fit, illa Dei: haec pro poena statuta est, illa pro aedi- ficatione Ecclesiae, et testimonio aeternae salutis. Caeterae obiectiones sunt leues et nullius mo- menti. Itaque a nobis praetermittentur hoc loco. 21 Obtestor in conspectu Deï, et Domi- ni Iesu Christi, et electorum Angelorum, ut haec serues absque eo, vt vnum alteri prae-  \pend
\section*{AD I. PAVL. AD TIM. }
\marginpar{[ p.628 ]}\pstart \phantomsection
\addcontentsline{toc}{subsection}{\textit{feras, nihil faciens in alteram partem decli- nando.}}
\subsection*{\textit{feras, nihil faciens in alteram partem decli- nando.}}feras, nihil faciens in alteram partem decli- nando. Alius canon, qui tamen ex superiori pendet. I- taque hic versiculus est ἀκολέθησις. Cum enim de iudicio aduersus Presbyteros instituendo su- pra praeceperit, docet, quale hoc iudicium esse et fieri debeat, ne in iudicando peccetur. Est autem huius canonis summa, summam in Ecclesiasticis iudiciis synceritatem adhibendam esse, ne vel in personarum iudicandarum acce- ptatione, vel ipsius iudicantis temeritate peccetur Hoc autem praeceptum quanquam a Paulo in iis iudiciis, quae aduersus Presbyteros et Pastores Ecclesiae exercentur traditum est, tamen in vni- uersum ad omnia iudicia, siue politica, siue Ec- clesiastica pertinet, de quocunque siue priuato, siue ἠγρμένῳ illa fiant. Atque hunc canonem in o- mnibus Ecclesiasticis iudiciis obtinere debere praeclare traditum est a Theodorito lib.  Histor. 2.cap.16. et a Libero Rom. Episcopo responsum aduersus Constantii imperatoris minas Athana- sium ex praeiudicio quodam deponi praecipientis. Hoc idem etiam praeceptum in politicis iudiciis locum habere apparet ex 2. Chronic.19. vers.6. a- deo vt maximi sit vsus in omnibus rebus, sitque ipsius iudicum conscientiae dirigendae tanquam norma, vt recte officium faciant. Nos tamen de Ecclesiasticis tantum iudiciis hic agemus Pauli vestigiis insistentes. Non nude autem, neque sim- pliciter hoc praecipit Paulus, sed cum obtestatione, eaque grauissima, et terribili, quemadmodum loquitur Chrysost. Ratio autem huius obtestatio-  \pend
\section*{CAPVT V. }
\marginpar{[ p.329 ]}\pstart nis tam seriae est duplex. Primum quod de re ma ximi momenti, et late fusa et patenti acturus exci- tare eos voluit quos affatur, et ad quos hoc prae- ceptum pertinet, quos hoc fulgure et stimulo per stringit. Agitur enim de Iudiciorum synceritate, ex qua vna re tum vitae et societatis humanae conser- uatio:tum ecclesiae ipsius vinculum, pax et disci- plinae omnis ratio pendet. Itaque hac synceritate neglecta, aut violata multam rerum confusionem et perturbationem tum in politicis, tum in ec- clesiasticis rebus sequi necesse est. Altera ratio ad dendae huius obtestationis est rei ipsius, quae prae cipitur, difficultas. Nihil enim tam arduum et dif ficile etiam optimo cuique etcordatissimo, quam illam synceram, aequam, et mediam viam in iudi- cando tenere, vt in nullam partem vel personarum acceptatione et respectu: vel ipse proprio animi motu a veritate vel officio deflectas. Admonet enim hic Paulus Timotheum, non gregarium quempiam et nouitium pastorem: sed egregium et insignem. Itaque proponit Paulus nobis alios nostrae sententiae iudices futuros, quorum reue- rentia et metu affici debemus, ne quid vel ho- minibus vel nostris affectibus tribuamus. Singula vero huius obtestationis verba suntdi ligenter obseruanda et perpendenda. Ac primum quod vel ludices vel testes quos producit pre- sentes esse testatur. Quod enim dicit perinde est, atque si spectatores, ac αυτόπτας eos esse affirmaret deinde testes adhibet grauissimos nempe, Deum ipsum, Iesum Christum, et Angelos sanctos. Dei nomine hic proprie patrem intelligimus, quod Iesu Christi Filii seorsum postea facta mentio  \pend
\section*{AD I. PAVL. AD TIM. }
\marginpar{[ p. ]}\pstart est. Patris enim et Dei voce complectitur Pau- lus quicquid generaliter ad diuinam naturam per- tinet, et est trium personarum commune: quo eo- dem dicendi genere passim vtitur scriptura. Er- go et Patrem et filium ipsum, qua Deus est, et Spi- ritum sanctum testem esse nostrarum actionum; et earum imprimis, quae in ecclesia et in iudiciis fiunt, quibus duabus nobis ipsa Dei maiestas ma- xime representatur spectatorem pronuntiat Pau- lus. Nam Deus etiam nudas cordis nostri cogita- tiones nouit, et intuetur non tantum ipsas actio- nes nostras perspicitque quae ab hominibus cerni non possunt. Inde illud est etiam apud profanos ho- mines vulgare ὁμμα θεο͂ πάντα καθεςα. Dei oculus omnia perspicit. Itaque recte testis et vindex ad uocatur Deus nostrarum actionum. qui vbique est, omnia videt, et omnium futurus est iudex. Si- miles autem huic scripturae loci sunt pene infini- ti veluti, 1. Samuel.2o.ver.42.2.Chron.24. vers.22. Atque hic primus est testis. Iesum Christum secundo loco testem citat et producit Paulus, quem Dominum vocat. Domi- nus est Christus quodsit constitutus a patre rex et Dominus omnium, dataque illi potestas in co- lum et in terram. Item quod omne iudicium pater concessit filio, neque iudicat quenquam vt est Ioan.5. vers.22. Sed et quatenus ecclesiae caput est Christus ecclesiasticis iudiciis praeest, neque tantum vt testis et spectator iis interest: sed vt iu- dex ipse futurus, ac rationem eorum a pastoribus, tam quam a Ministris a se delegatis, exacturus. Etsi ve ro Christus mediator et quatenus Iesus Christus est, homol est et Deus: tamen qua homo est,  \pend
\section*{CAPVT  V. }
\marginpar{[ p.3 ]}\pstart non est iam in his terris, neque in medio nostri versatur siue visibilis, siue inuisibilis praesentiae, vt nostras actiones oculis cernat, modo: sed tantum qua Deus est et omnia haec inferiora videt. Recte tamem Iesus Christus dicitur testis omnium actio num nostrarum, quia ille Deus et homo, est vnica tantum persona, quae hypostatica vnione vtra- que illa natura constat. Neque hic curiose dis- putandum est, Vtrum humana Christi natura, quae in coelos euecta est, omnia, quae hic in terris geruntur, ex sese sciat. Quanquam enim supra omnes angelos dignitate! est euecta, tamen illa non est in deitatem mutata, vel deificata. De so- lo Deo dicitur illum, quae hic fiunt, omnia nosse et videre: quanquam Christum, qui homo est ad Dei dexteram exaltatus, non tantum ea scire, quae angeli, fatemur: sed omnia, quae ad ecclesiae suae ad- ministrationem, totiusque huius mundi curam quam a patre acceptam habet, pertinent et spectant, habere nota et perspecta. Addit, electis Angelis. Primum electos vocat, vt eos distinguat a malis Angelis qui vulgo dicuntur, et qui a Christo Angeli Diaboli, id est, maledicti nominantur in Math.25. vers.41. deinde docet, pro hi ipsi angeli propter custodiam fidelium sibi a Deo demandatam intersint ecclesiae coetibus, et actionibus ecclesiasticis, tanquam testes et spe- ctatores pastorum negligentiae vel temeritatis: vel perfidiae, si quam admittant. Merito igitur ter tio loco citantur. Sunt enim spiritus Ministrato- rii, qui ad subsidium fidelium et piorum a Deo mittuntur. Itaque actiones nostras norunt. Inci- dunt autem in hunc locum variae quaestiones, quas  \pend
\section*{AD I. PAVL. AD TIM. }
\marginpar{[ p.33- ]}\pstart breuiter tantum perstringemus, et explicabimus Ac primum igitur quaeritur, velutia Basilio libro de spiritu sancto ca. 13. vtrum licuerit Paulo, Ange los Christo et Deo adiungere. Num vero ista sit blasphemia parem facere creaturam creatori, eamque illi honore adaequare. Id quod plane idololatricum est. Chrysostomus responder per modestiam id sieri a Paulo, vt pote qui non solum Deum testem, qui nobis est inaccessibilis longeque dignissimus adhibere voluerit: sed etiam creaturas ipsas, quae sunt nobis notiores ac familiariores: imprimis igi tur Angelos bonos commemorasse, qui sunt no- bis conciues Ephes.2.ver.19. Haec Chrysostomus. Respondeo vero distinguendum esse finem, pro- ter quem et Deus et Angeli in testimonium pro ducuntur. Etsi enim vtrique appellantur, vt sit ma- ioris momenti haec obtestatio, et percellantur ve hementius lectorum, et eorum ad quos haec ad- monitio spectat, animi, tamen et Deus et Christus non tantum producuntur vt testes a Paulo, sed etiam vt vindices futuri, et Iudices pastorum cul- pae, et negligentiae: Angeli vero proponuntur et appellantur tantum, vt testes. Ex quo fit, vt non adaequentur Deo: Angeli: neque habeat hic locus quicquam, quod idololatriam sapiat, aut in Deum blasphemiam. Testes vero harum rerum, quae in ecclesia ge- runtur, citari merito posse Angelos certum est, et eos imprimis, quibus a Deo ecclesiae cura de- mandatur vt ex Daniele. 1o. vers. 13. apparet. Et huic nostrae sententiae consentit Bernar. ser. 7. can- tic.cantic. Doleo perinde aliquos vestrum graui in sacris vigiliis deprimi somno, nec coeli ciues  \pend
\section*{CAPVT  V. }
\marginpar{[ p.333 ]}\pstart reuereri: sed in praesentia principum tanquam mor tuos apparere, cum vestra ipsi alacritate permoti vestris interesse solenniis delectentur. Angelos i- gitur inuisibiles cum piis versari scribit. Itaque Christus ipse venturus dicitur cum Angelis illis beatis Luc.9. vers.26. Iudae. vers.14 Et ipsi gaudent si quis peccator ad Deum conuertatur. Luc.15.ver. 1o. quia id sciunt, cum cura piorum quibusdam sit commissa. Quod idem de sanctis et in coelum re- ceptis hominibus dici non potest. Nam idem mu- nus non est illis a Deo impositum et de mandatum, vt nos in viis nostris conseruent. Quod tamen est a Deo Bonis angelis praeceptum. ps. 91. Denique Paulus in 1. Corinth.11. vers.1o. et Ephes.3.ver.1o. praesentes et spectatores eos interesse coetibus ec clesiasticis docet et asserit. Propter quos etiam ait honestum esse et decorum, vt foeminae Chri- stianae velentur in Dei ecclesia et conuentu. Secundo loco quaeritur, Cur hos electos ap- pellet: Num etiam alii, qui mali dicuntur Repro bati a Deo fuerint, et dici possint. Resp. Etsi illa electio et reprobatio, de qua tam saepe in scriptu ra fit mentio, proprie ad homines pertinet, non ad Angelos:tamen cum nihil praeter Dei volun- tatem acciderit etiam in lapsu angelorum, imo vero cum ex Angelis alii ceciderint, alii perstite rint, Deo ipso ita ordinante et praedestinante. Deni que cum ex ipso Dei decreto hoc totum ita eue- nerit, dubitari non posse, eos Angelos, qui persti- terunt in sua origine, a Deo electos quidem me ra ipsis gratiua fuisse: Eos aunt, qui ceciderunt, re- probatos ab eodem Deo: idque iusta de causa, quamquam nobis ignota. Voluit enim Deus hac ra-  \pend
\section*{AD I. PAVL. AD TIM. }
\marginpar{[ p.334 ]}\pstart tione, et in angelis, et in hominibus, suae tum misericordiae, tum etiam iustitiae diuitias ad glo- riam nominis sui, patefacere et demonstrare. Nam licet opifici facere de opere suo, quod ipsi libucrit Ergo boni Angeli electi fuerunt, etiam ante quam a Deo crearentur. Non autem ex vllis eorum meritis: sed ex sola Dei ipsius dignatione et gratia, vt loquitur Bernardus: quicquid Scho- lastici ex Augustino male intellecto dicant et philosophentur: Mali vero reprobati fuerunt, vt Boni electi. Tertio loco quaesitum est, ecquae sit istius ele- ctionis Angelorum causa: Num Christus quate- nus mediator ecclesiae: cuius etiam illi sunt mem bra: praesertim cum Paulus Coloss.1.vers.2o.scri- bat Deum sibi reconciliasse omnia per Christum. Ex quo colligunt quidam Christum esse etiam e- lectorum Angelorum Mediatorem, alio tamen modo, quam hominum piorum et electorum Resp. vero ipse, ceterorum rationibus breuitatis causa, omissis, Christum quatenus creator eorum est, non autem quatenus passus in carne, videri horum bonorum Angelorum etiam mediatorem esse. Cum enim Deus numquam iis fuerit offensus, nulla reconciliatione certe opus habuerunt. Dein de haec eorum electio a Iuda desinitur conserua- tio et confirmatio ipsius eorum primi status et ori ginis, adeo vt nihil prorsus illis restitutum fuisse per huiusmodi electionem dicatur, quoniam ni- hil prius fuerat ademptum, neque quicquam ab iis amissum. Ergo ipsi quales a Deo primum con- diti fuerunt, Deo autori suo certe placuerunt Ne que vera tamen huius eorum electionis causa est,  \pend
\section*{CAPVT  V. }
\marginpar{[ p.335 ]}\pstart qualitas illa et status in quo tantum creati sunt. ( Omnes enim angeli, etiam qui ceciderunt, dice- rentur et fuissent electi, quia in eodem statu feli- ci omnes conditi sunt) sed praecessit haec Dei e-s lectio etiam illorum creationem et ortum, et ab aeterno in Deo fuit. Hunc statum tam excellentem omnes quidem a Deo mera ipsius gratia erant con secuti, sed Dei illa gratia per quam sunt alii dicti e- lecti, et prior est, et maior gratia creationis. Hoc vero qui a Deo electi fuerunt, plus iam sunt adepti post malorum lapsum, qu de feliciss. illius conditionis suae statu perpetuo futuro iam a Deo certissimi effectit et in eo stabiliti sunt, et in aeter- num confirmati: Id quod, Deo illis reuelante, iam cognoscunt et intelligunt. Bernandus in Sermo. 5de festo omnium sanctor. Inter Angelos et ho- mines assignari potest ista diuersitas sanctitatis. Neque enim tanquam triumphantes honorari pos se videntur, qui nunquam pugnasse noscuntur. Ali- ter tamen honorandi sunt etiam ipsi, tanquam amici tui Deus, cuius nimirum voluntati semper adhaese- runt, tanta vtique felicitate quanta facilitate. Aliud igitur in Angelis sanctitatis genus. Et ser.5. de vi- sio. Non est innata eis sed a Deo collata iustitia, quae inferior est diuina iustitia. Idem Bernardus, serm.22. cantic. cantico. Si angeli, ait, numquam re dempti sunt, alii vtique non egentes, alii non pro merentes: illi quidem quia nec lapsi sunt, hi au- tem quia irreuocabiles sunt, quo pacto tu dicis Dominum Iesum Christum eis fuisse redemptio nem? Audi breuiter. Qui erexit hominem lapsum, dedit stanti angelo, ne laberetur: sic illum de ca- ptiuitate eruens, sicut hunc a captiuitate defendens.  \pend
\section*{AD I. PAVL. AD TIM. }
\marginpar{[ p.336 ]}\pstart Et hac ratione fuit aeque vtrique redemptio sol- uens illum hominem, et seruans istum angelum. Liquet ergo sanctis angelis Christum Dominum fuisse redemptionem, sicut iustitiam, sicut sapien tiam, sicut sanctificationem. Haec Bernard. Au- gusti in Enchirid. ad Laurentium cap. 28. idem sentit. Ac de obtestatione quidem haec dicta sunt. Vi deamus iam, a quibus vitiis pastori iudicium ec- clesiasticum exercenti sit cauendum, atque fu- giendum. Duo autem sunt nimirum.1. Praelatio personarum propter ipsarum dignitatem 2 Temeritas, prop- ter iudicantis affectum. Neminem enim propter dignitatem excusari debere censet Paulus. Ne- minem etiam propter affectum iudicis praecipi- tanter, et inconsiderate damnandum esse vult. Quae duplex ratio magnos in Iudicando erro- res inducit, ex quibus tandem euertitur eccle- sia. Caussae igitur propter quas peruertuntur iudicia sunt in nobis: vel in illis ipsis, de quibus iudicamus, a nobis quaeruntur. Vocem Προκρίμα- roς ad praeiudicia illa referunt, per quae non per- sona: sed causa potius Presbyteri, qui iudicandus est, vel grauatur, vel leuatur, id est, antequam cognitum sit a nobis de eo ( de cuius causa agitur) iam animi quadam praecipiti sententia domi con- cepta definitum est. πρόσκλισιν autem eam esse vo- lunt in iudicando rationem, quae leges ordinarias et statutas praetermittit, vt faueat illi, de quo agi- tur. Sed. Graecae vocis significatio non patitur hanc interpretationem, potiusque sequenda no- bis est D. Bezae explicatio. Vtcunque Paulus  \pend
\section*{CAPVT  V. }
\marginpar{[ p.337 ]}\pstart \phantomsection
\addcontentsline{toc}{subsection}{\textit{22 Manus cito ne cui imponito, neque communicato peccatis alienis: temetipsum pu- rum serua.}}
\subsection*{\textit{22 Manus cito ne cui imponito, neque communicato peccatis alienis: temetipsum pu- rum serua.}}non omnia iudiciorum peruersorum vitia nunc enumerat: sed ex iis aliqua, ex quibus paucis caete ra colligere licet. Gregorius Magnus.4 esse sum- mas iudiciorum corrumpendorum causas scribit scilices Timorem, Cupiditatem, Odium, Amo- rem, cuius sententia descripta est in Can. Quatuor modis. 11 quaest.3. Quae sane aliqua ex parte expo- nit, quae breuius hic Paulus recenset. 22 Manus cito ne cui imponito, neque communicato peccatis alienis: temetipsum pu- rum serua. Canon praestantissimus, qui etiam ad pastorum erga pastores officium pertinet. Nam superiores canones ad pastores iam creatos spectabant: hic autem de Pastoribus adhuc creandis agit. Hoc v- no breuiter complexus est Paulus optimam et totam pene eligendorum pastorum rationem, quae nobis hodie obseruanda quoque est, vt legi- time vocatos pastores habeamus. Quae res tan- ti est in ecclesia momenti, vt propter eam supe- riorem obtestationem a Paulo praemissam esse censeat Chrysosto. Certe cum de ea re monitum Timotheum esse oportuerit: et magni periculi plenam esse, et grauis momenti in ecclesia ordi- nanda, iam dubitari non debet. Haec autem prae- cepti summa est. Neminem in ecclesia Dei cito et temere promouendum esse ad aliquod munus eccle siasticum, quantumuis exiguum: de eo enim mune- ris genere, idest ecclesiastico, Paulum agere, non de politicis Magistratibus apparet ex phrasi ipsa qua vtitur. Ait enim Ne manꝰ imponas. Non quod  \pend
\section*{AD I. PAVL. AD TIM. }
\marginpar{[ p.338 ]}\pstart tamem et ipsi ciuiles magistratus θε͂ διάκονη et sa- cri etiam non sint, sed hic ecclesiastica tantum persequitur Paulus. Quaeritur autem, quae sit huius praecepti occa- sio, et cur de eo fuerint etiam prudentissimi cu- iusque ecclesiae pastores, qualis Timotheus, qua- lesque nobis sub Timothei persona proponun- tur, monendi. Resp. fuisse multiplicem necessita- tem ac tationem cur hoc praeceptum spiritus san- ctus adderet: sed imprimis fuit triplex, nimirum sci licet propter Impudentiam, Importunitatem tum Populorum tum Ordinandorum, et Nouitatis studium erat necessarium hoc prae ceptum. Nam fere homines nouis Ministris gaudent: Ergo ne qui semel arrisit, statim probetur et eligatur, ca- uendum est. Nam permulti sunt, in quibus egre- giae quidem animi dotes elucent, qualis est vel ze lus in Dei gloriam, vel etiam sumnma doctrina, quique propterea sunt commendabiles: qui ta- men ad munus ecclesiasticum sunt prorsus pro- pter alios defectus inepti, illiusque incapaces. Et certe qui ad huiusmodi muner a alios commen- dant, hoc notent diligenter, quod ait hoc loco Paulus, quodque Horatius Flaccus, quanquam poëta prophanus, recte monuit. Qualem commendes etiam atque etiam aspice, ne mox Incutiant aliena tibi peccata pudorem. Affert autem remedia aduersus omnes supe- riores rationes, quibus impulsi eligentes peccare solent. Primum igitur remedium primaque excue satio iustissima, quae contra afferri debet est haec, Neminem Dei praeceptis praeponendum, et idcir  \pend
\section*{CAPVT . V. }
\marginpar{[ p.339 ]}\pstart co vtcunque precibus instent et vrgeant nos a- lii, non esse alienis tamen peccatis communican- dum. Nam qui indignum eligit, vel eligi consen- tit, reus est omnium errorum et peccatorum, quae postea electus ille committit. Deinde consentit cum male agentibus, quatenus ipse temerariis eorum suffragiis consensum suum addit atque ad- iungit. Verum est enim illud, Agentes et Consen- tientes pari poena puniuntur: Iacob patriarcha Genes.49. vers. 6.negat se malo consilio Simeo- nis et Leui de necandis Sichemitis assensum esse. Ps.5o. vers.18. Qui cum furibus currunt, pro furi- bus habentur apud Deum. Denique ait Paul.1. Thess.5.ab omni malo malique specie esse nobis abstinendum. Neque hoc cuiusquam gratiae con- cedendum a nobis est, vt in illius fauorem pecce- mus, officium nostrum negligamus, et Deum ip- sum deseramus atque offendamus, cui nos nostri suffragii et vocis rationem reddere aliquando oportebit. Quod si iterum instent et premant nos alii, qu plura sint aliorum et contraria nobis suffragia, nos tamen puros seruemus, quid nobis videatur, libe- re declaremus, et nostra consilia ab huiusmodi temerario suffragio indignis hominibus dando se- paremus. Atque hoc est secundum remedium, siue secunda legitima excusatio et praemunitio boni pa- storis aduersus instantes quorundam preces et suf- fragia iam lata itemque aduersus calumnias mul- totum, qui nos putant, dum nostra suffragia qui- busdam iam ab aliis electis et probatis denega- mus, vel tam cito in illis non assentimur, duci ali- quo in eos odio vel inuidia, vel etiam nos esse  \pend
\section*{AD I. PAVL. AD TIM. }
\marginpar{[ p.340 ]}\pstart crudeles et nimium seueros. Sed valeat potius apud nos haec tutissima consolatio, quod puram conscientiam seruare volumus ac studemus. Ne- que tamen, si a maiori suffragiorum parte vinca- mur, solique dissentimus, seditionem in senatu, aut coetu aut ecclesia Dei mouere debemus: aut quicquam perturbate agere, vt vincat nostra v- nius vox: sedliberauerimus animas nostras atque officio functi erimus abunde, si primum nostram sententiam illiusque rationes exposuerimus: dein- de, si res sit tanti momenti, eo ordine vtamur ad huiusmodi electionem impediendam, vel retar- dandam, qui et legitimus est, et in ecclesia Dei con- stitutus et receptus, qualis est nimirum prouoca- tio a senatu ecclesiastico ad maiorem Ministro- rum vicinorum coetum, quem Classem vocant: a Classe ad Synodum ipsam prouincialem: a Synodo prouinciali ad synodum Generalem et Nationa- lem quam vocant. Ait Paulus, Cito. Fit autem Cito ex Pauli mente, quando vel non legitimum: vel non satis exactum et certum examen de eligendo siue promouendo pastore praecessit: sed vel nullum omnino, vel le- ue tantum et perfunctorium, quale fit in Papatu, et in multis etiam mediis ipsis Dei viui eccle- siis, neque de omnibus iis rebus, quae ad tale mi- nisterium obeundum rite desiderantur. Ergo quem admodum sunt diuersae vocationes in ecclesia Dei: et aliae aliis sunt digniores et grauiores: ita serius et maius in his, quam illis examen esse et fieri debet, veluti in Pastore ipso, quam in solo Presbytero vel Diacono plura inquirantur ne- cesse est, quia maius est pastoris quam Diaconio-  \pend
\section*{CAPVT  V. }
\marginpar{[ p.241 ]}\pstart nus.Ac quidem examen, quod in pastore requiri- tur, est fere triplex:fieri enim debet de Doctrina ipsius Vita et moribus, et Aptitudine ad docen- dum gregem et populum Dei. Id quod satis ex Tit. 1.et eadem epistola c.3. apparet. Et quia luculentus est hic locus, continetque maximum totius Politiae ecclesiasticae caput et pulcherrimum (quod est de electione et vocatio ne dignitatum ecclesiasticarum) de eo nobis bre uiter aliquid arbitramur esse dicendum. Habebit autem hic locus 4 capita scilicet r'Quid sit electio 2 Qui eligant 3 Quando 4 Quomodo fieri de- beat electio. In primis autem hoc pro certissimo fundamento constituatur, nempe quod Electio vocationem ordinariam praecedere debet, quemadmodum vocatio ipsa anteit mune- ris ipsius susceptionem et functionem, si modo bona conscientia volumus munus aliquod in ec- clesia Dei gerere ac administrare. Vocationem autem, de qua nunc agimus, κλησίαν aut χειροτονίαν potius quam ὀκλογὴν appellandam esse dicimus, quod ὀκλογὴ ad causas salutis nostrae pertineat, vti docet Paul. Rom.11.vers.5.quae disputatio est re- mota ab hoc argumento. Est enim vocatio haec sepositio designatioque tantum alicuius personae ad munus aliquod ecclesiasticum gerendum. Ac primum neminem sine vocatione legiti- main ecclesia Dei recte sanaque conscientia mu- nus vllum abtinere posse docet tum autoritas sa- crae scripturae: tum etiam ratio. Ac scriptura quidem Hier.23. vers.21. curre- bant, et ego non mittebam eos. Hebus 5.vers.5. Ne mo sibi sumit honorem, sed qui vocatur a Deo-  \pend
\section*{AD I. PAVL. AD TIM. }
\marginpar{[ p.46 ]}\pstart Roma. 1o. vers. 15 Quomodo praedicabunt, ni- si missi fuerint. 1.Timoth.4.ver. 14. Ne negligitodonum quod inte est, quod est tibi datum ad prophetandum per impositionem manuum Presbyterii. Tit.1. vers.5. Idcirco reliqui te, vt oppidatim constituas Presbyteros. Denique infiniti pene sunt loci his similes, quos illi studiose collegerunt, qui de ea re com- munes locos scripserunt: nos autem summa tantum rerum fastigia sequimur, et carpimus, vt simus breuiores. Ratio vero, cur sine legitima vocatione prae- cedente nemo bona conscientia vllo munere ecclesiastico fungi possit, est multiplex. Prima, quod in ecclesia Dei omnia fieri ordine certaque ratione, et decenter debeant 1. Corint. 14. Ergo nemo pro animi sui libito hanc vel illam dignitatem sibi in ecclesia sumere debebit, quia haec est summa ἀταξία et rerum confusio. Secunda. Quod in veteri ecclesia, quae Christi aduentum praecessit, nemo nisi ex Deo vocante et praecepto munus habuit: In ea etiam eccle- sia, quam Christus per euangelii praedicationem instaurauit, Apostoli ipsi a Christo vocati sunt. Ergo praeter et legalis, et Euangelicae eeclesiae for mam, praecepta, et instituta idest praeter ipsum Dei verbum in munere aliquo versantur, qui il- lud citra vocationem vllam legitimam vsurpant, et in ecclesia exercent. Tertia. Neque Moses, et Aaron, neque Aposto- liac ne Christus quidem ipse quanquam Dei fi- lius, nisi prius legitime vocatus, munus sibi assum-  \pend
\section*{CAPVT V. }
\marginpar{[ p.343 ]}\pstart psit in ecclesia Dei. Ergo nemo sibi honorem tri buere in ecclesia Dei, citra legitimam vocationem debet. 4. Dominus Deus est ecclesiae tanquam pe- culiaris suae domus et proprii peculii solus pater- familiâs et legislator, et dispensator. Ergo qui a- liquid in ea citra ipsius voluntatem vsurpat, fur est, et latro: non autem legitimus illius administra- tor et gestor. 5 Qui, praeter legitimam vocationem ecclesia- sticam dignitatem adit, perinde facit, atque qui domum alienam alio’aditu et modo, quam per o- stium ingreditur. Hoc autem furum est, et latro- num, non autem legitime ingredientium. ergo qui hac ratione, id est, sine legitima vocatione ec clesiasticum munus gerunt, sunt fures, non Pasto- res ecclesiae. Quod autem nobis obiicitut de Prophetis, eos sine vocatione saepe fuisse, et munus illud suum in ecclesia Dei exercuisse, falsissimum est, vt suo exem plo docet Amos, cap.7.vers.15. Item Hier.1.ver. 5. Sed vocatio illa prophetarum vt plurimum fuit extraordinaria. Eos enim Dominus ipse vocabat et excitabat, ex qua tribu volebat, quo tempore volebat, sed vocabat tamen. Deinde eorum vocatio- nem Deus vel ipso rerum ab iis praedictarum euentu : vel aliis modis aperte confirmabat, adeo, vt illi ab eo missi esse satis ab omnibus ad eorum pro- phetias attendentibus intelligerentur. Idem est respondendum de Christo cuius vocatio et a Deo ipso in ipso baptismo fuit patefacta, etiam olim a prophetis praedicta, etsi non fuit ordinaria. Aeriani haeretici damnati sunt: et nunc Anaba-  \pend
\section*{AD I. PAVL. AD TIM. }
\marginpar{[ p.3*f. ]}\pstart ptistae, et Enthousiastae quidam nebulones me- rito reiiciuntur, qui vocationes in ecclesia Dei se cure spernunt, vt se ingerant, atque intrudant ad ecclesiastica munera magno animi fastu et am- bitione. Ergo propter tria praecedere legitima vocatio debet, antequam quis munus aliquod in ecclesia Dei gerat vel sumat. I Propter Dei praeceptum. 2 Propter ecclesiae ipsius ordinem et disciplinam. 3 Propter ipsius, qui dignitatem illam accipit, conscientiam, quae alias nunquam tuta et quieta esse potest:non magis quam furis quandiu re a se furto ablata vtitur, vel eam retinet. Magnam e- nim animis nostris consolationem affert legiti- ma vocatio, quoties a peruersis, vel obstinatis hominibus vexamur, et negotia nobis praebet Satan. Quid est? Resp. Vocatio igitur haec, est ad munus aliquod fa- cta ab alio, quam a seipso, adoptio ac designatio. Haec igitur duplex est. Interna et Externa. Interna est a Deo, ac, ea quidem immediate, Externa vero, etsi a Deo est, quia ipse illius cele- brandae modum verbo suo praescripsit: mediate tamen est ab hominibus illam Dei ordinationem sequentibus. Neutra sine altera tuta satis esse po- test vel salutaris: Interna tamen potior est: sed externa nostri :atione tutior. Interna vocatio est interna et latens animi no- stri affectio atque inclinatio a Deo immissa ad a- liquod legitimum munus gerendum, quae illius desiderium in animo nostro parit. Igitur affectus infunditur a Deo, neque ex ambitione, neque ex auaritia, nec ex alio aliquo turpi vel carnali motu  \pend
\section*{CAPVT V. }
\marginpar{[ p.345 ]}\pstart in nobis inesse debet. Bernard serm.58.in Cant. sic eam definit, Est inuitatio et stimulatio quaedam charitatis pie nos sollicitantis aemulari fraternam salutem, aemulari decorem domus Dei, incremen- ta lucrorum eius, incrementa frugum iustitiae eius laudem et gloriam nominis illius Istiusmodi er- go erga Deum religiosis affecti bus quoties is qui animas regere, aut studio praedicationis ex officio intendere habet, hominem suum interio- rem senserit permoueri, toties pro certo spon- sum adesse intelligat: toties se ab illo ad vineas inuitari. Externa vocatio est, hominum qui iudicare possunt de nostra ad aliquod munus capacitate et idoneitate testimonium libero, non redempto non captato ipsorum suffragio, in Dei timore et via a Deo praescripta atque legitima patefactum et declaratum. Vt sit autem haec externa vocatio nostra legi- tima, duo haec concurrere debent:e quibus tamen vnum alterum praecedit, nimirum Electio, quae praecedit: Ordinatio seu approbatio electi, sequitur electionem. Electio est, eorum qui possunt et debent syn- cerum atque sanctum de nobis testimonium, per quod designamur ad certum munus gerendum. Ergo hic videndum 1 A quibus fieri debeat electio. 2 Quando, 3 Quomodo. Ac qui- dem primum illud statuendum est, quod Nemo a seipso eligi vel debet, vel potest. Verum enim illud est, Nemo iudex ferendus in propriae ratio- ne causae:item et illud, Laus in proprio ore sorde- scit. Neque obstat quod supra de interna eligen-  \pend
\section*{AD I. PAVL. AD TIM. }
\marginpar{[ p.340 ]}\pstart dorum vocatione diximus. Etsi enim esse debet, interna vocatio et affectus in eo qui munus ali- quod acceptat: ea tamen interna vocatio etiam ex- ternam, tanquam sensus illius interni testem, ad- hibet, et expectat, nisi quis extraordinarie a Deo ipso impelleretur. Interna enim vocatio, quae ex- ternam spernit, non tantum falsa est, et mera ar- rogantia, sed etiam violat Dei praeceptum: et turbat disciplinam Ecclesiae: neque a Deo est, sed est animi impia phrenesis. Id quod etiam in quadam sua ad Magnum epistola Cyprian. re- spondit. Sed et ipsum electionis et vocationis nomen huic a seipso de seipso factae ordinationi plane repugnat, cum is demum dicatur Vocari, qui ab alio accersitur. Sed neque a quolibet Electio facta legitima est, in muneribus praesertim Ecclesiasticis. Nam ab vno tantum aliquo facta alicuius ad dignita- tem Ecclesiasticam electio, rata et legitima non est, siue ab ipso principe illa fiat, siue ab eo, qui Ecclesiae patronus dicitur, siue ab eo solo qui in Dioe cesi Metropolitanus est, et Ecclesiae πey:σῶς et πρόεδρος. Neque enim Ecclesiae, quae in terris est, regimen et administratio nobis a Christo reli- cta, est Monarchia quaedam: ac multo minus ty- rannis: sed est Aristocratia, vt docemur parabo- la Christi, quae est Matth.24. vers.49. et 25.vers.14. Itaque vetat Petrus etiam ipsos Episcopos Ku- ρεέυαν in Ecclesia: et Christus ipse dissimilem pla- ne regum in suos subditos: et Pastorum in Eccle- sias suas potestatem esse dixit 1.Pet.5.Luc.22. Ita- que qui omnes similes faciunt, omnia perturbant. Quare ne a rege quidem ipso solo, vel ab vno dun-  \pend
\section*{CAPVT  V. }
\marginpar{[ p.347 ]}\pstart taxat Episcopo vel Archiepiscopo, facta electio legitima est, quantunuis magna censeatur inter homines istorum authoritas, vt praeclare annotat Theodorus lector lib. 2. Collecta. Sed in Dei Ec- clesia Christi vnius lex dominari debet: quae nul- la ratione infeingi potest, vt recte administrata Ecclesia dici possit. Praeter enim vnum Christum nemo est in Dei Ecclesia ἀυτοκράτωρ: nemo illius domus Dominus: sed solus Christus est ille vnus, extra cuius praeceptum nemo in ea dignitatem legitimam obtinet. Non vult autem Christus ab vno quodam solo patrono, vel Rege, vel Ponti- fice et Episcopo aliquos eligi et asciri ad munera Ecclesiae suae, vt perspicue postea apparebit. Quod ipsum recte quoque vetitum est fieri ab Adriano Rom. Pontifice, vt est in canone Nullus. et a Nicolao in cano. Porro, et can. seq. Distin.63. Atque inde orta sunt grauissima illa, et plus quam nefaria bella inter Henricos imperatores Rom. et Gregorios Pontif. Rom. cum vtrique ad se solum ius illud electionum trahere conarentur. Nec obstat, quod Titus, Timotheus, Paulus soli vi- dentur Presbyteros et Episcopos Ecclesiis praefe- cisse, vti ex hoc loco videtur colligi posse, et Act. 16. Tit. 1.Respondeo enim illud genus dicendi Con- stituas, imponas, etc. ex mente Pauli esse sic intel- ligendum: Non vt vni cuidam soli electio tota committatur iusque illud tribuatur: sed vt synec- dochicons complexus esse ea dicen di ratione in- telligatur omnes Paulus, et omnia, quae fieri in legitima vocationis Ecclesiasticae ratione et for- ma debent et solent (vt ex his verbis Tit.1.V.5. Vt cibi mandaui) facile concludi potest. Sed cum in  \pend
\section*{AD I. PAVL. AD TIM. }
\marginpar{[ p.043 ]}\pstart iis electionibus praecipuae quaedam partes essent, tum Titi, tum Timothei, tum ipsius Pauli, et Bar- nabae, vt pote qui Ecclesias ipsas quomodo id fieri oporteret, erudiebant, et qui huiusmodi e- lectionibus, tanquam totius actionis moderato- res intererant, atque praesidebant: idcirco tota a- ctio illis quodammodo tribuitur, quae tamen et totius Presbyterii, et totius ipsius Ecclesiae com- munis est, et tunc erat, vti apparet ex capite 4.V. 14. supra et Actor. 14. vers.23. Neque etiam obstat, quod Episcopus προεδ ρός appellatur a Patribus, quasi vnus ille totam Ec- clesiam pro arbitrio possit administrare, et in ea quos velit, praeficere. Nam vox illa πρόεδ poς or- dinem tantum declarat, quod inter collegas suos, eticompresbyteros sedere deberet ipse, qui Epi- scopus dicebatur: non autem regiam, aut summam praetoriamque potestatem in Ecclesiam illi tri- buit. Quod dicimus confirmat Ambrosius lib.  de Dignit. Sacer dot. cap.6. Solio in Ecclesia editio- re Episcopus iesidet, ait, vt cunctos aspiciat. Edi- ctum quoque Christi quod supra ex Luc.22. con- memorauimus, idem probat. Est enim solus Chri- stus Ecclesiae rex, non autem mortalis quispiam. Pertinet autem electio, et designatio ipsa ad totam Ecclesiam, in qua fit: sed diuersa ratione. Quemadmo dum enim totius Ecclesiae pastor est futurus: ita aly omnibus debet ordinari et appro- bari, ne quisquam gregi inuito pastor obtruda- tur. Praeterea etiam illud verissimum est, quod o- mnes tangit, et spectat, ab omnibus fieri debere. Huiusmod enim negotium totius Ecclesiae, non vnius autem illius partis negotium est, cum qui  \pend
\section*{CAPVT V. }
\marginpar{[ p.349 ]}\pstart pastor eligitur, Ecclesiae vniuersae detur. Ergo pessime faciunt, qui plebem quam vocant ab omni suffragio in Ecclesiasticarum dignitatum electionibus ferendo repellunt et semouent, tan- quam non sit et ipsa plebs pars Ecclesiae Dei, ea- que maxima. 120 Pessime quoque hallucinantur qui non distin- gunt quae sint Presbyterii, quae autem plebis, in electionibus istis partes: sed quod est illius, isti tribuunt, inducentes in Ecclesiam Dei magnam o- mnino confusionem. Praeclare enim Ambrosius in lib.  de Sacerdot. dignitate cap.3. Aliud est, ait, quod ab Episcopo requirit Dominus: aliud quod a plebe. Idem Chrysost. in libro 3. de Sacerdotio infra medium. Idem quoque Bernard.sermo 49. Cant. cuius aetate iam haec perturbata et popula- ris in electione ordinandorum ratio, a quibusdam videtur fuisse inuecta. Ordinauit me in charita- tem, ait. Factum autem est hoc cum in Ecclesia quosdam quidem dedit Apostolos, quosdam Pro- phetas, alios Euangelistas, etc. Oportet autem vt hos vnâ omnes charitas liget et contemperet in vnitatem corporis Christi. Quod minime omni- no facere potuerit, si ipsa non fuerit ordinata. Nam si suo quisque feratur impetu secundum spiritum, quem accepit: et ad quaeque volet, in- differenter, prout afficitur, et non rationis iudi- cio conuolarit, dum sibi assignato officio nemo contentus erit, sed omnes omnia indiscreta ad- ministratione attentabunt, uon plane vnitas erit, sed magis confusio. Haec Bernardus. Sic igitur tota haec res est distinguenda, vt in vocatione et electione duos quosdam actus di-  \pend
\section*{AD I. PAVL. AD TIM. }\pstart stinctos obseruemus et separemus, nempe Ele- ctionem ipsam siue praesentationem personae: et Acceptationem, quam alii Ordinationem, alii Confir mationem appellant. Haec igitur Electio siue praesentatio, quam dico est ipsius personae ad munus aliquod Ecclesiasti- cum vocandae prima agnitio, agnitae tenta- tio, degustatio, probatio, et examinatio, quae tum deipsius vita, tum de doctrina fieri debet, et ita examinatae atque consentientis denuntiatio et praesentatio, quae fit toti populo et Ecclesiae si- mul collectae. Hae vero sunt Presbyterii, et qui- dem totius, partes, non plebis aut populi. Approbatio autem eligendi est, de persona iam a Presbyterio examinata diligenter et proposita, Iiberum, sed tamen iusta aliqua ratione nitens totius populi et Ecclesiae suffragium, per quod post commodum aliquot dierum interuallum concessum, illam personam populus vniuersus, vel acceptat, vel reiicit atque repudiat. Haec igi- tur approbatio ad plebem et totum populum Ecclesiae sane pertinet. Hoc verum esse atque ita factitandum, vt sit legitima ordinariaque eligendorum et electorum vocatio, apparet tum exemplis veteris, et primi- tiue Ecclesiae: tum etiam exemplis illius, quae subsecuta est prima illa tempora. Denique fir- missimis rationibus idem comprobatur. Ac primum plebem non esse ab ordinationibus vocandorum ac praeficiendorummuneribus Ecclesia- sticis excludendam, demonstrant exempla veteris Ecclesiae, in quaproculdubio χεροτονία toti' Ec- clesiae interueniebat, vt ex Act.6.et 14 ostendi fa-  \pend
\section*{CAPVT  V. }
\marginpar{[ p.35  ]}\pstart cile potest. Itaque suo iure perfide priuant Eccle- siam, qui pastorem populo inscio et non assentien- ti, obtrudant. Quod fit in Papatu, faciunt e- nim maximam Ecclesiae iniuriam, cum eam iudi- cio et suo suffragio spolient. Qui propterea, vere sunt Sacrilegi nominandi. Nec enim est pastor legitimus qui inscio vel inuito, vel non assentienti gregi prae sidet et datur, vt ait Gregorius Magnus epistola 87. et 9o. Itaque olim etiam iam corru- ptis Ecclesiae moribus et inclinata disciplina, ta- men irritae factae sunt omnes Ecclesiasticae voca- tiones, quae ἀνδι λαο͂ συνέσεως, id est, vt vertit Cy- prianus, sine conscientla et assensu populi fierent. Quod confirmat Chrysostomus lib. 3. de Sacerd. Cyprianus lib. 1.epistola 4. lib. 2.epist.3 et 5. Theo- doret. in Histor. cap.2o. et 22.libus 4 Gregor. lib. 1. epist 5.libus 2.epist.69. Atque de hac ipsa electio- nis forma seruanda extat constitutio Caroli et Ludouici Imper. in canon. Sacrorum canonum distinct.63.Leo quoque primus Pontifex Rom. negat vllum verum esse Episcopum, nisi qui a cle- ro electus et a plebe expetitus fuerit. in canon. In nomine distinct.23. Quam ipsam etiam suo seculo adhuc fuisse obseruatam testatur variis in locis Bernardus epistola 2o2. et 150.164.13. et 27.id est, vt populus electioni assentiretur. Denique hoc ipsum verum esse iure canonico etiam confirma- tur, quemadmodum apparet ex canone, Plebis Diotrensis et canon. seq. et cano. Nolle distinct. 63. can. Factus, vbi dictum Cypriani aureum re- citatur.7.quaest.1. Idem testatur Ambrosius lib. 1o epistola 82. Vnde quam impii sint Trident. con- cilii canones huic contrarii manifeste apparet.  \pend
\section*{AD I. PAVL. AD TIM. }
\marginpar{[ p.552 ]}\pstart Neque tamen propter tumultum popularem, aut votum populi temerarium in vnum aliquem, qui erit indignus, debet ille a Presbyterio desi- gnari et eligi. Ostendit enim hoc loco Paulus nullo modo essecum alienis peccatis communi- candum. Quod etiam a Leone Episcopo Roma. responsum est, et refertur in cano. Miramur. di- stinctione 61. Sed aliud est denegare plebi iustum suumque suffragium: aliud autem illius temerita- ti obsequi, et obtemperare. Ac quidem haec sunt iura totius populi et Ecclesiae. Presbyterorum autem est, primum personam ipsam, quae idonea ad munus vacuum videri po- test, inquirere ex toto populo, eamque designa- re oculis, et de ea inter se agere atque conferre. Praeire enim debent toti Ecclesiae Presbyteri. I- taque non tantum eum agnoscere de vultu vel facie debent, quem designant: sed etiam de qua- litate. Vnde fit vt primum examen ipsius eligen- di ad eosdem Presbyteros pertineat, ne indignum proponant populo. Nam parum est personam i- psam nosse, nisi ingenii vires noueris. It aque do- ctrinam ordinandi debent Presbyteri tentare, tum quaestionibus illi propositis, tum periculum fa- cientes de illius dexteritate in tractanda propo- nendaque et interpretanda Sacra scriptura. Cur autem hoc primum suffragium ad Pres- byteros pertineat, non autem ad vniuersam Ec. clesiam et populum, quemadmodum tamen qui- dam censent, et contentiose disputant, ratio est: Prima, Quod Presbyteri et Pastores dicuntur ποιμένες et ηγέμινοι Ecclesiae. Hebr.13.vers.7.id est, Praepositi (vt loquitur Tertull. in Apolog cap.  \pend
\section*{CAPVT . V. }
\marginpar{[ p.353 ]}\pstart 9.) Vnde sua voce suoque consilio praeire popu- lo debent, et viam futurae deliberationis ostende- re. Ergo non plebs praeire debet sua voce Presby- ris: sed Presbyteri populo et plebi. Secunda, Quod illi ipsi sunt, vt est in cano. Ec- clesia habet 16. quaest.1. Ecclesiae senatus (vt ait Hieronymus) ad quem prima rerum Ecclesiasti- carum deliberatio et cura pertinet, atque etiam referenda est:itemque, vt idem Hieronymus ait, morum censura. Sic Paulus Philip.1.vers.1.ad eos scribit, vt per eos ipsius Epistola innotescat reli- quae Ecclesiae. Sic ad eos per manum Pauli et Bar- narbae eleemosynae Ecclesiarum deferuntur. Act. 11.vers3o. Sic illi ipsi Presbyteri Hierosolymis conueniunt, et primum audiunt Paulum, tum eum- dem Ecclesiae vniuersae sistunt Actor. 20. vers.28. 21.vers.18.11.vers.13. Tertia, Exemplum ipsius primitiuae Ecclesiae. Apostoli enim ipsi primi de Diaconis eligendis deliberant.i. de personis, quas ad eam dignitatem promouent. Ideo verba, quae sunt Actor o. ita explico, vt semper penes Apostolos ipsos prima totius actionis cogitatio, deliberatio, et admini- stratio fuerit, Secuta autem sit Ecclesia eorum consilium. Quanquam etiam in eo loco aliquid speciale fuisse videri potest, nimirum vt toti mul- titudini prima illa suffragiorum ferendorum po- testas deferretur, propter murmur Graecorum, qui conquerebantur Hebraeos sibi in Ecclesiae muneribus praeferri. Quarta, Ex ipsa loquendi ratione, qua vtitur scriptura Actor. 15. vers.22 16.vers.4.1. Corinth. 16. vers.3. ὁς ἀν δοκιμάσητε Ex qua phrasi luce cla-  \pend
\section*{AD I. PAVL. AD TIM. }
\marginpar{[ p.113 ]}\pstart rius intelligitur a populo et vniuersa Ecclesia probari tantum, vel improbari, quae Presbyteri et ipsi praepositi Ecclesiae e re Ecclesiae esse iudi- carant atque proposuerant. Quinta, Exempla posterioris Ecclesiae, quae tum canonibus suis sanxit, tum etiam vsu ipso et praxi demonstrauit a plebe tantum assensum re- quiri, vt quae a Presbyterio gesta sunt, iusta ali- qua ratione plebs vel rata haberet, vel irrita fa- ceret. Laodice. Synod.can.13. qui est relatus cap. Non est permittendum distinct. 63. Sic Cypria- nus electiones Ecclesiasticas celebrauit lib. 1.epi- stola 3. et 4.Sic suo quoque seculo factum esse scri- psit Socrates lib. 2, Histor. cap.6. Sed si erit summus Magistratus fidelis, in cuius territorio aliquis ad dignitatem Ecclesiasticam est asciscendus, debet certe specialis istius Ma- gistratus consensus praeter eum, qui a toto popu- lo praestatur, expectari et accedere. Id quod ex Socrate lib. 5.cap.2. Theod.libus 5.cap.9. can. Rea- tina dist.63, et Bernar. epist.282. Platina in Bonif. 3 facile colligi et confirmari potest. Nam ad eum ipsum Magistratum pertinet, si Presbyterium sit nimium negligens in substituendo aliquo in locum vacantis et mortui vel depositi pastoris, illud co- gere et edictis suis increpare, vt apparet ex Her- mio lib. 1.cap.13. Euagr. lib. 4.cap.37. Neque aliud (si recte intelligantur) in can. Porro, et cano. seq. distinct.63. constitutum est. Accedere quoque debet ipsius electi consen- sus, antequam populo denuntietur et proponatur et hoc recte in canone Sicut alterius 7. quaest. I. bonstitutum est, Ne inuitus quisqua m et omnino  \pend
\section*{CAPVT . Va }
\marginpar{[ p.355 ]}\pstart repugnans praeponatur, praeficiaturque Ecclesiae. Quod omnino periculosum est neque fieri vllo modo debet. Trinundinum autem siue spatium trium heb- domadarum populo siue plebi concedi solitum fuisse, intra quod de electo a Presbyrerio viro et illi oblato atque etiam audito populus iudicaret, apparet ex Socrate Scholast.libus  5.cap.5. Ipsius autem populi assensum χεροτονία, id est, manuum eleuatione praestari et fieri solitum fuisse apud Graecos ex more gentis illius certissimum est, idque constat tum ex profanis:tum etiam ex Ec- clesiasticis scriptoribus infiniris. Itaque et in Actis Apostolorum: et saepe apud patres electiones i- psae χεροτονίας voce nominantur. Videtur vero e- tiam haec consensus praebendi et testificandi ra- tio vsitata fuisse inter Hebraeos ex Nehem. 8.v.7 Apud nos Gallos in omnibus fere conuentibus publicis consensus noster solet modestissime so- lo silentio significari, post propositam negotii, de quo agitur, consultationem: modestiae, in- quam, causa. Id quod in electionibus Ecclesia- sticis hodie quoque in reformatis Galliae Eccle- siis obseruamus. Negant autem quidam obstinatissime, liberum populo suffragium eligendi relinqui, nisi duo aut tres populo presententur, siue proponantur a Pre- sbyterio, quorum sit libera populi optio. Nam si vnus, isque duntaxat offeratur et praesentetur, ec- quae electionis libertas populo restat, aiunt, cum sit electio de duobus vt minimum vnius optio? Respondeo vero vbi copia atque facultas plurium idoneorum populo offerendorum datur, et sup-  \pend
\section*{AD I. PAVL. AD TIM. }\pstart petit, non vnum tantum, me authore, populo proponendum esse: sed duos aut plures. Vbi au- tem vnus tantum idoneus esse potest, etiam in eo ipso liberam esse populi electionem deliberatio- nem, et assensionem. dico adeo, vt si quem etiam magis aptum agnouerit populus, aut vnus aliquis e populo, de eo Pres byterium monere et possit et debeat. Atque haec omnia quae diximus in ordina- ria vocatione locum habent. Nam extraordina- ria vocatio alia est, quae nulla habet praecepta. Secundum caput est. Quando electio sit cele- branda. Responsio vero facilis est, nimirum vbi prior pastor vel defunctus est: vel legitime depo- situs. Nam vel viuente adhuc, vel manente et of- cio fungente priori, secundus eligi non debebit nisi fortasse in subsidium defessi, veluti si nimia sit iam superstitis infirmitas senectus morbus, cae- citas, aut aliud quippiam simile, alius aduocari potest, qui illi primo tantum accedit, non autem succedit, vt docet Socra.libus 7. cap.4o. Herun lib.  8. cap.26. Bernard.epistola 126. et tota 7.quaest.I. in Decretis responsum est. Quod si prior pastor defunctus erat, illud fere fuit obseruatum, vt non nisi demum post tertium sepulturae diem in locum defuncti de alio eligendo ageret Ecclesia, vt est in canone Nullus Pontifice distinctione 79. Viri etiam graues, et Episcopi suarum Ecclesiarum amantissimi saepe sibi Successores ipsi designa- runt, etiam viui quod futurum periculum prospi- cerent in futura post suum obitum electione. Hos autem ita ab ipsis designatos, postea populus at- que vniuersa Ecclesia comprobabat ant equam le- gitimi fierent pastores vti Augustinus et Athana-  \pend
\section*{CAPVT  Ve }
\marginpar{[ p.357 ]}\pstart sius, et alii sibi successores elegerunt. August. epi. 110. Theo. Hist. lib. 4. cap.2o. Eusebus lib. 7. cap.32. Atque haec desecundo capite et de tempore electionis Tertium caput est, Quomodo sit electio cele- lebranda. Respond. Sic esse celebrandam, vt et Persona idonea reperiatur, et Ratio animusque eligentium sit syncerus. Persona vero cognoscetur idonea si fiat serium examen ac diligens ipsius tum in Doctrina ipsa, et illius explicandae methodo, tum Vita et motibus, sintne in eo tales, quales antea praescripsit Paulus cap.3. Imperitia vero psalmodiae et cantus, doctum et probum hominem a munere Episcopi exclu- dere minime vel debet, vel potest, quicquid a Gregorio Magno contra responsum sit, et pro- batum in cano. Florentinum Archidiaconum di- stinct. 85. Sed iam Episcopi officium in cantile- nas conuersum fuerat, relicto abdicatoque ab il- lis munere et onere concionandi. Fuit etiam o- lim diligenter et sancitum, et obseruatum, vt con- stitutionum Ecclesiasticarum nemo ignarus in Pastorem Ecclesiae ordinaretur, vt tradit Sozom. libus 4. cap. 24. Quae sane conditio non est in futuro Pastore negligenda, si quis talis reperiri potest. Sin minus, eligi tamen poterit, qui non erit omni- no ordinis Ecclesiastici ignarus, modo reliquas animi dotes habeat, quae sunt ad pastorale munus necessariae. Neque enim cantus Ecclesiastici neque istarum regularum Synodicarum ignoratio adi- tum piis et doctis viris ad pastoris officium ex- cludere debent, quia illa duo vel vsu, vel exerci- atione facile postea edisci possunt.  \pend
\section*{AD I. PAVL. AD TIM. }
\marginpar{[ p.338 ]}\pstart Animus autem eligentium erit syncerus, si neque Auaritia (Bracarensi Synodo 3. cap.6.) Fa- uore, Ambitione, vi et armis et carnali aliquo affectu ad electionem progrediantur, vel impellan- tur qui ferunt suffragia: sed solam Dei gloriam, et Ecclesiae vocantis aedificationem spectent, sibique pro scopo vere animo proponant. Ieiunium etiam publice indici solitum fuisse, cum fiebat aliqua Ecclesiastica electio, testatur Bernard.epistola 202. Et certe quatuor illa tem- pora, quae dicuntur in Papatu (quibus publicum ieiunium toti Ecclesiae indictum est) erant eae an- ni tempestate, squibus passim et publice in to- ta Ecclesia electiones Ecclesiasticorum munerum celebrari cum precibus et ieiunio solebant. Ap- paret ex can. Ordinationes dist. 75. atque etiam ex eo quod clericis adhuc hodie in Papatu Episcopi ordines quos vocant conferunt illo tempore, quic- quid aliter Leo, et Calixtus sentiant et scribant, in canone Ieiunium et cano. Huius obseruantiae distinct.76. Post Presbyterii electionem et populi ap- probationem, olim quidem (vt ex Cypria. lib. 2. epistola II. apparet) qui electus fuerat publico scripto proposito publice significabatur, vt si quod esset impedimentum, quod obiici posset, intelligeretur. Hoc male praetermissum est hoc tempore. Post talem igitur electionem ordinatio siue confirmatio, siue manuum impositio fieri solebat electo. His enim tot nominibus vna res et eademappellata est a veteribus. Est autem ordinatio haec ipsius electae atque approbatae personae in munus illud, ad quod ido-  \pend
\section*{CAPVT  V. }
\marginpar{[ p.359 ]}\pstart ne a visa est, receptio et missio, quae fit praesente vniuerso populo, cum inuocatione Dei nominis. Itaque tunc in illius muneris professionem mit- titur ab vniuersa Ecclesia electus. Haec etiam cum ieiunio et publicis precibus fieri certe debet, ap- pellaturque fere in scriptura, Impositio manuum veluti hoc loco Actor.6.verso. supra 3.vers.14.2. Timoth. 1.vers. 6. a caeremonia quae in Ecclesia obseruabatur. Ergo electus Presbyter, vel Pastor vel Episcopus in possessionem mittendus est, ordi- nandus vel confirmandus, vt loquuntur, quod vt fieret more vsitato tempore Apostolorum ma- nus imponebantur Electo atque approbato can. Presbyter distinct.23. et hoc loco. Fuit enim haec caeremonia ex veteribus et legalibus ritibus ad tempus retenta, vel potius ad imitationem im- positionis manuum, quae fiebat super hostiam sa- crificandam, recepta in Christiana Ecclesia, vt ap paret ex Leuitic.1. et 8.Iacob bene dicturus filios Iosephi, manus illis imponit. vt est Genes.48.v. 14. Christus etiam ipse paruulos benedicturus manum illis imponit Matth.19. vers.15. Atque e- tiam illa in Baptismo locum habuit, vti apparet Hebr.6.vers.2. In aegrotis etiam et dono miracu- lorum Mar.16.vers.18. Cur autem ista manuum impositio fieret du- plex ratio est. I Vt intelligeret is, cui manus imponeban- tur, se totum Deo asseri et destinari. Denique in se a Deo manus iniici, vt iam abrenuntians rebus omnibus suis et mundi negotiis totus muneri suo ad Dei gloriam incumberet atque inseruiret. Vn- de preces adiungebantur, vt eum Dominus ipse  \pend
\section*{360 AD I. PAVI. AD TIM. }\pstart praestaret, et efficeret talem. 2 Ratio est, vt Spiritus sancti dona in ele- ctum effunderentur, eumque quatenus, quaque mensura Dominus ipse videbat conducere, im- pleret. Quod ipsum etiam tota Ecclesia pro electo summis votis a Deo petebat. Vnde ab August. manuum impositio nihil aliud esse definitur, quam oratio super hominem. de Baptismo lib. 4. Haec tamen caeremonia, cum sit tantum ritus quidam, non vsque adeo certe necessaria est, mo- do preces adhibeantur in confirmando electo, et Ecclesia vniuersa vota sua eum eo coniungat, vt spiritu Dei corroboretur. Id verum esse appa- ret, quod Paulus ipse non semper hac voce et cae- remonia vtitur, cum de electione Episcoporum agit. vti Tit.1. vers.5. Deinde quod etiam ipsis Papistis et canoni- bus testibus, non requiritur haec impositio ma- nuum in omnibus ordinibus et dignitatibus Ec- clesiasticis veluti in acolytho, et Subdiaconis or- dinandis, vt expresse refertur can. Subdiaconus et cano.seq. distinct. 24. quicquid ibi glossa glosset, et distinguat. Tertio, quod etiam ipsis Papistis testibus, vis tota huius impositionis manuum a precibus pen- deat, non ab ipso ritu et iniectione manuum in or- dinandos, adeo vt sit tantum illa caeremonia si- gnum quoddam externum commendationis cuius- dam, quae a tota Ecclesia fit Deo, illius personae, quae iam electa est, et quae in possessionem sui muneris est mittenda Vnde praeter missa preca- tione nihil omnino valet et efficit ipsa manuum impositio, etiam ab Episcopo facta vt est in cano  \pend
\section*{CAPVT . V. }
\marginpar{[ p.361 ]}\pstart Quorundam. Dist. 24. Ergo cum preces retinen- tur, ipsis canonibus Papisticis autoribus ipsa ma- nuum impositionis vis, finis, et substantia conser- uatur. Quod est satis. Neque est manuum impo- sitio essentialis pars et ritus legitimae vocationis Denique dum hanc caeremoniam praecise vr- gent homines, videmus eos incidere in vanas quae- stiones et ineptas, veluti De triplici manuum im positione Vna Ordinatoria, altera Confirmatoria tertia Curatoria vt est in canone. Manus. I. quaest. I. Item vtrum, repeti in eodem possit manuum impositio. Item vtrum sic manus imponi debeant, vt ca put ipsum electi attingant, de quo anxie quaeri- tur in cano. Episcopus dist 23. Ergo fatui sunt, qui haerent in ea nuda caeremonia, et proram et pup- pim legitimae vocationis ecclesiasticae constituunt quod faciunt Papistae, cum illa sit rerum indifferem tium numero:neque sit pars vlla essentialis legi- timae vocationis: sed tantum illius testificandae adminiculum externum et signum quoddam confirmatae electionis. Ergo et retineri et omitti potest pro more re- gionis, in qua electus ordinatur. Additum est denique, vt esset legitima ordina tio episcopi, nimirum vt tres episcopi vicini or- dinationi siue confirmationi interessent siue ma nus vna cum vniuersa ecclesia, de cuius pastore e- ligendo agitur, imponerent. Id quod ex Apostolo- rum canonibus sumptum putant, et studiose re- tinent Papistae, quod tamen et onerosum est pau peribus ecclesiis, et ad ambitionem et pompam  \pend
\section*{302 AD I. PAVL. AD TIM. }\pstart fit ab illis videturque inuentum. Quanquam ta- men operaepretium est ad conseruandam fidei concordiam, et vnionem tum doctrinae tum chari- tatis electum episcopum quaerere et habere ma- nus associationem a vicinis episcopis et pastori- bus: quam se ab aliis Apostolis accepisse testatur Paulus Galat.2.vers. 9. et concilio Mileuit cano. 15.ordinati in vna ecclesia aliis episcopis vicinis statim significabantur publico scripto misso ab ordinatoribus: sed praecise tres alios episcopos accersere etaduocare est plane ecclesiis graue et onerosum, tamen Graecanicae ecclesiae studiose, vti fastum omnem externum, et alia pompernalia retinuerunt, quemadmodum apparet ex Historia Euagrii lib. 2.cap.8. Socrat.libus 5.cap.7.libus 4.cap. 3.cano. episcopus dist.23. Item constitutum erat et obseruatum in ista ordinationevt episcopus non nisi in vrbe metro- politana confirmaretur, vt ex concilio Toletano apparet et est in cano. Qui in aliquo Dist.62. Denique nisi electus curaret sese intra quin- que menses consecrandum ab Archiepiscopo suo amitteret ius sibi quaesitum cano. Quoniam Dist. 100. Sed haec omnia cuiusque ecclesiae iudicio funt relinquenda, quando quis sit ordinandus, et cligendus. Nam velle praecise vt ab Archiepisco- po quis confirmetur et ordinetur, vtque in vrbe metropolitana, id plane redolet veterem tyran- nidem Antichristi, quae prorsus est in ecclesia Christiana abolenda: et quilibet pastor melius in- ter gregem sibi destinandunconfirmatur. Demum additae sunt quaedam vestes a communibus diuer sae, et ornamenta, vt augustior haberetur inaugu-  \pend
\section*{CAPVT  V. }
\marginpar{[ p.363 ]}\pstart ratio, Item oleum chirothecae et similes quaedam nugae plane superstitiosae et pueriles. Quae sunt omnia reiicienda. Nam dum formosam ecclesiam illiusque dignitates speciosas repraesentare nobis conantur, meretricios illam cultus inducunt: et ex ca- sta et Christi sponsa verum scortum et foetidum effecerunt. Postremo circa annum 7ooa Christo passo in Italicis et occidentalibus ecclesiis reli- qui episcopi ceperunt confirmationem suae digni tatis petere a Roma. episcopo. Vide quae Isych. scribit. lib. 6.in cap.21. et 22.Leuitic. Ex his autem omnibus apparet quam nulla sit, vel non legiti- ma eorum verbi Dei Ministrorum, vel ecclesiae pastorum Vocatio, qui solius regis, vel reginae: vel patroni, vel episcopi, vel Archiepiscopi auto- ritate, diplomate bullis, iussu, et iudicio fit vel e- ligitur. Id quod dolendum est adhuc fieri in iis ecclesiis, quae tamen purum Dei verbum habent et sequuntur, veluti in media Anglia. Nam An- glos homines alioqui sapientissimos, acutissimos, et pientissimos in istis tamen Papisticae idolola- triae et tyrannidis reliquiis agnoscendis et tollen dis, scientes. prudentesque caecutire, mirum est. Ita- que praeclare sentiunt, qui omnem illam chartula riam et episcopaticam curionum, et pastorum ec clesiae creandorum rationem item ex solo episcopi con sensu et diplomate ministrorum verbi caelestis vo çationem, approbationem, et inaugurationem damnant, tollendamque ex reformata ad Dei verbum ecclesia censent: quod ordo Dei verbo praescri- ptus in ordinatione huiusmodi personarum sit praetermissus, ac violatus sicut i perspicue appa- reat. Denique et Senatui ecclesiastico et Populo  \pend
\section*{364 AD l PAVL. AD TIM. }\pstart Christiano ius omne suum atque suffragium mi- sere sit hac ratione et in hoc genere vocationum ecclesiasticarum ademptum, et in vnum quendam episcopum magna tyrannide atque abusu transla- tum. Dominus Deus talibus corruptionibus, quae adhuc in ecclesiis ipsius supersunt et deffendun- tur, mederi magna sua misericordia dignetur et velit, quae tandem certe magnam ecclesiae Dei ruinam secum trahent, et ipsum sacrosanctum ver- bi Dei ministerium reddent efficientque vel mer cenarium: vel omnino contemptibile et abiectum. Quod Dominus auertat. Vna modo quaestio superest in hoc argumen- to, Nimirum, quid de eorum administratione sit sentiendum, qui non sunt legitime vocati: at- que vtrum iis sana conscientia adhaerere debea- mus. In quo certe distinguendum est. Aut enim est Omnino illegitima vocatio eius, qui ecclesiam aliquam administrat: vel Non omnino illegitima, sed tantum ex parte. Illegitima omnino est ea, in qua omnia Dei verbo prescripta, sunt praetermissa vel violata, vt si quis seipsum intruserit priuata autoritate. Non omnino sed tantum ex parte illegitima est ea, in qua tantum quaedam ex illis non obseruata fue- runt, sed praetermissa. Ergo cuius omnino illegiti ma vocatio est, is certe pro priuato habendus est, ac huiusmodi vocatio pro nulla. Itaque ne- que sacramenta conferre potest, neque reli- qua negotia ecclesiae gerere: et quae gessit pro nullis habentur, nisi fortasse esset extraordinaria illius vocatio, id est, quae signis testimoniisque certis a Deo confirmata esset. Sola autem propa-  \pend
\section*{CAPVT . V. }
\marginpar{[ p.365 ]}\pstart gati euangelii, et fructus, qui inde multus emer- git, con sideratio non confirmat huiusmodiomnino illegitimam vocationem, vti neque foetus enatus il- legitimam copulam, quae intercessit inter adulteros. Vitium autem istud tolli potest, si postea ordi ne et legitime vocetur is, qui primum illegitime in ministerio versabatur. Qui autem ex parte tantum est illegitime ele- ctus et vocatus, veluti: si per Simoniam quis mu- nus adeptus sit, et non legitimo populi vel sena- tus ecclesiastici consensu, consensu tamen gradum obtinet, non est habenda pro omnino nulla huius modi vocatio, sed vitium illud est corrigendum Itaque isti sacramenta conferre possunt, quia vtcum que non omnino veram et satis legitimam vocatio nem habeant: habent tamen aliquam. Sic Scribae et Pharisaei in Cathedra Mosis sedebant. Sic Caia- phas in pontificatu summo erat, quem pretio re- demerat. Itaque exemplo et Apostolorum, et Chri sti ipsius, qui eos monuit et reprehendit non autem secessionem a corpore ecclesiae fecit, ab iis nos se gregare in alium coetum non debemus, nec a to- to ecclesiae corpere. ( Id enim esset Donatistarum haeresim reuocare) sed pacem ecclesiae, quantum in nobis est, colere nos oportet, etsi istud vitium quod in eorum personis et electione inhaeret, damnare oportet: et quantum possumus tum e- mendare, tum etiam emendatum velle. Hac ratione fit, vt etiam a sacrificulis Papisticis collatus Baptismus non sit in ecclesia Dei repeten dus: a priuato autem collatus idem baptismus sit tamen repetendus, et pro nullo habendus. Etsi e- nim in ecclesia Dei illegitimam vocationem Pa-  \pend
\section*{AD I. PAVL. AD TIM. }
\marginpar{[ p.326 ]}\pstart \phantomsection
\addcontentsline{toc}{subsection}{\textit{23 Ne amplius esto abstemius: sed vino}}
\subsection*{\textit{23 Ne amplius esto abstemius: sed vino}}pistici sacrificuli habent: tamen ex consensu po- puli habent aliquam. Est autem aliud, vt ait Augu- sti lib. 2. contra epistolam Parmeni. cap. 13 aliquid prorsus non habere: aliud autem aliquid perni- ciose habere: aliud aliquid salubriter habere. Quod tertium solum eligendum nobis est. Sed tamen vti magistratus vitio creatus, Magistratus tamen est propter populi consensum, quemadmo- dum.lib 5. de lingua latina docet Varro: ita isti sa- crifici etsi vitio creati sunt episcopi et presbyteri, vt, παρεπίσκοποι potius, qua m ἐπσκοποι sint merito appellandi, tamen quadamtenus presbyteri sunt et Episcopi, praesertim apud eum populum, qui in eo consensit. Caeterum obiter monendi sunt lectores, quos- dam referre totum hunc Pauli locum, vbi de non imponendis cito manibus agitur, ad reconciliatio- nes publicas lapsorum, quae fieri solitae sunt in ec- clesia per impositionem manum quemadm. ex Cy- priano lib. 1.epist.14. facile colligi potest, et ex ve terum scriptis et canonibus Synodorum. Sed ta- men aptius videtur, si referamus hunc locum ad electiones ecclesiasticarum functionum: impri- mis autem Pastorum. Illae enim reconciliationes lapsorum veluti idolis sacrificantitum, et libelario- rum, qui dicebantur, quae per manuum impositio- nem fiebant, videntur demum post obitum et tem- pora Apostolorum notae et vsitatae fuisse ea for- ma in ecclesia, quam refert Cyprianus: vti et multae aliae ceremoniae post Apostolos in ecclesiam sunt inductae, et obseruatae. 23 Ne amplius esto abstemius: sed vino  \pend
\section*{CAPVT  V. }
\marginpar{[ p.123 ]}\pstart \phantomsection
\addcontentsline{toc}{subsection}{\textit{pauculo vtere, propter stomachum tuum et crebras tuas infirmitates.}}
\subsection*{\textit{pauculo vtere, propter stomachum tuum et crebras tuas infirmitates.}}pauculo vtere, propter stomachum tuum et crebras tuas infirmitates. Etsi hunc versiculum quidam subiiciunt sequenti ta men nossequemur ordinem quem codices commu nes habent. Est igitur hic locus quoque continuatio siue Α’κολό θησις superioris argumentum, in quo de Presbyterorum officio agitur continet autem hic versiculus praeceptum, quod ad omnes in vni- uersum Presbyteros et praepositos ecclesiae perti net, quatenus ii exemplar gregis non tantum in doctrina: sed maxime in vita esse debent, iuxta dictum Christi. Math.5. Vos estis sal terrae. et Pe- tri in 1.epistola cap.5. Quamquam enim hic locus quiddam peculiare, et quod ad Timotheum sigillatum spectat, trade re videtur: vis tamen et ratio praecepti est gene- ralis, et ad omncs, qui aliquo fungantur munere in ecclesia Dei, extenditur. Hunc autem ordinem in huius versiculi explicatione sequemur, vt primum quae ad specialem huius loci explicationem fa- ciunt, dicamus: deinde praeceptum Pauli ad om- nes generaliter producamus ex ipsius mente et scopo. Iubet igitur Paulus, vt Timotheus vtatur pauculo vnio, idque propter ipsius valetudinem et stomachi debilitatem, vt suo muneri melius supereesse possit. In quo quibusdam videtur nimis abrupta et perturbata orationis huius et styli Pauli ratio, quod a praecepto generali, de electione presbyte rorum, statim transeat, aut potius transuolet ad praecepta medicinae: et ea, quae peculiariter ad  \pend
\section*{AD I. PAVL. AD TIM. }
\marginpar{[ p.368 ]}\pstart valetudinem et stomachum Timothei pertinent. Quae igitur seriei et continuationis huius loci afferri ratio potest? Respondent quidam apte hunc locum cum superiori connecti. Cum e- nim superiori proxime versiculo voluerit Pau- lus Timotheum esse ἀγνὸν, id est, castum, vt interpretantur illi:merito quoque de vini potio- ne praeceptum subiici, quae castitati repugnare vi- deri poterat. Itaque ne castitatis retinendae prae- textu nimia et immodica abstinentia ab eo suscipe retur, vtilem hanc admonitionem Paulum sub- iunxisse volunt. Haec Chrysosto: haec Anselmus: haec pene coeterorum omnium sententia est. Nos tamen aliud existimamus. Nam quemadmodum voxαγνὸ ad castitatem corporis siue coelibatum quemadmodum tamen illi fingunt, minime per tinet: sic neque hoc praeceptum additum est pro- pter castitatem. Quid igitur? Cum oporteat om- nes fideles ecclesiae pastores et ministros verbi Dei vitae suae luce et sanctitate praeire reliquo gregi, docet hoc loco Paulus exemplo Timothei quatenus eos victus sobrietatem et rationem ha bere deceat, vt in cibo ipso caeteris quoque vitae exemplum pastores praebeant. Timotheum autem vult Paulus vti vino, sed mouieb: ex quo apparet eum fuisse abstemium, nontam quidem natura:quam vitae magna conti nentia. Non vetat enim nude, ne Timoth. aquam bibat ὀλίγῳ. Haec vox non tam ad qualitatem vi- ni, quam ad quantitatem referenda est, vt ὀλιγον οινον intelligamus, non ὀλεγορόρν. ι. quod aquae minimum ferat: sed paucum, et quo quis inebriari non pos- sit. Diserte ver o vocem hanc, ὀλίγον, addidit, vt  \pend
\section*{CAPVT V. }
\marginpar{[ p.369 ]}\pstart hoc suum praeceptum atque doctrinam Euangelii ab omni calumnia vindicaret. Nam videri poterat ex hac Pauli iussione euangelium incitare, ac im- pellere homines ad vinum hauriendum, seseque ingurgitandum, nisi hoc tempera mentum atque hanc velut exceptionem subtexuisset Paulus. Id quod sobrii homines inter Ethnicos vituperas- sent, et merito quidem. Nam et Os.7. vers.5. et Prouerbus 31. quanta, quamque grauia Dei iudicia in ἀνοπότας id est, vinolentos et insobrios homi- nes denuntientur, apparet. Denique Luc.21. vers. 34. Christus et crapulam et ebrietatem damnat. Quare merito moderationem atquesobrietatem hac voce ὀλίγον praescripsit Paulus, etiam Timo- theo sua iam sponte sobrio et temperanti, nedum vt praeceptum de sobrietate seruanda cuiquam graue aut intempestiuum videri possit. Cur autem praescripserit vsum vini Timo- theo, ratio est, non quod medicum agat Paulus: sed quod in suae ipsius valetudinis detrimentum et virium corporis debilitationem nimiam ab- stemius esset, idque sponte, hac sibi lege imposi- ta, ad domandos carnis impetus et libidinosos motus, qui in iuuenili illa aetate magni et feruen- tes esse potuerunt, nisi hac tanquam frigida suf- fusa, id est, vini detractione, temperarentur et ex- tinguerentur. Itaque Timotheus habita tum suae Ætatis, tum Vocationis ratione, vino sponte absti- nebat: non superstitione aliqua, quasi vini vsu Deus offenderetur: non haeresi, quasi hanc Dei crea turam damuaret, quod postea fecerunt Aquarii et Seueriani: Non voto aliquo suscepto, quale fuit veterum Nazaraeorum votum. Non denique  \pend
\section*{AD I. PAVL. AD TIM. }\pstart odio et auersatione ipsius gustus et odoris, quod faciunt ii, qui natura sunt abstemii, et vinum gustare non possunt: sed certo animi proposito et instituto, vt carnis libidinem in tam feruente aetate, tanquam aestate, reprimeret, at que compes ceret: vti qui equis, ne nimium ferociant, pabuli partem adimunt, ne nimium impinguentur. Ergo qui superstitione aliqua ducti, qui haeresi alicui addicti, qui voto constricti, vino abstinent hoc Timothei exemplo iuuari vel deffendi non possunt, cuius factum hic a Paulo commemora- tum est ab illis dissimillimum. Denique ratio addita est a Paulo, et modus, ex quo Timotheum aestimare vult, quatenus et vino et caeteris cibis abstinere debeat. Nimirum quatenus ratio valetudinis et corporis sanitas patitur id fieri: non vt longaeui tantum fiamus: sed vt commodius et facilius Deo inseruiamus. At que haec quidem particulariter pertinent ad Ti- motheum: Videamus iam quae ex his colligi de- beant, et ad omnes Presbyteros et Ecclesiae prae- positos diffundantur. Duo vero canones ex hoc versiculo a nobis excerpentur, iique generales et vtiles. I Presbyteri eam victus rationem ac sobrie- tatem instituunto, quae et ipsorum muneri facile obeundo conducat, et caeteris ad exemplum tem perantiae esse et praelucere possit. 2 Presbyteri at que praeposiri Ecclesiae in ea vi ctus sui sobrietate ac ratione instituenda, et bo- nae valetudinis suae, et virium stomachi atque to- tius corporis sui rationem habeuto. Atque hic vterque canon est exp licandus.  \pend
\section*{CAPVT V. }
\marginpar{[ p.37 ]}\pstart Ac primum quidem primus. Iam inde a primis Apostolorum temporibus propter et Muneris eminentiam, et functionum ipsarum Ecclesiasticarum finem receptum est, vt maior quaedam sanctitas a praepositis Ecclesiae requireretur, quam a reliquo populo Christiano. Qui enim in eo gradu collocantur, vitae exemplo caeteris praelucere debent, quemadmodum docet Petrus.1.Epist.5.vers.3. Vnde a Christo appellan- tur et Lux et sal terrae, in quo caeteri saliuntur. Math.5. Neque modo iniis, quae ad cultum Dei, veramque vitae reformationem et sanctimoniam pertinent, quale est vt scortatione abstineant, fur to, caede, idololatria, haec seuerior vitae ratio in praepositis Ecclesiae requirebatur: sed etiam in re bus ἀδιαφρόροις et mediis: et quae etiam licent reli- quis Christianis, nempe propter muneris ipsius Ecclesiastici dignitatem et sanctitatem, maio- remque commendationem. Vnde Paulus 1. Co- rinth. 9. vers. 27. ait se contundere corpus suum, et subiicere, ne cum aliis praedicauerit, ipse re- probus fiat, cedatque illiin exitium, quod ab aliis requirit. Et quae secuta sunt Aposto- los tempora austeriorem, rigidiorem, seueriorem que semper praepositis Ecclesiae, quam reli- quo populo, disciplinam indixerunt, quemadmo dum et ex canonibus Apostolorum, qui dicuntur et decretis synodorum apparet, maximeque ex eo, quod Thelesphorus Episcopus Roma. ieiu- nium Quadragesimale primum solis clericis in- dixerat, a quibus seruabatur, non etiam a reliquo Ecclesiae populo, vt apparet etiam ex ipso decre- to.  \pend
\section*{AD I. PAVL. AD TIM. }
\marginpar{[ p.372 ]}\pstart Ratio huius seuerioris regulae illis imponen- dae fuit duplex. Prima, quod debent propter doctrinae (quam ardentius et purius profitentur quam reliqui) confirmationem et commendationem, quemad- modum diximus, in omnibus quae non modo ho nesta sunt, sed speciem honesti habent, praeluce- re reliquo gregi. Itaque non modo omnem cra- pulam, luxum, lasciuiam fugere debent praepositi Ecclesiae, sed vel solam illorum vitiorum vmbram et speciem, vt alios suo exemplo aedificent, caele- stemque doctrinam commendent. Nam sic vitia aliorum liberius reprehendere possunt. Est enim haec et verissima et maxime notabilis Hierony- mi sententia, An non confusio et ignominia est Iesum Christum crucifixum et pauperem, et esu- rientem fartis praedicare corporibus, et ieiunio- rum doctrinam rubentes buccas, tumentiaqueora proferre? Si in Apostolorum loco sumus, non so- Ium sermonem eorum imitemur, sed etiam con- uersationem et abstinentiam. Haec Hierony- mus in Mich. Secunda, quod fere homines, quale est huma- num ingenium, iis rebus externis plurimum tri- buunt a quibus si quis abstinet, is et Dei timen- tior, et sanctior a vulgo iudicatur. Coloss. 2. in fin. Tertia quoque ratio addi potest. Quod etiam (quae est nostra vitiositas et corruptela) in vsu re rum indifferentium nimium libero, minusque seuero caro fit lasc iuior, et intemperantior, quam tamen contundere necesse est. Id quod fieri nisi rerum earum, etiam concessarum, detractione sae  \pend
\section*{CAPVT  V. }
\marginpar{[ p.373 ]}\pstart pe non potest. Adeo, vt ei, qui vere temperanter, et sobrie viuere velit, carnisque obscoenos motus comprimere, omnino sit necesse multis rebus ad tempus, vel in perpetuum abstinere, quae tamen licent: sed earum vsu carebit ipse sponte, non quidem propter conscientiam, quasi vetita sint edi, sed quod hoc fraeno carnis libido et rebellio domatur, et rationi subiicitur. Partiar autem totam hanc disputationem in tria capita, nempe vt agam de fine huius abstinentiae, De Cibis: et De Modo illius. Finis autem, propter quem quis iis cibis et re bus abstinet, varius esse potest, vnus tamen, is- que solus probandus est, quem secutus est Ti- motheus, vt ex hoc loco apparet, nempe vt Deo liberius inseruiamus. Ac primum quidem Ethnici ipsi et gentiles quibusdam cibis abstinuerunt propter super- stitiosas opiniones, et plane in Deum blasphe- mas. Ægyptii porris, caulibus, cepis et aliis oleri- bus id genus seuere abstinuerunt, quemadmodum M. Tullius, Diodorus Siculus, Iuuenalis, aliique plures authores tradiderunt. Pythagorici philoso phi carnibus abstinuerunt. Philistinis, qui et Palae stini, piscium genus nullum edere concedebatur, et quidem hi omnes, quod eos cibos quibus absti- nebant, Deos esse crederent, aut ita diis consecra tos at que sanctos, vt in hominum mortalium vsu et commercio esse minime deberent. Hic finis prorsus et blasphemus est in Deum, qui e creatu ris creatorem facit: et proculdubio damnandus. Deinde Haeretici successerunt, qui quibusdam cibis abstinent, sed non ea ratione, propter quam  \pend
\section*{AD I. PAVL. AD TIM. }
\marginpar{[ p.374 ]}\pstart Timotheus, et pii debent. Nam Seueriani et A- quei vino abstinuerunt, quod opus diaboli esse contenderent. Manichaei carne: alii aliis cibis, qu eos esse nefandas et diabolicas creaturas crede- rent, vt in lib.  de Haeresibus docuimus. Hic finis damnandus quoque est, et impugnat dictum Pau Ii sup.cap.4.vers.4.omnis cibus creatura Dei est, et bonum quiddam. Denique etiam cano. Si quis Presbyter Dist. 3o.improbatur. Tertii etiam et ipsi peccant in fine, propter quem cibis abstinendum est. Etsi enim omnes ci bos a Deo creatos, productosque fatentur:ita ta- men eos distinguunt, vt ex iis alios aliis sanctiores et ad vitam aeternam comparandam aptiores e- xistiment, abusi eo Dei consilio, per quod in ve- teri lege quosdam cibos interdixit suo populo, vt est Deut. 14. Leuit.11. Tales fuerunt Tatiani, Encratitae, et eorum discipuli:tales hodie Mona- chi. Et hic quoque finis est improbus, et transfert ad terrena mundi elementa quod vnius Dei Spi- ritus est proprium, nimirum sanctificare nostras conscientias. Itaque damnandus est iuxta illud Pauli sup.4. vers.4.cibus quilibet a Deo conditus in vsum hominis minime reiiciendus est, et Ro- ma 4.vers. 17. Non est regnum Dei esca et potus, sed iustitia et pax, et gaudium per spiritum san- ctum et I.Corinth. 8.vers.8. esca nos non commem dat Deo. August. in lib.  quaest. Euange. nec in ab- stinendo, ait, nec in manducando est iustitia. Et ne quis sibi nimium in ea ciborum abstinentia pla- ceat, haeretici etiam plerique ea excelluerunt, quem admodum Socrat es lib. 7.cap.17. tradit. Alii voto, eoque perpetuo, hanc abstinentiam  \pend
\section*{CAPVT  V. }
\marginpar{[ p.375 ]}\pstart ciborum sibi indicunt propter illud Pauli in ea- dem epistad Roma.14. vers.21. Bonum est non ves ci carnibus: neque bibere vinum: et illud etiam quod est 1.Corinth.8.vers.13 si esca facir, vt offen- dat frater meus, non vescar carnibus in aerernum. Vtroque modo, hoc nimirum, et superiore in Chri- stiana Ecclesia grauiter peccatum est, et pene primis ecclesie temporibus, dum inconsiderato quodam sanctitatis zelo homines ducuntur. At- que hic finis proxime a nobis commemoratus ex eo satis refutatur, quod Pauli dictum conditio naliter et κατὰ τι prolatum tanquam simpliciter sit dictum et effatum isti accipiunt. Docet enim Paul. quibusdam cibis abstinendum tunc, cum pro- ximi aedificatio, et iusta ratio id exigit: non autem ex voto id fieri vult, neque simpliciter et omni- no. Verus igitur et legitimus huiusmodi ciborum abstinentiae finis, isque probandus, est ille, quem scriptura praescribit. Est autem is duplex vel Car- nis nostrae extraordinaria mortificatjo, vel Infir- morum fratrum ratio et aedificatio, qui ad tempus sunt tolerandi a nobis. Primus finis commemoratur a Paulo 1.Corin. 9. vers.27. contundo corpus meum, et in seruitu- tem redigo. Item Coloss2.vers.23. Sic ieiunium institutum et vsum habuit in Dei Ecclesia vel a tota, vel ab vno aliquo priuatim susceptum. Atque certe ibi recte obseruat Chrysostomus dici a Pau lo Castigo: et in seruitutem redigo:non dixit, per- edo et Punio. Non enim inimica caro est, sed casti go, et redigo in seruitutem, quod Domini est, non hostis: magistri, non inimici: exercitatoris, non adu ersarii Bernad. sermo. 66.in Cantico. docet.  \pend
\section*{AD I. PAVL. AD TIM. }
\marginpar{[ p.310 ]}\pstart Nos cibis posse abstinere triplici de causa vel ex medicorum praeceptis, quod non reprehenditur, si tamen nimia non sit cura corporis: vel ex regu lis abstinentiae, vt mortificetur caro, quod etiam si sine superstitione fiat laudandum est: vel ex in- sania haereticorum, quod est omnino damnan- dum. Et in epistola ad Coloss idem Chrysosto. Deus, ait, corpori honorem suum contribuit, ipsi vero (de nimium abstinentibus loquens) non cum honore vtuntur corpore. Denique concilio Gan- grensi, quod vicinum fuit primo Niceno, nimia ista abstinentia, et si quavis in ea ponatur est graui- ter reprehensa. Quod vero scribit Gregor. Mag. pios a licitis rebus fere abstinere, vel vt sibi me- rita apud Deum omnipotentem augeant, vel vt culpas vitae anteactae deleant, sitque haec absti- nentia pro poena, est falsissimum, etsi probatur cano. Sunt qui 27.quaest.2. Secundus finis expressus est saepe a Paulo in epistolis, praesertim in ea, quae est ad Romanos cap.14, et 15. 1. ad Corinthios 9. atque etiam in Actor. ca.15. a Synodo Hierosolymitana. Propter aedisicationem enim infirmorum, qui non pote- rant videri peccare malitia, sed nuda tantum i- gnorantia, Paulus se omnia omnibus factum esse dicit: et iubent Apostoli collecti quibusdam ci- bis Christianos abstinere ob eandem causam. Alterum caput huius disputationis est de re- bus, quibus abstinere solerent veteres, et nunc debeamus ipsi abstinere. In quo nullum est prae- ceptum diserte ab ipso Dei verbo positum: sed debet quisque iis cibis abstinere, quibus libidi- nem in sese maxime vel accendi, vel excitari sen-  \pend
\section*{CAPVT . V. }
\marginpar{[ p.277 ]}\pstart tit et experitur. Itaque qui carnis esu se prorita- ri vident, carne abstinere debent. Qui vero pi- scium esitatione se lasciuiores fieri sentiunt (est enim cibi genus delicatissimum, vt testatur in Symposiac. Plutarchus piscium esus) piscibus ab- stinebunt. Qui vino se libidinosiores effici com- perient, vinum ne bibant, quia quo maxime ca- ro nostra proritatur, illud est subtrahendum, quem- admodum ferocienti equo nimium pabulum (quen- admodum in lib.  de Arte equestri tradit Xeno- phon) est subducendum, vt contundatur, et do- metur facilius. Qui vero esu leguminum se ad peccandum magis incitari sentiunt, leguminibus potius quam carne debent abstinere: quia fon- tes et incentiuum peccati est tollendum, non illud, quod libidinem in nobis non accendit. Irem quorum ciborum esu maxime offenduntur infir- mi adhuc fratres, quibuscum versamur, illis est abstinendum, non est autem par ratio in aliis ci- bis Fieret enim temere et sine causa. Neque vero hic nobis vel Nazaraeorum votum, de quo est Numeror. 6. quisquam in vsum et praeceptum reuocet. Fuit enim illud caeremoniale, et iam per Christum implementum atque finem habuit. I- taque desiit esse in vsu. Neque etiam quisquam regerat, reponatque vetus illud discrimen ci- borum, quod in lege sua caeremoniali Dominus instituit Deuter. 14. Leuitic. 11. quia illud criam fuit plane caeremoniale. Vnde Apostoli ipsi de vsu ciborum interrogati minime illud inter ho- mines retinuerunt in concilio Hierosolymitano Actor. 15. Quin etiam illud ciborum discrimen saepe Paulus oppugnat, imprimis autem in Epi-  \pend
\section*{AD I. PAVL. AD TIM. }
\marginpar{[ p.378 ]}\pstart stola ad Galatas. Denique neque etiam aliquis nobis obtrudat illud, quod est Leuitic. 1o.vers.9. quia et hoc ipsum plane antiquatum est a Chri- sto, vti et altare et Sacerdotium. Fuerunt tamen certa quaedam cibortim gene- ra, quibus maxime ἐγκρατέυοντες et μονάζοντες Chri stiani abstinere solebant, partim ex nimia iam superstitione, partim ex inepta virorum piorum imitatione. Ac quidem imitatione, quod antiqui- ssimi Christiani propter persecutiones in deserta ac vastas solitudines fugere coacti, saepe herbis tantum, aqua pura, secundo pane, et siliquis, simi- libusque rebus illic vixerunt, cum neque caro, neque vinum, neque alii cibi propter incultum locum suppeterent illis, neque illic reperirentur. Horum vitam imitari se putauerunt, qui tam vi- libus cibis postea, etiam sine necessitate, et in me- diis vrbibus victitarunt. Superstitione vero quidam impellebantur, quod iam plus satis externae isti vitae austeritati, et curiositati quadam verae sanctitatis ignoratio- ne homines tribuebant. Ergo qui se reliquis pu- riores atque mundiores esse inter Christianos volebant, vt ex Eusebio lib. 2.cap.17 lib.  6. cap.3. Socrate lib.  4. cap.23. aliisque veterum locis col- ligitur, primum fere astinuerunt his cibis (quan- quam tamen, vt ait Paulus Tit. I.omnia sunt mun- da mundis) abstinuerunt, inquam, Carne, Oleo, Vino: Oleribus interim, et reliquis rebus indi- scriminatim vescentes, puram etiam aquam po- tantes, non conditam, non mistam, non saccaro, non herbis vim illius augentibus, aut vsum gu- stumque gratiorem facientibus temperatam. Post-  \pend
\section*{CAPVT  V. }
\marginpar{[ p.879 ]}\pstart ea creuit superstitio, et haec initia latius postea sunt propagata, adeo vt postea, et iam tempore Gangrensis Synodi, quae fuit sub Constantino Magno, Sagimen omne et pinguedo, et non tan- tum caro ipsa Monachis et abstinentibus interdice retur. Demum accessit decretum Gregorii Ma- gni, quod abstinere cupientibus prohibet et in- terdicit esu omnis rei, quae sementinam carnis originem habeat, adeo, vt iam non solum caro i- psa sit prohibita: sed et caseus, et ouum, et lac, et butyrum, quia sementinam carnis, id est, a carne originem habent. Hoc iam sequuntur omnes Papistae, et est in canon. Denique distinct.4. tan- quam hoc sit perfectissimum abstinentiae genus. Bernardus tamen sermo 66. Cantic. videtur sen- tire hanc sententiam esse ex Manichaeorum hae- resi deriuatam. Caeterum, ait, quid sibi vult, quod ita generaliter omne, quod ex coitu generatur, vitatur? Nempe horrent lac, et quicquid ex eo conficitur, vt in eo Bernardo neque cum Grego- rio Magno, neque cum superstitiosis nostri tem- poris Papistis conueniat, qui generaliter omnes cibos istos prohibent. Mirum tamen cum et vo- tum Nazaraeorum vini interdictionem contine- ret, et Timotheus vino potius, quam aliis reli- quis cibis ad carnem domandam abstinuisse dica- tur: et veteres Monachi vino potissimum non v- terentur. Praeterea cum tota distinct. 35. vini in- commoda atque pericula varia enumerentur, maximeque Episcopis, et sanctioribus viris ca- uendum praedicetur. Denique cum Prouerbus 31. serium aliquid acturis vinum interdicatur, mirum, inquam, est, nullos tamen Monachos etiam abstinen- tissimos et rigidissimae vitae regulam professos.  \pend
\section*{AD I. PAVL. AD TIM. }
\marginpar{[ p.300 ]}\pstart Vino tamen abstinere voluisse, nisi fortasse Berna. excipias, qui se vino docet abstinuisse. Atqui ta- men Paulus Ephes.5. vers.18. in vno hoc cibi ge- nere lasciuiam et ἀσωτίαν inesse testatur potius, quam in reliquis omnibus cibis. Ex quo apparet Monachismum nostri temporis, addo etiam le- suismum, esse meram hypocrisin, in quo non vera abstinentia praescripta est, sed sola tantum ostentatio et abstinentiae vmbra, cum vino potius, quam carne esset illis abstinendum, si vere carnem contundere studebant. Atque haec de cibis. Sequitur tertium et vltimum huius disputa- tionis caput, in quo altera regula a nobis in vitae ratione, quae a praepositis Ecclesiae seruanda est, constituta continetur et exponitur. Quaeritur e- nim De modo suscipiendae istius abstinentiae, et instituendae sobriae victus rationis, quis esse et statui debeat. Breuiter autem respondet Paulus, eum esse modum legitimum, atque laudandum, in quo valetudinis ratio habetur a nobis. Ex quo fit, vt tria colligamus. primum non esse nobis o- mnino omnibus cibis abstinendum. Neque enim tunc, nisi modo plane miraculoso, viuere posse- mus. Deinde cousequitur, non semper, aut in perpetuum extraordinaria illa ciborum abstinen- tia vtendum nobis esse, sed tantum ad tempus. Nam deficerent vires, et paulatim ita debilitare tur natura, vt nulli negotio superesse possemus. Sobrietas quidem nobis est per totum vitae cur- riculum seruanda: sed haec quoque non sen- per in iisdem cibis fugiendis versatur. Praeterea cum de extraordinaria quaeritur, quae ad doman- dum magnum carnis impetum adhibetur, ea cer-  \pend
\section*{CAPVT V. }
\marginpar{[ p.381 ]}\pstart te perpetuo obseruanda et tuenda non est. Vt igi- tur non sunt idem impetus senilis, atque iuuenilis aetatis: ita neque eadem vitae ratio ab vtrisque est suscipienda, vel vtrisque praescribenda. Mutantur enim haec pro tempore, aetate, locis, personis, vt etiam praeclare scripsit, sensitque Augustinus in quaestion. Euangelii. Non enim, ait, interest o- mnino quid alimentorum sumas, vt succurras necessitari corporis, dumenodo congruat in ge- neribus alimentorum cum his, cum quibus ti- bi viuendum est: neque quantum sumas interest multum, cum videamus aliorum stomachum ci- tius saturari, etc. Deinde ira concludit idem Au- gustinus, Magis ergo interest, non quid vel quan- tum alimentorum pro congruentia hominum at- que personae suae, et pro suae valetudinis necesli- tate quis accipiat: sed quanta facilitate, et seueri- tate animi iis careat, cum iis vel oportet, vel ne- cesse est carere, vt illud in animo Christiani com- pleatur, quod ait Apostolus, Scio et minus ha- bere: scio et abundare. Tertium denique quod ex hoc Pauli dicto colligitur, est huiusmodi. Ni- hil in iniuriam naturae, et nimiam corporis de- bilitationem vel instituendum, vel institurum et semel susceptum retinendum nobis esse. Quod Timothei exemplo docemur hoc loco, qui, quem- admodum verisimile est, post hanc Pauli com- monefactionem vino vsus est, quo tamen absti- nere decreuerat. Sed quia ea abstinentia debilita- bantur vires corporis naturales, postea vinum in victum adhibuit. Est enim non modo vita haec donum Dei: itaque a nobis minime spernenda: sed virium naturalium et sanitas et integritas quo-  \pend
\section*{502 A D I. IAVE. AD TIM. }\pstart que donum Dei est, quae quantum quidem in nobis est, fouenda, sustentanda, tuendaque est, vt nostro muneri satis esse possimus. Aliâs enim nostra culpa fit, quod inutiles aut minus idonei euadimus. Praeclarum est exemplum veteris Ec- clesiae, quae Alcibiadem quendam nimis austere viuentem monuit, vt ea viuendi ratione abstine- ret, vinumque biberet. At que ille sapienti Eccle- siae consilio paruit libens, vt refert Eusebus lib. 5. Histo. cap.3. Vnde falsa et plane respuenda est illa laus, quam Bernardus serm.6.in Psal. Qui ha- bitat, quibusdam Monachis tribuit, nimirum quod variis corporis exercitiis et abstinentiis contere- rentur supra vires, supra naturam, supraque consue- tudinem. Male quoque idem Bernard. sermo.3o. Cantic. Monachum te fore professus es, non me- dicum: nec de complexione tibi iudicandum, sed de professione. Verior illa est eiusdem Bern. sen- tentia ser. 1o.in Ps. Qui habitat. Nec sane dixerim vt vel ipsam odio habeas carnem tuam. Dilige eam tanquam tibi datam in adiutorium, et ad aeterne beatitudinis consortium praeparatam: caeterum sic amet anima carnem, vt non ipsa in carnem transiisse putetur. Sed et illa quoque eiusdem sen- tentia sermo. 4. in eundem Psalmum verissima est Dominum voluisse, vt in manna caelitus com- pluta omnis saporis, omnisque odoris delectatio inesset, vt liberum esse ciborum vsum atque a Deo sibi concessum homines intelligerent. Nam hic vini vsum praescribit Paulus, in quo suauitas vo- luptasque gustus vnâ cum vtilitate coniuncta est, vt est Psal.1o4.vers.16.nec ignorauit Bernar.tam seueram vitae rationem merito illo etiam seculo  \pend
\section*{CAPVT V. }
\marginpar{[ p.383 ]}\pstart \phantomsection
\addcontentsline{toc}{subsection}{\textit{24 Quorundam hominum peccata an- te manifesta sunt, praeeuntia damnationi: quosdam vero subsequuntur:}}
\subsection*{\textit{24 Quorundam hominum peccata an- te manifesta sunt, praeeuntia damnationi: quosdam vero subsequuntur:}}dici crudelitatem, vt ex eius epistolis apparet, de qua purgare quidem Monasticam vitam niti- tur: sed non potest, vti neque ad hoc argumentum respondit Thomas in Apologia et defensione Monasticae vitae, quia est ineuitabile aduersus cam telum. Modus tamen is adhiberi debet in cibis, in quo nullum ebrietatis, ac intemperantiae spe- cimen aut vestigium appareat. Atque eo pertinet, quod ait Paulus, modieo vino vtere: non enim ad ebrietatem et satietatem sed ad sanitatem sumi et vinum, et reliquos cibos vult Deus cum gratiarum actione. Denique ex hoc ipso loco futilis et inepta pror- sus deprehendi porest illa Bernardi sententia, qui epist.321.negat honestati et puritati vitae Mo- nasticae, id est, sanctae conuenire, vti medicamen- tis et potionibus ad expellendum morbum ido- neis. Nimium certe fuit ille seuerus, et praeter aequitatem ac rationem. Æquior Chrysosto. qui et hic ait Timotheum ad medicos remissum, non autem miraculo curatum a Paulo: et in Homil. 14.in hanc epistolam de vita Monastica agens, v- sum medicamentorum Monachis, cum est opus, non adimit, sed concedit: quanquam tamen rigidior eorum censor habitus est Chrys. Sed Satan semel via su- perstitionibus patefacta, eas in immensum post- ea adauxit, et in iugum intolerabile tandem ag- gregauit ad torquendas hominum conscientias. 24 Quorundam hominum peccata an- te manifesta sunt, praeeuntia damnationi: quosdam vero subsequuntur:  \pend
\section*{AD I. PAVL. AD TIM. }
\marginpar{[ p.904 ]}\pstart \phantomsection
\addcontentsline{toc}{subsection}{\textit{25 Itidem et bona opera ante manife- sta sunt: et ea quae secus habent, occultari non possunt.}}
\subsection*{\textit{25 Itidem et bona opera ante manife- sta sunt: et ea quae secus habent, occultari non possunt.}}25 Itidem et bona opera ante manife- sta sunt: et ea quae secus habent, occultari non possunt. Diuersa quoque est huius versiculi cum supe- riore nectendi ratio. Putant enim quidam cum vecs.22. coniungendum, vers.autem 23.per paren- thesin insertum et legendum quoque. Alii aliud. Nos autem hunc locum tum ωθακολόθησιν, tum etiam νοθέτησιν esse censemus, et idcirco cum su- periore argumento recte conuenire atque con- iungi posse, quod praecepta duo contineat ad Pa- storum et Presbyterorum munus spectantia, ex quibus vniuersa pene totius censurae Ecclesiasti- cae ratio definiri potest. Id quod maxime ad Pre- sbyterorum officium pertinet. Atque id iam hoc toto capite persequitur Paulus. Sic igitur statuo canonem tradi iam a Paulo, per quem quomodo praepositi Ecclesiae se in censuris Ecclesiasticis gerere aduersus gregem suum debeant, instruan- tur, id est, quale sit eorum munus erga vniuer- sum populum sibi commissum et demandatum a Deo. Quae sane tertia pars est muneris pasto- ralis, vti supra diximus, maxima. Prima enim do- cet quomodo erga seipsos gerere debeant. Se- cunda quomodo aduersus collegas suos. Tertia haec, quomodo erga vniuersum Ecclesiae suae coetum. In qua ipsa parte posita est ratio et di- sputatio de censura Ecclesiastica. Quanquam huius praeceptionis, quae nunc traditur, tantam vim esse tamque late patentem fateor, vt etiam ad supe- rius praeceptum, quod ad electionem at que ordi- nationem pastorum Ecclesiae pertinet, referri possit  \pend
\section*{ChRVIV. }
\marginpar{[ p.385 ]}\pstart et in eo magnum vsum habeat. Quare hic totus locus hunc canonem vtilissimum, et valde neces- sarium ex nostra interpretatione continebit. In censuris Ecclesiasticis praepositi Ecclesiae ne temere progrediuntor:sed Deum ipsum prae- euntem sequuntor, qui suo tempore bonos et malos detegit. Malos, antequam a Deo patefacti fuerint, prae- positi ne eiiciunto. Bonos autem, antequam i- psis operib' tales apparuerint, ne speciali hono- re praepositi decoranto: aut caeteris praeferunto. Horum autem omnium canonum ratio est illa quam affert Paulus, quod nimirum multorum peccata saepe latent idque diu: multorum etiam saepe bona opera et candor animi et synceritas diu occulta manet, ita Domino dispensante res humanas e asque administrante.  Cur autem id accidat magna sapientique Dei prouidentia fit. Nam si quicunque sunt improbi et hypocritae in Dei Ecclesia, statim apparerent, neque piorum fides explorari et probari tam certo posset: neque pii ipsi tam officiosi in hoc mundo viuerent, neque etiam tuti essent, sed ab impiis opprimerentur. Facit igitur haec commi- stio, vt melius, certiusque qui sunt probi et veri Dei serui manifesti fiant, cum, tam multis diffluen- tibus et eiectis, illi firmi tamen maneant. Sic hae- reses probant piorum fidem, quemadmodum Pau- lus ait ICorinth.11.vers.19. Praeterea si pii omnes agnoscerentur, impii vero aperte quoque statim ostenderentur a Deo, nulla spes resipiscentiae re- licta illis videretur, omnisque nostrae in delin- quentibus conuertendis industriae et charitatis  \pend
\section*{AD I. PAVL. AD TIM. }
\marginpar{[ p.386 ]}\pstart effectus cessaret. Dominus igitur vult aliorum statum et animum nobis occultum esse, vt cum iis et lubentius versemur: et magis sollicite nobis ipsis caueamus. Sed etiam illis ipsis, qui peccarunt, haec dilatio datur a Deo, tanquam occasio matu- rius resipiscendi, atque ad Deum redeundi. Denique Dominus ipse, qui rerum omnium maturitates et tempora nouit, atque etiam con- stituit videt quando et quae manifestari tum ad ipsius gloriam, tum etiam ad nostram aedifica- tionem oporteat. Neque vult nos omnia illa sci- re quae scit ipse. Huc pertinet quod Ecclesia Dei comparatur a Christo ipso sagenae siue reti, quo boni et mali pisces capiuntur Matth.13.vers.47. Areae, in qua bonum granum et paleae promiscue conduntur. Matth.3.vers.12.Item agro, in quo bonum semen et lolium quoque nascitur Matth.13.vers.25. De- nique domui amplae, in qua sunt quaedam vasa honesta, quaedam sordida 2. Timoth.2. vers.2o. Nam quos domi suae adhuc tolerat Dominus ipse, non est nostrum eos expellere vel arcere:ne- que Deo ipso vel sapientiores, vel iustiores, vel ardentiores sumus ad ipsius gloriam promouen- dam. Hoc etiam ipsum pertinet tum ad Pastorum ipsorum excusationem, tum ad Bonorum con- solationem. Ac quidem excusantur pastores, qui quanquam diligentiam, quam potuerunt, sum- mam adhibuerunt, vident et animaduertunt ta- men in Ecclesia Dei adhuc latere quosdam im- probos et hypocritas, qui paulatim deteguntur, et a Deo ex suis tenebris in lucem pertrahuntur, pa-  \pend
\section*{CAPVT  V. }
\marginpar{[ p.383 ]}\pstart tefactis eorum sceleribus. Ergo non sunt accusandi fideles et diligentes Ecclesiae praepositi, si ex iis, qui ab ipsis electi sunt et ordinati, aliqui postea ap parueunt esse nebulones, et pestes Ecclesiae: aut si quosdam in Ecclesia Dei fouerint et retinuerint qui postea detecti sunt improbissimi, et pessimi. Nam ad tempus multorum peccata latent, et tempore patent, vt ait etiam ille Ethnicus. Neque statim produntur improbi. Bonis etiam viris haec eadem commixtio et occultatio consolationem affert, si quando videntur sperni, et prae aliis ne- gligi, quod cogitant vnius Dei esse hominum a- nimos et synceritatem mentis prodere atque pa- tefacere, id est, in lucem proferre, cum voluerit. Interim igitur sua sorte et obscuritate contenti viuunt, neque murmurant, vel in Deum ipsum, vel in homines. Quod autem ait Paulus vers.25. Que secus sunt id est quae praua sunt, non posse occultari, et in perpetuum celari, ita accipiendum est, vt hoc fie- ri saepissime, et vt plurimum, intelligamus: non autem vt semper. Saepe enim multorum et bona opera, et turpissima scelera latent nos et occulta manent perpetuo, quia ita Deo visum est. Sed vt plurimum Dominus detegit improbos: et piorum syncerum animum prodit, vt est in Psal.37. vers. 6. et Psal.112. Ex hoc igitur loco colligimus iudicium de mo- ribus: et censuram, per quam peccata hominum Ecclesiastica poena coercentur, pertinere ad Prae- positos Ecclesiae, quos hoc loco alloquitur Pau- lus. Quam sententiam confirmat Christus Matt.16 Quaecunque ligaueritis in terra, erunt ligata et  \pend
\section*{AD I. PAVL. AD TIM. }
\marginpar{[ p.388 ]}\pstart in caelis. Item et illud Bernardi, pastorum est vi- gilare super gregem propter tria necessaria, ni- mirum ad disciplinam, ad custodiam, ad preces. In disciplina rigor iustitiae: in custodia spiritus consilii: in precibus affectus compassionis requi- ritur. Censura vero haec est multiplex, vti supra est demonstratum. Restat autem vt de excom- municatione dicamus, quia de reliquis generibus censurae Ecclesiasticae quid statuendum sit, ex iis, quae hoc ipso capite dicta sunt a nobis, facile po- test intelligi. Ergo de excommunicatione dica- mus, de qua septem haec faere quaeri solent. 1 Vtrum excommunicatio habere locum in Dei Ecclesia debeat. 2 A quo fieri debeat. 3 Propter quas causas fieri debeat. 4 Quomodo fieri debeat. Quando fieri debeat. 6 Quis eius finis esse debeat, id est, ad quem finem fiat. 7 Quis illius factae sit effectus. Haec singula ordine explicemus. Ac quidem, vtrum excommunicatio locum in Ecclesia habere debeat, dubitari non potest. Pri- mum, propter Dei ipsius praeceptum ita fieri iu- bentis: deinde propter perpetuum ipsius Eccle- siae vsum. Ac praeceptum Dei tum in veteri, tum nouo testimonio extat variis in locis. Iam inde ab orbe condito, et ab ipsis mundi initiis excon- municationem fuisse ex Dei praecepto constitu- tam apparet ex Genes.4.v.14. et 16.17.item Iosue 6.vers.18. Exod.1o.vers.8. et aliis locis veteris Te- stamenti. In nouo etiam extat praeceptum Chri-  \pend
\section*{CAPVT  V. }
\marginpar{[ p.389 ]}\pstart sti, quod Matt.18. et Ioan.2o.recitant. Ex quibus i- psis locis eam spiritualis censurae et castigationis spe- ciem in vsu fuisse in Ecclesia docetur, vti ex Ioan. cap.9. et 1.Cor.5 et 2.7. et hoc ipso loco. Neque de eo dubitari potuit, nisi duo quaedam argumenta contra afferrentur. Primum neminem inuitum cogendum esse, vt recte viuat aut sentiat. Nam, vt vulgo dicitur, Inuitum qui seruat idem facit occidenti. Id autem faciunt, qui inobsequentes Deo excommunicant. Huic argumento saepe re- spondit Augustinus in disputationibus contra Donatistas et negat veram esse hanc sententiam. Saepe enim qui primum inuitus ab errore vitae vel doctrinae abductus est, postea volens et lubens in recto itinere permansit. Praeterea cum quis corrigitur inuitus, si hoc ipsum illi non prodest, quia inuitus id facit: aliis tamen prodest, qui metu poenae coercentur, ne ad eadem scelera exemplo pertrahantur. Denique haec eadem ratio facit ne in sceleratos homines vllos, quantumuis impios et derestabiles, poenae vllae etiam a Magistratu decernantur, quia ea ratione inuiti resipiscere et modeste viuere coguntur. Itaque haec prima ra- tio est absurdissima. O Alterum argumentum, quod contra vsum ex- communicationis affertur, ducitur ex loco Pauli 2 Corinth. 1o. vers.8. Nempe datam esse pastori- bus, et Ministris verbi Dei potestatem ad aedifi- cationem, non ad subuersionem et destructionem. At excommunicatio destruit gregem Domini, quem minuit, et eum ipsum qui excommunica- tur, perdit, quia exponit Diabole:ergo ea vti pa- storibus non licet in Ecclesia. Respond. Aliquid  \pend
\section*{AD I. PAVL. AD TIM. }
\marginpar{[ p.30 ]}\pstart fieri in aedificationem Ecclesiae duobus modis, vel Per se, vel Per consequens. Per se, cum ali- quid in ea plantatur: per consequens, cum quod illi noxium est, euellitur. Est enim similis agro Ecclesia Dei, vt est apud Matth.13. At in agro si- ue quid noxiarum herbarum (quae bonas et vtiles suffocant) extirpemus: siue quod vtile semen sera- mus, agro prosumus. Ergo eodem modo de Ec- clesia statuendum, qui ex ea eiicit improbos mul- tum iuuat et prodest Ecclesiae. Quod autem obiicitur saepe quosdam ea abu- sos esse ad priuatam animi sui vindictam: saepe bonos ex Ecclesia Dei depulsos ea ratione, non oppugnat nostram sententiam. Ea enim sunt vi- tia persouarum male sua potestate vtentium, non autem rei ipsius. Sed distinguunt sic quidam, nimi- rum, vt velint quibus in locis et Ecclesiis fidelis est Magistratus, in iis nullum excommunicatio- nis vsum esse debere: in quibus autem non est Magistratus fidelis, in iis tantum debere: quia fi- delis Magistratus si fungitur officio, punit scelera, quae contra Dei verbum admittuntur. Nemo autem vnius criminis et peccati ratione duplicem poenam ferre et pati debet. Quod tamen accide- ret, si idem etiam ab Ecclesia excommunicare- tur. Res. Peccatum omne quod contra Dei prae- cepta fit, esse huiusmodi, vt ex Dei ipsius defini- tione et corporis et animi poenam mereatur, quia vtriusque partis nostrae creator est Deus. Et cum dupliciilla poena afficitur peccator, vna tantum poena afficitur, id est, ea, quae debetur, et indicta est a Deo ipso: etsi in diuersis partibus a nobis vna illa poena sentitur. Itaque Esdrae cap.1o.  \pend
\section*{CAPVT . V 2 }\pstart vers.8.vtraque poena in rebelles indicitur: ciuilis et Ecclesiastica siue excommunicario, quia vtriusque partis nostrae Dominus author est. Est enim pec- catum totius suppositi actio, itaque totum sup- positum ad poenam ipse exigit. Ergo in Dei Ecclesia vsum habere debet ex- communicatio. Quaeritur autem, quid sit excommunicatio. Ac dubitari non potest, eam esse censurae et coer- citionis Ecclesiastieae speciem quandam. Ergo sic eam definiunt nonnulli, vt generaliter esse ve- lint a qualibet societate fidelium hominum se- parationem: itemque ab aliquo, vel omnibus do- nis Dei in Ecclesiam effusis interdictionem et publicam hominis scelerati prohibitionem. Ve- rum haecdefinitio generalior est. Vnde sic rectius definiri potest excommunicatio, vt sit vel ab ipso Ecclesiae coetu toto et societate: vel ab aliquo Dei dono, quod quidem omnium Christianorum commune sit, publica et legitima interdictio siue separatio hominis impuri et non poenitentis. Ex quo fit, vt a quibusdam, velutia Scholasticis duplex excommunicatio statuatur, vna nimirum Minor, quae est interdictio tantum a Sacramentis, vel ab vno eorum, non autem a coetu Ecclesiae: altera Maior, per quam non tantum a Sacramentis, sed ab ipso to- to Ecclesiae coetu quis separatur, quam ipsam e- tiam diuidunt in Excommunicationem quae ap- pellatur hoc nomine: et Anathema, cuius etiam fit mentio apud Paulum Galat.1.et 1.Corinth.16. et alibi in scriptis Patrum. Qui inter Hebraeos ista scrupulosius et diligen- tius scrutantur, sanctionis legis constituunt ge-  \pend
\section*{AD I. PAVL. AD TIM. }
\marginpar{[ p.332 ]}\pstart nus triplex. 1 Auersationem, quam appellant un quae est interdictio ab ingressu templi ad tempus pro- ptercorporis immunditiem. 2 Deuotionem, quam vocant uun quae est ex- tremo exitio addictio et respondet nostrae excon- municationi, per quam quis a coetu Ecclesiae se- ponitur donec resipiscat. Hoc ipsum aliis verbis dicitur exterminari e populo Dei, item ἀ ποσυνά. γωyoν fieri Ioan.12.vers.42.16.vers.2. Adiudicationem, quam dicunt nnov est si- milis anathemati. De secundo genere tantum loquitur hic Pau- lus, et nos quoque, quod etiam vtranque excom- municationem continet et maiorem, et minorem. Ethnici homines videntur simile excommu- nicationi quiddam habuisse ex praua imitatione Ecclesiae Dei: quam plane simiarum more fa- ciebant. Quosdam facris suis interdicebant, quos ὠαγεῖς, id est, detestabiles appellabant, et ipsam sententiam ἐναγός. Sic Alcibiades factus est, et excommunicatus Athenis, vt Plutarchus docet. Ergo quaesitum est secundo loco, a quo ex- communicatio fieri debeats Vtrum ab vno tantum, an a senatu Ecclesiastico, an vero a tota ipsa Ecclesia. Dixit Christus. Matth. 18.vers.17. Dic Ecclesiae. Dicit contra Paulus, Hunc notate, tan- quam rem totam ad se deferri velit, sibique ipse authoritatem eam vendicet. Et refertur a Theodo- rito Cyri Episcopo historia. lib 5. cap.37. de quo- dam Heremita, qui solus Theodosium impera- torem Constantinopolitan. huius nominis secum- dum excommunicauerit. Scholastici dicunt Epi-  \pend
\section*{CAPVT V. }
\marginpar{[ p.393 ]}\pstart scopos, et eo superiores omnes ordines posse excommunicare: at qui sunt Episcopo interiores non posse, adeo vt et Presbytero et Diacono idem non concedant quasi ex vnius Episcopi nutu et sententia id fieri debeat. Sic igitur illi censent. Cer- te ab vno solo, quanquam Episcopo, fieri neque debet, neque potest excommunicatio. Dicit e- nim Christus, dic Ecclesiae: et Paulus de excom- municatione agens, vult Ecclesiam cum suo spi- ritu congregatam esse 1.Corinth.5. vers.4. Ratio est, quod differunt regna politica ab Ecclesia, et Ecclesiastico regimine. Nam regna vnius regis nutu administrantur, at Ecclesia non item. Nec enim pastor est rex et dominus Ecclesiae (vt ait Petrus) 1. Pet.5.vers.3. Ait igitur Christus reges et principes domi- nantur inter suos subditos: vos autem non sic in gregem Domini et Ecclesi.am Quare vnus quidam quantumuis magna in Ecclesia praeditus potesta- te non potest solus excommunicare. Nec aliud sibi vult Paulus cum ait, Notate per Epistolam. Vult quidem rebelles sibi significari, vt pudore maio- ri afficiantur, quod tam probrose innotescant a- pud eum, quem reuereri debent. Quod si dicamus et legamus, Notate, referentes vocem (Epistola, ad verba sermonem nostrum, haec erit mens Pauli vt velit rebelles a tota Thessalonicensium Eccle- sia excommunicari. Et quod dicitur pastor πρόεδρως non tribuit illi ius et imperium regium in Ecclesiam: sed in pu- blicis coetibus eminentiorem tantum locum, et primam in decernendo deliberandoque vocem. Sed vtrum a senatu Ecclesiastico, an vero a to-  \pend
\section*{AD I. PAVL. AD TIM. }
\marginpar{[ p.334 ]}\pstart ta ipsa multitudine ecclesiae et a quoque fideli soffragium sigillatim ferente fieri debeat, quaesitum est. Res. Senatu suffragium ferente in consistorio fieri debet excommunicatio, ad quem senatum morum et doctrinae cura, diiudicatio, et aestima- tio pertinet, conscia tamen et approbante postea ecclesia tota, cui illa res, actio, sententia sena- tus et caussa nota fieri debet, vt eam approbet ip sa, vel reprobet vt supra docuimus. Quod enim omnes tangit, ab omnibus fieri debet, sed a quo- que pro eo gradu (vt docet Ambrosius lib.  de Di- gnitare sacerdotali. cap.31) et ordine, quem habet in Dei Ecclesia. Haec sententia editis libris con- firmata est, et abunde antea a nobis probata. Obi- ter autem, quid in hoc dicto Christi, dic Ecclesiae, vox Ecclesiae valeat et significet explicatum est etiam in ca. Nullus in verbo Ecclesia dist. 63.item cano. Ecclesia de consecratio. dist.1.et cano. scire. 7.quaest.1. ne putent qui nobis contradicunt, no- uam aliquam interpretationem a nobis excusam, et fabrefactam. Est igitur illic Ecclesia accepta pro ipso coetu praepositorum Ecclesiae, id est, pro pastoribus et Presbyteris. sed et Dauid Kimhi in Ol5 ait nomine Domus Israel saepe solum Syne- drion contineri, et designari in sacra scriptura. Quaeritur autem tertio loco. Propter quas caus- sas fieri debeat excommunicatio. Scholastici di- stinguunt, et volunt Minorem quidem propter peccatum quodlibet fieri posse. Maiorem autem excommunicationem non nisi propter crimen, id est, tale peccatum, quod et graue sit, et poenam publicam a magistratu mereatur, quale est adul terium, homicidium voluntarium, falsi crimen et  \pend
\section*{CAPVT . V. }\pstart caet. Vtranque tamen non nisi propter rem et deli ctum iam notorium, fieri debere. Haec illi. Alii aliter distinguunt. Volunt.n ex iis tantum caussis et criminibus excommunicationem iaci posse in aliquem, ex quibus quis damnatus lege Dei lapidaba tur et moriebatur. Ex quibus autem caussis damna- tus neque lapidaretur, neque vltimo capitis sup- plicio afficeretur, non posse quenquam excom- municari. Vtraque distinctio non nititur autho- ritate scripturae. Ac quidem proxima ciuilem Iudaeorum politiam cum Ecclesiastica plane con- fundit: illa autem altera Scholasticorum distin- ctio causas excommunicationis veras non ani- maduertit, quas in 2. Thessaloni.3. vno verbo tra- dit Paulus, dum ait, ὠτις ὀυχ ὐπακόει. Ergo et in peceato, et in crimine contumacia animi et ob- stinatio in perseuerando vel non poenitendo est ve- rissima excommunicarionis causa: vti resipiscen- tia et dolor peccati est causa absolutionis aut re- uocationis in Dei coetum et Ecclesiam. Confirmatur nostra sententia dicto Christi Matth. 18.vers.17. Si te non auditdic Ecclesiae, si non audit Ecclesiam, sit tibi Ethnicus. Ergo contumaciam, et pertinaciam animi damnat Christus, quia qui corrigi non vult etiam in re leui, per quam tamen offendiculum praebet Ecclesiae, sed in ea perseuerat, monitus saepe saepius, poterit tandem excommunicari si aedifica- tio Ecclesiae id requirat. Notorium autem esse delictum oportet, quia propter scandalum fit excommunicatio. Quod autem est ignotum, offem diculum praebere cuiquam non potest. Ergo ignotum peccatum, vel crimen quantumuis magnum, non est excommunicabil e-  \pend
\section*{AD I. PAVL. AD TIM. }
\marginpar{[ p.MB. ]}\pstart Quaeritur, Quid si ea res, propter quam quis reprehenditur, verbo Dei non est expresse dan- nata et definita, Vtrum perseuerans in ea excom- municari possit, velut propter vestitum nimis sumptuosum, nimias mensae lautitias. Respond. Cum rebellio sit excommunicationis causa, re- bellio autem esse non possit, si in nulla re Dei praeceptum violetur, certe quando illud quod facimus, et quomodo facimus, lege Dei non est vetitum, ex eo neque offendiculum, neque iusta reprehensio nascitur, neque etiam excommuni- catio fieri vlla debet vel potest. Sed quoniam in tota vita nostra ex Dei verbo, Dei et proximi ratio est habenda, siue in re ipsa quam facimus, siue in modo, quo quid gerimus, potest dupliciter anobis peccari, vt in eo peccato nostro obstinate et contumaciter perseuerantes possumus excom- municari, tamen temeraria Ecclesiae iudicia esse non debent. Quartum autem quaeritur, quomodo excommuni- catio fieri debeat. Primum vtrum Voce ipsa, an potius ex Scripto. Deinde qua verborum forma. Ac quidem videntur ex scripto excommunicati ii, de quibus agit in 2. Thess.3.Paulus, non autem viua voce. Ait enim, notate per Epistolam. Et for- tasse ad euitandam excommunicatorum inuidiam et odium hoc prodest. Sed contra vsus Ecclesiae obtinuit. Fieri enim vina voce solet: scripto ta- mem potest. Vtcunque ea postea debet aliis inno- tescere, quorum intererit scire. Et hoc sibi vult in 2. Thessalonis. Paulus. Sic autem, siue verbis si- ue scripto fiat, concipienda est, vt et persona ipsa excommunicanda: et factum, propter quod eiici-  \pend
\section*{CAPVT . V. }
\marginpar{[ p.397 ]}\pstart tur quis e Dei Ecclesia, populo per pastorem, vel Presbyteros significetur aperte, non autem oc- cultetur, ne sit aliis inutilis ista censura. Qua autem forma verborum debeat fieri, disputatum est. Et certe varia et diuersa verborum formula vsa est semper Ecclesia pro locorum, rerum, et factorum, de quibus agebatur, varia ratione. Eius rei exempla extant apud Synesium Episcopum epistola 58.et 66. et apud Euagrium historicum lib. 4 cap.38. Qui enim putant his Pauli verbis vtendum prae- cise, Tradimus Satanae ad interitum carnis, vti est 1.Corinth.5. vers.5.1. Timoth.1. vers.2o. sunt ni- mium in verbis religiosi. Deinde vim illorum verborum Tradimus Satanae ad interitum carnis, non recte capiunt et intelligunt. Ideo enim tradit Sa- tanae Paulus, vt caro, id est, vitiosa animi libido in iis intereat prorsus: non (vt interpretatur Au- gustinus lib. 3 cap.1o. contra Parmenia. et Greg. in Leuit. Homil.15)in carnis et corporis afflictio- nem. Nanque hoc falsum saepe apparet. Neque e- nim omnes excommunicati in carne et corpore suo hic viuentes cruciantur et vexantur a Diabo- lo. Vixit enim Iulianus apostata sanus corpore, et validis membris etiam excommunicatus. Sed ostendit Paulus finem excommunicationis, Ni- mirum, vt praua animi libido euellatur in iis, per quam Dei verbo sunt immorigeri: et spiritus, id est, vis et afflatus spiritus Dei sit in illis ipsis po- tens, et viuus. Ergo haec verborum forma libera est, modo vis excommunicationis et istius cen- surae exprimatur. Quinto loco quaeritur, quando fieri debeat ex- communicatio, an statim atque quis publice de  \pend
\section*{AD I. PAVL. AD TIM. }
\marginpar{[ p.330 ]}\pstart liquit:an demum quando emendari iam amplius non potest, vel quia non est viuus, vel quia est obstinatus et desperatus. Respond. Quanquam huiusmodi quaedam vitia saepe committuntur, vt statim videantur eüciendi, qui ea perpetrauêre, sintque tanquam Ecclesiae pestes, tamen si qui commisit, ex animo dolet, poenitet, et emenda- tur, non debet excommunicari. Ergo statim fieri non debet, sed postquam agnitus est is, qui deli- quit, contumax et obstinatus perdite. Iuberi qui- dem potest etiam poenitens sceleris, vt si ita sit ex aedificatione Ecclesiae Coena abstineat, sed ad tempus tantum, non in perpetuum. Praeterea il- lud sit explorandae huiusmodi hominis poeniten- tiae causâ, vt quantum fieri potest, sciatur, verâne sit, an falsa resipiscentia eius, qui se poenitere fa- cti sui testatur. Hoc ipsum docet Paulus saepe suo exemplo. Nam quos excommunicari iubet, ii sunt qui post plures admonitiones vitam tamen in meliorem frugem commutare spreuerant. Ergo ne porcis margaritae contra Christi praeceptum dentur, examinari et explorari eorum poeniten- tia debet. Quod autem quaeritur de mortuis hominibus et iam vita functis, vtrum excommunicari possint. Responsio est, eos posse excommunicari. Exen- pla sunt non illa vulgata Stephani 6. et Formosi primi et Sergii primi Pontificum Romanorum, qui suos prae decessores iam mortuos excommu- nicarunt et persecuti sunt, vti L Sylla C. Marium iam defunctum: sed illa potius, de quibus historiae extant apud Euagrium lib.  4. cap. 38. item apud Socratem Scholasticum lib. 7. cap.45. Et ita etiam  \pend
\section*{CAPVT  V. }
\marginpar{[ p.399 ]}\pstart scribunt et sentiunt Scholastici: atque huius o- mnis excommunicationis vis in mortuo ad illius nomen et doctrinam, non autem ad personam, quae iam nobiscum non versatur euitandam, per- tinet. Ter vero an quater prius debeat moneri ex- communicandus, definiri certo non potest. Ex rerum enim, et locorum circunstantia hoc totum diiudicandum est. Quanquam quidam leges Ro- man.imit ati et sequentes existimant, vt minimum bis monendos esse, qui excommunicari debebunt. Ter autem admoneri eos satis esse putant, quia trina denuntiatio facta ex iuris ciuilis regulis suf- ficit, et satis interpellat, vt quis contumax esse iu- dicetur. Afferunt Paulum Tit.3.v.1o. et Theophy- lact.in Matth.18. Sed tota haec quaestio est facti, et relinquenda prudentiae senatus Ecclesiastici. Sextum quaeritur, quis excommunicationis sit finis, quem spectare debeamus. Ac multi quidem spectari possunt, imprimis autem tres debent obseruari. Vnus est ipsius peccatoris sanitas, id est, excommunicatum, ipsiusque salutem respicit nempe vt hac ratione pudore afficiatur, atque ab errore reuocetur. Hunc finem annotat Paulus, cum ait, vt eum pudeat 2. Corinth.2. vers.7.7. vers.12. Pudore enim solent homines ad resipi- scentiam adduci. Hoc est quod aliis verbis dicit Paulus vti 1. Corinth.5. vers.5. ad interitum car- nis et vitiosae naturae, quae in nobis inest Neque tamen quod homines vereri magnus sit profe- ctus, si ipse per se spectetur: sed qu est optimus et maxime nostrae naturae consentaneus modus pec- catorum corrigendorum, pudor et verecundia  \pend
\section*{An I. TAVE. AD TIM. }
\marginpar{[ p.403 ]}\pstart peccati. Volunt enim omnes recte de se et ho- neste ab aliis sentiri et iudicari. Secundus finis, est ipsius Ecclesiae incolumitas id est, spectat ipsum Ecclesiae coetum, ne propter vnius priuati scelus male audiat tota Ecclesia, vi- deaturque, si talem retineat, potius sentina ma- lorum, et porcorum esse hara, quam Dei sacra- tissima ara et templum. Sic improbos et impios e Dei Ecclesia eiici vult et praecipit Paul. Ephes. 5.vers.3.4. Nihil enim immundum in Dei Eccle- sia tolerari debet, quantum quidem percipi et deprehendi potest. Tertius excommunicationis finis est aliorum fidelium vtilitas, vt hoc exemplo territi ab huius- modi peccatis et criminibus abstineant. Corrum- punt enim facile bonos mores colloquia et ex- empla aliorum praua. Ergo hac seueritate per- specta etiam mali in officio continentur, et ti- ment. Septimo autem et vltimo loco quaeritur, quae sit legitimae et iustae excommunicationis poena et effectum. Et certe, vti diuini cuiusdam et coele- stis fulminis, magnus est fragor et vis magna. Duo autem effecta solent enumerari excom- nicationis praecipua et potissima. Vnum, qui ratio- ne Dei:alterum qui Hominum, spectari debet. Ac quidem Dei ratione excommunicatio legitime et iuste facta excommunicatum separat a Chri- sto et vita aeterna. Dicit enim Christus quaecunque ligaueritis in terra Matth.18. erunt ligata in caelis ne huius teli vis imbecilla esse iudicetur: aut Dei verbi authoritas vilescat apud homines. Certe excommunicatio conscientiam proprie spectat,  \pend
\section*{CAPVT  VS }
\marginpar{[ p.401 ]}\pstart et animarum salutem, quam adimit obstinato: non est autem poena politica, vt quidam putant. Vti enim distincta est iurisdictio ciuilis, et Eccle- siastica:ita sunt diuersi vtriusque fines, poenae di- uersae, et effectus diuersi. Ciuilis enim iurisdictio in corpus aut bona haec terrena exercetur: Eccle- siastica in animam siue conscientiam, et in bona aeterna, quibus nos priuat. Itaque a pastoribus infligitur, non a ciuili magistratu. Hominum autem ratione priuat excommuni- catio excommunicatum non omnium quidem hominum: sed Fidelium tantum et Christiano- rum communione. In quo quaeritur, quanam con- munione Fidelium, vtrum omni prorsus com- munione cum Fidelibus: an certarum tantum rerum et bonorum cum iis participatione et vsu communi. Respondent Scholastici maiorem ex- communicationem priuare excommunieatum o- mni licita communione cum Fidelibus, siue in bonis spiritualibus, siue corporalibus, terrenis, et ciuili- bus. Nos in vniuersum statuimus, excommunica- tionem legitimam arcere per se excommunica- tum ab omnibus spiritualibus et communibus bonis, quae Dominus suae Ecclesiae, id est, omni- bus Fidelibus largitus est. Cum enim excommu- nicati extra Dei Ecclesiam eiecti sint, non sunt iis magis, quam canibus et porcis danda illa bona, quae Dominus suorum filiorum propria et pecu- liaria esse vult, illisque solis reseruata, iuxta dictum Christi. Non sunt margaritae obiiciendae porcis: itemque illud, Non est bonum dare panem filio- rum canibus Matrh.15. vers.26.7. vers.6. Aliis autem donis, quae non sunt generaliter omnibus veris et  \pend
\section*{402 AD I. PAVL. AD TIM. }\pstart piis Christianis data non priuat excommunica- tio excommunicatum, qualis est donum linguarum, sanationum, et alia huiusmodi. Vnde priuantur excommunicati vsu Sacramentorum, Coenae, et Baptismi, si non fuerint adhuc baptizati, quan- quam nondum baptizatos excommunicari mo- ris non fuit, quia nondumin Ecclesia esse huiusmo- di censentur.i. qui baptizati non sunt, etsi iam Ca- techumeni fuerunt, et tempus catecheseos im- pleuerunt. Augustinus lib.  11. de Genesi ad liter. cap.4o. Solent, ait, in hoc paradiso, id est, Eccle- sia a Sacramentis altaris visibilibus homines ex- communicati disciplina Ecclesiastica remoueri. De filio excommunicati quaeritur, vtrum ba- ptizari possit, si pater maneat obstinatus, neque satagat et curet Ecclesiae reconciliari. Respon. Etsi dubia et agitata fuit haec quaestio, in qua in diuersas alii ab aliis sententias iuerunt, ex Augu- stino tamen lib. 2. de Adulter. coniugiis et lib. con- tra Parmenia. satis intelligitur etiam esse bapti- zandos excommunicatorum filios, quia filius non debet portare iniquitatem patris: et in ipso ex- communicato baptismi character non est per ex- communicationem deletus, etsi vis eius nulla est in excommunicato, quandiu quidem in suo sce- lere manet obstinatus. Vide de hac quaest. Cal. epi. 136. et Bezae epi.1o. Sed quaesitum est, vtrum excommunicatio arceat excommunicatum ipso ingressu et aditu Ecclesiae, vt is ne in coetum qui- dem admitti debeat, vt audiat Dei verbum. Pu- tant Scholastici arcendum, adeo vt in regno in- terdicto iuste nolint sacral fieri et celebrari publice et assa voce. Irenaeus lib. 2. cap.4. scribit Cerdonem haereticum, postquam excommuni-  \pend
\section*{CAPVT  V. }\pstart catus fuit, ipso Ecclesiae ingressu prohibitum esse D. Caluinus tamen lib.  4. Institutio. cap.11. sect. 1o. docet diuersam fuisse in eo Ecclesiae discipli- nam et consuetudinem, neque semper prohibi- tos excommunicatos ingredi Ecclesiam. Quan- do enim Christianis palam conuenire non licuit per imperatorum edicta ad audiendum Dei ver- bum, tunc excommunicati, tanquam lupi, arce- bantur ab ingressu Ecclesiae, ne eam proderent hostibus. Postquam autem coeperunt publice et sine metu coadunari Christiani homines. i. qui Christum profitebantur, neque sibi metuerent a Magistratibus, tam seuera disciplina in excom- municatos non est vsurpata, vt arcerentur ab in- gressu Ecclesiae, et abipsa auditione verbi Dei, ne fores et spes omnis resipiscentiae illis praeclu- di videretur. Vis tamen ipsa verborum et excom- municationis hoc habet et notat, vt excommu- nicati aditu Ecclesiae arceantur, sed poenae haec pars illis ex quadam Ecclesiae indulgentia remit- titur Itaque excommunicati dice bantur ἀποσυνά- γωγον, id est, segreges, et sunt eorum loco, quos Iudaei pro Ethnicis habebant. Itaque aditu tem- pli arcebant, et is etiam vetus Ecclesiae mos fuit Tertul. in apologet. cap.39. canone Cum excom- municato 11. quaest.3. Sic Ambrosius Theodosium imperatorem a se excommunicatum ex regulis Ecclesiasticae disciplinae arcuit ingressu templi Mediolanensis. Quod ad terrena autem haec et temporalia, distinguendum est. Aut enim de Publica commu- nione, aut de Priuata et familiari nostra conuersa- tione cum excommunicatis quaeritur. In iis o-  \pend
\section*{AD I. PAVL. AD TIM. }
\marginpar{[ p.404 ]}\pstart nibus licet nobis communicare cum excommu- nicato, quae publica pax et ciuitatis politia et ra- tio cum eo agere postulat, veluti habet suffragium in creando Magistratu vna cum reliquis ciuibus. Itaque licet cum eo, tanquam cum ciue, agere et communicare: non autem tanquam cum fratre, vel Christiano. Manet enim excommunicatus e- tiam ciuis, quanquam frater proprie non manet. Nec obstat quod ait 2. Thess.3. vers.15. Paulus monete vt fratrem. Id enim quem sensum habeat, postea explicabitur. Vnde perperam Pontificii excommunicatis omnem sui iuris in iudicio ciui- li persequendi potestatem adimunt. Etsi enim potest excommunicatus conueniri et appellari de debito, et si qua nobis in eum actio competit: ipse tamen ex Pontificiorum iure et sententia debitores suos conuenire non potest, neque vllum ius suum iudicio et in foro ciuili persequi. De priuata autem nostra, qui sumus fideles et Christiani, cum excommunicato communicatione etiam quaeritur, quae esse debeat. In quo etiam i- pso est distinguendum. Nam cum eo communi- care volumus vel tanquam cum homine, vel cum ami- co, fratre et socio. Omnis quidem arctior cum eo societas, familiaritas, priuataque amicitia nostra dissoluenda, at que dissuenda est. Quae sunt autem ad vitae huius societatem tuendam necessaria, et quae humanitatis, cum eo fieri et haberi possunt. Vnde et ab eo emere possumus, item cum eo con- trahere, communem haereditatem habere, ex con- muni cum eo puteo et fonte aquam haurire, illi item possumus nostras merces vendere. Sed etiam si sit vxor, parens, cognatus, aut noster filius ex-  \pend
\section*{CAPVT  V. }
\marginpar{[ p.405 ]}\pstart communicatus, possumus et officia praestare quae illi debemus: et ab eo exigere, quae debet ipse. Denique quaecunque homo homini debet, ea non sunt deneganda. Quae autem amicus amico ἐκ πρναιρέσεως pollicetur, aut offert, ea nobis non sunt vel ab excommunicato quaerenda, vel cum eo contrahenda et ineunda. Veteres tribus ma- xime Symbolis istam cum excommunicato fami- liaritatem tam arctam describebant, quibus iube- bant pios abstinere, neque cum eo communia habere. Ea autem Symbola erant, lectus, mensa, domus, quae cum excommunicato communicare nos pro- hibent, quia ea fere amicitiae, et arctioris familia- ritatis signa sunt inter homines Synes.epist.66. et Plutarch. Symposia. lib.  8.quaest.7. ea putant esse sacra. Nam ea iis tantum a nobis communicanda, et offerenda esse, qui communibus nobiscum sacris vtuntur: non autem excommunicatis, non quod tamen vetent isti patres hominem excommu- nicatum egentem alere, errantem lecto excipere aegrotantem lecto accommodare: sed ita loque- bantur, ne vllam amicitiam cum iis contrahamus non autem vt iis vel communia humanitatis vel priuata officia denegemus. Eodem sensu est acci- piendum, quod est 2 Ioan. vers.1o. Ne dicamus iis tiur, non vt nos aduersus eos inciuiles et inhu- manos ostendamus: sed ne iis blandiri videamur et ita eos in errore foueamus. Ex quo apparet, quae sit significatio vocis eυναναμίγνυθς, qua vti tur Paulus et 1. Corinth.5.vers.9. et II. Quanquam enim plus est non habere commercium, vel non misceri, quam se subducere: vtrumque tamen ita est explicandum, vt ad priuatae amicitiae foe-  \pend
\section*{AD I. PAVL. AD TIM. }
\marginpar{[ p.400 ]}\pstart dera, iura, et commercia tantum pertineat. Id quod satis docet et interpretatur Paulus 1. Cori. 5.vers.11. voce σωεθίεη. Ratio est, quia hac voce et prohibitione Paulus non damnat publica com- mercia, quae cum iis necessario fiunt, quale est ab iis emere, conducere, stipulari, et reliqua huius- modi. Nec item vetat, si cuius mercis societas antea cum iis contracta fuerit, fidem iis datam seruare, et in eo contractu manere: sed vetat fa- miliares nos iis esse, ne eos in peccatis obstina- tiores reddamus. Vnde quaesitum est, vtrum quae ex Dei praece- to priuatim quisque alteri, pro vocationis vario genere debet, excommunicatis praestare teneamur: veluti vtrum subditi principi suo excommunicato parere, teneantur, vxor marito, filius patri, seruus domino. Papistae non putant eis haec officia prae- standa esse, sed bona conscientia denegari et pos- se et debere obsequium huiusmodi excommuni- catis volunt, quasi soluta sint vincula subiectio- nis et iurisiurandi propter contumaciam princi- pis in suum ipsius principem, et eum quidem sum- mum, id est, Deum, vt quemadmodum ille non paret Deo:ita neque subditi ipsi principi suo con- tumaci in Deum et excommunicato parere te- neantur. Falsa est haec sententia, et omnino impia et blasphema. Primum enim per excommunica- tionem non tolluntur iura naturae et sanguinis, veluti cognationum et affinitatum : ergo filius manet filius patris etiam excommunicati, vxor manet vxor: cognati manent illius cognati. Deinde non tolluntur iura et officia humani- tatis per eandem poenam.Itaque excommunica-  \pend
\section*{CAPVT  V. }
\marginpar{[ p.40 ]}\pstart tis etiam omnia humanitatis officia sunt praestan- da.Ad quod affero quod est Romanor. 12.V.20. Tertio non tollit excommunicatio iura poli- tica et ciuilis societatis, quemadmodum nec ipsa infidelitas vel apostasia aperta a Deo tollit ea. Nam infidelis et apostata imperator, qualis Iu- lianus olim, manet Imperator, et Magistracus, et pro tali a Christianis est agnitus. Ergo falsa est Scholasticorum sententia, qui sentiunt non esse parendum principibus excommunicatis: et con- tra eos ipsos scripsisse videtur Thomas Aquinas ea, quae extant in 2.22 quaest. 1o. Sane Sigibertus author probatus et Monachus scribit in Odone fol.1o1. hanc doctrinam esse et nouam et haereti- cam, quae iubet ne hypocritis principibus, id est, excommunicatis pareamus, quam Hildebrandus qui et Gregor.7. Pontifex Roman. primus pro- mulgasse videtur, cum veteres Christiani Iulia- no apostatae paruissent. Ambrosius Theodosio a se excommunicato: et ipsi apostoli etiam omni- no infidelibus et persecutoribus Ecclesiae princi- pibus, quales Nero et Domitianus, paruerunt. Denique quaesitum est, vtrum bonorum con- fiscatio in excommunicatum sanciri debeat et possit, vt etiam ea poena multetur. Huius rei exen- plum extat Esdrae. 1o.ver.8. Respond. Minime. Nam est excommunicatio conscientiae poena:non autem corporis vel bonorum. Et quod illic de Esdra di- citur nihil mouet. Erat enim Magistratus Esdras non solum Leuitarum, et coetus Ecclesiastici prae- ses. Ergo vtriusque iurisdictionis, id est, ciuilis et Ecclesiasticae authoritate illud edictum sancitum est, quo celerius Iudaei dispersi conuenirent.  \pend
\section*{AD I. PAVL. AD TIM. }
\marginpar{[ p.Reo. ]}\pstart Haec autem omnia locum demum habent postquam publice quis ex Ecclesia eiectus est: non autem prius quam publice sit excommunicatus. Et quod affertur ex August. lib.  de Vnitate Ec- clesiae cap. 20. etiam antequam quis visibiliter excommunicatus sit, praecisum esse ab Ecclesia, pertinet ad mentis et conscientiae terrorem in- cutiendum et piam commonefactionem haereti- corum hominum: non autem vt eos pari loco ha- beamus, atque qui iam palam et totius Ecclesiae suffragio recte sunt excommunicati. Videtur tamen omnibus, quae hactenus dicta sunt, de effectis excomunicationis obstare quod ait Paulus, etiam excommunicatum, vt fratrem esse agnoscendum. Cui ipsi repugnat quod ait Christus Matth. 18. excommunicatum habendum esse pro Ethnico et infideli. Ergo non pro fratre. Hae sententiae in speciem pugnantes ita primum sunt inter se conciliandae, vt quod ait Christus, non interpretemur ἀπῶς et κτι πάντα. Nec enim omnino par aut aequalis debet iudicari excom- municati et infidelis conditio, sed in quibusdam tantum. Est enim dissimilis vtriusque status in eo quod excommunicatus Baptizatus est: itaque Christianus et Prope est. At infidelis et Ethnicus Baptizatus non est, et Procul est. Sed in eo est si- millima vtriusque conditio, quod legitime et vere excommunicatus. 1 vti et Ethnicus in Ecclesia non est, nec de Ecclesia 2 Quod nulla priuata et familiaris nostra cum vtroque conuer- satio esse debet. 3 Quod si vterque talis esse perseueret, regno Dei vitaque aeterna excludi- tur.  \pend
\section{CAPVT  VI. }
\marginpar{[ p.409 ]}\pstart \phantomsection
\addcontentsline{toc}{subsection}{\textit{Q VICVNQVE sub iugo sunt serui, suos dominos omni honore dignos ducunto, ne nomen Dei et doctrina blasphemetur.}}
\subsection*{\textit{Q VICVNQVE sub iugo sunt serui, suos dominos omni honore dignos ducunto, ne nomen Dei et doctrina blasphemetur.}}Quod autem ad dictum Pauli, sane et ciuili- ter intelligendum est. Ex charitate enim dictum est:non autem ex vera, exacta, et propria fratris, id est, Christiani hominis definitione. Sed quan- do illud verum est, Nemo desperet meliora lapsis et potest resipiscere qui hodie desipit. Praeterea quos excommunicat Ecclesia, non odio persona- rum ipsarum, sed eorum tantum sceleris et con- tumaciae ratione eiicit, fit, vt qui excommunica- ti sunt, quatenus et homines sunt, et aliquando nobiscum in Dei Eccesia versati sunt, eorum de- syderio, amore, zelo, affectuque salutis ipsorm, moueri debeamus, neque sint prorsus a nobis abiiciendi, sed vt fratres agnoscendi, admonendi atque etiam, vt in viam reducantur, vti oues per- ditae, sedulo a nobis conquirendi. Ergo etiam ex- communicati χτ τι fratres nostri sunt, non tamen omnino et absolute, nec pares iis habendi, qui in Dei metu perstiterunt. Ex quibus omnibus expli- catus videtur vberrimus et vtilissimus ille de ex- communicatione locus. Sequitur, CAP. VI. 
\textbf{Q}VICVNQVE sub iugo sunt serui, suos dominos omni honore dignos ducunto, ne nomen Dei et doctrina blasphemetur. Videri potest abrupta huius capitis et loci se- ries, et minime consentiens cum toto superio-  \pend
\endnumbering
\end{pages}
\end{document}
        