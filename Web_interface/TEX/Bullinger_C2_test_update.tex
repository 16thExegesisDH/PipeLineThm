% !TeX TS-program = lualatex
\documentclass{article}
\usepackage[T1]{fontenc}
\usepackage{microtype}
\usepackage[pdfusetitle,hidelinks]{hyperref}

\usepackage{polyglossia}
\setmainlanguage{english}
\setotherlanguages{latin,greek}
\usepackage[series={},nocritical,noend,noeledsec,nofamiliar,noledgroup]{reledmac}
\usepackage{reledpar}

\usepackage{fontspec}
\setmainfont{TeX Gyre Termes}

\usepackage{sectsty}
\usepackage{xcolor}

\usepackage{fancyhdr}
\pagestyle{fancy}
\fancyhf{}
\fancyhead[LE,RO]{\nouppercase{\leftmark}}  
\cfoot{\thepage}
\renewcommand{\headrulewidth}{0.4pt}

% Redefine \section to remove numbering
\usepackage{titlesec}
\titleformat{\section}[hang]{\normalfont\scshape\color{gray}}{}{0pt}{} % no number in heading
\titleformat{\subsection}[hang]{\normalfont}{}{0pt}{} % also remove subsection number
\titleformat{\subsubsection}[hang]{\normalfont\footnotesize\color{black}}{}{0pt}{}

% Modify how section marks are stored to exclude numbers
\makeatletter
\renewcommand{\sectionmark}[1]{%
	\markboth{#1}{}} % Only store the section title, without number
\renewcommand{\subsectionmark}[1]{%
	\markright{#1}} % Only store the subsection title, without number
\renewcommand{\numberline}[1]{} % Hide the section number in TOC
\makeatother

\begin{document}

\date{}
        \title{Commentarii In Epistolas D. Pavli Ad Timotheum : [Bullinger, Heinrich], [1536]}
\maketitle
\tableofcontents
\clearpage
\begin{pages} 
\beginnumbering
\section{CAP. II.}
\section*{AD TIM. CAP II. }
\marginpar{[ p.107 ]}\pstart rum pelago est ueritas canonica fides pura et charitas syncera. In his qui perstant et pura simplicitate ad portum tendunt salutis, salui portum attingunt foe licitatis aeeternae. Porro qui flatibus arrogantiae et con tentionis uela dant abreptique malarum cupiditatum fluctibus a uia recta recedunt, et fidei lucidam cynosuram negligunt, impingunt in perfidiae scopulos atque intereunt. Quid uero sit sathanae tradere expositum est 1. Cor.5. cap. Finis huius traditionis est ut pudore probroque correcti, discant ab impiis parumque Christianis dogmatis temperare. Hactenus tractalauimus nego tium de sana doctrina tradenda et uitandis fabulis, di uersis impiisque dogmatis atque opinionibus.  \pend 
\phantomsection
\addcontentsline{toc}{subsection}{\textit{INSTRVITVR COETVS ECclesiasticus. }}
\subsection*{\textit{INSTRVITVR COETVS ECclesiasticus. }} 
\phantomsection
\addcontentsline{toc}{subsection}{\textit{Adhortor igitur ut ante omnia fiant deprecationes, obsecrationes, interpellationes, gratiarum actiones pro omnibus hominibus, pro regibus, et omnibus in eminentia constitutis, ut placidam ac quietam uitam degamus cum omni pietate et honestate. CAP.II. }}
\subsection*{\textit{Adhortor igitur ut ante omnia fiant deprecationes, obsecrationes, interpellationes, gratiarum actiones pro omnibus hominibus, pro regibus, et omnibus in eminentia constitutis, ut placidam ac quietam uitam degamus cum omni pietate et honestate. CAP.II. }}
\section*{COMMENT. IN I. EPIST. }\pstart Proximum sanae doctrinae quod ecclesia Christi ha bet preces sunt sacrae, sub quas refero Coenam gratulatoriam sacrificium scilicet laudis et gratiarum acti onis. Et in Actis cap.2.legimus de primitiua ecclesia, Erant autem perseuerantes in doctrina apostolorum et communicatione et fractione panis et precationibus. Baptismi autem paulo ante haec uerba meminerat Lucas. Proinde satis fuerit ecclesiae sanctae si habeat doctrinam sanam id est euangelicam, praedicantem in nomine Christi poenitentiam et peccatorum remis sionem, item prophetiam id est lectionem ac expositi onem scripturarum piam, preces denique et coenam mysticam, quibus addimus eleemosynam atque Baptismum. De quibus diximus et in 2. et 20.cap. Act. aposto. et in 1. ad Corin.11.cap. Certe apud uetustissimas ecclesias in coetibus sacris plura istis non habuere Christiani ad iustam institutionem ordinationemque ecclesiasticam necessaria. Temporum successione et pastorum negligentia factum est ut istorum uestigia uix appareant, de doctrina prophetia precibus sacris et coena mystica loquor. Nam haec corrupit auaritia superstitio et innouandi studium. Inde enim irrepsere ab usus missarum, sacrificia pro mortuis, murmur et cantus templorum. Olim in coetibus sacris paucissimae conspiciebantur Ceremontae, et quae erant uel ad doctrinam uel ad preces pertinebant. Omnia uero uul\pend
\subsubsection*{Coetus ecclesiastici ueterum qua les fuerint. }
\section*{AD TIM. CAP II. }
\marginpar{[ p.108 ]}\pstart gari lingua fiebant et recitabantur ad utilitatem om nium. Nihil hic erat uenale. Nullus uel emptor uel uenditor negotiabatur in aede sacra. Nullum erat discrimen inter plebem et sacerdotem, nisi quod hic pro ministerio suo et inflitutione sacris praeerat. Et ne quid hac in re dissimulem coetus ecclesiaflici ante annos mille fuerunt huiusmodi. Confluebat in aedem sacram cultus diuini gratia populus. Dum autem ingrederetur aedem, ab iis qui iam conuenerant Psalmi aliquot in aliis quidem ecclesiis canebantur in aliis recitabantur tantum, donec uniuersus coetus plene coisset. Hoc uero sacrorum initium ab introeundo appellarunt introitum. Iam ubi conuenisset ecclesia acclamabatur ab omnibus unanimi concentu κύριε ἐλέησον kyrie eleison, Domine miserere. Addebatur a quibusdam Hymnus qui dicitur angelicus, cuius initium est Gloria in excelsis Deo. Hic est gratulatorius et deprecatorius. Hoc finito collectam recitabat minister aliꝗs ecclesiae. Ea erat orationis species qua totius ecclesiae uota et necessitates collectae Deo recitabantur. Deinde praelegebatur a Diaconis eruditioribus locus aliquis siue ex libris propheticis siue literis apostolicis pro tempo ris loci et populi ratione delectus. His absolutis in aeditiorem locum per gradus ascendebat episcopus euangelium Christi praedicaturus. Plebs interim carmine concepto sancti spiritus gratiam implorabat, quod a  \pend
\section*{COMMENT. IN I. EPIST. }\pstart gradibus Graduale appellari coepit. Hic uero magna cum authoritate praelegebat euangelium episcopus, deinde uero interpretabatur. Ad finem contionis sacrae recitabat symbolum quod apostolorum uocant uel Nicenum uel Secundo conditum. Praeterea totam ecclesiam inuitabat ad misericordiam, ut quilibet pro suis facultatibus aliquid in aerarium conferret pauperum: eam sacri portionem nuncupabant Offertorium. Et Pontius Paulinus docet mensam in ecclesia poni solitam pro pauperibus cibo reficiendis quam et domini mensam uocat, et a domino positam. Hac inquam ratione hisce ceremoniis et ritibus exercebant preces et doctrinam sanam ueteres. Attamen hic ritus non per omnia omnibus erat communis. Quidam enim sacrum coetum non a Psalmis auspicabantur, sed ab acclamatione κύριε ελέησον. Idem non habebant inusu uel Hymnum angelicum, uel collectas, uel carmen graduale. Huic rursus alii succinebant Hebraeorum Alleluiah ut alibi indicat Hieronymus. Apud quosdam episcopus ipse sine istis omnibus sacra contione et adiectis publicis precibus conuentum et in cipiebat et dimittebat. Nulla uero ecclesia cogebatur in alterius ritus et inftituta iurare, dummodo pre ces et sacrae contiones saluae essent, sancteque exercerentur. Et optimae perpetuo quam paucissimis conten tae potissimas impenderunt doctrinae et precibus. Por  \pend
\section*{AD TIM. CAP II. }
\marginpar{[ p.109 ]}\pstart ro in Coena mystica hic ritus apud omnes penee ecclesias seruabatur. Prodibat in coetum sacerdos. Instructa pane et uino stabat in prospectu populi mensa mystica, huic iste assistens populo benedicebat dicens, Dominus uobiscum. Is uero respondebat, Et cum spiritu tuo. Deinde hunc ad res maximas excitans mentesqué singulorum praeparans clamabat, Sursum corda. Respondebat plebs, Habemus ad dominum. Cyprianus enim in sermone de oratione Dominica, Ideo et sacerdos inquit ante Orationem praefatione praemissa parat fratrum mentes dicendo, Sursum corda, dum respon det plebs, Habemus ad dominum, admoneatur nihil aliud se quam dominum cogitare debere. Haec Cyprianus. Hinc uero ad gratiarum actionem inuitans sacerdos ait, Gratias agamus domino deo nostro. Responde bat plebs, Dignum et iustum est. Hic subiiciebat sacerdos, Vere dignum et iustum est aequum et salutare, nos tibi semper et ubique gratias agere domine sancte pater omnipotens aeterne Deus per Christum dominum nostrum. Horum enim pene omnium meminit et Augustinus liber  de Bono perseuerantiae cap.13, dicens, Quod ergo in sacramentis fidelium dicitur, ut sursum corda habeamus ad dominum, munus est domini, de quo munere ipsi domino deo nostro gratias agere a sacerdote post hanc uocem quibus hoc dicitur, admonentur, et dignum et iustum esse respondent etc  \pend
\subsubsection*{Coena mystica ueterum. }
\section*{COMMENT. IN I. EPIST. }\pstart Post ista autem uerba subiiciebat sacerdos, Qui pridie quam pateretur accepit panem, gratias egit, fregit, de ditque discipulis suis, dicens, Accipite edite, Hoc est corpus meum quod pro uobis datur. Et reliqua quae leguntur in euangelio. His magna cum religione peractis, dicebatur oratio Dominica. Id quod Hieronymus quo que testatur liber  aduersus Pelagianos3. Sed et uulgo receptum est Apostolos tantum ad orationis dominicae preces consecrasse, hoc est celebrasse coenam mysticam. Post orationem dominicam sacra mysteria per cipiebat populus, communioneque sacramentorum cor poris et sanguinis dominici in unum corpus coalescebat ecclesiasticum cuius caput Christus est. Caeterum rite ad hunc modum peractis omnibus dimittebatur coetus. Ex his autem omnibus facile colligit Lector non omnino stupidus unde abusus Missae papisticae fluxerit, et quam nihil hodie antiquae pietatis in hac nisi profunda obsoletaque prope nominum uestigia cernere liceat. Canonis cuius initium est, Te igitur clementissi me pater, hactenus nullam feci mentionem eo quod certum sit hunc sero tandem receptum esse a quibusdam ecclesiis. Et proculdubio melius hodie haberet coetus ecclesiasticus si nunquam fuisset receptus. Verum non est institutum nostrum de his in praesentiarum agere prolixius, cum quaedam et in episto. ad Hebr. annotarim. Haec in hunc finem hunc in locum annotare libuit  \pend
\section*{AD TIM. CAP II. }
\marginpar{[ p.110 ]}\pstart ut intelligeret Lector quales fuerint olim coetus ecclesiastici, quodque piae antiquitatis pro se nihil habeant, qui hodie abusum Missarum omnibus ecclesiis ut ritum apostolicum obtrudere uolunt, adeoque et contendunt haereticos esse quotquot Missas suas uulgo receptas non adorant. Imo adiicio his quod illis nouum fore nil ambigo, Gregorium uidelicet eius nominis primum antistitem Romanum annis ab hinc ni fallor noningentis, cui et Missas maxime debemus, nullam tamen ecclesiam ad ritum coëgisse uel suum uel praescriptum alias. Id quod inde liquet quod Augustino episcopo Cantuariensi scribenti percontantiqué, Cum una sit si des cur sunt ecclesiarum diuersae consuetudines et aliter consuetudo Missarum in sancta Romana ecclesia atque aliter in Galliarum ecclesiis tenetur? Gregorius in haec uerba respondet, Nouit fraternitas tua Romanae ecclesiae consuetudinem in qua se meminit nutritum quam ualde amabilem te habeat. Sed mihi placet ut siue in Romana, siue in Galliarum, siue in qualibet ecclesia aliquid inuenisti quod plus omnipotenti Deo possit placere sollicite eligas in Anglorum ecclesia quae adhuc ad fidem noua est institutione prae cipua quae de multis ecclesiis colligere potuisti infundas. Non enim pro locis res sed pro bonis rebus loca amanda sunt. Ex singulis ergo quibusque ecclesiis quae pia quae religiosa quae recta sunt elige et hac qusi  \pend
\subsubsection*{Olim nemo ad praescriptum Missarum ritum seruandum coactus. }
\section*{COMMENT. IN I. EPIST. }\pstart in fasciculo collecta apud Anglorum mentes in consuetudinem depone. Haec Gregorius. Ea non profero quod praecipue iis innitar sed ut collatione istorum et nostrorum temporum euincam quam iniquis conditio nibus hodie nobiscum agatur, et quod contra uetus exemplum ritus nobis obtruditur qui non modo nihil utilitatis et pietatis habet sed impietatis et superstitionis plurimum. Neglecta igitur improbitate et con uitiis aduersariorum teneamus obsecro ea in coetu sanctorum sine quibus sanctimonia non perficitur, nenpe doctrinam sanam, et preces puras assiduasqué, de doctrina sana disputatum hactenus, de precibus quoq sacris cum aliâs saepe dictum sit praesentia Pauli uerba perstringemus modo. Primum et ante omnia iubet fieri preces, hoc ipso ostendens precibus sacris nil per inde necessarium et utile esse. Praeterea tempus designat commodum, ut scilicet ipso statim diluculo conueniant et communes ad Deum profundant preces. Plinius certe in epistola quadam ad Traianum Imperatorem quae extat liber  episto. Plinii 10. meminit moris fuisse Christianis ut stato die ante lucem conuenirent carmenque Christo deo dicerent secum inuicem, seque sacramento aliquo non in scelus aliquod obstringerent sed ne furta latrocinia et adulteria committerent. Et Tertullianus conuentus Christianorum non semel ante lucanos coetus appellat. Quanquam hic nullam statua\pend
\subsubsection*{Antelucani coetus. }
\section*{AD TIM. CAP II. }
\marginpar{[ p.111 ]}\pstart mus necessitatem neque preces certo tempori affigamus. Liberum enim est Christianis et locum et tempus designare precibus maxime idoneum. Certe matutinum tempus nemo improbare potest, ut scilicet ieiuni ac so brii simul cum luce Deo nostras preces offeramus et uerbo eius eruditi protinus ad operas et negotiationes nostras digrediamur. Deinde enumerat Paulus certas orationis species, sub his nimirum omnes intelligens. Erasmus ex Graecanicis scholiis tria priora sic discriminat ut Paulus δέησιν uocarit qua rogamus ut liberemur a malis et ideo Ambrosium eleganter uertisse deprecationem, προσευχήν autem esse qua precamur ut nobis contingant bona, ἔντευξιν uero cum querimur de his qui nos laedunt. Et multo uenustius haec omnia Paraphrasi explicans, Primum deprecentur, inquit, Deum ut auertat omnia quae turbant et inquietant statum religionis ac reip. Deinde postulent ab eo quae faciunt ad negotium pietatis et tranquilli tatem reipublicae. Sub haec aduersus eos qui persequuntur gregem Christi, nihil aliud quam praesidium eius implorent. Postremo pro his quae Dei beneficio conti gerunt agantur gratiae. Habes in his quatuor speciebus orationis opinor reliquas omnes. Habes item quid et pro quibus orandum sit, quanquam ea ipsa consequentibus copiosius exprimantur, nempe pro omnibus hominibus. Nam Christianorum est imitari exemplum  \pend
\subsubsection*{Orationie species. }
\subsubsection*{Pro quibus et quid orandum sit, }
\section*{COMMENT. IN I. EPIST. }\pstart Dei patris et Iesu Christi filii sui. Ille autem sinit solem suum oriri super bonos et malos. Hic uero etiam pro persequutoribus orauit. Gentiles in sacris suis de uouent Christo addictos. At Christiani pro uniuersis hominibus etiam inimicis et peccatoribus orare debent. Variae item sunt necessitates hominum, sunt temtationes multae, infirmitates quoque cum animi tum corporis crebrae quibus affliguntur tum boni cum mali. Pro his omnibus orandus est dominus. Addit, Pro regibus et omnibus ἐμ ὑπεροχῆ id est in sublimitate et dignitate constitutis: qui uidelicet in republica proceres sunt dignitateque et functione aliqua ciuili eminent. Loquitur autem et de regibus ethihcis et de magistratu a religione Christiana alieno. Ita enim et Ieremias ad eos qui apud idololatras in Babylone te nebantur captiui scribens, unde haec Pauli desumpta creduntur, praecipit, Et quaerite pacem ciuitatis in quam migrare feci uos illuc et orate pro ea ad dominum quia in pace eius erit pax uestra. Et Tertullianus ad Scapulam, Sacrificamus inquit pro salute Imperatoris, sed Deo nostro et ipsius, sed quomodo praecepit Deus pura prece. Non enim eget Deus conditor uniuersitatis odoris aut sanguinis alicuius. Haec enim daemoniorum pabula sunt. Et in Apologetico cap.39. Coimus, inquit, in coetum et aggregationem ut ad De um quasi manu facta precationibus ambiamus oran\pend
\section*{AD TIM. CAP II. }
\marginpar{[ p.112 ]}\pstart tes. Haec uis deo grata est. Oramus etiam pro imperatoribus pro ministris eorum ac potestatibus, pro statu saeculi, pro rerum quiete, pro mora finis. Item3o.ca pite eiusdem libri, Oramus semper pro omnibus imperatoribus, uitam illis prolixam, imperium securum, domum tutam, exercitus fortes, senatum fidelem, populum probum, orbem quietum, et quaecunque hominis et Caesaris uota sunt. Sed et ipse Paulus quid oremus significautissime adiiciens ait, Vt placidam siue tranquillam ac quietam uitam degamus: non equidem in uoluptatibus et impuritate sed in omni pietate siue uero dei cultu genuinaque religione in iustitia καί σεμνότητι et honestate. Nam σεμνότοῦς honestatem significat, uerecundiam, sanctimoniam, et grauitatem, item morum praecipue seueritatem in adolescentia probe instituta. Opponuntur haec seditionibus turbis atque bellis quibus labefactantur omnia honesta studia, corrumpuntur mores, perit honestas, soluitur fides et pudicitia. Potissimum itaque pro tranquillitate publica orabit ecclesia, aut si omnino bella pro defensione legum patriae et iustitiae gerenda suscipiuntur a magistratibus oret ecclesia dominum ut uictoria suis concessa iustam pacem oppressa omni iniustitia restituat. Harum rerum exempla innumera inuenias in Psalmis quibus omnium orantium uota plenissime comprehenduntur. Habes pro quibus et quid omnino orandum sit.  \pend
\section*{COMMENT. IN I. EPIST. } 
\phantomsection
\addcontentsline{toc}{subsection}{\textit{Nam hoc bonum est et acceptum coram seruatore nostro Deo, qui cunctos homines uult saluos fieri et ad agnitionem ueritatis ue nire. }}
\subsection*{\textit{Nam hoc bonum est et acceptum coram seruatore nostro Deo, qui cunctos homines uult saluos fieri et ad agnitionem ueritatis ue nire. }}\pstart Quia uero preces pro incredulis et ethnicis fusae uideri poterant inutiles contraque decorum et uoluntatem dei fieri, ideo iam superiorum ueluti causam reddens addit, Nam hoc est bonum et acceptum coram seruatore nostro Deo, qui cunctos homines uult saluos fieri et ad agnitionem ueritatis uenire. Hinc enim sequitur quod nostrum sit pro omnibus orare, nemini ueritatis doctrinam denegare. Qui enim sumus ut resistamus uoluntati dei? Vult ille omnes homines saluos fieri, neminem excludit a salute. Vult omnes ad agnitionem ueritatis uenire. Inde scriptum est in prophetis, In omnem terram exiuit sonus eorum et in fines or bis terrae uerba eorum. Cui respondet illud domini praeceptum, Ite in orbem uniuersum et praedicate euangelium omni creaturae. Diuersa sunt cultoribus dei designata tempora. Non enim uniuersi prima statim hora ueniunt, quidam sero tandem uineam domini ingrediuntur et eundem cum primis denarium recipiunt. Ne ergo desperemus aut iudicemus de infidelibus temere. Solet hoc loco necti quaestio spinosior  \pend
\subsubsection*{Quaestio praedestina tionis et liberi arbitrii. }
\section*{AD TIM. CAP II. }
\marginpar{[ p.113 ]}\pstart quidem quam utilior, Cum deus uelit omnes homines saluos fieri, qui fiat ut non omnes saluentur? Nec ignoro quid hic a plerisque responsum et quam odiose concertatum sit maximo etiam cum dispendio et scandalo piarum mentium. Alii enim gratiam illam dei qua sola beamur prope ad blasphemiamusque defenderunt, reiicientes in Deum omnium peccatorum culpam. Alii uero istud offendiculum uitare uolentes huc prolapsi sunt ut et ipsi non citra blasphemiam dei tribuerint homini quae dei sunt. Vt enim illi nihil relinquunt homini ne peccata quidem: ita hi nihil non tribuunt uiribus hominis. Vtrinque peccatum est et utrisque usu uenit quod prouerbio dicitur, Incidit in Scyllam cu piens uitare Charybdim. Multo prudentius et religi osius uidentur mihi hanc quaestionem expediisse ueteres, qui de ipsa aduersum Pelagianos disputarunt copiosissime, atque ita ut quaecunque hominis sunt relique rint homini, quae uero dei sunt Deo. D. Ambrosius siue alius quispiam uir sanctus et doctus de hac re aedidit libros duos inscriptos De uocatione omnium Gentium, in iis tandem negotium totum in hoc contraxit compendium, remotis abdicatisque omnibus concertationibus quas intemperantium disputationum gignit animositas tria esse perspicuum est quibus in hac quae flione debeat inhaereri. Vnum quod profitendum est, Deum uelle omnes homines saluos fieri et in agnitio  \pend
\section*{COMMENT. IN I. EPIST. }\pstart nem ueritatis uenire. Alterum quod dubitandum non est ad ipsam cognitionem ueritatis et perceptionem salutis non suis quenquam meritis sed ope atqueopera diuinae gratiae peruenire. Tertium quod confitendum est Altitudinem iudiciorum dei humanae intelligentiae penetrabilem esse non posse, et cur non omnes saluet qui omnes uult saluos fieri non oportere disquiri. Quo niam si quod cognosci non potest non quaeratur inter primam et secundam definitionem non remanebit cau sa certaminis, sed secura ac tranquilla fide utrumque prae dicabitur utrumque credetur. Deus quippe apud quem non est iniquitas et cuius uniuersae uiae misericordia et ueritas omnium hominum bonus conditor iustus est ordinator, neminem indebite damnans neminem in debite liberans, nostra plectens cum punit noxios, sua tribuens cum facit iustos ut obstruatur os loquentium iniqua et iustificetur Deus in sermonibus suis et uincat cum iudicatur. Nec damnati iusta quaere monia nec iustificati uerax est arrogantia, si uel ille dicat non meruisse se poenam uel iste asserat meruisse se gratiam. Congruunt his Ambrosianis Augustiniana quae leguntur liber  de Bono perseuerantiae cap.19. Hic tamen non negarim rationem commodiorem inueniri posse qua hic nodus dissoluatur. Neque hoc dissimulat Aurel. August. cuius uerba quod mire huc faciant ex 53. Tra cta. in Ioannem nunc ascribo. Si quis istam quaesionem  \pend
\section*{AD TIM. CAP II. }
\marginpar{[ p.114 ]}\pstart inquit liquidius et melius nouit se posse et confidit exponere, absit ut non sim paratior discere quam docere, tantum ne audeat quisquam liberum arbitrium sic defendere ut nobis orationem qua dicimus, Ne nos inferas in tentationem, conetur auferre: rursus ne quis neget uoluntatis arbitrium et audeat excusare peccatum. Sed audiamus dominum et praecipientem et opitulantem et iubentem quid facere debeamus, et adiuuantem ut implere possimus. Nam et quosdam nimia uoluntatis suae fidutia extulit in superbiam, et quos dam nimia uoluntatis suae diffidentia deiecit in negligentiam. Illi dicunt, Vt quid rogamus Deum ne uinca mur tentatione quod in nostra est potestate? isti dicunt, Vt quid conamur bene uiuere quod in dei est po testate? O domine ô pater qui es in coelis ne nos inferas in quamlibet istarum tentationum, sed libera nos a malo. Audiamus dominum dicentem, Rogaui prote Petre ne deficiat fides tua. Nec sic existimemus fidem nostram esse in libero arbitrio ut diuino non egeat ad iutorio. Audiamus et euangelistam dicentem, Dedit eis potestatem filios dei fieri ne omnino existimemus in nostra potestate non esse quod credimus uerum in utroque illius beneficia cognoscamus. Nam et agendae sunt gratiae quia data est potestas et orandum ne succumbat infirmitas. Ipsa est fides quae per dilectionem operatur sicut eius mensuram dominus cuique partitus  \pend
\section*{COMMENT. IN I. EPIST. }\pstart est, ut qui gloriatur non in seipso sed in domino glorietur. Hactenus Augustini uerba recensui. Ex quibus omnibus liquet quod bonam illam dei uoluntatem attinet ipsum quidem uelle cunctos homines saluos fieri et ad agnitionem ueritatis uenire, et proinde quod alii ueniunt non esse humani meriti sed diuinae uoluntatis et gratiae, quod uero alii non ueniunt humanae malitie et corruptionis. Salus igitur et iustificatio non erit neque uolentis neque currentis sed miserentis dei, credentque quotquot ordinati sunt ad uitam aeternam. Sunt autem ad aeternamuitam ordinati quicunque in Christum credunt. Deus enim omnia agit lege certa et ordine iusto. Elegit nos antequam iacerentur fundamenta mundi, sed in Christo et ad sanctimoniam. ltaque qui Christo fidunt et student sanctimoniae electi et ad uitam ordinati sunt. Electio enim dei a uoluntate non est separanda neque sola praeter ordinem et contra media est iactanda. Petrus enim electis scribit dei secundum praefinitionem dei patris per sanctificationem spiritus in obedientiam et aspersionem sanguinis Iesu Christi. Et Paulus, Quos inquit praesciuit Deus eos et praefiniuit ut conformes fiant imagini filii eius, etc. lam cum constet non omnes conformes fieri imagini filii eius, colligunt alii non omnes ad uitam esse praescitos, proinde tropo hunc Pauli locum exponunt Deum uelle omnes credentes saluos facere. Quod et  \pend
\section*{AD TIM. CAP II. }
\marginpar{[ p.115 ]}\pstart apud Augustinum contra Pelagianos inuenies, et plane uerum est. Impii enim non saluantur sed pii tantum. Caeterum illi non habent quod Deo qui pro bonitate sua uoluit eos ad agnitionem ueritatis uenire, suae dannationis et impietatis culpam impingant. Non enim sequitur praescientiam dei nos ad res impias affigereut culpam peccati in Deum reiicere possimus. Recte enim D. Augustinus Tracta. in Ioan.53. Non propterea quenquam deus ad peccandum cogit quia futura hominum peccata iam nouit. Ipsorum enim praesciuit peceata, non sua non cuiusquam alterius sed ipsorum. Fecerunt ergo peccatum Iudaei, quod eos non compulit facere, cui peccatum non placet, sed facturos esse praedixit quem nihil latet. Et ludas non ideo prodidit dominum quod ita de eo scriptum esset, sed ideo de eo scriptum erat quod deus praesciret auaritia et desperatione animum ipsius fore corrumpendum. Obiicis, Atqui ratione pari consequetur ut quia uiderit deus fidelium bona opera praedestinarit eos ad uitam aeternam, unde consequatur electionem dei pendere a nostris meritis. Nequaquam. Nam mala quae ex corruptione et labe primaria uere nostra sunt puniuntur in nobis iustitia dei: sed gratia dei electi sumus ad uitam uon propter opera nostra, sed propter sanguinem Christi. 2. Timoth. 1. Haec in praesenti ex uetustis scriptoribus praecipuisque religionis nostrae antistitibus pro  \pend
\section*{COMMENT. IN I. EPIST. }\pstart ferre libuit. Quod si alii melius quiddam reuelatum fuerit certe cum Augustino discere quam docere malo, dummodo Deus non blasphemetur et in illum omnis culpa malitiae et scelerum nostrorum non reiiciatur, aut nostris uiribus tribuatur quod reuera illius gratiae est. Dominus enim apud Oseam planis uerbis pronuutiat, Perditio tua Israel, sed in me est auxilium tuum. Et in euangelio dicit Christus, Hierusalem Hierusalem, quoties uolui te congregare sicut gallina congregat pullos suos sub alas, et tu noluisti. Item Daniel, Tibi domine inquit iustitia, nobis autem confusio facierum. Et Dauid, Ne intres in iudicium cum seruo tuo quia non iustificabitur in conspectu tuo omnis uiuens. Non nobis domine non nobis, sed nomini tuo da gloriam. Iu stus est dominus in omnibus uiis suis, et sanctus in omnibus operibus suis. Et Paulus, Si adhuc uelatum est euangelium nostrum in his qui pereunt uelatum est in quibus Deus huius saeculi excaecauit sensus incredu lorum ne illucesceret illis lumen euangelii. Vt ergo uno uerbo dicam omnia, Lux et salus a sola dei gratia est, Excaecatio et damnatio a corruptione nostra. Quod uero deus interim excaecare dicitur iustitiae eius est, secundum quam merito punit inobedientiam et perfidiam nostram induratione desperatione et caecitate. Ipsi soli gloria in saecula, Amen.  \pend
\section*{AD TIM. CAP II. }
\marginpar{[ p.116 ]} 
\phantomsection
\addcontentsline{toc}{subsection}{\textit{Vnus enim Deus, unus etiam mediator dei et hominum homo Christus lesus, ꝗ dedit semetipsum precium redemptionis pro omni bus, ut esset testimonium temporibus suis in quod positus sum ego praeco et apostolus? ueritatem dico in Christo non mentior doctor gentium cum fide et ueritate. }}
\subsection*{\textit{Vnus enim Deus, unus etiam mediator dei et hominum homo Christus lesus, ꝗ dedit semetipsum precium redemptionis pro omni bus, ut esset testimonium temporibus suis in quod positus sum ego praeco et apostolus? ueritatem dico in Christo non mentior doctor gentium cum fide et ueritate. }}\pstart Ad confirmationem superiorum etiam haec pertiuent. Nam si unus Deus est qui neque Iudaeorum Deus tantum est, neque huius uel illius gentis peculiaris, sed omnium ex aequo communis, si unus totius carnis mediator est inter Deum et hominem, siis semetipsum precium redemptionis impendit pro omnibus, profecto orandum nobis erit pro omnibus. Quando quidem uero insignis inter discipulos contrauersia fuerat de eo num et gentibus natus esset Christus, et an Deus Gentium quoque Deus esset, ideo uocationem et mini sterium suum hic praedicat apostolus, ne quis testimonio eius, quo testabatur Christum omnibus redimendis uenisse et Deum non Iudaeorum tantum sed et Gentium esse Deum, diffideret. Haec in summa tractat  \pend
\section*{COMMENT. IN I. EPIST. }\pstart praesens Pauli locus, iam ut singula repetamus et explicemus diligentius, Vnus inquit est Deus, quo non excludit uel spiritum sanctum uel filium qui non separatur a substautia patris, substantia autem unus est deus, persona trinus, unitas ergo illa nontam ad substantiam dei refertur quam ad homines, qui unum id est communem habent Deum. Ita etiam unus id est conmunis omnium mediator est Christus Iesus, ut hunc nemo sibi peculiariter uendicare possit. Non enim pro una aliqua gente, sed pro omnibus sese impendit. Verum uidentur haẽc quod ipsauis euangelii in ipsis lateat altius esse repetenda. Mediator non est inter consentientes sed dissidentes. Qui enim concordant nullo opus habent conciliatore. Et mediator et conciliator equipollent, ubi itaque mediator est ibi dissidium atque partes sunt. Et in praesentiarum partes sunt Deus et Homo. Hae inter se dissident. Causa uero dissidii est peccatum. At hoc non expiatur citra mortem et sanguinem. Oportet itaque huius dissidii mediatorem cum hominibus participare carne et sanguine adeoque et uerum hominem esse qui mori et sanguinem effundere possit. Iam si nihil aliud quam homo sit et sanguis eius nihil differat a reliquo corruptorum hominum sanguine, non expiabit peccatum dissidii causam, imon nisi Deus quoque sit hic mediator ad deum nequaquam accedere poterit. Oportet enim mediatorem utrique par\pend
\subsubsection*{Vnus deus }
\subsubsection*{Vnus mediator. }
\section*{AD TIM. CAP II. }
\marginpar{[ p.117 ]}\pstart ti communem esse. Vere enim Chrysostomus, Homo inquit purus mediator nunquam penitus fieret. Oportebat enim huiusmodi mediatorem cum Deo colloqui. Deus item solus mediator esse non posset. Neque enim suscepissent eum hi quorum mediator accederet. Pro inde diuino consilio uerbum caro factum est. Et Paulus in praesenti non citra emphasim dixit, Homo Christus Iesus. Quasi dicat, In hoc hominem induit Christus Iesus ut mediatorem agere posset. Verus ergo De us et uerus homo est Christus. Hic dedit semetipsum praecium redemptionis pro omnibus. Semetipsum inquam. Non enim per alium offerri potuit. sed per spiritum aeternum seipsum obtulit immaculatum deo. Hebrae.7.9. Et propheta, Oblatus est inquit quoniam ipse uoluit. Denique ipse dominus, Panis quem ego dabo ait caro mea est quam ego impendam pro mundi uita. Et uerbum quo hic utitur apostolus ἀντίλυτρον id significat praecium quo redimuntur captiui ab hostibus, eamque commutationem qua capite caput uita redimitur uita. Ita ergo impendit se pro nobis dominus ut sua morte nos uiuificaret, suo sanguine expiaret, et ex maledictis faceret benedictos dei filios. Neminem exclusit hic nisi qui sese excludit sua ingratitudine et perfidia. Hinc Paulus Impendit se ait pro omnibus. Quod autem alii hinc colligunt omnes etiam perfidisfimos fore saluandos, petulanter blasphemant. Qui e\pend
\section*{COMMENT. IN I. EPIST. }\pstart nim eredit habet uitam aeternam, qui non credit dannabitur. Hoc quidem ab aeterno praeuidit Deus, uoluit autem praedicari suis temporibus. De quibus et prophetae, et Petrus 2. Pet. 2.Et Paulum delegit ut omnibus praedicaret salutem. Id uero iuramenti quadam specie confirmans addit, Veritatem dico per Christum, non mentior quod Deus uidelicet me delegit ut sim Doctor omnium gentium, ut omnibus annuntiem gratiam euangelicam, idque fideliter et uere. Haec uero eximia Pauli laus est.  \pend 
\phantomsection
\addcontentsline{toc}{subsection}{\textit{Volo igitur orare uiros in omni loco, sustollentes puras manus absque iracundia et disceptatione. }}
\subsection*{\textit{Volo igitur orare uiros in omni loco, sustollentes puras manus absque iracundia et disceptatione. }}\pstart Redit ad preces unde paulo ante fuerat digressus, tollitque loci superstitionem et ostendit qualis debeat esse oratio. Quasi dicat, Quod iubeo conuenire in coetum atque in hoc publicas profundere preces non est quod inde quis colligat preces nullibi quam in conuentu efficaces esse, imo uolo orare uiros in omni loco ubicunque res postulat. Omnis enim locus purus est si animus non sit impurus. Et oratio ex mente non ex loco estimatur. Legitur Christus saepius in precibus pernoctasse sane non in templo, frequentius orauit in monte. Tollitur ergo hisce Pauli uerbis loci superstitio, ut  \pend
\subsubsection*{De orationis loco. }
\section*{AD TIM. CAP II. }
\marginpar{[ p.118 ]}\pstart et apud Matthaeum in 6. et Ioan.in 4. Porro inde non sequitur, quod tamen uaesani quidam colligunt, in templo non esse orandum. Non enim temere pronuntiauit dominus, Domus patris mei domus orationis est. Et Paulus in templo non impie legitur in Actis orasse. Quod si in templo et aede sacra hoc est doctrinae precibus et sacramentis dicata orare non licet, certe oratio erit loco affixa. Atqui libera est, licet itaque et in aede sacra et in quouis loco ubicunque res et decorum ipsaque necessitas postulat orare. Audiamus nunc qualis esse debeat oratio. Volo inquit uiros orare sustollentes puras manus absque ira et disceptatione. Caeterum exteriori orantis habitu adumbrat qualis esse debeat pie precantis animus. Eleuantur manus ad coelum, eleuetur et mens. Obliuiscamur rerum terrena rum sola coelestia cogitemus. Proferimus et excutimus manus quoties palam testari uolumus nos a furto et rapina esse quam alienissimos, orantes igitur excutiamus ex pectore omnem iniustitiam crudelitatem et impuritatem, oremus autem mente pura et syncera. Nam apud Isaiam in cap.2.clamat ad Israelem dominus, Cum extenderitis manus uestras abscondam oculos meos a uobis et cum multiplicaueritis orationem non exaudiam, eo quod manus uestrae plenae sint sanguinibus. Lauamini igitur et mundi estote et tollite malitiam conatuum uestrorum e conspectu meo et  \pend
\subsubsection*{Qualis esse debeat oratio. }
\section*{COMMENT. IN I. EPIST. }\pstart cessate malefacere. Apud Ieremiam in cap.n.dicit dominus, in singulis ciuitatibus et plateis posuerunt aras ad offerendum Baal. Tu igitur noli orare pro populo hoc, quoniam ego non exaudiam. Huc pertinet quod Paulus adiungit χωρις ὀργῆς καί διαλογισμoῦ sine ira et haesitatione siue disceptatione. Postremum illud referunt quidam ad fidutiam inconstantio rem et uacillantem, ut hic locus congruat cum illo 14 cobi, Postulet cum fidutia nihil haesitans, etc. Alii retulerunt ad animum implicatiorem odio inuidiaque uitiatum. Certe congruunt huc quae dominus apud Matthaeum hisce uerbis docuit. Si obtuleris munus tuum ad aram, et ibi recordatus fueris quod tibi cum fratre male conuenit, relinque munus, abi et reconciliare fratri tuo. Munus nostrum oratio est. Pura enim prece uult sibi litari dominus. Ara autem nostra cor nostrum est. Si igitur constituamus apud nos precari, tollamus prius ex animo quicquid inuidiae et malitiae contra fratrem concepimus, ut ex pura bonaque; conscientia dicere queamus, Dimitte nobis peccata uostra sicut et nos remittimus debitoribus nostris. Tertullianus in 3o. cap. Apolog. describens exteriori quoque orantium habitu mentis iustam compositionem. Ad coelum inquit sufpicientes Christiani manibus expansis, quia innocuis, capite nudo, quia non erubescimus, deniq sine monitore, quia de pectore oramus precantes sumus  \pend
\section*{AD TIM. CAP II. }
\marginpar{[ p.119 ]}\pstart omnes semper pro omnibus. Et paulo post, Offerimus deo opimam hostiam orationem de carne pudica, de anima innocenti, de spiritu sancto profectam. Haec imitemur obsecro qui hodie Christiani uideri uolumus et inuicem batalogia lacessimus. Non enim in uerborum copia non in numero, sed in pietate et fide in assi duitate et charitate consistit efficatia et uirtus orationis.  \pend 
\phantomsection
\addcontentsline{toc}{subsection}{\textit{Consimiliter et mulieres in amictu modesto cum uerecundia et castitate ornantes semetipsas, non tortis crinibus, aut auro, aut margaritis aut uestitu sumptuoso, sed quod decet mulieres profitentes pietatem per opera bona. }}
\subsection*{\textit{Consimiliter et mulieres in amictu modesto cum uerecundia et castitate ornantes semetipsas, non tortis crinibus, aut auro, aut margaritis aut uestitu sumptuoso, sed quod decet mulieres profitentes pietatem per opera bona. }}\pstart Nihil in hunc locum adfertur a receptis interpretibus quod non in breuem summam contraxerit Paraphrastes, nec quicquam dici potest necessarium quod idem non sit hisce uerbis complexus, quae hic malui ascribere quam post planam illam expositionem prolixius cornicari. Ad exemplum uirorum inquit precentur et mulieres. Si quid est in animo muliebrium affectuum id prius eiiciant, pro ludaicis purgationibus morum innocentiam adferentes, ad hanc uicti\pend
\subsubsection*{Mulierum ornatus et habitus in coetu sacro. }
\section*{COMMENT. IN I. EPIST. }\pstart mam animi non corporis cultum adhibeant, non nuditate corporis uirorum oculos ad libidinem sollicitantes, sed amictu tectae et eo amictu qui modestiam qui pudorem qui pudicitiam prae se ferat. Absit autem ut Christianae mulieres eo cultu prodeant in coetum sacrum quo uulgus prophanarum mulierum solet ad nuptias aut theatrum exire, quae se multo studio prius ornant ad speculum arte contortis crinibus, aut auro intertexto, aut pendulis ab auribus collonue margaritis, aut alioqui holoserica purpureaue ueste quo simul et formam spectatoribus lenocinio commendent et suas ostentantes opes tenuioribus suam inopiam exprobrent. Quin is sit potius Christianarum habitus qui uitae respondeat qui dignus uideri possit iis mulieribus quae ueram pietatem ac dei cultum profiteantur, non ostentatione diuitiarum sed benefactis quibus opibus unice delectatur Deus, cuius oculis foedum est quod mundo uidetur splendidum et magnificum. Haec paraphrastes. Paria propemodum reperies in 2.ad Corinth.12.et 1.Pet.3. cap. Tertullianus scripsit libellum de Cultu foeminarum eruditissimum et utilissimum, quem hisce uerbis mire uidelicet ad praesentia facientibus concludit, Prodite uos iam medicamentis et ornamentis extructae apostolorum, sumentes de simplicitate candorem, de pudicitia ruborem, depictae oculos uerecundia et spiritus taciturnitate, inseren\pend
\section*{AD TIM. CAP II. 120 }\pstart tes in aurem sermonem dei, annectentes ceruicibus iugum Iesu Christi. Caput maritis subiicite et satis ornatae eritis. Manus lanis occupate, pedes domi figite et plus quam in auro placebunt. Vestite uos serico probitatis, byssino sanctitatis, purpura pudicitiae. Taliter pigmentatae Deum habebitis amatorem.  \pend 
\phantomsection
\addcontentsline{toc}{subsection}{\textit{Mulier in silentio discat cum omni subiectione. Caeterum mulieri docere non permitto, neque authoritatem usurpare in uiros sed esse in silentio. Adam enim prior formatus est, deinde Eua, et Adam non fuit deceptus, sed mulier seducta fuit per praeuaricationem: salua tamen fiet per generationem liberorum, si manserit in fide ac dilectione et sanctificatione cum castitate. }}
\subsection*{\textit{Mulier in silentio discat cum omni subiectione. Caeterum mulieri docere non permitto, neque authoritatem usurpare in uiros sed esse in silentio. Adam enim prior formatus est, deinde Eua, et Adam non fuit deceptus, sed mulier seducta fuit per praeuaricationem: salua tamen fiet per generationem liberorum, si manserit in fide ac dilectione et sanctificatione cum castitate. }}\pstart Quo facilius a mulierculis quod praescripserat impetraret Legem dei profert admonens illas sui officii et humilitatis. Lege autem dei praecipitur mulieribus obedientia silentium et humilitas. Est autem hoc ge\pend
\subsubsection*{Lex de mulierum disciplina. }
\section*{COMMENT. IN I. EPIST. }\pstart nus hominum natura garrulum fastuosum et recalcitrans. Hinc Paulus, Mulier inquit in silentio discat cum omni subiectione. Id est mulieres silere et per omnia subesse et a uirorum authoritate pendere debent. Sequitur enim huius legis expositio, Mulieri docere non per mitto, neque; ἄυθεντεῖν ἀνδρός id est authoritate uti in uiros et illos ad suum cogere affectum legesue praescribere authenticas, sed omnino esse in silentio. Hanc legem repetiit et in 1. ad Corinth.14. cap. et transsumpsit ex Gene. 3. cap. ubi in haec uerba pronuntiat dominus, Multiplicando multiplicabo dolorem tuum et conceptum tuum. In dolore paries filios, et ad uirum tuum erit desyderium tuum, et ipse dominabitur tibi. Id est nihil constitues priuato affectu sed ad nutum uiri omnia ages. Proinde cum in rebus priuatis reuerenter debeant auscultare mulieres, multo minus publicis functionibus praefici debent. Inuertunt enim naturae et dei ordinem qui in iudiciis exercendis inqué dicendis sententiis, in ferendis item legibus, denique et in ministeriis et functionibus ecclesiasticis ullum locum permittunt mulieribus. Vnde consequens est et humanarum et diuinarum rerum ex aequo fuisse ignaros qui Abbatissas (sic enim uocarunt principes inter uirgines Vestales, in quarum numero tamen non deerant scorta Corinthiaca) praefecerunt rei pubus  legibus adde et sacris functionibus. Causas di\pend
\section*{AD TIM. CAP II. }
\marginpar{[ p.121 ]}\pstart uinae legis subiicit apostolus ne quis muliercularum causam agens quereretur istam ex audacia et libidine uirorum esse promulgatam. Prior causa est, Adam prior formatus est deinde Eua, adeoque ex uiro formata est foemina, adde quod in adiutorium uiri a Deo condita, ut portio uerius uiri sit quam caput:non e capite sed latere uiri sumpta. Posterior causa est, Adam non fuit deceptus sed mulier seducta fuit. Vir enim neque serpentis promissionibus, neque pomi illecebra decipi potuisset: sola charitas uxoris pertraxit ad pernitiosum obsequium. Denique Adam homo a quo nihil humani alienum non saxum rigidum erat, ut nihil dicant qui obiiciunt uirum debuisse constantiorem fuisse quam quem illecebrae fregissent uxoris. Fons ergo praeuaricationis a foemina existit. Hinc non temere ad mulierem Tertullianus de habitu muliebri, Tu inquit es diaboli ianua, tu es arboris illius resignatrix, tu es diuinae legis prima desertrix, tu es quae eum suasisti quem diabolus aggredi non ualuit, tu imaginem dei hominem tam facile elisisti, propter tuum meritum id est mortem etiam filius dei mori habuit etc. Quando ergo magisterium Euae non uni Adae sed toti mortalium generi semel cessit pessime, subesse posthac et iure ad uirum respicere iubetur mulier quaelibet. Agnoscat imbecillitatem et lapsum ueterem, ac satis habeat quod quae olim fuit dux ad impietatem et ruinam  \pend
\section*{COMMENT. IN I. EPIST. }\pstart nunc erroris memor sequatur uiri pietatem et resurgat in pia obedientia. Non enim haec dicuntur in contumeliam omnium mulierum, neque ideo excluduntur a consortio sanctorum, nec quisquam putet sibi ideo in istas licere quidlibet. Imo cum muliercularum genus alias sit timidulum, ne in desperationem prorsus praeceps rueret, addit nunc consolationem, dicens, Salua tamen fiet per generationem liberorum etc. Atqui certum est salutem non deberi ullis meritis nostris, tantum abest ut debeatur procreationi liberorum. Sola enim dei gratia per Christum saluamur. Abutitur ergo saluandi uerbo pro reparandi, quemadmodum et Germanis mos est usurpare uerbum heylen für widerbzingen vnd ersetzen. Quasi dicat, Audistis quantam muliebris sexus sibi inusserit maculam, sed ea eluetur si non desit suo officio. Officium autem mulierum est liberos gignere, labores et dolores quos secum trahit partus et educatio liberorum pacienter tolerare, liberis omnem operam et diligentiam impendere, fidem item cum erga deum tum maritum seruare, utrumque diligere ex animo. Sic autem fiet ut excus sa omni superstitione cui tamen hoc genus maxime ob noxium est uere et indiuulse uera fide Deo et Christo adhaereant, fidem quoque coniugalem non uiolent, deum item ante omnia, mox maritos diligant quemadmodum ecclesia solet diligere Christum sponsum suum. Huc  \pend
\subsubsection*{Gene.3. }
\section*{AD TIM. CAP II. }
\marginpar{[ p.122 ]}\pstart pertinet sactimonia et castitas siue pudicitia et sobrietas. Debent enim mulieres animo et corpore integrae et sanctae esse, deinde castae et sobriae in dictis factis moribus denique toto corporis gestu et habitu, Titum 2. Mulier enim impudica monstrum uerius hominis quam homo est. Hisce aunt caeterisque uirtutibus sarcient inquit Paulus, quod in primo illo lapsu diminutum est, quod scilicet hominis partes et officia attinet, alias nemo negare potest omne reparationis precium et meritum passioni dominicae deberi. Doceant ergo episcopi sancti quod non puduit sanctissimum Christi apostolum Paulum docere. Quidam non sustinentes doctrinam sanam ad fabulas conuertuntur et nugantur de uita monastica ac uirginitate quam interim sanctissimam deoque gratissimam esse credimus, si modo eiusmodi sit qualem describit apostolus 1. Cor. 7. Caeterum coactam illam et inuoluntariam adeoque et immundam non damnare non possumus, quod certo sciamus deo probari placereque sanctimoniam cum animi tum corporis. Vnde et apostolus praecepit corpora nostra possideamus in sanctimonia et honore et non in concupiscentia, alibi quoque commendans connubium sanctum et immunditiam damnans. Honora bile ait est connubium apud omnes et cubile impollutum, scortatores autem et adulteros iudicabit deus.  \pend 
\phantomsection
\addcontentsline{toc}{subsection}{\textit{DESCRIPTIO EPISCOPI ET familiae eius. }}
\subsection*{\textit{DESCRIPTIO EPISCOPI ET familiae eius. }}
\subsubsection*{Coelibatus monasticorum. }
\section*{COMMENT. IN I. EPIST. }\pstart In ecclesia dei hoc est sanctorum coetu ueluti anima est minister uerbi, utpote quem Paulus alibi appellat cooparium dei, et per cuius ministerium deus populum suum gubernat. Est igitur hic a deo constitutus ut ab ore eius cumcti expectent et audiant uerbum dei, idem a domino ueluti lucerna quae piam toti domui est expositus ad quem intendentes qui uersantur in domo quid agant norint. Nam in euangelio dominus ad uerbi ministros, Vos, inquit, estis lux mundi. Sic luceat lux uestra coram hominibus ut uideant uestra bona opera glorificentque patrem qui in coelis est. Proinde Paulus uitam et mores Episcopi informat ne ex his ullae in ecclesiam offundantur tenebrae.  \pend 
\phantomsection
\addcontentsline{toc}{subsection}{\textit{Indubitatus sermo, Si quis episcopi munus appetit bonum opus desiderat. Oportet igitur episcopum irreprehensibilem esse, unius uxoris maritum, uigilantem, sobri um, modestum, hospitalem, aptum ad docendum, non uinolentum, non percussorem, non turpiter lucri cupidum, sed aequum, alienum a pugnis, alienum ab auaritia, }}
\subsection*{\textit{Indubitatus sermo, Si quis episcopi munus appetit bonum opus desiderat. Oportet igitur episcopum irreprehensibilem esse, unius uxoris maritum, uigilantem, sobri um, modestum, hospitalem, aptum ad docendum, non uinolentum, non percussorem, non turpiter lucri cupidum, sed aequum, alienum a pugnis, alienum ab auaritia, }}\pstart Chrysostomus et huius abbreuiator Theophylactus hanc particulam πιςτὸς ὁ λόγος retulerunt ad  \pend
\endnumbering
\beginnumbering
\section{CAP. III. }
\pstart
\pend
\endnumbering
\end{pages}
\end{document}
        