
%%%%%%%%%%%%%%%%%%%%%%%%%%%%%%%% SCRIPT FOR E-RARA AND MDZ FILES     %%%%%%%%%%%%%%%%%%%%%%%%%%%%%%%%%%%%%%%%%%%%%%%%
%%%%%%%%%%%%%%%%%%%%%%%%%%% fini le 30.04.2025 par F. GOY            %%%%%%%%%%%%%%%%%%%%%%%%%%%%%%%%%%%%%%%%%%%%%%%%
% !TeX TS-program = lualatex
\documentclass{article}
\usepackage[T1]{fontenc}
\usepackage{microtype}
\usepackage[pdfusetitle,hidelinks]{hyperref}

\usepackage{polyglossia}
\setmainlanguage{english}
\setotherlanguages{latin,greek}
\usepackage[series={},nocritical,noend,noeledsec,nofamiliar,noledgroup]{reledmac}
\usepackage{reledpar}

\usepackage{fontspec}
\setmainfont{TeX Gyre Termes}

\usepackage{sectsty}
\usepackage{xcolor}

\usepackage{fancyhdr}
\pagestyle{fancy}
\fancyhf{}
\fancyhead[LE,RO]{\nouppercase{\leftmark}}  
\cfoot{\thepage}
\renewcommand{\headrulewidth}{0.4pt}

% Redefine \section to remove numbering
\usepackage{titlesec}
\titleformat{\section}[block]{\normalfont\scshape\color{gray}}{}{0pt}{} % no number in heading
\titleformat{\subsection}[hang]{\normalfont}{}{0pt}{} % also remove subsection number
\titleformat{\subsubsection}[hang]{\normalfont\footnotesize\color{black}}{}{0pt}{}

% Modify how section marks are stored to exclude numbers
\makeatletter
\renewcommand{\sectionmark}[1]{%
	\markboth{#1}{}} % Only store the section title, without number
\renewcommand{\subsectionmark}[1]{%
	\markright{#1}} % Only store the subsection title, without number
\renewcommand{\numberline}[1]{} % Hide the section number in TOC
\makeatother

\begin{document}

\date{}
        \title{Commentarium in epistolam D Pauli Apostoli ad Timotheum primam : [Pellicanus Conrad], [1539]}
\maketitle
\tableofcontents
\clearpage
\begin{pages} 
\beginnumbering
        
\section*{AD TIMOTH. CAP. I. }
\marginpar{[ p.479 ]}\pstart lata ac religiosa conuersatione, coram omni ecclesia, quae eum audacem reddere poterat, ad animose uiriliterque  occurrendum, etiam apostolatus sui authoritate, et spiritus sancti uirtute, qua perfusus erat, omnibus illis aduersariis potestatibus, doctrina et moribus, uolentibus corrumpere commissam sibi ab Apostolo ecclesiam Ephesiorum, exemplum futurus episcopis successoribus.  \pend
\phantomsection
\addcontentsline{toc}{subsection}{\textit{Quam quidam repellentes circa fidem naufragauerunt, ex quibus est Hymenaeus et Alexander, quos tradidi satanae, ut discant non blasphemare. }}
\subsection*{\textit{Quam quidam repellentes circa fidem naufragauerunt, ex quibus est Hymenaeus et Alexander, quos tradidi satanae, ut discant non blasphemare. }}\pstart Erant Ephesi, qui contra conscientiam et agnitam ueritatem Apostolum traducebant, confingebant in eum mendacia, et obloquebantur: qui et semel susceptam euangelii fidem abiiciebant, Hymenaeus et Alexander. Hii uerbis Apostoli pertinaciter restiterant, seditiones mouerant contra Paulum, Iudaei ex genere, et a fide apostatae, Pauli persecutores subdoli, et non semper in manifesto aduersantes. Vnde et Paulus Timotheum admonet de Hymenȩo, ut caueat, nihil illi fidat, in secunda epistola. Satanae tradere, est excommunicare, et eiicere a coetu fidelium, ut a morbidis agnis sibi sani et boni Christiani caueant et non inficiantur. Prodidit ergo illorum malitias, et ecclesiae patefecit, ut et illi resipiscerent, et reliqui sibi de periculo prospicerent. Talis olim in ecclesia excommunicatio agebatur, sic publicorum criminum propalatio, et exomologesis in aedificationem ecclesiae fiebat. Quas utinam consuetudines probatissimas, iusta forma, secundum apostolicas traditiones ecclesia recuperet, ut reformentur mores, et disciplina non cesset, ut discant non blasphemare. Blasphemia nimirum est, peruersa docere, secreta diuina temeraria curiositate disquirere. Nihil enim commune habet humana ratio collata diuinis. Compescendi diligenter in ecrlesia sunt, qui dominum blasphemant, uerbi illius authoritati detrahunt, qui gentilium errores postliminio reducunt, qui Iudaicis fabulis et haereticorum uenenis inficiunt ecclesiam, qui religionis praetextu gentilium ritus commutant in speciem religionis, quos satius fuisset in totum reiici, ne gentilium ceremoniis ecclesia plus quam Iudaica grauetur, puritasquedoctrinae et operum commaculetur, et tandem opprimatueritatem.  \pend
\section{CAPVT II.}
\phantomsection
\addcontentsline{toc}{subsection}{\textit{\huge\textbf{O}\normalsize Bsecro igitur primum omnium fieri obsecrationes, orationes, postulationes, gratiarumactiones pro omnibus hominibus. }}
\subsection*{\textit{\huge\textbf{O}\normalsize Bsecro igitur primum omnium fieri obsecrationes, orationes, postulationes, gratiarumactiones pro omnibus hominibus. }}\pstart Cum igitur necesse habeas militare periculosam militiam, cumqueside tibi et conscientia opus sit, ut imminentibus undique  et semper malis occurrere possis, et officio satisfacere tuo, spiritu sancto te uocante commisso, quae sine dei gratia ac uirtute, nec a te praelato haberi possunt, neque  ab alio. Ideo adhortor, non impero, non importunus urgeo, ut dominus, sed ut fidelis tibi pater filium obsecro, quem pro syncero affectu et filiali amore, scio non minus complecti hortamentum, quam seruus imperium. Non enim patris praeceptum filius expectare debet et exigere, sed solo nutu insinuasse filio beneplacitum patris, abunde suffecerit, ut protinus perficiat sedulo, quod patrem optare crediderit. Sic non mandatis, imperiis et coactionibus in dei negotiis est agendum. Nec Christus nec apostoli id tentarunt, sed usi sunt admonitionibus, consiliis, orationibus, et adhortationibus, quibus in ecclesia proficere studuerunt. Vt nihil mandauerint, nisi quae pro charitate seruanda seriosius proponebant. Vt ante omnia fiant orationes.) Cum sanctum ac deo acceptum, uel probandum, nihil prorsus ualeamus, nostris tantum freti uiribus, sine dei auxilio, nec uerba, nec opera, nec exempla, nec aliqua solicitudo nostra aliquid prosit uel promoueat, necesse est ante omnia  \pend
\section*{COMMENT. IN I. EPIST. }
\marginpar{[ p.480 ]}\pstart nostra conamina, dominum implorare, auxilium illius petere ac assistentiam, ut ipse bonum inspiret, desiderari faciat ac perfici: nihil enim aliunde obtinebimus.  \pend\pstart Deprecationes.) Dicuntur heae orationes ad deum, quibus aduersa quae merui mus peccatis nostris, per dei misericordiam amoueri precamur, iram scilicet dei, et poenam condignam, conscientiae angustias, incommoda corporis et spiritus, tentationes perniciosas, et proximas periculo, ac casus peccatorum. Obsecrationes.) Sunt, quibus per omnia sacra et deo grata, consequi desideramus dona dei nobis necessaria, spiritualia in primis, deinde et corporalia. Vt quod nostris meritis obtinere non possumus, eius gratia, et Christi domini meritis et intercessione a domino postulemus nobis concedi in gloriam dei. Interpellationes.) Sunt cum unus pro alterius necessitate ex charitate solicitus intercedit, et profratre orat, ut uel aduersa amoliantur, uel optabilia consequantur, cum scilicet pro inuicemoramus, ut pariter saluemur. Gratiarumactiones.) Sunt, eum pro acceptis beneficiis dominum laudamus, et omnia eidem accepta reddimus, et a nobisipsis nihil nos habuisse uel potuisse fatemur. Vicissim nos offerentes uoluntati suae, cuius beneplacito per omnia subiiei desideramus. Hoc est enim eucharistiae uerum ac Christianissimum sacrificium, laudis scilicet et confessionis, ac gratiarumactionis. Pro omnibus hominibus.) Non solum orandum pro amicis, fidelibus, iustis, ac familiaribus, ac benefactoribus, uerum omnium maxime proeis qui magis in digent, dei maiori opus habent misericordia. Vt sunt inimici nostri, ut charitate imbuantur Christiana, infideles, Iudaei, haeretici, ut respectu gratiae diuinae, lumine uerae fidei et ueritatis illustrentur, et domino credant, iniusti ac impii ut resipiscant: ignoti nobis, ut domino miserante a peccatis liberentur. Nihil unquam inter orandum occurrat discriminis, quo minus affectuose pro uno hominum genere oremus, quam pro alio, uel excludere quoscunque  praesumamus, uel intendamus frigidius. Nec tantum oremus obbeneficia, ut hypocritae solent, et incommodum nostrum, sed sola fraterna charitate ad orandum permoueamur pro egentibus: et quia id domino placere credimus, similia abaliis orationum suffragia desiderantes. Non iubet Apostolus orari pro defunctis, sed pro hominibus, non pro animabus fidelium mortuorum, quae non iam nobis proximae sunt, sed deo. Viui no bis ubique  commendantur, mortui deo:nos contra egimus, totam fere nostram solicitudinem impendentes in eos, demortuos scilicet, de quibus nobis dominus nihil praecepit. Pro his non piget nimium multa dare, agere, pati, interim tamen fratres indigentes, maximopere a deo nobis commissos, dissimulamus, et nihil iuuamus. Sed haec excogitauit auaritia cleri, ut morituri multa legarent pro se mortuis de diuitiis suis, quas posthac, ut non suas, relinquere cogebantur. Haeredibus debita iure naturae ea intercessoribus pro animabus facile largiri poterant, maxime quia purgatorii poenas his mediis euadere posse instruebantur praeter uerbum dei, et sine sanctorum exemplo.  \pend
\phantomsection
\addcontentsline{toc}{subsection}{\textit{Pro regibus, et omnibus qui in sublimitate constituti sunt, ut quietam et tranquillam uitam agamus, in omni pietate et castitate. }}
\subsection*{\textit{Pro regibus, et omnibus qui in sublimitate constituti sunt, ut quietam et tranquillam uitam agamus, in omni pietate et castitate. }}\pstart Apostoli temporibus reges mundi infideles erant et ethnici, aduersarii fidelibus, sed tamen pro eis orari uoluit, imo tanto solicitius instetit, quanto magis clementiae diuinae influxus illis pro illorum regimine necessarius erat, ut boni fierent, ut tractabiles, mansuete praesint, benigne dominentur, quo ualeant sustineri, ut cor illorum regatur a domino, quo subditi tranquillius regantur: ut illis desuper praestetur sapientia et timor dei, ut ament iustitiam, et tueantur innocen\pend
\section*{AD TIMOTH. CAP. II. }
\marginpar{[ p.481 ]}\pstart tiam. Et omnibus in sublimitate constitutis.) Addita coniunctio expositiue accipitur. Regibus, id est sublimibus. Sunt em reges qui in magistratu aliquo constituti, regunt, moderantur et praesunt. Constituri uero non ab alio que ̃ a deo. Omnis enim potestas a deo est, etiam mala, propter peccata populi, et imminens subditis dei iudicium. Per eos enim deus non raro sua exequitur iudicia. Dabo, inquit, tibiregem in furore meo. Effoeminati erunt principes eorum, etc. Maximam itaque  nobis poenam deprecabimur, si bonum et religiosum principem orationibus a deo promerebimur. Propter peccata enim populi dominus regnare facit hypocritam, id est impium. Sublimitas autem est, regere subiectos, iuxta stilum spiritus sancti, et a deo ordinata: non tantum ut honorentur praefecti, sed multo magis ut onerentur, solicitudine maiori prodesse cupientes eis, quibus ut per charitatem et sapientiam seruiant, a deo dati sunt. Reges gentium dominantur, et benefici dicuntur, non sic reges fidelium. Hi enim sciunt sese tanto modestiores gerere debere, quanto sunt plurium commodis et profectibus destinati, et obsequio charitatis ac fidei deuincti. Debetur ipsis honor, reuerentia, timor, obedientia, ac sustentatio a subditis: ipsi uicissim obligantur amorem, fidem, curam, tutelam, solicitudinem, instructionem, consolationem, regimen, coërcionem a malis, et discolis poenam. Nec obmittendum negligentius, quod non prece, pretio et instantia, sublimitas sit ambienda, uel regimen aliorum, sed constitui oportet ab aliis, nihil nisi subditorum bonum respicientibus, neque  sine consensu eorum, si tamen rationabiliter sibi consultum cupiant, et non omnem malint subterfugere disciplinam, ut non tam seruitus habeatur, que ̃ administratio disciplinae. Vt placidam et quietam uitam agamus.) Annitendum omni studio regibus, ut populus beneuolenter et cum gaudio, non ex tristitia obediat, sed placide, syncero affectu, sine murmure et displicentia, si fieri possit, et sine uiolentia:ut sint omnia sine tumultu, odio, seditione et rebellione. Sed Christiana pace, tranquille conuersentur, ut fratres ac domestici discipuli Christi, pacifici, modesti, mansueti et humiles. Haec est enim uita felicissima, planequebeata, et Christiana charitate iucunda. Cum omni pietate et honestate.) Faciunt haec duo perfectam quietem, si scilicet deum nostrum colamus fide, spe et charitate, ut totos nos committamus ei ut patri: et omnia quae occurrunt, patris prouidentia et pietate nobis euenire certo credamus. Tunc enim murmurabimus nunquam, temere non iudicabimus illius iudicia et dispensarionem, sedut sunt omnia optima cordis pace, ut patris pientissimi beneficia amplectemur. Castitas hic pro omni morum grauitate accipienda, ut honestas. In conuersatione enim ad proximum, morum honestas et comitas admodum nos commendat fratribus. Cauebit quisquis huiusmodi est, offensam et odia, perpetuam autem tranquillitatem fouebit et amicitiam, quae inter inhonestos ac morosos aut leues personas diu constare non potest.  \pend
\phantomsection
\addcontentsline{toc}{subsection}{\textit{Hoc enim bonum est, et acceptum coram saluatore nostro deo, qui omnes homines uult saluos fieri, et ad agnitionem ueritatis uenire. }}
\subsection*{\textit{Hoc enim bonum est, et acceptum coram saluatore nostro deo, qui omnes homines uult saluos fieri, et ad agnitionem ueritatis uenire. }}\pstart Quieta conuersatio nostra, et tranquillitas huius uitae in simplici obedientia, etiam ethnicis regibus exhibenda docetur, ut ecclesia careat dominantium mole stia, neque  grassentur, ut magis opprimant, ne irritentur contra nos et euangelicam doctrinam, non scandalizentur nostris moribus, non ansam arripiant mouendi persequutiones, et blasphemandi nomen Christi. Cedit enim in gloriam religionis Christianae, imo domini dei nostri, qui per filium suum saluauit nos, si pacifice, mansuete ac sedulo dominis obsequamur, subiecti omnibus in timore del, osten\pend
\section*{COMMENT. IN I. EPIST. }
\marginpar{[ p.482 ]}\pstart dentes facto, optimi dei optimas leges scriptas esse in cordibus nostris. Vt ex bonis operilbus nos considerantes, glorificent deum, si quando respectu misericordiae a deo fuerint uisitati, et corde pio religiosoque  donati, dominumquedeum nostrum ex fidelium religione cognouerint, nobiscumque ; uenerari inceperint, orationibus nostris interim domino commendati, et exemplo acquisiti. Omnes, inquit, homines uult saluos fieri.) Est enim dominus deus summe bonus, nihil eorum odiens quae condidit, neminem uult perire. Adest sua gratia omnibus, ut ueniant ad agnitionem ueritatis, quam per propheticas euangelicas et apostolicas literas, communem fecit omnibus. Vbi denuo coniunctio. Et exposstiue sumenda uidetur, ut nihil aliud sit uelle saluos fieri, que  ad agnitionem diuinae uoluntatis perducere, ut agnoscere ueritatem sit fieri saluum. Ipsa enim ueritas liberat a peccato, si illi credatur. Denique  uocauitomnes, annunciatur omnibus, Christus pro omnibus mortuus est, omnes ad se trahit beneficiis, proomnibus tradidit semetipsum. Contemptores autem dei et negatores non computandi ueniunt inter omnes, sed redigentur in nihilum coram deo. Si itaque sermoincidat, Quare non omnes beatitudinem assequantur, leum omnes homines uelit saluos fieri ?Responderi potest, Quare non omnes capiant gratiam, quam sibi porrigi nemo ambigit? Quare non credant uerbo dei omnes? Quare nemo etiam cogi uelit ad bonum suum, nec debeat ? Nec se quispiam excusare potest de ignorantia, deque  tractu instinctuum bonorum in conscientia, qui non cessant. Quod si gratiam sequendi quis abesse causetur, non tamen eam optare ac petere se non posse uere dicet. Domino autem dicente, Petite et accipietis. Is re uera nonnihil recipiet, qui sedulo dominum fuerit deprecatus. Vltima tamen responsionis anchora tandem erit, uoluntas dei, quae regula est iustitiae, cuius non sit inquirenda ratio, qui et nobis nihil obligatur, et bonorum nostrorum non eget. Voluntate signi et antecedente, uult omnes saluari, dicunt Theologi post Augustinum, et distributione accommoda euadunt. Denique  agnitio ueritatis, sensus est Christi, qui ueritas est, qui euangelium praedicauit mundo, qui ueram fidem ac uiuam, portum salutis statuit, et iustitiae summam. Lux mundi est, qui illuminat omnem hominem. Humanae scientiae, et ueritas intellectus naturalis, ad salutem inutiles sunt, et mendacium dici possunt. Christum autem scire, sanctorum est, ideo stultus in hoc mundofiat, qui uoluerit fieri sapiens in deo et saluus.  \pend
\phantomsection
\addcontentsline{toc}{subsection}{\textit{Vnus enim deus, unus et mediator dei et hominum, homo Christus lesus, qui dedit redemptionem semetipsum pro omnibus. }}
\subsection*{\textit{Vnus enim deus, unus et mediator dei et hominum, homo Christus lesus, qui dedit redemptionem semetipsum pro omnibus. }}\pstart Breuis hic expressa est summa fidei Christianae, et ueritatis irrefragabilis, cuius solius cognitio firmiter persuasa, salutem operatur hominibus. Si, inquam, non simplici id notitia sciant, sed potius uera sapientia: qui scilicet sic norunt, ut sapiant q̃ suauis sit dominus. Hunc ergo tenorem fidei, et breuissimam regulam, simpliciter et breuibus discutiamus. Vnum esse deum religiose cognoscitur, quum praeter eum qui omnia condidit, nihil aliud, ut alicubi summum bonum desideratur, colitur, amatur, antefertur, uel confertur nihil. Non scilicet uenter, non caro et sanguis, aurum, honor, laus, et totus hic mundus. Omnia haec postponuntur, non tantum uerbo et lingua, ut facile possunt hypocritae, sed corde et sensu. Dum illud unum semper intendimus et attendimus, propter illud omnia agimus, non circa plurima occupamur et distrahimur. Hoc est uere agnoscere unum deum, praeter eum nihil desiderare, quod non propter illum uelimus. Vnus mediator.) uel conciliator. Secundus fidei huius breuissimae articulus est, credere et scire  \pend
\section*{AD TIMOTH. CAP. II. }
\marginpar{[ p.485 ]}\pstart Christum esse unicum mediatorem, quem pater coelestis statuit inter se et nos,ut cum essemus peccatores, et a deo proiecti et inimici, per Christum reconciliemur. Hic omnes homines peccatores esse, et reconciliatione egere ad deum, manifeste docetur, qui ut aduersarii et hostes, deum offensum sibi patiuntur, et abiectos esse damnatosqueex primi hominis transgressione significantur, sola Christi gratia conciliandos: et non id solum, sed et filiationis et haereditatis ius accepisse per Christi mysterium, et in Christo omnia bona, ut nostra sint, quae sunt fratris nostri. Vnus autem, non multi mediatores uel conciliatores docentur: unus enim Christus, unus saluator. Non ergo praeter eum interponendi sunt alii, quantumlibet sancti et pii homines, omnes enim peccatores sunt, et gratia dei indigent. Solus Christus suapte innocentia, dignitate et meritis proprie dictis, non ex gratia, nos peccatores conciliat patri, pro officio suo quod a patre accepit. Omnia denique  pater dedit in manus filio unigenito, ad illum nobis currendum, quoties quomodo cunque  urgemur, qui fons est pietatis, et fidelissimus frater, qui sese pro nobis totum obtulit, in acerbissimam simul et probrosissimam mortem crucis, non nisi ex maxima charitate et fide ad genus humanum. Non est ergo ut eum ueluti morosum nobis promereamur, sanctorum aliorum quorumcunque  intercessionibus, benignior nobis est,  omnes sancti esse possint, quantumlibet charitate fulgentes. Solus pro nobis mortuus est, et nos toties inuitat, pulsat, expectat, hortatur, allicit, rogat, et cogit aliquando. Venite, ait, ad me omnes etc. Venite edite absque  argento etc. Audit nos semper per suam sibi soli propriam omniscientiam, adest per immensitatem, qua patri consubstantialis est: intendit singulis perfectissima charitate, quandoquidem sui sunt omnes: non distrahitur multitudine curandorum uel saluandorum, omnia nostra nouit melius  nosipsi, quibus egeamus considerat, nobis dormientibus. Praecepit seriose, ut in solo suo nomine precaremur patrem. Honorem hunc gloriosissimum suum alteri prorsus nulli cedere potest: nec patitur communem ab aliis usurpari unigenitus filius. Homo, inquit, Christus Iesus.) Mirum quomodo dubitauerunt hominem non fuisse quidam haereticorum, qui sacras literas non negant, et hanc epistolam, qui hic simpliciter homo dicitur. Caro enim et frater noster est, tentatus per omnia, absque  peccato, et pro nobis hominum nouissimus sese exhibuit, ac uermem nominauit, ut hominem perditum redimeret. Et primogenitus quoque  dicitur ex multis fratribus, ut chirographum peccati deleret per crucem, uere pro nobis mortuus est, utquehomo esset, qui pro humano satisfaceret genere. Christus Iesus, unctione sua consecrans omnes gentes, et Christianos faciens sua gratia, et spiritus sui unctione: passione quoque  sua saluans gentiles et Iudaeos, faciens utraque  unum. Qui dedit semetipsum redenptionem.) Obtulit seipsum in remissionem omnium peccatorum, sacrificium seipsum perfectum et maximum, omnium quoque  praecedentium sacrificiorum finem et scopum, proquenobis captiuis commutationem sufficientissimam. Non raptus est ad iudicium mortis inuitus, sed sua sponte obediuit, et seipsum obtulit pretium deo patri, pro salute humani generis, et reconciliatione perfecta. Cum essemus inimici, et tales inimici, ut etiam eum nos ipsum crucifixerimus, homines scilicet, pro quibus se pretium et satisfactionem mediator obtulit. Nec dicamus, non esse nos posteros Iudaeorum, qui hodie quoque  in glorificatum dominum nostrum et regem simile facinus non refugeremus, simili occasione permoti, qua Iudaeorum primores, quibus meliores inter nos non ita multos inuenire licebit, quin et acerbius ipsum iterum nostris uitiis crucifigimus. Pro omnibus.) Exposuit ac tradidit se in mortem pro omibus, nemo nisi nolens saluari, relinquitur. Sufficiens  \pend
\section*{COMMENT. IN I. EPIST }
\marginpar{[ p.484 ]}\pstart fuit, et satisfecit deo patri pro omibus, nihil obmisit de suo, quo minus omnes redimi potuerint, et hodie possint. Excusatio nulla nobis relinquitur, qui plane scimus quid uel facere, uel desiderare, uel quae obmissa dolere, uel non indolentiam deprecari, et humiliter agnoscere debeamus, proquefide huiusmodi et charitate deum orare. Quis haec non se posse nisi contumax uere fatebitur ? Ad hanc uerita tis agnitionem, si quis illuminante deo peruenerit, ut agnoscat ac credat certa fide, unum suum deum, et conciliatorem unicum, et sibi deuotissimum dominum Iesum Christum, quodqueseipsum dederit ex maxima gratia redemptionis pretium pro omnibus credentibus, sine dubio saluus erit, perfectus fidelis habebitur, et Christianus coram deo et Christo suo. Quae illi fides reputabitur ad iustitiam, sicut Abrahae. Hanc breuissimam summam totius noui testamenti et euangelicae doctrinae, dum olim apostoli praedicarent facile orbis attrahi ad credendum poterat, idolisqueabiectis, Christiana religione imbui. Cum nondum tam incredibilia obtruderentur multa, qualia interim praeter sacram scripturam fidelibus, de plus que ̃ bis duodecim articulis neotericae fidei imponuntur necessario credenda, de Papatu multa, de indulgentiis, purgatorio, diuis, et diuorum intercessionibus, miraculis, delectu ciborum, excommunicationibus, confessionibus, casibus episcopalibus, uotis, peregrinationibus, Conciliis, sacramentis, impedimentis, Missis, horis canonicis, ceremoniis, festis, ordinibus, imaginibus, monasticis regulis, dispensationibus, fraternitatibus, consecrationibus. Et quis ea numerare possit, quibus hodie grauantur conscientiae piorum, et impiorum conuersio non iuuatur? Quorum nihil continetur in Symbolo apostolorum uel canonicis scripturis, imo nec priscis sacrosanctis sanctorum patrum conciliis: de quibus multa scribi possent, sed sufficiunt quae nunc passim leguntur. Nobis abunde iamdudum suffecit uerissimum diui Hieronymi dictum, Quicquid (credendorum et necessario agendorum) de sacris scripturis non producitur, eadem facilitate contemnitur qua probatur.  \pend
\phantomsection
\addcontentsline{toc}{subsection}{\textit{Cuius testimonium temporibus suis confirmatum est, in quo positus sum ego praedicator et apostolus. Veritatem dico, non mentior, doctor gentium in fide et ueritate. }}
\subsection*{\textit{Cuius testimonium temporibus suis confirmatum est, in quo positus sum ego praedicator et apostolus. Veritatem dico, non mentior, doctor gentium in fide et ueritate. }}\pstart Ad hoc Christus ueniens sese patri pro nobis obtulit, ut suo illo tempore toti mundo testaretur diuinae charitatis erga genus humanum beneficium, dudum promissum patriarchis ac prophetis, in redemptionem omnium hominum, ut credentes per eum saluentur. Noluit enim occultum esse, quod sacrificiorum figuris et propheticis promissionibus in ueteri testamento uoluiresse tam multis testatum. Quin et ego Paulus ad hoc nostrae redemptionis euangelium praedicandum assumptus sum ab eodem Christo domino, segregatus ac electus post alios, ut toto orbe gentibus hoc praedicem, et hanc dei tantam gratiam adnunciem. Nuncius missus a Christo et apostolus, quanquam cum Christo non uixerim, ex misericordia tamen sua me conuertit, ex persecutore et blasphemo in apostolum ac ueritatis doctorem ad omnes gentes. Veritatem dico, non mentior.) Modestissime iurat, quod notorium esset, et negari non poterat. Celebris, inquit, fama iactatur insaniae meae, cum spirans minarum et caedis in discipulos, subito in multorum comitatu e coelo uocatus sum, et a Christo conuersus. Deinde cum Barnaba a spiritu sancto segregatus in gentes, attestatione non modica miraculorum, profectu praedicationis maximo, a Iudaea incipiens per Asiam, usque  ad Illyricum omnia repleui euangelio Christi. Non potest esse dubia apostolatus mei uocatio, et quod me Christus  \pend
\section*{AD TIMOTH. CAP. II. }
\marginpar{[ p.485 ]}\pstart gentium doctorem dederit, testantur signa apostolatus mei: qualem denique  me gesserim in gloriam Christi, non in meum compendium aliquod temporale, nihil miscendo doctrinae, quam in me Christus non fuerit loquutus. Omnia sine hypocrisi et figmento, dextre ac plane ueraciter tradidi, ut accepi a domino. Doctorem se nominat, non in Vniuersitate, ut loquuntur, promotum: non enim Aristotelem audierat antea, et parua logicalia, sed spiritu sancto promotus, docebat sine inter missione uerbo, exemplo et scripturis, non tam nomine doctor q̃officio, assiduoque ́ docendi studio. Tales etiam non sic promoti, doctores erant Origenes, Auguslinus, Hieronymus, et similes. Quanquam eruditionis et fidei testimonium habaere, et factis consequi non fuerit inutile, sed sine quaestu, fastu et hypocrisi: utqueprofessionis suae specimen ostendant studio, fide et exemplis, uerbo et facto, non serico uel annulis, ne inter comessationes et pocula pro gratia nummorum, et personarum respectu, officium et authoritas docendi in ecclesia commendetur indignis, cum fide et ueritate doctores agant, ut scripturis canonicis edocti, fidem fideliter doceant, sine permixtione falsorum dogmatum, quae resipiant sensus car, nis, ambitionem honoris, et quaesitum.  \pend
\phantomsection
\addcontentsline{toc}{subsection}{\textit{Volo ergo uiros orare in omni loco, leuantes puras manus, sine ira et disceptatione. }}
\subsection*{\textit{Volo ergo uiros orare in omni loco, leuantes puras manus, sine ira et disceptatione. }}\pstart Docuerat orandum pro consequenda uitae tranquillitate in hoc mundo, et ut ad agnitionem ueritatis omnes peruenirent saluandi. Nunc uiros docet ac mulieres instruit, quomodo orandum, et de aliis necessariis, incidenter et obiter, foeminarum mores instituendo: primum autem uiros, Quos, inquit, uolo orare in omni loco. Nondum apostolorum temporibus Christianis erant amplissima templa, sufficiebat locus profamilia quacunque , longo tempore in criptis praedicatum: ubicunque  ecclesia erat, duo scilicet aut tres uel plures, ibi locus orationis et praedicationis, in cubiculis, in agro, in naui, in montibus ac syluis: ubique  orationis locus, ubi spiritus sancti templum consecratum habetur, quod sunt fidelium animi. Idipsum docuit ueritas Christus, Cum, inquit, oraueris, ingredere cubiculum, et ora patrem in abscondito. Multi hodie asueuerunt, ut non nisi in ecclesia orare ualeant, et exaudiri credant: si tamen uere orant, et non potius multa sensuum et mentis distractione adnumerent potius uerba, que ̃ ut corde, affectu feruorequespiritus subleuentur in deum, in considerationem fragilitatum, defectuum ac miseriarum, quarum consideratione legitima oratio sit promouenda. In omni, inquit, loco.) Consecratus est a deo mundus totus, et homo orationi. Vt non in templo Hierosolymis, nec Romae, sed in spiritu et ueritate sciamus orandum. Id quod Christus docuit dominus, quiquese sic orantes exauditurum promisit. Caetera diuinae potius prouidentiae committantur. Leuantes puras manus.) Non lotas, Pharisaeorum more, sed puras ab omni uiolentia, a sanguine innocentum, a munerum acceptione, usura, furto, a foedis libidinosisquecontactibus, et ab omi malo opere. Nec uult Paulus extendi manus pharisaico more, sed desiderantium ex petitionis ardore. Non obsunt, sed prosunt quaedam ceremoniae orantibus, sed non nimiae, et quae mentis distractionem uel hypocrisim mouere possint. Sine ira et disceptatione.) Vbi docet nihil a deo postulandum, nisi prius orans proximo conciliatus sit charitate Christiana. Solet uirorum animos occupare furor et ira, ut superbia mulieres, incontinentia ac liuor. Ira autem charitati contraria, orationi locum omnino praecludit. Non ignoscit deus nobis, nec dona pietatis largitur, nisi peccantibus in nos antea indulserimus, nisi fratri offensam remiserimus, et ex animo dilexe\pend
\section*{COMMENT. IN I. EPIST. }
\marginpar{[ p.486 ]}\pstart rimus propter deum.  \pend
\phantomsection
\addcontentsline{toc}{subsection}{\textit{Similiter et mulieres in habitu ornato, cum uerecundia et sobrietate ornantes se, et non in tortis crinibus, aut auro, aut margaritis, uel ueste pretiosa, sed quod decet mulieres, promittentes pietatem per opera bona. }}
\subsection*{\textit{Similiter et mulieres in habitu ornato, cum uerecundia et sobrietate ornantes se, et non in tortis crinibus, aut auro, aut margaritis, uel ueste pretiosa, sed quod decet mulieres, promittentes pietatem per opera bona. }}\pstart Mulieres Apostolus docet post uiros, de idoneis earundem moribus, ne de forma superbiant amictu uel ornatu, quod agnatum mulieres patiuntur et gaudent, alioqui non pessimae, imo et bonae, castae ac prudentes, quae de speciei ornatuumque ́ cura uix sibi praecipere possunt. Admittit quidem Paulus, ut habitu ornato sint pro ingenio muliebri, non denique  sordido amictu, uel nimium neglecto, quo abominatio sint maritis, et occasio forsitan exteras adamandi, sed modum praescribit, ut decet castas, uerecundas, et minime irritatrices libidinis. Munde quidem, eleganter ac honeste uestiantur, hoc est, ne nuditate et ornatu superfluo, pretioso uel curioso meretricibus assimilentur, appareantqueaffectu placendi in publico, maxime uero in coetu uel ecclesia, quod Christianis maritis nec placere potest neque  debet. Ornentur autem legitimis monilibus, uerecundia scilicet et sobrietate, ex uultu, oculis, incessu, uerbis, subindicando moribus ipsis, inuitas sese offerri uirorum aspectibus, nolle ab aliquo desiderari ad malum, malle semper contegi, et aegre uidere uiros, multo molestius uideri. Quin et propter maritum uerecundia non nudentur: gratiosissimum enim ornamentum uerecundiam habet, quisquis est maritus sapiens et probus. Sobrietate.) Nulla uinolentia aut crapula aliquatenus notetur mulier, quae lubricitatis et libidinis certi sunt testes, teste Hieronymo. Quae et impedimenta praebent foetibus optatissimis, insipientiae et garrulitatis uitia secum adferunt, mendaciorum et litium matres, nihilque  secretum seruaturae, seminaria malorum. Nascitur e sobrietate castitas, ut sunt sorores, continent inhonesta desideria, et libidines impediunt. Ornantes se, non in tortis crinibus.) Non compositione capillorum, et implicatura argenti et auri, non adulterino colore, uel cerussa fucoque  peregrino, quibus iniuriam inurunt formae naturali, magis autem uirtutum exercitiis ornentur. Non auro aut margaritis.) Non onerent impensis maritos, colligant potius, et in thesauros reponant pro haeredibus, uel certe sanctius subueniant indigentibus, in opera pietatis uerae conseruent. Non addant superbiae naturali fomenta luxuriae, non certare foeminae uicissim optent de sumptibus, sed de modestia, sapientia et maturitate: pretium margarirarum et auri felicius expendetur in pauperes. Veste (inquit) non uestibus: quandoquidem superfluae uestes Christianos dedecent, et excusari non possunt, etiam in diuitibus. Non ad luxum amiciantur, sed ad honestatem et necessitatem. Vnica quidem non sufficit, sed superflua non decet, pretiositas intolerabilis. Scandalizat, irritat, zelum suscitat et exemplum, extenuat rem familiarem, difficile defenditur a tineis, et non sine negotio curatur ac solicitudine. Sed quod decetmulieres profitentes pietatem, aemulentur diligentius. Propensiores ipsas foeminas natura constituit ad cultum dei, ad multas ceremonias ac superstitiosas. Has discant non esse pietatem, frequentare ecclesias, gestare oratoria uel corollas, preculas mussitare, admirari cantus, et organis delectari, aspicere picturas idolorum, inclinari, incendere cereos, aqua conspergere sepulchra et seipsas. Haec quidem pietatis signa habita sunt iamdudum etiam apud ethnicos, sed minime sunt pieras, prorsus autem uersa sunt in impios usus. Pietatem magis profiteantur fideles foeminae et uiri, recordatione suorum peccatorum, suspiriis pro gratia et indulgentia, emendatione morum, et sancto proposito, attentius auscultari uerbis di\pend
\section*{AD TIMOTH. CAP. II. }
\marginpar{[ p.487 ]}\pstart uinis, audita meditari et ruminare, ut deinde liberis ea domi recenseant ac inculcent, ne quid aduersi uerbo dei admitti sinant in pueris, nec exemplis propriis in citent ad irreligiosa uerba uel opera, idquea tenera aetate et primis annis, nondum deprauatis animis. Orationes breues, sed feruidas dicant ex consideratione necessitatum profusas, et ex animo humiliato: non multas, nec toties repetitas pro more et praefixo numero. Non enim quantitatem precum iuxta numerum et tempus, sed qualitatem deus exigit et deuotionem. Offerant monetam probatam charitate, non loquacitate: domi orent, familiae prospiciant, marito obsequantur, et auxilientur in cura familiae. Caueant ab aspectu inuerecundo et uago. Lumen non lapidibus uel mortuis accendant, sed operi necessario, utili ac tenebris. Fides sine aquarum aspersione suffecerit. Pro omnibus istiusmodi exercitiis arripiant bona opera. Haec autem sunt, ut ministrent liberis, marito, familiae, uicinis, pauperibus, praegnantibus, parientibus, puerperis, infirmis, domui incumbant ornandae, lauandae, purgandae, prospiciendae die noctuque , iuxta Salomonis canticum. Omnia autem haecin fide peragant, persuasae certo talia deo placere, esseque  ipsissimae pietatis officia, et opera a deo imposita, obedientiaeque  sanctae merita, et fructus optimi. Si quando uacat, maxime aunt feriis, lectioni incumbant, pueros ididem doceant, moneant, corrigant, et a malo exemplo in uerbis et operibus solicite sibi temperent.  \pend
\phantomsection
\addcontentsline{toc}{subsection}{\textit{mulier in silentio discat cum omni subiectione. Docere autem mulieri non permitto, neque  dominari in uirum, sed esse in silentio. }}
\subsection*{\textit{mulier in silentio discat cum omni subiectione. Docere autem mulieri non permitto, neque  dominari in uirum, sed esse in silentio. }}\pstart Viris quidem permittitur, imo iniungitur ut in coetu, quae de auditis non intelligunt, inquirant a senioribus et presbyteris in ecclesia docentibus: et si quid eis sedentibus spiritus reuelat, loquantur cum modestia et sine disceptatione, ad profectum ecclesiae. Non prohibetur a uerbo ueritatis proponendo uir qualiscunque  scripturas quacunque  in lingua, sane intelligens. Voluit em̃ Paulus, ut passim quini aut seni pariter prophetarent, etiam intra ecclesiam. Cantica autem et organa psallentiumqueclamores, ac tumultus inconditos, et studium assiduum canendi non docuit. In cordibus psallere, et spiritualibus hymnis domino laudes dicere, et canere mente, non tantum spiritu, hoc est sono, Christianos omnes docet. Mulieres uero in ecclesia sileant, domi uiros interrogent, si quid discendum fuerit, et in silentio:ne sensu suo abundare contingat, et litigare incipiant, uerbis et sensu proprio sibi complacentes. Docere mulieres non permitto, Paulus inquit: non ego ausim huic sententiae uerboqueadiicere, in coetu uirorum, eatenus ut liceat foeminis docere in ecclesia foeminarum: sensu enim infirmae sunt, et non satis stabiles. Sed uiros presbyteros senes annis et moribus habeant agmina foeminarum, a quibus publice huiusmodi congregatio instruatur. Nemo uero negauerit, aliquot nonnunquam foeminas conferre posse laudabiliter et debere quoque  de auditis ac praelectis. Et legere sacra scripta piorum doctorum etiam poterunt in coetu foeminarum, sic tamen ut intelligant, et non ignota in lingua. Verum sine intellectu orare, legere, canere, psallere, opus est sine omni fructu ac laude, maxima et perniciosa temporis perditio, et operum uel piorum uel alias utilium grauis iactura, qualia introduxit in ecclesiam sola pigritia ignorantiaqueet superstitio monachorum, Olim et in Italia et in Germania quoque , uel teste Hieronymo, doctae erant sanctimoniales, quibus lectiones et cantica Latina lingua seruire poterant, et deuotioni conferre. Hodie in Germania et Gallia nihil intelligunt eorum quae iactant, uel sine iudicio aliquantulum suspicantur: quas non tam obedientia, quam hominibus et stultis monachis promittunt, excusas, que  dementia culpar. Quid enim  \pend
\section*{COMMENT. IN I. EPIST. }
\marginpar{[ p.488 ]}\pstart inutilius institui poterat? quid stultius excogitari, quam ut tantundem temporis die noctuque  expendant infelicissimae muliereulae in recitationibus, imo laboriosis cantibus uerborum, nihil prorsus illis significantium? Reliquum autem temporis ab huiusmodi tam molesta operis fatigatione, otio et remissione resolui et deperdi, ut omne earum tempus longe sit illis infructuosius, que ̃ somnium iners: interim tamen sibi de cultu diuino uehementer complacent, iactant merita gloriosa, quae et datis literis et sigillis uenditant benefactoribus: nec se patiuntur aliquatenus ab huiusmodi institutis diuelli. Loquimur nunc de eis monialibus, quae optimae habentur et religiosiores, misere deceptis a stultis monachis. Qui curae suae demandatas mulierculas, fidelius regere, et secundum canonicas scripturas docere, de domini dei sui certissima uoluntate, et de mandatorum illius obseruantia, qua domino seruuli seruire deberent ac possent. Quis enim domini sui seruitium seruus iactare potest, dum obmissis eis negotiis quae dominus prosequenda serio iniunxit, illis potius incumbit quae minime iussit, et suapte natura charitati non prosunt, quae est finis legis. De eis monialibus, quae sine disciplina uiuunt genio, miror quid mundus cogitet quantumlibet malus, et quid cogitent parentes, dum filias sic prostituunt miserrime, idquetitulo diuini cultus, ubi tantum carni seruitur, et stolidis monachis confessoribus. Haec et similia maiore querela opus habent, quam ego ualeam consequi uerbis quibuscunque . Nec dominari in uirum.) Id enim foedum marito est et ridiculum, ac fomentum superbiae uxoribus, uirique ́ contemptus. Monere quidem maritum potest et debet, rogare, blandiri, et uincere malum in bono: interim tamen esse in silentio, parce loqui et breuiter, cum reuerentia ac timore: illico a uiro monita tacere, taceat: irato marito non respondeat, tempus captet opportunius: semel non crebrorepetat quae dicenda uiderit. Lingua mulieribus tanto diligentius moderanda est, quanto natura propensiores sunt ad cauilla et multiloquium. Nec circuncursari eas delectet per alienas aedes, non com morari in plateis, non foris plurimos ac in triuiis miscere sermones, domi agant potius cum silentio.  \pend
\phantomsection
\addcontentsline{toc}{subsection}{\textit{Adam enim primus formatus est, deinde Eua: et Adam non est sedu ctus, mulier autem seducta in praeuaricatione fuit. }}
\subsection*{\textit{Adam enim primus formatus est, deinde Eua: et Adam non est sedu ctus, mulier autem seducta in praeuaricatione fuit. }}\pstart Ne quopacto superbiendi ansam mulier sibi uendicet dominandi uiro, admonet Apostolus ut meminerint suae originis, quam a uiro sumpta accepit: qui primo conditus a deo, natura quoque  perfectior, absolutior, sapientia clarior, et ad seductiones intentatas cautior atque  fortior erat. Quo sopore merso, ex costa illius Eua diuina uirtute formata est:ut unum corpus de uno corpore, et os ex ossibus prodiens, amoris uinculo arctius astringerentur. Adam quidem prior uirtute et tempore, Eua posterior utroque :ideoque  duci ea, instrui, parere et sequi debet, agno. scerequese dei ordinatione subiectam. Adam non fuit deceptus, non eum serpens aggredi primum audebat, qui eum fortassis minime decepisset sapientiorem foemina: non fefellisset tam impudenti mendacio, non ambiuisset excellentiam, non colloquium tam friuolum tulisset, nisi in gratiam mulieris consortis, amicae. Cui cum haec placere uideret Adam, delicias suas contristare non poterat, sed acquie uit, roganti mulieri attendens, oblitus praecepti uel negligens. Mulier autem seducta mendaciis, inuidia, dolo, blasphemia diaboli, permissione ac dispensatione diuina, nedum iusta, sed iustitiae originem nacta, qua proficere homo debuisset et progredi gratia dei praesente, acceptoquepraecepto, retrocessit ipsa Eua, transgressa mandatum, et uerbis dei non credens, sed infidelitatis primum peceato su  \pend
\section*{AD TIMOTH. CAP. II. }
\marginpar{[ p.489 ]}\pstart perata, superbiaque  et auaritia, gula et luxuria decepta est, et uiro insuper insidiata, blanditiis et instantia uicit, et in prȩuaricationem traxit, ut pariter cum ea coeperit dubitare, si uere haec praedixeric uxor, scilicet imminere mortem, quam post gustum uideret superstitem ac uiuentem, ut et ipse peccaret. Cui semel seductrix facta, posthac directricis et doctoris munus ac officium praesumere et praestare nun quam par fuerit. Semel tam perniciosae suasioni consentiens Adam et seductus, meritosuspectam habuit omnem mulieris imormanenem et igemuim; ex ipsa experimento sumpto fragilitatis et insipientiae, cum perfecta esset, nunc bis misera et sui admonita, praesumere doctrinam non audeat solidae uernalis et sapientiae.  \pend
\phantomsection
\addcontentsline{toc}{subsection}{\textit{Saluabitur autem per filiorum generationem, si permanserit in fide et dilectione, et sanctificatione, cum sobrietate. }}
\subsection*{\textit{Saluabitur autem per filiorum generationem, si permanserit in fide et dilectione, et sanctificatione, cum sobrietate. }}\pstart Subdita quidem mulier sit uiro, et se humiliter gerere studeat inter uiros ac pudice, in ecclesia quoque  nihil praesumat, agnoscens mentis imbecillitatem et inconstantiam, ut sibi docendi munus in ecclesia non arroget. Verum neutiquam ob id contemnenda est uiro, quam per fillorum et liberorum generationem beatam fecit dei munificentia, et insigni munere illustrauit. Quid enim admirabilius generatur in rerumnatura, ubi omnia tam sunt miraculosa,  in utero soeminae: non tantum humanum corpus, sed et anima immortalis et rationalis creando in funditur, et infundendo creatur ?Cuius generationis, nutritionis, educationis pariter et instructionis officium dominus in matrem contulit et mulierem, incunabulaque  tradendae pietatis et profectus matribus commisit circa prolem:ut sicut lacte pascunt corpusculum, sic solido cibo nutriantur a uiris: in fide ac sacris dogma tibus uiros foeminae perfectos audiant ac sequantur. Curet igitur liberorum mater, ut in ea quam a teneris unguiculis a se fidem et dilectionem des imbiberunt filii, in eis permaneant, iugiter inculcando fidem et timorem dei, ut sancti sint moribus, ab omni criminum contagione perpetuo sibi caueant. Carni et sanguini non indulgeant, dei mandata quibus eos mater edocuit, semper memoriter cogitantes, ut sancti sint et permaneant corpore et spiritu. Sobrietatem in primis soli cite curent, ut cibo, potu, somno, otio, lusibus, indulgentia, sodalitate, non nimis nec periculose imcumbant. Potius autem matres studeant liberis persuadere, quos domino genuerunt et proximis, fidem, dilectionem, sanctimoniam, pudicitiam, sobrietatem in uerbis et moribus. Ad hoc necessarius est foeminis matribus rerum illarum usus et studium, ut puellae didicerint quod matres docere possint: perniciosissima itaque  negligentia est, si uirgines insolentius educentur, si sine timore dei, uerecundia, modestia, pudicitia, sobrietate et religione succrescamnt. Quomodo enim docebunt, quod nunquam didicerunt ? Quid sperandum de uxoribus, quae insolentia puellari deprauatae sunt, ut nesciant quid sit uerecundia uel honestas?  \pend
\endnumbering\beginnumbering\section{CAPVT III.}
\phantomsection
\addcontentsline{toc}{subsection}{\textit{\huge\textbf{F}\normalsize Idelis sermo. }}
\subsection*{\textit{\huge\textbf{F}\normalsize Idelis sermo. }}\pstart Quae tibi dilecto filio meo ad populum in ecclesia docendum scribout Christi apostolus, et tibi in eodem pater filio meo scribo, certissima habeas: et ea quae tam nunc tibi dixi de uiris et mulieribus, q̃ nunc scribenda cogito, scias a domino reuelata atque  fidelia. Christi domini doctrinam esse, qui in me loquitur, non hominum uerba, id est mendacia. Nihil ex me loqui in ecclesia audeo, non opiniones ac fabulas comminiscor, fidelis est omnis dei sermo et efficax, quem mihi disseminandum credidit in orbem fidelibus.  \pend
\phantomsection
\addcontentsline{toc}{subsection}{\textit{Si quis episcopatum desiderat, bonum opus desiderat. }}
\subsection*{\textit{Si quis episcopatum desiderat, bonum opus desiderat. }}
\endnumbering
\end{pages}
\end{document}
        
